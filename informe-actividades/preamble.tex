%%%%%%%% PREÁMBULO %%%%%%%%%%%%
\usepackage{pdfpages}
\usepackage{lmodern} % Provides scalable fonts to avoid font shape issues
\title{FORMATO INFORME AUTOMÁTICA}
\usepackage[spanish,es-tabla]{babel} % Indica que escribiremos en español
%\usepackage[latin1]{inputenc}
\usepackage[utf8]{inputenc} % Indica la codificación: ISO-8859-1(latin1) o utf8  
\usepackage{amsmath} % Comandos extras para matemáticas
\usepackage{mathrsfs}
\usepackage{mathtools}
\usepackage{physics}
% \usepackage{geometry}
% \geometry{a4paper, margin=1in}
\usepackage{amssymb} % Símbolos matemáticos
\usepackage{graphicx} % Incluir imágenes
\usepackage{color} % Para colorear texto
\usepackage{subfigure} % Para subfiguras
\usepackage{float} % Permite usar el especificador [H]
\usepackage{capt-of} % Etiquetas fuera de elementos flotantes
\usepackage{sidecap} % Para poner el texto de las imágenes al lado
\sidecaptionvpos{figure}{c} % Alineación vertical del texto
\usepackage{caption} % Para controlar la numeración de figuras
\usepackage{commath} % Funcionalidades extras para diferenciales, integrales, etc.
\usepackage{graphicx, amsmath, amsthm, latexsym, amssymb, amsfonts, epsfig, float, enumerate, color, listings, graphicx, fancyhdr, array, bm}
\usepackage{tikz,circuitikz,pgfplots}
\pgfplotsset{compat=1.8}
\usepackage{cancel} % Para cancelar expresiones (\cancelto{0}{x})
% \usepackage{xcolor} % Agregar color a los cuadros
\usepackage{anysize} % Para personalizar los márgenes
\marginsize{2cm}{2cm}{2cm}{2cm} % Márgenes: izquierda, derecha, arriba, abajo
\usepackage{appendix}
\renewcommand{\appendixname}{Apéndices}
\renewcommand{\appendixtocname}{Apéndices}
\renewcommand{\appendixpagename}{Apéndices} 
\usepackage[colorlinks=true,plainpages=true,citecolor=blue,linkcolor=blue]{hyperref}
%\usepackage{hyperref} 
\usepackage{fancyhdr} 
\pagestyle{fancy}
\setlength{\footskip}{35.11942pt}
\fancyhf{}
\fancyhead[L]{\footnotesize AUTOMÁTICA} % Encabezado izquierda
\fancyhead[R]{\footnotesize UTP}   % Encabezado derecha
\fancyfoot[C]{\thepage }  % Pie de página central
\fancyfoot[L]{%
	\includegraphics[width=0.1\textwidth]{imagenes/LogoU.png}%
	\includegraphics[width=0.2\textwidth]{imagenes/logo_automatica.png}%
} % Pie de página izquierda
\renewcommand{\footrulewidth}{0.4pt}
\usepackage{booktabs}
\usepackage{listings} % Para incluir código fuente
\definecolor{dkgreen}{rgb}{0,0.6,0} % Definir colores
\definecolor{gray}{rgb}{0.5,0.5,0.5} 
\usepackage{multirow}
\lstset{language=Python,
	keywords={import,from,as,def,class,return,if,elif,else,for,while,break,continue,try,except,finally,with,lambda,pass,yield,global,nonlocal,assert,async,await},
	basicstyle=\ttfamily,
	keywordstyle=\color{blue},
	commentstyle=\color{gray},
	stringstyle=\color{dkgreen},
	numbers=left,
	numberstyle=\tiny\color{gray},
	stepnumber=1,
	numbersep=10pt,
	backgroundcolor=\color{white},
	tabsize=4,
	showspaces=false,
	showstringspaces=false}
\newcommand{\sen}{\operatorname{\sen}} % Definir comando para el seno


\usepackage[capitalise, noabbrev]{cleveref}
\crefname{table}{\spanishtablename}{\spanishtablename}

% Define custom colors
\definecolor{MyDarkBlue}{RGB}{44,62,80}    % Titles and accents
\definecolor{MyAccent}{RGB}{52,152,219}      % Accent for elements
\definecolor{MyLightGray}{RGB}{245,245,245}  %
