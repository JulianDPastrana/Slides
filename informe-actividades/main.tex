\documentclass[10pt,xcdraw, spanish]{article}

\usepackage{graphicx}
\usepackage{subcaption}
\graphicspath{ {figures/} }
 % Carga el preámbulo
\begin{document}
	%%%%%%%%%%%%%%%%%%%%%%%%%%%%%%%%%% PORTADA %%%%%%%%%%%%%%%%%%%%%%%%%%%%%%%%%%%%%%%%%%%%
\vspace*{0.5cm}
\begin{center}
	\newcommand{\HRule}{\rule{\linewidth}{0.5mm}}
	\vspace*{-1.5cm}
	% \textsc{\huge Universidad Tecnológica\\ \vspace{5px} de Pereira}\\[1.5cm]
	% \includegraphics[width=0.2\textwidth]{imagenes/Logo_UTP.png}
	% \vspace{1.5cm}
	\textsc{\LARGE Informe de actividades en el marco del programa: \\ 
		\vspace{0.4cm}
		XXXXXXXXXXX}\\
	\vspace{2cm}
	\HRule \\[0.4cm]
	{ \bfseries Nombre: Julián David Pastrana Cortés }\\[0.4cm]
	{ \bfseries Contrato de servicios: 5551 }\\[0.4cm]
	\vspace{1.5cm}
	\includegraphics[width=0.28\textwidth]{imagenes/LogoU.png}
	\includegraphics[width=0.35\textwidth]{imagenes/logo_automatica.png}
	\HRule \\[1.5cm]
	\vspace{1.5cm}
	\vspace*{0.5cm}
	\textsc{\textbf{\Large Grupo de investigación en Automática } }\\
	\vspace{3.5cm} 
	\begin{center}
		{\large \today}
	\end{center}
\end{center}
\newpage
   % Carga la portada y otros elementos de layout
	
	\tableofcontents 
	\newpage
	
	\section{Información general del contrato}
	\begin{table}[H]
		\centering
		\begin{tabular}{>{\arraybackslash}m{8cm} >{\arraybackslash}m{6.5cm}}
			\toprule
			\textbf{Rol} & \vspace{2mm}Contratista - Estudiante de Doctorado\vspace{2mm}\\\hline
			\vspace{2mm}
			\textbf{Contrato de servicios No.}\vspace{2mm} & \vspace{2mm} 5551 de 2025\vspace{2mm}\\\hline
			\vspace{2mm}
			\textbf{Objeto del contrato} \vspace{2mm} & \vspace{2mm} Prestación de servicios profesionales para el Desarrollo de metodología de gamificación para entrenamiento de niños con trastornos de impulsividad ALIANZA CIENTÍFICA CON ENFOQUE COMUNITARIO PARA MITIGAR BRECHAS DE ATENCIÓN Y MANEJO DE TRASTORNOS MENTALES RELACIONADOS CON IMPULSIVIDAD EN COLOMBIA - ACEMATE MINCIENCIAS CONTRATO 790-2023 
			\vspace{2mm}\\\hline
			\textbf{Período del informe} & \vspace{2mm} 01 de marzo de 2025 al 31 de marzo de 2025\vspace{2mm}\\\hline
		\end{tabular}
	\end{table}
	
	\subsection{Descripción general de la vinculación}
	Como resultado de la vinculación se contribuye al alcance de el objetivo 3 del programa de investigación: ALIANZA CIENTÍFICA CON ENFOQUE COMUNITARIO PARA MITIGAR BRECHAS DE ATENCIÓN Y MANEJO DE TRASTORNOS MENTALES RELACIONADOS CON IMPULSIVIDAD EN COLOMBIA
	
	\subsection{Objetivo general}
	Desarrollar una metodología de gamificación para entrenamiento de niños con trastornos de impulsividad ALIANZA CIENTÍFICA CON ENFOQUE COMUNITARIO PARA MITIGAR BRECHAS DE ATENCIÓN Y MANEJO DE TRASTORNOS MENTALES RELACIONADOS CON IMPULSIVIDAD EN COLOMBIA - ACEMATE MINCIENCIAS CONTRATO 790-2023.
	
%	\section{State of Art}

%%%%% Emotion detencton form wearable devices

Wearable devices allows to measure physiological biomarkers in a non-intrusive fashion. The collected data may correlate with stress levels, emotion states and biological responses. Those measurement often include Hear Rate Variability (HRV), Electrodermal Activity (ADA), Heart Rate (HR) and three axis acceleration (ACC) \cite{Vos2023}. Advance in machine learning techniques have allowed to construct models that predict sentimental state based on recorded biomarkers, understanding that human health extends to mental well-being.

Authors in \cite{Zhu2023} evaluate a set of traditional machine learning algorithms to predict people's stress based on EDA activity, including K-Nearest Neighbor, Support Vector Machine (SVM), Naive Bayes, Logistic Regression and Random Forest. Those model were trained fed by a feature vector containing statistic of EDA data, and raw data itself for comparing classification results. Researchers reported best performance for SVM model, and no consistency training with raw data or feature vector.

However, traditional machine learning models lack of expressiveness and capacity to generalize well \cite{Yang2023}. Moreover, feature often rely on statistics, forgetting sequential dependencies in the data. A closer overview dive us to a multi-modality scenario, that is, different measurement could be taken at its own frequency sample.

Deep learning techniques allow us to handle multi-modal measurement, leveraging data structure like time dependencies for sequential recordings, or spatial patters for images. That can be done through feature representation from multiple data entities. However, that arise the challenge of how to combine data structures from different sources \cite{Baltrusaitis2019,Liang2024}. 

The work developed by \cite{Venugopalan2021} integrate multiple data modalities using auto-encoders for genetic data and 3D convolutional neural network (CNN) for imaging data, demonstrating outperform over shallow models and single modality.




	\section{Metodología}
La metodología propuesta para el desarrollo del proyecto se estructura en cuatro actividades principales, distribuidas a lo largo de siete meses, con entregables específicos para cada una:

\subsection{Actividad 1: Planteamiento de la metodología, técnicas y modelos a implementar (Meses 1-2)}
En esta etapa inicial se realizará la conceptualización y diseño de la metodología de trabajo. Se definirán las técnicas de análisis y los modelos computacionales que se implementarán para el procesamiento de señales EEG. Esta actividad incluye la revisión de fundamentos teóricos, selección de herramientas y establecimiento del marco de trabajo.

\textbf{Entregable:} Informe técnico que documenta la metodología propuesta, las técnicas seleccionadas y los modelos a desarrollar.

\subsection{Actividad 2: Revisión técnica del estado del arte (Meses 3-4)}
Se llevará a cabo una revisión exhaustiva y sistemática del estado del arte en análisis de señales EEG, identificando las metodologías existentes, técnicas avanzadas y tendencias actuales en el procesamiento de señales cerebrales. Esta revisión permitirá contextualizar el trabajo y validar las decisiones metodológicas.

\textbf{Entregable:} Informe técnico con el análisis del estado del arte y la fundamentación teórica del proyecto.

\subsection{Actividad 3: Desarrollo e implementación de modelos de análisis de EEG (Meses 5-6)}
Esta etapa constituye el núcleo del proyecto, donde se desarrollarán e implementarán los modelos de análisis de señales EEG. Se realizará el montaje y configuración de los modelos para diferentes tareas específicas, incluyendo pruebas, validación y optimización de los algoritmos desarrollados.

\textbf{Entregable:} Código fuente de los modelos implementados, documentado y funcional.

\subsection{Actividad 4: Documentación de resultados consolidados (Meses 6-7)}
En la fase final se consolidarán todos los resultados obtenidos durante el proyecto. Se documentarán los hallazgos, análisis de desempeño de los modelos, conclusiones y recomendaciones. Esta documentación servirá como soporte integral del proyecto.

\textbf{Entregable:} Informe técnico final con la documentación completa de resultados, análisis y conclusiones del proyecto.
	\section{Conjunto de Datos}

\subsection{Toadstool: Un conjunto de datos para el entrenamiento de máquinas de inteligencia emocional que juegan a Super Mario Bros}

El conjunto de datos de libre acceso Toadstool es una colección de registros en video, sensores e información demográfica obtenidos de diez individuos mientras jugaban Super Mario Bros. La selección de los participantes se realizó buscando una amplia variedad en cuanto a experiencia previa con videojuegos, incluyendo desde personas que apenas habían jugado alguno en su vida hasta aquellas con una extensa trayectoria desde la infancia, con edades comprendidas entre los 26 y 48 años. También se buscó mantener un balance en el género, participando cinco hombres y cinco mujeres. Los autores en \cite{toadstool} señalan la presencia de anomalías, tales como ninguna o muy poca actividad detectada por los sensores, así como diferencias significativas entre la actividad registrada al inicio y al final de las sesiones de juego.

El desempeño de los participantes fue evaluado según el número de niveles completados en un tiempo determinado y el número de muertes ocurridas durante la partida. Dicho puntaje se mantuvo en secreto entre los participantes para evitar que se rindieran o relajaran durante el juego. Adicionalmente, se les incentivó mediante una recompensa con el fin de mantener un desempeño competitivo. Un resumen de las características de los participantes se presenta en la \cref{tab:datos_participantes}.


\begin{table}[h!]
	\centering
	\resizebox{\textwidth}{!}{%
		\begin{tabular}{|c|c|c|c|c|c|c|c|}
			\hline
			\textbf{ID} & \textbf{Edad} & \textbf{Sexo} & \textbf{Mano dominante} & \shortstack{\textbf{Horas}\\\textbf{por semana}} & \shortstack{\textbf{Años}\\\textbf{de actividad}} & \shortstack{\textbf{Experiencia}\\\textbf{previa}} & \shortstack{\textbf{Puntaje}\\\textbf{del juego}} \\
			\hline
			0 & 26 & Hombre & Derecha  & 4--8 & 22 & Mucha  & 17,100 \\
			1 & 48 & Hombre & Izquierda & 0--1 & 1  & Poca   & 3,000  \\
			2 & 28 & Hombre & Derecha  & 0--1 & 0  & Ninguna& 300    \\
			3 & 32 & Hombre & Derecha  & 4--8 & 4  & Algo   & 13,300 \\
			4 & 32 & Mujer  & Derecha  & 0--1 & 5  & Algo   & 6,400  \\
			5 & 30 & Mujer  & Derecha  & 0--1 & 5  & Poca   & 2,700  \\
			6 & 35 & Hombre & Izquierda & 1--4 & 30 & Mucha  & 14,300 \\
			7 & 34 & Mujer  & Derecha  & 1--4 & 14 & Algo   & 3,800  \\
			8 & 31 & Mujer  & Derecha  & 0--1 & 2  & Poca   & 200    \\
			9 & 27 & Mujer  & Derecha  & 0--1 & 5  & Poca   & 10,600 \\
			\hline
		\end{tabular}%
	}
	\caption{Características demográficas y experiencia de los participantes en relación con su desempeño en el juego. Se incluyen datos como edad, sexo, lateralidad, dedicación semanal, años de experiencia, nivel de experiencia previa y puntaje obtenido.}
	\label{tab:datos_participantes}
\end{table}

En el conjunto de datos, se incluye para cada participante un video de su rostro durante la sesión de juego, grabado a una resolución de 640×480 píxeles y a 30 fotogramas por segundo. Además, se registran las acciones realizadas mediante el mando de juego para controlar al personaje, así como los datos recopilados por la pulsera Empatica E4. La Empatica E4 es un dispositivo que permite la adquisición en tiempo real de datos fisiológicos, como la actividad electrodérmica (EDA), el pulso de volumen sanguíneo (BVP), la temperatura de la piel y la aceleración en tres ejes \cite{garbarino2014empatica}.




%	\section{Models}

\subsection{Problem Setting}

Consider a time-series vector collecting sequential data observed across $P$ output at some time instant \( t \), denoted as \( \boldsymbol{v}_t \subseteq \mathbb{R}^{P}\). We want to include \( T \in \mathbb{Z}^{++} \) sequential observations simultaneously back as a vector \( \boldsymbol{x} \) belonging to some input space \( \mathcal{X} \), that is

\[
\boldsymbol{x}= [ \boldsymbol{v}_{t-1}^\top, \boldsymbol{v}_{t-2}^\top, \cdots, \boldsymbol{v}_{t-T}^\top  ]^\top,
\]

and therefore \( \boldsymbol{\mathcal{X}} \subseteq \mathbb{R}^{L} \), with \( L = PT \), and \( T \) is model order. We are interesting to predict next \( H \in \mathbb{Z}^{++} \) sequential values, giving the output target \( \boldsymbol{y} \in \boldsymbol{\mathcal{Y}} \) as

\[
\boldsymbol{y} = [ \boldsymbol{v}_{t}^\top, \boldsymbol{v}_{t+1}^\top, \cdots, \boldsymbol{v}_{t+H-1}^\top  ]^\top,
\] 

thus, the output space \( \boldsymbol{\mathcal{Y}} \subseteq \mathbb{R}^{D} \), with \( D = PH \), and \( H \) is model Horizon. The above formulation enables the model to leverage sequential data for accurate future predictions. We build a train dataset compound by \( N \) input-output i.i.d. pair observations as \( \boldsymbol{\mathcal{D}} = \{\boldsymbol{x}_n, \boldsymbol{y}_n\}_{n=1}^N = \{ \boldsymbol{X}, \boldsymbol{Y}\} \).

\subsection{Chained LMC GP}

\subsubsection{Likelihood Model}

We start by setting the likelihood function assuming the distribution over \(  \boldsymbol{y}_n \) as the product of \( D \) conditionally independent distributions, one by output as follows:

\begin{equation}\label{eq:likelihood_funciton}
	p(\boldsymbol{Y} \mid \boldsymbol{\theta}(\boldsymbol{X})) = \prod_{n=1}^N p\left(\boldsymbol{y}_n \mid \boldsymbol{\theta}(\boldsymbol{x}_n)\right) = \prod_{n=1}^N \prod_{d=1}^D p\left(y_{d,n} \mid \boldsymbol{\theta}_d(\boldsymbol{x}_n)\right), 
\end{equation}

begin \( \boldsymbol{\theta}(\boldsymbol{X}) = \{ \boldsymbol{\theta}_d(\boldsymbol{x}_n \}_{n=1,d=1}^{N, D} \), with \( \boldsymbol{\theta}_d(\boldsymbol{x}) \subseteq \mathbb{R}^{J_d} \) as a vector containing \( J_d \) parameter for \( d \)-th output distribution. Each element of \( \boldsymbol{\theta}_d(\boldsymbol{x}) \), denoted as \( \theta_{d,j}(\boldsymbol{x}) \), could be restricted to some subset of \( \mathbb{R} \). To handle that, we model \( \theta_{d,j}(\boldsymbol{x}) = h_{d,j}(f_{d,j}(\boldsymbol{x})) \) as a transformation of an unrestricted latent variable \( f_{d,j}(\boldsymbol{x}) \) via a link function \( h_{d,j} \).

\subsubsection{Linear Model of Coregionalization (LMC) prior}

Our main task is reduced to model each \(f_{d,j}\) and plug-in it in \cref{eq:likelihood_funciton}. Consider a set of \( Q \) independent GPs \( \{ g_q \}_{q=1}^Q \) over the input space such that \( g_q(\boldsymbol{x}) \sim \mathcal{GP}(0, k_q(\boldsymbol{x}, \boldsymbol{x}'))\) with kernel parameters \( \Phi_q \) that will be linearly weighed via \( a_{(d,j),q} \in \mathbb{R} \) coefficients to generate \( f_{d,j} \) by:

\begin{equation}\label{eq:lmc_process}
	f_{d,j}(\boldsymbol{x}) = \sum_{q=1}^Q a_{(d,j),q} g_{q}(\boldsymbol{x}).
\end{equation}

With that in mind, the cross-covariance function of the latent variable \( f_{d,j} \) is as follows:
\begin{equation}
	\begin{split}\label{eq:cross-covariance_function}
		k_{\boldsymbol{f}_{d,j}, \boldsymbol{f}_{d',j'}}(\boldsymbol{x}, \boldsymbol{x}') &= \text{cov}\{f_{d,j}(\boldsymbol{x}), f_{d',j'}(\boldsymbol{x}')\}\\
		&=\sum_{q=1}^Q \sum_{q'=1}^Q a_{(d,j),q}a_{(d',j'),q'}\text{cov}\{u_{q}(\boldsymbol{x}), u_{q'}(\boldsymbol{x}')\}\\
		&=\sum_{q=1}^Q a_{(d,j),q}a_{(d',j'),q} k_{q}(\boldsymbol{x}, \boldsymbol{x}').
	\end{split}
\end{equation}


In this way, output dependencies are shared via the mixing process described in \cref{eq:cross-covariance_function}, allowing each \( f_{d,j} \) to be composed as a set of features granted by \( g_q \) through by coefficients \( a_{(d,j),q} \).

\subsection{Variational Inference}

So far, the LMC model does not take into account data observation set \( \boldsymbol{\mathcal{D}} \) and only relies on prior knowledge. Let us group all latent variables located at all train points \( f_{d,j}(\boldsymbol{x}_n) \) into a vector \( \boldsymbol{f} \in \mathbb{R}^{NJ} \) where \( J = \sum_{d=1}^D J_d \), and the likelihood function in \cref{eq:likelihood_funciton} can be notated as \( p(\boldsymbol{Y} \mid \boldsymbol{f}) \). According to \cref{eq:lmc_process}, \( \boldsymbol{f} \sim \mathcal{N}\left(\boldsymbol{0}, \boldsymbol{K}_{\boldsymbol{f},\boldsymbol{f}} \right) \), and \( \boldsymbol{K}_{\boldsymbol{f},\boldsymbol{f}} \in \mathbb{R}^{NJ \times NJ} \) is filled out by the kernel function in \cref{eq:cross-covariance_function} evaluated at each pair of elements in \( \boldsymbol{X} \). The posterior distribution \( p(\boldsymbol{f} \mid \boldsymbol{Y}) = p(\boldsymbol{Y} \mid \boldsymbol{f}) p(\boldsymbol{f}) / p(\boldsymbol{Y}) \) can be intractable whether due to non-Gaussian likelihood assumption and do not achieve a close form or do it and bear a high complexity of \( \mathcal{O}(N^3J^3) \).


\subsubsection{Inducing variables method}

The main idea behind inducing variables method is to generate a new vector \( \boldsymbol{u} \in \mathbb{R}^{MQ} \) with elements \( g_q(\boldsymbol{z}_m) \) corresponding the \( Q \) independent processes evaluated at \( M \ll N \) inducing points denoted as \( \boldsymbol{Z} = \{ \boldsymbol{z}_m \}_{m=1}^M \) such as \( \boldsymbol{z}_m \in \mathcal{X} \). Inasmuch as consistence of GP, \( \boldsymbol{u} \sim \mathcal{N}\left(\boldsymbol{0}, \boldsymbol{K}_{\boldsymbol{u},\boldsymbol{u}} \right) \), and \( \boldsymbol{K}_{\boldsymbol{u},\boldsymbol{u}} \in \mathbb{R}^{MQ \times MQ} \). Now we focus on the joint Gaussian prior \( p(\boldsymbol{f}, \boldsymbol{u}) = p(\boldsymbol{f} \mid \boldsymbol{u}) p(\boldsymbol{u}) \). Due to independence property of \( \boldsymbol{u} \) process, \( p(\boldsymbol{u}) = \prod_{q=1}^{Q} p(\boldsymbol{u}_q) = \prod_{q=1}^{Q} \mathcal{N}\Bigl( \boldsymbol{u}_q \mid \boldsymbol{0}, \boldsymbol{K}_{\boldsymbol{u}_q,\boldsymbol{u}_q}  \Bigr) \), 

\begin{equation}\label{eq:conditional_f_given_u}
	\begin{split}
		p(\boldsymbol{f}\mid \boldsymbol{u})
		&= \prod_{d=1}^{D}\prod_{j=1}^{J_{d}}
		p(\boldsymbol{f}_{d,j}\mid \boldsymbol{u})\\
		&= \prod_{d=1}^{D}\prod_{j=1}^{J_{d}}
		\mathcal{N}\Bigl(
		\boldsymbol{f}_{d,j} \mid
		\boldsymbol{K}_{\boldsymbol{u}, \boldsymbol{f}_{d,j}}^{\top}
		\boldsymbol{K}_{\boldsymbol{u},\boldsymbol{u}}^{-1}
		\boldsymbol{u},
		\boldsymbol{K}_{\boldsymbol{f}_{d,j}, \boldsymbol{f}_{d,j}} -
		\boldsymbol{K}_{\boldsymbol{u}, \boldsymbol{f}_{d,j}}^{\top}
		\boldsymbol{K}_{\boldsymbol{u},\boldsymbol{u}}^{-1}
		\boldsymbol{K}_{\boldsymbol{u}, \boldsymbol{f}_{d,j}}
		\Bigr) ,
	\end{split}
\end{equation}

and notation is as follows: \( \boldsymbol{u}_q = [g_q(\boldsymbol{z}_1), \cdots, g_q(\boldsymbol{z}_M)  ]^\top \in \mathbb{R}^{M} \) is a random vector, with covariance matrix \(  \boldsymbol{K}_{\boldsymbol{u}_q,\boldsymbol{u}_q} \in \mathbb{R}^{M \times M} \) filled out by the \( q \)-th kernel function of the independent process \( k_q \) evaluate at each pair of elements in \( \boldsymbol{Z} \). Similarly, \( \boldsymbol{f}_{d,j} = [f_{d,j}(\boldsymbol{x}_1), \cdots, f_{d,j}(\boldsymbol{x}_N) ]^\top \in \mathbb{R}^{N} \) is a zero-mean random vector, with covariance matrix \(  \boldsymbol{K}_{\boldsymbol{f}_{d,j},\boldsymbol{f}_{d,j}} \in \mathbb{R}^{N \times N} \) with elements given by kernel function in \cref{eq:cross-covariance_function} evaluate at each pair of points in \( \boldsymbol{X} \) that is also Gaussian. 


Our focus is placed on approximate the joint posterior distribution $p(\boldsymbol{f}, \boldsymbol{u} \mid \mathcal{D})$ as follows:

\begin{equation}
	p(\boldsymbol{f}, \boldsymbol{u} \mid \mathcal{D}) \approx q(\boldsymbol{f}, \boldsymbol{u}) = p(\boldsymbol{f}\mid \boldsymbol{u}) q(\boldsymbol{u}) = \prod_{d=1}^D \prod_{j=1}^{J_d} p(\boldsymbol{f}_{d,j} \mid \boldsymbol{u}) \prod_{q=1}^Q q(\boldsymbol{u}_q),
\end{equation}

where \( q \) distribution refers to a posterior approximation. Provided the above, \( q(\boldsymbol{u}) = \mathcal{N}(\boldsymbol{u} \mid \boldsymbol{\mu}, \boldsymbol{S}) \) and \( q(\boldsymbol{u}_q) = \mathcal{N}(\boldsymbol{u}_q \mid \boldsymbol{\mu}_{q}, \boldsymbol{S}_{q}) \) are called variational distributions assumed Gaussian-shaped. Also, \( \boldsymbol{\mu} = [\boldsymbol{\mu}_1^\top, \boldsymbol{\mu}_2^\top, \cdots, \boldsymbol{\mu}_Q^\top]^\top \in \mathbb{R}^{MQ} \), and \( \boldsymbol{S} \in \mathbb{R}^{MQ \times MQ} \) is a block-diagonal matrix with blocks given by \( \boldsymbol{S}_q \in \mathbb{R}^{M \times M} \).

\subsubsection{Latent Posterior}

The approximate marginal posterior for $\boldsymbol{f}_{d,j}$ denoted $q(\boldsymbol{f}_{d,j}) = \int p(\mathbf{f}_{d,j} \mid \boldsymbol{u})q(\boldsymbol{u})d\boldsymbol{u}$ has a close form giving by 

\begin{equation}\label{eq:latent_posterior_approximation}
	q(\boldsymbol{f}_{d,j}) = \mathcal{N}\left(\boldsymbol{f}_{d,j} \mid \boldsymbol{K}_{\boldsymbol{u}, \boldsymbol{f}_{d,j}}^\top \boldsymbol{K}_{\boldsymbol{u}, \boldsymbol{u}}^{-1} \boldsymbol{\mu}, \boldsymbol{K}_{\boldsymbol{f}_{d,j}, \boldsymbol{f}_{d,j}} - \mathbf{K}_{\boldsymbol{u}, \boldsymbol{f}_{d,j}}^\top \boldsymbol{K}_{\boldsymbol{u}, \boldsymbol{u}}^{-1} (\boldsymbol{S} - \boldsymbol{K}_{\boldsymbol{u}, \mathbf{u}}) \mathbf{K}_{\mathbf{u}, \boldsymbol{u}}^{-1} \boldsymbol{K}_{\boldsymbol{u}, \boldsymbol{f}_{d,j}} \right).
\end{equation}

The complexity of this model is now \( \mathcal{O}(JNQM^2) \).

\subsubsection{Predictive distribution}

Consider a test point \( \boldsymbol{x}_{*} \in \mathcal{X} \). Assuming a good approximation of the variational posterior $p(\mathbf{u} \mid \mathbf{y}) \approx q(\mathbf{u})$, the posterior of latent parameter function vector at test point \( \boldsymbol{f}_* \in \mathbb{R}^{J} \) takes the form

\begin{equation}
	q(\boldsymbol{f}_*) = \int p(\boldsymbol{f}_* \mid \boldsymbol{u}) q(\boldsymbol{u}) d\boldsymbol{u},
\end{equation}

which is filled out similarty to \cref{eq:latent_posterior_approximation}. The predictive distribution at the test input \( \boldsymbol{y}_* \in \mathcal{Y} \), given the observed data \( \mathcal{D} \), can thus be approximated as:

\begin{equation}\label{eq;predictive_distribution}
	p(\boldsymbol{y}_* \mid \mathcal{D}) \approx \int p(\boldsymbol{y}_* \mid \boldsymbol{f}_*) q(\boldsymbol{f}_*) \, d\boldsymbol{f}_*,
\end{equation}

which integrates over the latent function vector $\boldsymbol{f}_*$ to account for its uncertainty in the prediction of $\boldsymbol{y}_*$. Compute \cref{eq;predictive_distribution} could be intractable. However, statistics like mean and variance can be approximated via Monte Carlo methods.

\subsection{Variational Bounds}

The learnable parameters in our model encompass the variational parameters \( \boldsymbol{\mu}_q \) and \( \boldsymbol{S}_{q} \), inducing points \( \boldsymbol{Z}\), and kernel parameters \( a_{(d,j), q}\) and \( \Phi_q \). To select it optimally, we use the lower bound $\mathcal{L}$ for $\log p(\mathbf{y})$ loss function, obtained as follows:

\begin{equation}
	\begin{split}
		\log p(\boldsymbol{y})
		&= \log \int\!\!\int p\bigl(\boldsymbol{y}\mid\boldsymbol{f}\bigr)\,
		p\bigl(\boldsymbol{f}\mid\boldsymbol{u}\bigr)\,
		p(\boldsymbol{u})
		\,d\boldsymbol{f}\,d\boldsymbol{u}
		\\[6pt]
		&\ge
		\int\!\!\int q\bigl(\boldsymbol{f},\boldsymbol{u}\bigr)\,
		\log \frac{p\bigl(\boldsymbol{y}\mid\boldsymbol{f}\bigr)\,
			p\bigl(\boldsymbol{f}\mid\boldsymbol{u}\bigr)\,
			p(\boldsymbol{u})}
		{q\bigl(\boldsymbol{f},\boldsymbol{u}\bigr)}
		\,d\boldsymbol{f}\,d\boldsymbol{u}
		\;=\;\mathcal{L}\,.
	\end{split}
\end{equation}


We can further simplify $\mathcal{L}$ to get

\begin{equation}
	\begin{split}
		\mathcal{L}
		&= \int\!\!\int q(\boldsymbol{f},\boldsymbol{u})
		\log \frac{p(\boldsymbol{y}\mid\boldsymbol{f})\,
			p(\boldsymbol{f}\mid\boldsymbol{u})\,
			p(\boldsymbol{u})}
		{q(\boldsymbol{f},\boldsymbol{u})}
		\,d\boldsymbol{f}\,d\boldsymbol{u}
		\\
		&= \int\!\!\int q(\boldsymbol{f},\boldsymbol{u})
		\log p(\boldsymbol{y}\mid\boldsymbol{f})
		\,d\boldsymbol{f}\,d\boldsymbol{u}
		\;-\;
		\int q(\boldsymbol{u})
		\log \frac{q(\boldsymbol{u})}{p(\boldsymbol{u})}
		\,d\boldsymbol{u}
		\\
		&= \int q(\boldsymbol{f})
		\log p(\boldsymbol{y}\mid\boldsymbol{f})
		\,d\boldsymbol{f}
		\;-\;
		\sum_{q=1}^{Q}
		\mathrm{KL}\bigl\{q(\boldsymbol{u}_{q})\parallel p(\boldsymbol{u}_{q})\bigr\}
		\\
		&= \mathbb{E}_{q(\boldsymbol{f})}\bigl\{\log p(\boldsymbol{y}\mid\boldsymbol{f})\bigr\}
		\;-\;
		\sum_{q=1}^{Q}
		\mathrm{KL}\bigl\{q(\boldsymbol{u}_{q})\parallel p(\boldsymbol{u}_{q})\bigr\}
		\\
		&= \sum_{d=1}^{D}
		\mathbb{E}_{q(\boldsymbol{f}_{d,1}),\dots,q(\boldsymbol{f}_{d,J_d})}
		\bigl\{\log p(\boldsymbol{y}_{d}\mid\boldsymbol{f}_{d,1},\dots,\boldsymbol{f}_{d,J_d})\bigr\}
		\;-\;
		\sum_{q=1}^{Q}
		\mathrm{KL}\bigl\{q(\boldsymbol{u}_{q})\parallel p(\boldsymbol{u}_{q})\bigr\}
		\\
		&= \sum_{n=1}^{N}\sum_{d=1}^{D}
		\mathbb{E}_{q(f_{d,1,n}),\dots,q(f_{d,J_d,n})}
		\bigl\{\log p(y_{d,n}\mid f_{d,1,n},\dots,f_{d,J_d,n})\bigr\}
		\;-\;
		\sum_{q=1}^{Q}
		\mathrm{KL}\bigl\{q(\boldsymbol{u}_{q})\parallel p(\boldsymbol{u}_{q})\bigr\},
	\end{split}
\end{equation}

being \( \text{KL}\{q(\boldsymbol{u}_q)\parallel p(\boldsymbol{u}_q)\} \) the Kullback-Leibler divergence between two Gaussian distributions, with a closed form solution, working as a regularization factor.

\subsection{Adam + Natural Gradient Optimization}

The optimization step for the Chained LMC GP involves several key parameters: kernel parameters (such as characteristic lengthscales and output scale for the exponential quadratic kernel), the set of inducing point locations, and the variational distribution parameters. According to \cite{giraldo2021fully}, there is a strong dependency between the variational parameters and the others, making the model extremely sensitive to any changes in these variables. Additionally, since the ELBO loss function is generally non-convex, stochastic gradient optimization methods often converge to poor local minima.

Recent research has found a beneficial use of Natural Gradient (NG) optimization to overcome the issue mentioned above. The work developed by \cite{pmlr-v84-salimbeni18a} demonstrates the advantages of NG for variational GPs optimization with minimal effort and how to combine it with the Adam optimizer, providing a hybrid optimization framework. In the same spirit, we propose an optimization scheme where the variational parameters are governed by Natural Gradient (NG) optimization, while the others are optimized using the Adam optimizer and we refer to this Adam + NG. That approach allows us to achieve a similar optimization performance as Adam's as if we were optimizing an exact model.


\subsection{Adaptative Multi-Rate LSTM (AMR LSTM)}

\subsubsection{Problem Setting}

Consider a train dataset, comprising \(N\) input-output pairs of examples denoted as  $\mathcal{D} = \{(\boldsymbol{X}{i},\,y{i})\}_{i=1}^N$, where 
$\boldsymbol{X}{i}\in\mathcal{X}$ and $y{i}\in\mathcal{Y}$ belongs to some input and output space respectively. Here $\mathcal{X}$ is the space of collections of, at most, \(P\) vector sequences data observed within a fixed window of \(T\) fundamental time steps:
\[
\boldsymbol{X}_i
= \bigl\{\boldsymbol{x}{i}^{(1)}, \dots, \boldsymbol{x}{i}^{(t)}, \dots,\boldsymbol{x}_{i}^{(T)}\bigr\}.
\]

At each time step \(t \in \{1, \cdots, T\}\}\), the observed vector is

\[
\boldsymbol{x}{i}^{(t)} \in \mathbb{R}^{P_t},\,P_t \in \{1, \cdots, P\}.
\]

The dimensionality $P_t$ at each time step varies depending on which signals are present at that particular instant due to their individual sampling rates, as shown in \cref{fig:input_space_structure}.

\begin{figure}[ht]
	\centering
\begin{tikzpicture}[x=1.5cm,y=1cm]
	% user‐changeable parameters:
	\newcommand{\Tmin}{0}    % lowest time index
	\newcommand{\Tmax}{5}   % highest time index
	\newcommand{\Pdim}{3}    % number of rows (p = 1…Pdim)
	\newcommand{\Ndim}{3}    % number of depth‐layers (N = 1…Ndim)
	% how far out braces sit (tweak if you want more/less clearance):
	\newcommand{\Xoff}{0.3}  
	\newcommand{\Yoff}{0.3}  
	% Draw each layer N = 1…Ndim (1 is backmost, Ndim is front)
	\foreach \N in {1,...,\Ndim} {
		% how much to shift layer \N back in x,y
		\pgfmathsetmacro{\shift}{0.5*(\Ndim-\N)}
		\begin{scope}[shift={(\shift,\shift)}]
			% for each row p
			\foreach \p in {1,...,\Pdim} {
				% compute its y‐coordinate so p=1 is the top
				\pgfmathtruncatemacro{\y}{\Pdim-\p}
				% and for each time index i
				\foreach \i in {\Tmin,...,\Tmax} {
					% pattern: one blue then (p-1) reds
					\pgfmathtruncatemacro{\modres}{mod(\i,\p)}
					\ifnum\modres=0
					\def\fillcolor{MyAccent!20}
					\def\labeltext{$x_{\i,\p}^{(\N)}$}
					\else
					\def\fillcolor{red!20}
					\def\labeltext{}
					\fi
					\fill[fill=\fillcolor,draw=MyDarkBlue]
					(\i,\y) rectangle ++(1,1);
					\node[text=black,font=\footnotesize]
					at (\i+0.5,\y+0.5) {\labeltext};
				}
			}
		\end{scope}
	}
	
	% ─────── Braces ─────────────────────────────────────────────
	% T‐brace above, spanning all time‐cells:
	\draw[decorate,decoration={brace,amplitude=4pt}]
	(\Tmin+1,\Pdim+\Yoff+1) -- (\Tmax+2,\Pdim+\Yoff+1)
	node[midway,above=4pt] {$T$};
	
	% p‐brace at right, spanning all rows:
	\draw[decorate,decoration={brace,amplitude=4pt,mirror}]
	(\Tmax+2+\Xoff,1) -- (\Tmax+2+\Xoff,\Pdim+1)
	node[midway,right=4pt] {$P$};
	
	% N‐brace (diagonal) from front layer to back layer:
	\pgfmathsetmacro{\Nshift}{0.5*(\Ndim-1)}
	\pgfmathsetmacro{\startX}{\Tmin-\Xoff}
	\pgfmathsetmacro{\startY}{\Pdim}
	\pgfmathsetmacro{\endX}{\Tmin+\Nshift-\Xoff}
	\pgfmathsetmacro{\endY}{\Pdim+\Nshift}
	\draw[decorate,decoration={brace,amplitude=4pt}]
	(\startX,\startY) -- (\endX,\endY)
	node[midway,above left=2pt] {$N$};
\end{tikzpicture}
	\label{fig:input_space_structure}
\caption{Structure of input space}
\end{figure}

	
	\section{Resultados}
	Se presenta la versión definitiva de la propuesta de doctorado sobre modelos funcionales y procesos Gaussianos aplicados a la detección y tratamiento de trastornos mentales, documentada en la sección de anexos.
	
	Se implementó en Python un modelo de inferencia basado en procesos Gaussianos encadenados para ampliar la metodología de gamificación mediante un enfoque estocástico.
	
	El conjunto de datos Toadstool 2 fue seleccionado por su variedad en datos provenientes de múltiples fuentes, cada una con sus propios parámetros de adquisición, registrados dentro de una ventana de 4 segundos. Todos los datos recolectados durante este intervalo fueron etiquetados con una emoción, pudiendo ser enojado, neutro, triste, sorprendido, entre otras.
	
	El problema del conjunto de datos anterior puede abordarse como uno de fusión asincrónica de múltiples fuentes. Una revisión del estado del arte resalta a las redes recurrentes como candidatas para construir un modelo que integre y explote datos temporales, con el fin de generar una representación latente que condense las características más importantes de las secuencias en un conjunto finito de variables. Finalmente, esta representación será la encargada de alimentar el modelo basado en procesos Gaussianos para generar una distribución predictiva como pronóstico.
	
	Se desarrolló e implementó un modelo basado en redes neuronales recurrentes LSTM que alternan sus celdas ocultas dependiendo de las características y del número de entradas, manteniendo sus estados latentes y memorias compartidas, llamado Multi-Rate LSTM, como se muestra en la \cref{fig:mrlstm}. Aqui un tipo particular de celda se encarga de analizar una configuración de entrada en cada instante de tiempo, actualizar los estados ocultos y posteriormente transferir la información relevante a la siguiente celda en el siguiente instante de tiempo.
	

\begin{figure}[ht]
	\centering
	\begin{tikzpicture}[item/.style={rectangle, minimum width=1.5cm,draw,thick,align=center,fill=blue!20},itemblock/.style={rectangle,draw,thick,align=center},
		itemc/.style={item,on chain,join},]
		\begin{scope}[start chain=going right,nodes=itemc,every
			join/.style={-latex,very thick},local bounding box=chain]
			\path node (A0) {Celda $A$} node (A1) {Celda $B$} node (A2) {Celda $A$} node[xshift=2em] (At)
			{Celda $B$};
		\end{scope}
		\node[left=1em of chain,scale=2] (eq) {$=$};
		\node[left=2em of eq,item] (AL) {Modelo};
		\path (AL.west) ++ (-1em,2em) coordinate (aux);
		\draw[very thick,-latex,rounded corners] (AL.east) -| ++ (1em,2em) -- (aux) 
		|- (AL.west);
		\foreach \X in {0,1,2,t} 
		{\draw[very thick,-latex] (A\X.north) -- ++ (0,2em)
			node[above,item,fill=gray!20] (h\X) {Estado $\X$};
			\draw[very thick,latex-] (A\X.south) -- ++ (0,-2em)
			node[below,item,fill=gray!20] (x\X) {Entrada\\Instante $\X$};}
		\draw[white,line width=0.8ex] (AL.north) -- ++ (0,2em);
		\draw[very thick,-latex] (AL.north) -- ++ (0,2em)
		node[above,item,fill=gray!20] {Estado $t$};
		\draw[very thick,latex-] (AL.south) -- ++ (0,-2em)
		node[below,item,fill=gray!20] {Entrada\\Instante $t$};
		\path (x2) -- (xt) node[midway,scale=2,font=\bfseries] {\dots};
	\end{tikzpicture}
	\caption{Representación gráfica del modelo Multi-Rate LSTM.}
	\label{fig:mrlstm}
\end{figure}
	 
	 
Sin embargo, la realización de varias pruebas mostró que el conjunto de datos Toadstool 2 presenta un desbalance entre el número de muestras disponibles y la complejidad de predecir etiquetas basadas en datos provenientes de múltiples fuentes.

Para solucionar este problema, se está integrando un nuevo conjunto de datos denominado EPA-air, el cual incluye mediciones ambientales detalladas provenientes de múltiples sensores distribuidos geográficamente, proporcionando información temporal y multimodal que facilita la sincronización y el análisis de secuencias temporales para apoyar el entrenamiento del modelo.

Se identificó como problema latente en las técnicas de gamificación la integración asincrónica de múltiples fuentes. El problema radica en cómo mantener las dependencias a lo largo de las series de tiempo y entre las mismas, a pesar de la asincronía de adquisición, presente cuando a un paciente se le toman múltiples métricas provenientes de diferentes dispositivos. 

Se propone el uso de parches de longitud variable para mantener la secuencialidad de los datos ante este escenario. Esto alimenta un modelo secuencial, como una red LSTM, para generar un espacio embebido que posteriormente será mapeado al espacio de salida mediante un modelo probabilístico. 


Una revisión profunda del estado del arte evidencia el uso de modelos basados en redes convolucionales y arquitecturas transformer, capaces de capturar características temporales y espaciales a nivel local y global. En esta línea, se han implementado modelos EEGNet y EEG Conformer para el análisis de señales EEG, las cuales registran la actividad eléctrica cerebral mediante electrodos sobre el cuero cabelludo. Estos modelos han mostrado resultados satisfactorios con la base de datos BCI Competition IV 2a, que contiene registros multicanal de EEG durante tareas de imaginación motora. Actualmente, se realizan pruebas con una base de datos propia orientada a la clasificación de estados de inhibición y no inhibición, con el fin de evaluar la robustez y capacidad de generalización del modelo.

	
	\section{Anexos}
% Include first page with a title overlay
\includepdf[pages=1,
pagecommand={%
	\thispagestyle{plain}
		\Large\bfseries Propuesta doctorado
}
]{../propuesta_doctorado/main.pdf}
% Include remaining pages without any overlay
\includepdf[pages=2-last, pagecommand={}]{../propuesta_doctorado/main.pdf}

	\bibliographystyle{apalike}
	\bibliography{refs}
	
\end{document}
