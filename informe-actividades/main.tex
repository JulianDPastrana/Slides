\documentclass[10pt,xcdraw]{article}

\usepackage{graphicx}
\usepackage{subcaption}
\graphicspath{ {figures/} }
 % Carga el preámbulo
\begin{document}
	%%%%%%%%%%%%%%%%%%%%%%%%%%%%%%%%%% PORTADA %%%%%%%%%%%%%%%%%%%%%%%%%%%%%%%%%%%%%%%%%%%%
\vspace*{0.5cm}
\begin{center}
	\newcommand{\HRule}{\rule{\linewidth}{0.5mm}}
	\vspace*{-1.5cm}
	% \textsc{\huge Universidad Tecnológica\\ \vspace{5px} de Pereira}\\[1.5cm]
	% \includegraphics[width=0.2\textwidth]{imagenes/Logo_UTP.png}
	% \vspace{1.5cm}
	\textsc{\LARGE Informe de actividades en el marco del programa: \\ 
		\vspace{0.4cm}
		XXXXXXXXXXX}\\
	\vspace{2cm}
	\HRule \\[0.4cm]
	{ \bfseries Nombre: Julián David Pastrana Cortés }\\[0.4cm]
	{ \bfseries Contrato de servicios: 5551 }\\[0.4cm]
	\vspace{1.5cm}
	\includegraphics[width=0.28\textwidth]{imagenes/LogoU.png}
	\includegraphics[width=0.35\textwidth]{imagenes/logo_automatica.png}
	\HRule \\[1.5cm]
	\vspace{1.5cm}
	\vspace*{0.5cm}
	\textsc{\textbf{\Large Grupo de investigación en Automática } }\\
	\vspace{3.5cm} 
	\begin{center}
		{\large \today}
	\end{center}
\end{center}
\newpage
   % Carga la portada y otros elementos de layout
	
	\tableofcontents 
	\newpage
	
	\section{Información general del contrato}
	\begin{table}[H]
		\centering
		\begin{tabular}{>{\arraybackslash}m{8cm} >{\arraybackslash}m{6.5cm}}
			\toprule
			\textbf{Rol} & \vspace{2mm}Contratista - Estudiante de Doctorado\vspace{2mm}\\\hline
			\vspace{2mm}
			\textbf{Contrato de servicios No.}\vspace{2mm} & \vspace{2mm} 5551 de 2025\vspace{2mm}\\\hline
			\vspace{2mm}
			\textbf{Objeto del contrato} \vspace{2mm} & \vspace{2mm} Prestación de servicios profesionales para el Desarrollo de metodología de gamificación para entrenamiento de niños con trastornos de impulsividad ALIANZA CIENTÍFICA CON ENFOQUE COMUNITARIO PARA MITIGAR BRECHAS DE ATENCIÓN Y MANEJO DE TRASTORNOS MENTALES RELACIONADOS CON IMPULSIVIDAD EN COLOMBIA - ACEMATE MINCIENCIAS CONTRATO 790-2023 
			\vspace{2mm}\\\hline
			\textbf{Período del informe} & \vspace{2mm} 01 de marzo de 2025 al 31 de marzo de 2025\vspace{2mm}\\\hline
		\end{tabular}
	\end{table}
	
	\subsection{Descripción general de la vinculación}
	Como resultado de la vinculación se contribuye al alcance de el objetivo 3 del programa de investigación: ALIANZA CIENTÍFICA CON ENFOQUE COMUNITARIO PARA MITIGAR BRECHAS DE ATENCIÓN Y MANEJO DE TRASTORNOS MENTALES RELACIONADOS CON IMPULSIVIDAD EN COLOMBIA
	
	\subsection{Objetivo general}
	Desarrollar una metodología de gamificación para entrenamiento de niños con trastornos de impulsividad ALIANZA CIENTÍFICA CON ENFOQUE COMUNITARIO PARA MITIGAR BRECHAS DE ATENCIÓN Y MANEJO DE TRASTORNOS MENTALES RELACIONADOS CON IMPULSIVIDAD EN COLOMBIA - ACEMATE MINCIENCIAS CONTRATO 790-2023.
	
	
	\section{Metodología}
La metodología propuesta para el desarrollo del proyecto se estructura en cuatro actividades principales, distribuidas a lo largo de siete meses, con entregables específicos para cada una:

\subsection{Actividad 1: Planteamiento de la metodología, técnicas y modelos a implementar (Meses 1-2)}
En esta etapa inicial se realizará la conceptualización y diseño de la metodología de trabajo. Se definirán las técnicas de análisis y los modelos computacionales que se implementarán para el procesamiento de señales EEG. Esta actividad incluye la revisión de fundamentos teóricos, selección de herramientas y establecimiento del marco de trabajo.

\textbf{Entregable:} Informe técnico que documenta la metodología propuesta, las técnicas seleccionadas y los modelos a desarrollar.

\subsection{Actividad 2: Revisión técnica del estado del arte (Meses 3-4)}
Se llevará a cabo una revisión exhaustiva y sistemática del estado del arte en análisis de señales EEG, identificando las metodologías existentes, técnicas avanzadas y tendencias actuales en el procesamiento de señales cerebrales. Esta revisión permitirá contextualizar el trabajo y validar las decisiones metodológicas.

\textbf{Entregable:} Informe técnico con el análisis del estado del arte y la fundamentación teórica del proyecto.

\subsection{Actividad 3: Desarrollo e implementación de modelos de análisis de EEG (Meses 5-6)}
Esta etapa constituye el núcleo del proyecto, donde se desarrollarán e implementarán los modelos de análisis de señales EEG. Se realizará el montaje y configuración de los modelos para diferentes tareas específicas, incluyendo pruebas, validación y optimización de los algoritmos desarrollados.

\textbf{Entregable:} Código fuente de los modelos implementados, documentado y funcional.

\subsection{Actividad 4: Documentación de resultados consolidados (Meses 6-7)}
En la fase final se consolidarán todos los resultados obtenidos durante el proyecto. Se documentarán los hallazgos, análisis de desempeño de los modelos, conclusiones y recomendaciones. Esta documentación servirá como soporte integral del proyecto.

\textbf{Entregable:} Informe técnico final con la documentación completa de resultados, análisis y conclusiones del proyecto.
	\section{Conjunto de Datos}

\subsection{Toadstool: Un conjunto de datos para el entrenamiento de máquinas de inteligencia emocional que juegan a Super Mario Bros}

El conjunto de datos de libre acceso Toadstool es una colección de registros en video, sensores e información demográfica obtenidos de diez individuos mientras jugaban Super Mario Bros. La selección de los participantes se realizó buscando una amplia variedad en cuanto a experiencia previa con videojuegos, incluyendo desde personas que apenas habían jugado alguno en su vida hasta aquellas con una extensa trayectoria desde la infancia, con edades comprendidas entre los 26 y 48 años. También se buscó mantener un balance en el género, participando cinco hombres y cinco mujeres. Los autores en \cite{toadstool} señalan la presencia de anomalías, tales como ninguna o muy poca actividad detectada por los sensores, así como diferencias significativas entre la actividad registrada al inicio y al final de las sesiones de juego.

El desempeño de los participantes fue evaluado según el número de niveles completados en un tiempo determinado y el número de muertes ocurridas durante la partida. Dicho puntaje se mantuvo en secreto entre los participantes para evitar que se rindieran o relajaran durante el juego. Adicionalmente, se les incentivó mediante una recompensa con el fin de mantener un desempeño competitivo. Un resumen de las características de los participantes se presenta en la \cref{tab:datos_participantes}.


\begin{table}[h!]
	\centering
	\resizebox{\textwidth}{!}{%
		\begin{tabular}{|c|c|c|c|c|c|c|c|}
			\hline
			\textbf{ID} & \textbf{Edad} & \textbf{Sexo} & \textbf{Mano dominante} & \shortstack{\textbf{Horas}\\\textbf{por semana}} & \shortstack{\textbf{Años}\\\textbf{de actividad}} & \shortstack{\textbf{Experiencia}\\\textbf{previa}} & \shortstack{\textbf{Puntaje}\\\textbf{del juego}} \\
			\hline
			0 & 26 & Hombre & Derecha  & 4--8 & 22 & Mucha  & 17,100 \\
			1 & 48 & Hombre & Izquierda & 0--1 & 1  & Poca   & 3,000  \\
			2 & 28 & Hombre & Derecha  & 0--1 & 0  & Ninguna& 300    \\
			3 & 32 & Hombre & Derecha  & 4--8 & 4  & Algo   & 13,300 \\
			4 & 32 & Mujer  & Derecha  & 0--1 & 5  & Algo   & 6,400  \\
			5 & 30 & Mujer  & Derecha  & 0--1 & 5  & Poca   & 2,700  \\
			6 & 35 & Hombre & Izquierda & 1--4 & 30 & Mucha  & 14,300 \\
			7 & 34 & Mujer  & Derecha  & 1--4 & 14 & Algo   & 3,800  \\
			8 & 31 & Mujer  & Derecha  & 0--1 & 2  & Poca   & 200    \\
			9 & 27 & Mujer  & Derecha  & 0--1 & 5  & Poca   & 10,600 \\
			\hline
		\end{tabular}%
	}
	\caption{Características demográficas y experiencia de los participantes en relación con su desempeño en el juego. Se incluyen datos como edad, sexo, lateralidad, dedicación semanal, años de experiencia, nivel de experiencia previa y puntaje obtenido.}
	\label{tab:datos_participantes}
\end{table}

En el conjunto de datos, se incluye para cada participante un video de su rostro durante la sesión de juego, grabado a una resolución de 640×480 píxeles y a 30 fotogramas por segundo. Además, se registran las acciones realizadas mediante el mando de juego para controlar al personaje, así como los datos recopilados por la pulsera Empatica E4. La Empatica E4 es un dispositivo que permite la adquisición en tiempo real de datos fisiológicos, como la actividad electrodérmica (EDA), el pulso de volumen sanguíneo (BVP), la temperatura de la piel y la aceleración en tres ejes \cite{garbarino2014empatica}.




	
	\section{Resultados}
	Versión final del estado del arte sobre modelos de detección del TDAH basados en modelos funcacionales y procesos Gaussianos redactado en una propuesta de doctorado. El documento se encuentra en la sección de anexos.
	
	Avance en el desarrollo de un modelo de inferencia basado en Procesos Gaussianos encadenados para la amplicación de la metodología de gamificación. Consiste en un conjunto de códigos en lenguaje Python.
	
	\section{Anexos}
% Include first page with a title overlay
\includepdf[pages=1,
pagecommand={%
	\thispagestyle{plain}
		\Large\bfseries Propuesta doctorado
}
]{../propuesta_doctorado/main.pdf}
% Include remaining pages without any overlay
\includepdf[pages=2-last, pagecommand={}]{../propuesta_doctorado/main.pdf}

	\bibliographystyle{apalike}
	\bibliography{refs}
	
\end{document}
