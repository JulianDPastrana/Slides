\section{Models}

\subsection{Problem Setting}

Consider a time-series vector collecting sequential data observed across $P$ output at some time instant \( t \), denoted as \( \boldsymbol{v}_t \subseteq \mathbb{R}^{P}\). We want to include \( T \in \mathbb{Z}^{++} \) sequential observations simultaneously back as a vector \( \boldsymbol{x} \) belonging to some input space \( \mathcal{X} \), that is

\[
\boldsymbol{x}= [ \boldsymbol{v}_{t-1}^\top, \boldsymbol{v}_{t-2}^\top, \cdots, \boldsymbol{v}_{t-T}^\top  ]^\top,
\]

and therefore \( \boldsymbol{\mathcal{X}} \subseteq \mathbb{R}^{L} \), with \( L = PT \), and \( T \) is model order. We are interesting to predict next \( H \in \mathbb{Z}^{++} \) sequential values, giving the output target \( \boldsymbol{y} \in \boldsymbol{\mathcal{Y}} \) as

\[
\boldsymbol{y} = [ \boldsymbol{v}_{t}^\top, \boldsymbol{v}_{t+1}^\top, \cdots, \boldsymbol{v}_{t+H-1}^\top  ]^\top,
\] 

thus, the output space \( \boldsymbol{\mathcal{Y}} \subseteq \mathbb{R}^{D} \), with \( D = PH \), and \( H \) is model Horizon. The above formulation enables the model to leverage sequential data for accurate future predictions. We build a train dataset compound by \( N \) input-output i.i.d. pair observations as \( \boldsymbol{\mathcal{D}} = \{\boldsymbol{x}_n, \boldsymbol{y}_n\}_{n=1}^N = \{ \boldsymbol{X}, \boldsymbol{Y}\} \).

\subsection{Chained LMC GP}

\subsubsection{Likelihood Model}

We start by setting the likelihood function assuming the distribution over \(  \boldsymbol{y}_n \) as the product of \( D \) conditionally independent distributions, one by output as follows:

\begin{equation}\label{eq:likelihood_funciton}
	p(\boldsymbol{Y} \mid \boldsymbol{\theta}(\boldsymbol{X})) = \prod_{n=1}^N p\left(\boldsymbol{y}_n \mid \boldsymbol{\theta}(\boldsymbol{x}_n)\right) = \prod_{n=1}^N \prod_{d=1}^D p\left(y_{d,n} \mid \boldsymbol{\theta}_d(\boldsymbol{x}_n)\right), 
\end{equation}

begin \( \boldsymbol{\theta}(\boldsymbol{X}) = \{ \boldsymbol{\theta}_d(\boldsymbol{x}_n \}_{n=1,d=1}^{N, D} \), with \( \boldsymbol{\theta}_d(\boldsymbol{x}) \subseteq \mathbb{R}^{J_d} \) as a vector containing \( J_d \) parameter for \( d \)-th output distribution. Each element of \( \boldsymbol{\theta}_d(\boldsymbol{x}) \), denoted as \( \theta_{d,j}(\boldsymbol{x}) \), could be restricted to some subset of \( \mathbb{R} \). To handle that, we model \( \theta_{d,j}(\boldsymbol{x}) = h_{d,j}(f_{d,j}(\boldsymbol{x})) \) as a transformation of an unrestricted latent variable \( f_{d,j}(\boldsymbol{x}) \) via a link function \( h_{d,j} \).

\subsubsection{Linear Model of Coregionalization (LMC) prior}

Our main task is reduced to model each \(f_{d,j}\) and plug-in it in \cref{eq:likelihood_funciton}. Consider a set of \( Q \) independent GPs \( \{ g_q \}_{q=1}^Q \) over the input space such that \( g_q(\boldsymbol{x}) \sim \mathcal{GP}(0, k_q(\boldsymbol{x}, \boldsymbol{x}'))\) with kernel parameters \( \Phi_q \) that will be linearly weighed via \( a_{(d,j),q} \in \mathbb{R} \) coefficients to generate \( f_{d,j} \) by:

\begin{equation}\label{eq:lmc_process}
	f_{d,j}(\boldsymbol{x}) = \sum_{q=1}^Q a_{(d,j),q} g_{q}(\boldsymbol{x}).
\end{equation}

With that in mind, the cross-covariance function of the latent variable \( f_{d,j} \) is as follows:
\begin{equation}
	\begin{split}\label{eq:cross-covariance_function}
		k_{\boldsymbol{f}_{d,j}, \boldsymbol{f}_{d',j'}}(\boldsymbol{x}, \boldsymbol{x}') &= \text{cov}\{f_{d,j}(\boldsymbol{x}), f_{d',j'}(\boldsymbol{x}')\}\\
		&=\sum_{q=1}^Q \sum_{q'=1}^Q a_{(d,j),q}a_{(d',j'),q'}\text{cov}\{u_{q}(\boldsymbol{x}), u_{q'}(\boldsymbol{x}')\}\\
		&=\sum_{q=1}^Q a_{(d,j),q}a_{(d',j'),q} k_{q}(\boldsymbol{x}, \boldsymbol{x}').
	\end{split}
\end{equation}


In this way, output dependencies are shared via the mixing process described in \cref{eq:cross-covariance_function}, allowing each \( f_{d,j} \) to be composed as a set of features granted by \( g_q \) through by coefficients \( a_{(d,j),q} \).

\subsection{Variational Inference}

So far, the LMC model does not take into account data observation set \( \boldsymbol{\mathcal{D}} \) and only relies on prior knowledge. Let us group all latent variables located at all train points \( f_{d,j}(\boldsymbol{x}_n) \) into a vector \( \boldsymbol{f} \in \mathbb{R}^{NJ} \) where \( J = \sum_{d=1}^D J_d \), and the likelihood function in \cref{eq:likelihood_funciton} can be notated as \( p(\boldsymbol{Y} \mid \boldsymbol{f}) \). According to \cref{eq:lmc_process}, \( \boldsymbol{f} \sim \mathcal{N}\left(\boldsymbol{0}, \boldsymbol{K}_{\boldsymbol{f},\boldsymbol{f}} \right) \), and \( \boldsymbol{K}_{\boldsymbol{f},\boldsymbol{f}} \in \mathbb{R}^{NJ \times NJ} \) is filled out by the kernel function in \cref{eq:cross-covariance_function} evaluated at each pair of elements in \( \boldsymbol{X} \). The posterior distribution \( p(\boldsymbol{f} \mid \boldsymbol{Y}) = p(\boldsymbol{Y} \mid \boldsymbol{f}) p(\boldsymbol{f}) / p(\boldsymbol{Y}) \) can be intractable whether due to non-Gaussian likelihood assumption and do not achieve a close form or do it and bear a high complexity of \( \mathcal{O}(N^3J^3) \).


\subsubsection{Inducing variables method}

The main idea behind inducing variables method is to generate a new vector \( \boldsymbol{u} \in \mathbb{R}^{MQ} \) with elements \( g_q(\boldsymbol{z}_m) \) corresponding the \( Q \) independent processes evaluated at \( M \ll N \) inducing points denoted as \( \boldsymbol{Z} = \{ \boldsymbol{z}_m \}_{m=1}^M \) such as \( \boldsymbol{z}_m \in \mathcal{X} \). Inasmuch as consistence of GP, \( \boldsymbol{u} \sim \mathcal{N}\left(\boldsymbol{0}, \boldsymbol{K}_{\boldsymbol{u},\boldsymbol{u}} \right) \), and \( \boldsymbol{K}_{\boldsymbol{u},\boldsymbol{u}} \in \mathbb{R}^{MQ \times MQ} \). Now we focus on the joint Gaussian prior \( p(\boldsymbol{f}, \boldsymbol{u}) = p(\boldsymbol{f} \mid \boldsymbol{u}) p(\boldsymbol{u}) \). Due to independence property of \( \boldsymbol{u} \) process, \( p(\boldsymbol{u}) = \prod_{q=1}^{Q} p(\boldsymbol{u}_q) = \prod_{q=1}^{Q} \mathcal{N}\Bigl( \boldsymbol{u}_q \mid \boldsymbol{0}, \boldsymbol{K}_{\boldsymbol{u}_q,\boldsymbol{u}_q}  \Bigr) \), 

\begin{equation}\label{eq:conditional_f_given_u}
	\begin{split}
		p(\boldsymbol{f}\mid \boldsymbol{u})
		&= \prod_{d=1}^{D}\prod_{j=1}^{J_{d}}
		p(\boldsymbol{f}_{d,j}\mid \boldsymbol{u})\\
		&= \prod_{d=1}^{D}\prod_{j=1}^{J_{d}}
		\mathcal{N}\Bigl(
		\boldsymbol{f}_{d,j} \mid
		\boldsymbol{K}_{\boldsymbol{u}, \boldsymbol{f}_{d,j}}^{\top}
		\boldsymbol{K}_{\boldsymbol{u},\boldsymbol{u}}^{-1}
		\boldsymbol{u},
		\boldsymbol{K}_{\boldsymbol{f}_{d,j}, \boldsymbol{f}_{d,j}} -
		\boldsymbol{K}_{\boldsymbol{u}, \boldsymbol{f}_{d,j}}^{\top}
		\boldsymbol{K}_{\boldsymbol{u},\boldsymbol{u}}^{-1}
		\boldsymbol{K}_{\boldsymbol{u}, \boldsymbol{f}_{d,j}}
		\Bigr) ,
	\end{split}
\end{equation}

and notation is as follows: \( \boldsymbol{u}_q = [g_q(\boldsymbol{z}_1), \cdots, g_q(\boldsymbol{z}_M)  ]^\top \in \mathbb{R}^{M} \) is a random vector, with covariance matrix \(  \boldsymbol{K}_{\boldsymbol{u}_q,\boldsymbol{u}_q} \in \mathbb{R}^{M \times M} \) filled out by the \( q \)-th kernel function of the independent process \( k_q \) evaluate at each pair of elements in \( \boldsymbol{Z} \). Similarly, \( \boldsymbol{f}_{d,j} = [f_{d,j}(\boldsymbol{x}_1), \cdots, f_{d,j}(\boldsymbol{x}_N) ]^\top \in \mathbb{R}^{N} \) is a zero-mean random vector, with covariance matrix \(  \boldsymbol{K}_{\boldsymbol{f}_{d,j},\boldsymbol{f}_{d,j}} \in \mathbb{R}^{N \times N} \) with elements given by kernel function in \cref{eq:cross-covariance_function} evaluate at each pair of points in \( \boldsymbol{X} \) that is also Gaussian. 


Our focus is placed on approximate the joint posterior distribution $p(\boldsymbol{f}, \boldsymbol{u} \mid \mathcal{D})$ as follows:

\begin{equation}
	p(\boldsymbol{f}, \boldsymbol{u} \mid \mathcal{D}) \approx q(\boldsymbol{f}, \boldsymbol{u}) = p(\boldsymbol{f}\mid \boldsymbol{u}) q(\boldsymbol{u}) = \prod_{d=1}^D \prod_{j=1}^{J_d} p(\boldsymbol{f}_{d,j} \mid \boldsymbol{u}) \prod_{q=1}^Q q(\boldsymbol{u}_q),
\end{equation}

where \( q \) distribution refers to a posterior approximation. Provided the above, \( q(\boldsymbol{u}) = \mathcal{N}(\boldsymbol{u} \mid \boldsymbol{\mu}, \boldsymbol{S}) \) and \( q(\boldsymbol{u}_q) = \mathcal{N}(\boldsymbol{u}_q \mid \boldsymbol{\mu}_{q}, \boldsymbol{S}_{q}) \) are called variational distributions assumed Gaussian-shaped. Also, \( \boldsymbol{\mu} = [\boldsymbol{\mu}_1^\top, \boldsymbol{\mu}_2^\top, \cdots, \boldsymbol{\mu}_Q^\top]^\top \in \mathbb{R}^{MQ} \), and \( \boldsymbol{S} \in \mathbb{R}^{MQ \times MQ} \) is a block-diagonal matrix with blocks given by \( \boldsymbol{S}_q \in \mathbb{R}^{M \times M} \).

\subsubsection{Latent Posterior}

The approximate marginal posterior for $\boldsymbol{f}_{d,j}$ denoted $q(\boldsymbol{f}_{d,j}) = \int p(\mathbf{f}_{d,j} \mid \boldsymbol{u})q(\boldsymbol{u})d\boldsymbol{u}$ has a close form giving by 

\begin{equation}\label{eq:latent_posterior_approximation}
	q(\boldsymbol{f}_{d,j}) = \mathcal{N}\left(\boldsymbol{f}_{d,j} \mid \boldsymbol{K}_{\boldsymbol{u}, \boldsymbol{f}_{d,j}}^\top \boldsymbol{K}_{\boldsymbol{u}, \boldsymbol{u}}^{-1} \boldsymbol{\mu}, \boldsymbol{K}_{\boldsymbol{f}_{d,j}, \boldsymbol{f}_{d,j}} - \mathbf{K}_{\boldsymbol{u}, \boldsymbol{f}_{d,j}}^\top \boldsymbol{K}_{\boldsymbol{u}, \boldsymbol{u}}^{-1} (\boldsymbol{S} - \boldsymbol{K}_{\boldsymbol{u}, \mathbf{u}}) \mathbf{K}_{\mathbf{u}, \boldsymbol{u}}^{-1} \boldsymbol{K}_{\boldsymbol{u}, \boldsymbol{f}_{d,j}} \right).
\end{equation}

The complexity of this model is now \( \mathcal{O}(JNQM^2) \).

\subsubsection{Predictive distribution}

Consider a test point \( \boldsymbol{x}_{*} \in \mathcal{X} \). Assuming a good approximation of the variational posterior $p(\mathbf{u} \mid \mathbf{y}) \approx q(\mathbf{u})$, the posterior of latent parameter function vector at test point \( \boldsymbol{f}_* \in \mathbb{R}^{J} \) takes the form

\begin{equation}
	q(\boldsymbol{f}_*) = \int p(\boldsymbol{f}_* \mid \boldsymbol{u}) q(\boldsymbol{u}) d\boldsymbol{u},
\end{equation}

which is filled out similarty to \cref{eq:latent_posterior_approximation}. The predictive distribution at the test input \( \boldsymbol{y}_* \in \mathcal{Y} \), given the observed data \( \mathcal{D} \), can thus be approximated as:

\begin{equation}\label{eq;predictive_distribution}
	p(\boldsymbol{y}_* \mid \mathcal{D}) \approx \int p(\boldsymbol{y}_* \mid \boldsymbol{f}_*) q(\boldsymbol{f}_*) \, d\boldsymbol{f}_*,
\end{equation}

which integrates over the latent function vector $\boldsymbol{f}_*$ to account for its uncertainty in the prediction of $\boldsymbol{y}_*$. Compute \cref{eq;predictive_distribution} could be intractable. However, statistics like mean and variance can be approximated via Monte Carlo methods.

\subsection{Variational Bounds}

The learnable parameters in our model encompass the variational parameters \( \boldsymbol{\mu}_q \) and \( \boldsymbol{S}_{q} \), inducing points \( \boldsymbol{Z}\), and kernel parameters \( a_{(d,j), q}\) and \( \Phi_q \). To select it optimally, we use the lower bound $\mathcal{L}$ for $\log p(\mathbf{y})$ loss function, obtained as follows:

\begin{equation}
	\begin{split}
		\log p(\boldsymbol{y})
		&= \log \int\!\!\int p\bigl(\boldsymbol{y}\mid\boldsymbol{f}\bigr)\,
		p\bigl(\boldsymbol{f}\mid\boldsymbol{u}\bigr)\,
		p(\boldsymbol{u})
		\,d\boldsymbol{f}\,d\boldsymbol{u}
		\\[6pt]
		&\ge
		\int\!\!\int q\bigl(\boldsymbol{f},\boldsymbol{u}\bigr)\,
		\log \frac{p\bigl(\boldsymbol{y}\mid\boldsymbol{f}\bigr)\,
			p\bigl(\boldsymbol{f}\mid\boldsymbol{u}\bigr)\,
			p(\boldsymbol{u})}
		{q\bigl(\boldsymbol{f},\boldsymbol{u}\bigr)}
		\,d\boldsymbol{f}\,d\boldsymbol{u}
		\;=\;\mathcal{L}\,.
	\end{split}
\end{equation}


We can further simplify $\mathcal{L}$ to get

\begin{equation}
	\begin{split}
		\mathcal{L}
		&= \int\!\!\int q(\boldsymbol{f},\boldsymbol{u})
		\log \frac{p(\boldsymbol{y}\mid\boldsymbol{f})\,
			p(\boldsymbol{f}\mid\boldsymbol{u})\,
			p(\boldsymbol{u})}
		{q(\boldsymbol{f},\boldsymbol{u})}
		\,d\boldsymbol{f}\,d\boldsymbol{u}
		\\
		&= \int\!\!\int q(\boldsymbol{f},\boldsymbol{u})
		\log p(\boldsymbol{y}\mid\boldsymbol{f})
		\,d\boldsymbol{f}\,d\boldsymbol{u}
		\;-\;
		\int q(\boldsymbol{u})
		\log \frac{q(\boldsymbol{u})}{p(\boldsymbol{u})}
		\,d\boldsymbol{u}
		\\
		&= \int q(\boldsymbol{f})
		\log p(\boldsymbol{y}\mid\boldsymbol{f})
		\,d\boldsymbol{f}
		\;-\;
		\sum_{q=1}^{Q}
		\mathrm{KL}\bigl\{q(\boldsymbol{u}_{q})\parallel p(\boldsymbol{u}_{q})\bigr\}
		\\
		&= \mathbb{E}_{q(\boldsymbol{f})}\bigl\{\log p(\boldsymbol{y}\mid\boldsymbol{f})\bigr\}
		\;-\;
		\sum_{q=1}^{Q}
		\mathrm{KL}\bigl\{q(\boldsymbol{u}_{q})\parallel p(\boldsymbol{u}_{q})\bigr\}
		\\
		&= \sum_{d=1}^{D}
		\mathbb{E}_{q(\boldsymbol{f}_{d,1}),\dots,q(\boldsymbol{f}_{d,J_d})}
		\bigl\{\log p(\boldsymbol{y}_{d}\mid\boldsymbol{f}_{d,1},\dots,\boldsymbol{f}_{d,J_d})\bigr\}
		\;-\;
		\sum_{q=1}^{Q}
		\mathrm{KL}\bigl\{q(\boldsymbol{u}_{q})\parallel p(\boldsymbol{u}_{q})\bigr\}
		\\
		&= \sum_{n=1}^{N}\sum_{d=1}^{D}
		\mathbb{E}_{q(f_{d,1,n}),\dots,q(f_{d,J_d,n})}
		\bigl\{\log p(y_{d,n}\mid f_{d,1,n},\dots,f_{d,J_d,n})\bigr\}
		\;-\;
		\sum_{q=1}^{Q}
		\mathrm{KL}\bigl\{q(\boldsymbol{u}_{q})\parallel p(\boldsymbol{u}_{q})\bigr\},
	\end{split}
\end{equation}

being \( \text{KL}\{q(\boldsymbol{u}_q)\parallel p(\boldsymbol{u}_q)\} \) the Kullback-Leibler divergence between two Gaussian distributions, with a closed form solution, working as a regularization factor.

\subsection{Adam + Natural Gradient Optimization}

The optimization step for the Chained LMC GP involves several key parameters: kernel parameters (such as characteristic lengthscales and output scale for the exponential quadratic kernel), the set of inducing point locations, and the variational distribution parameters. According to \cite{giraldo2021fully}, there is a strong dependency between the variational parameters and the others, making the model extremely sensitive to any changes in these variables. Additionally, since the ELBO loss function is generally non-convex, stochastic gradient optimization methods often converge to poor local minima.

Recent research has found a beneficial use of Natural Gradient (NG) optimization to overcome the issue mentioned above. The work developed by \cite{pmlr-v84-salimbeni18a} demonstrates the advantages of NG for variational GPs optimization with minimal effort and how to combine it with the Adam optimizer, providing a hybrid optimization framework. In the same spirit, we propose an optimization scheme where the variational parameters are governed by Natural Gradient (NG) optimization, while the others are optimized using the Adam optimizer and we refer to this Adam + NG. That approach allows us to achieve a similar optimization performance as Adam's as if we were optimizing an exact model.


\subsection{Adaptative Multi-Rate LSTM (AMR LSTM)}

\subsubsection{Problem Setting}

Consider a train dataset, comprising \(N\) input-output pairs of examples denoted as  $\mathcal{D} = \{(\boldsymbol{X}{i},\,y{i})\}_{i=1}^N$, where 
$\boldsymbol{X}{i}\in\mathcal{X}$ and $y{i}\in\mathcal{Y}$ belongs to some input and output space respectively. Here $\mathcal{X}$ is the space of collections of, at most, \(P\) vector sequences data observed within a fixed window of \(T\) fundamental time steps:
\[
\boldsymbol{X}_i
= \bigl\{\boldsymbol{x}{i}^{(1)}, \dots, \boldsymbol{x}{i}^{(t)}, \dots,\boldsymbol{x}_{i}^{(T)}\bigr\}
\].

At each time step \(t \in \{1, \cdots, T\}\}\), 

\begin{center}
	\begin{tikzpicture}[x=1.5cm,y=1cm]
		% user‐changeable parameters:
		\newcommand{\Tmin}{0}    % lowest time index
		\newcommand{\Tmax}{5}   % highest time index
		\newcommand{\Pdim}{3}    % number of rows (p = 1…Pdim)
		\newcommand{\Ndim}{3}    % number of depth‐layers (N = 1…Ndim)
		% how far out braces sit (tweak if you want more/less clearance):
		\newcommand{\Xoff}{0.3}  
		\newcommand{\Yoff}{0.3}  
		% Draw each layer N = 1…Ndim (1 is backmost, Ndim is front)
		\foreach \N in {1,...,\Ndim} {
			% how much to shift layer \N back in x,y
			\pgfmathsetmacro{\shift}{0.5*(\Ndim-\N)}
			\begin{scope}[shift={(\shift,\shift)}]
				% for each row p
				\foreach \p in {1,...,\Pdim} {
					% compute its y‐coordinate so p=1 is the top
					\pgfmathtruncatemacro{\y}{\Pdim-\p}
					% and for each time index i
					\foreach \i in {\Tmin,...,\Tmax} {
						% pattern: one blue then (p-1) reds
						\pgfmathtruncatemacro{\modres}{mod(\i,\p)}
						\ifnum\modres=0
						\def\fillcolor{MyAccent!20}
						\def\labeltext{$x_{\i,\p}^{(\N)}$}
						\else
						\def\fillcolor{red!20}
						\def\labeltext{}
						\fi
						\fill[fill=\fillcolor,draw=MyDarkBlue]
						(\i,\y) rectangle ++(1,1);
						\node[text=black,font=\footnotesize]
						at (\i+0.5,\y+0.5) {\labeltext};
					}
				}
			\end{scope}
		}
		
		% ─────── Braces ─────────────────────────────────────────────
		% T‐brace above, spanning all time‐cells:
		\draw[decorate,decoration={brace,amplitude=4pt}]
		(\Tmin+1,\Pdim+\Yoff+1) -- (\Tmax+2,\Pdim+\Yoff+1)
		node[midway,above=4pt] {$T$};
		
		% p‐brace at right, spanning all rows:
		\draw[decorate,decoration={brace,amplitude=4pt,mirror}]
		(\Tmax+2+\Xoff,1) -- (\Tmax+2+\Xoff,\Pdim+1)
		node[midway,right=4pt] {$P$};
		
		% N‐brace (diagonal) from front layer to back layer:
		\pgfmathsetmacro{\Nshift}{0.5*(\Ndim-1)}
		\pgfmathsetmacro{\startX}{\Tmin-\Xoff}
		\pgfmathsetmacro{\startY}{\Pdim}
		\pgfmathsetmacro{\endX}{\Tmin+\Nshift-\Xoff}
		\pgfmathsetmacro{\endY}{\Pdim+\Nshift}
		\draw[decorate,decoration={brace,amplitude=4pt}]
		(\startX,\startY) -- (\endX,\endY)
		node[midway,above left=2pt] {$N$};
	\end{tikzpicture}
\end{center}