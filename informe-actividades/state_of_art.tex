\section{State of Art}

%%%%% Emotion detencton form wearable devices

Wearable devices allows to measure physiological biomarkers in a non-intrusive fashion. The collected data may correlate with stress levels, emotion states and biological responses. Those measurement often include Hear Rate Variability (HRV), Electrodermal Activity (ADA), Heart Rate (HR) and three axis acceleration (ACC) \cite{Vos2023}. Advance in machine learning techniques have allowed to construct models that predict sentimental state based on recorded biomarkers, understanding that human health extends to mental well-being.

Authors in \cite{Zhu2023} evaluate a set of traditional machine learning algorithms to predict people's stress based on EDA activity, including K-Nearest Neighbor, Support Vector Machine (SVM), Naive Bayes, Logistic Regression and Random Forest. Those model were trained fed by a feature vector containing statistic of EDA data, and raw data itself for comparing classification results. Researchers reported best performance for SVM model, and no consistency training with raw data or feature vector.



