\section[\empty]{Objective 3: Long Short-Term Memory (LSTM) Architecture}

\begin{frame}{Objective 3: Long Short-Term Memory (LSTM) Architecture}
\begin{columns}[t]
\begin{column}{0.65\textwidth}


\begin{tikzpicture}[
	% GLOBAL CFG
	font=\sf \scriptsize,
	>=LaTeX,
	% Styles
	cell/.style={% For the main box
		rectangle, 
		rounded corners=5mm, 
		draw,
		very thick,
	},
	operator/.style={%For operators like +  and  x
		circle,
		draw,
		inner sep=-0.5pt,
		minimum height =.2cm,
	},
	function/.style={%For functions
		ellipse,
		draw,
		inner sep=1pt
	},
	ct/.style={% For external inputs and outputs
		circle,
		draw,
		line width = .75pt,
		minimum width=1cm,
		inner sep=1pt,
	},
	gt/.style={% For internal inputs
		rectangle,
		draw,
		minimum width=4mm,
		minimum height=3mm,
		inner sep=1pt
	},
	mylabel/.style={% something new that I have learned
		font=\scriptsize\sffamily
	},
	ArrowC1/.style={% Arrows with rounded corners
		rounded corners=.25cm,
		thick,
	},
	ArrowC2/.style={% Arrows with big rounded corners
		rounded corners=.5cm,
		thick,
	},
	]
	
	%Start drawing the thing...    
	% Draw the cell: 
	\node [cell, minimum height =4cm, minimum width=6cm] at (0,0){} ;
	
	% Draw inputs named ibox#
	\node [gt] (ibox1) at (-2,-0.75) {$\sigma$};
	\node [gt] (ibox2) at (-1.5,-0.75) {$\sigma$};
	\node [gt, minimum width=1cm] (ibox3) at (-0.5,-0.75) {Tanh};
	\node [gt] (ibox4) at (0.5,-0.75) {$\sigma$};
	
	% Draw opérators   named mux# , add# and func#
	\node [operator] (mux1) at (-2,1.5) {$\times$};
	\node [operator] (add1) at (-0.5,1.5) {+};
	\node [operator] (mux2) at (-0.5,0) {$\times$};
	\node [operator] (mux3) at (1.5,0) {$\times$};
	\node [function] (func1) at (1.5,0.75) {Tanh};
	
	% Draw External inputs? named as basis c,h,x
	\node[ct, label={[mylabel]Cell}] (c) at (-4,1.5) {$\boldsymbol{c}_{t-1}$};
	\node[ct, label={[mylabel]Hidden}] (h) at (-4,-1.5) {$\boldsymbol{h}_{t-1}$};
	\node[ct, label={[mylabel]left:Input}] (x) at (-2.5,-3) {{$\boldsymbol{x}_{t}^{(i)}$}};
	
	% Draw External outputs? named as basis c2,h2,x2
	\node[ct, label={[mylabel]\empty}] (c2) at (4,1.5) {$\boldsymbol{c}_{t}$};
	\node[ct, label={[mylabel]\empty}] (h2) at (4,-1.5) {$\boldsymbol{h}_{t}$};
%	\node[ct, label={[mylabel]left:\empty}] (x2) at (2.5,3) {$\boldsymbol{h}_{t}$};
	
	% Start connecting all.
	%Intersections and displacements are used. 
	% Drawing arrows    
	\draw [ArrowC1] (c) -- (mux1) -- (add1) -- (c2);
	
	% Inputs
	\draw [ArrowC2] (h) -| (ibox4);
	\draw [ArrowC1] (h -| ibox1)++(-0.5,0) -| (ibox1); 
	\draw [ArrowC1] (h -| ibox2)++(-0.5,0) -| (ibox2);
	\draw [ArrowC1] (h -| ibox3)++(-0.5,0) -| (ibox3);
	\draw [ArrowC1] (x) -- (x |- h)-| (ibox3);
	
	% Internal
	\draw [->, ArrowC2] (ibox1) -- (mux1);
	\draw [->, ArrowC2] (ibox2) |- (mux2);
	\draw [->, ArrowC2] (ibox3) -- (mux2);
	\draw [->, ArrowC2] (ibox4) |- (mux3);
	\draw [->, ArrowC2] (mux2) -- (add1);
	\draw [->, ArrowC1] (add1 -| func1)++(-0.5,0) -| (func1);
	\draw [->, ArrowC2] (func1) -- (mux3);
	
	%Outputs
	\draw [-, ArrowC2] (mux3) |- (h2);
%	\draw (c2 -| x2) ++(0,-0.1) coordinate (i1);
%	\draw [-, ArrowC2] (h2 -| x2)++(-0.5,0) -| (i1);
%	\draw [-, ArrowC2] (i1)++(0,0.2) -- (x2);
	
\end{tikzpicture}
\end{column}

    % equations column, also top‐aligned
\begin{column}[t]{0.45\textwidth}

	\small
	\begin{align*}
\boldsymbol{i}_t &= \sigma\bigl(\boldsymbol{W}_{ii}\,\boldsymbol{x}_t^{(i)} 
+ \boldsymbol{W}_{hi}\,\boldsymbol{h}_{t-1}
+ \boldsymbol{b}_i\bigr),\\
\boldsymbol{f}_t &= \sigma\bigl(\boldsymbol{W}_{if}\,\boldsymbol{x}_t^{(i)} 
+ \boldsymbol{W}_{hf}\,\boldsymbol{h}_{t-1}
+ \boldsymbol{b}_f\bigr),\\
\boldsymbol{g}_t &= \tanh\bigl(\boldsymbol{W}_{ig}\,\boldsymbol{x}_t^{(i)} 
+ \boldsymbol{W}_{hg}\,\boldsymbol{h}_{t-1}
+ \boldsymbol{b}_g\bigr),\\
\boldsymbol{o}_t &= \sigma\bigl(\boldsymbol{W}_{io}\,\boldsymbol{x}_t^{(i)} 
+ \boldsymbol{W}_{ho}\,\boldsymbol{h}_{t-1}
+ \boldsymbol{b}_o\bigr),\\
\boldsymbol{C}_t &= \boldsymbol{f}_t \odot \boldsymbol{C}_{t-1}
+ \boldsymbol{i}_t \odot \boldsymbol{g}_t,\\
\boldsymbol{h}_t &= \boldsymbol{o}_t \odot \tanh\bigl(\boldsymbol{C}_t\bigr).
	\end{align*}
	
\(\odot\) is the Hadamard product, and \(\sigma\) the sigmoid function.

	
\end{column}
\end{columns}


\end{frame}
	
\begin{frame}[t]{Missing‐Value Imputation}
	\begin{columns}[t]
		\begin{column}{0.55\textwidth}
			\begin{block}{}
				At each time step \(t\), missing channel entries in \(\boldsymbol{x}_t^{(i)}\) can be imputed sequentially from the hidden state via an affine map:
				\[
				\hat{x}_{t,j}^{(i)}
				= \boldsymbol{w}_j^{\mathsf T}\,\boldsymbol{h}_t + b_j,
				\quad j = 1,\dots,p.
				\]
			\end{block}

		\end{column}
		\begin{column}{0.45\textwidth}
			\[
			\begin{aligned}
				\{\boldsymbol{W}_{ii},\,\boldsymbol{W}_{if},\,\boldsymbol{W}_{ig},\,\boldsymbol{W}_{io}\}
				&\in \mathbb{R}^{H \times p},\\
				\{\boldsymbol{W}_{hi},\,\boldsymbol{W}_{hf},\,\boldsymbol{W}_{hg},\,\boldsymbol{W}_{ho}\}
				&\in \mathbb{R}^{H \times H},\\
				\{\boldsymbol{b}_{i},\,\boldsymbol{b}_{f},\,\boldsymbol{b}_{g},\,\boldsymbol{b}_{o}\}
				&\in \mathbb{R}^{H},\\
				\boldsymbol{w}_j &\in \mathbb{R}^{H},\\
				b_j &\in \mathbb{R},\\
				\boldsymbol{h}_t,\,\boldsymbol{C}_t &\in \mathbb{R}^{H}.
			\end{aligned}
			\]
			Here \(H\) denotes the dimensionality of the hidden state.
		\end{column}
	\end{columns}
\end{frame}

\begin{frame}[fragile]{Architecture}
	\resizebox{\textwidth}{!}{
		\tikzset{trapezium stretches=true}
		\centering
		\begin{tikzpicture}[font=\sffamily, >=stealth, node distance=8mm]
			% Latent variable (z)
			\node[rectangle, fill=MyAccent, text=white, font=\sffamily\large, minimum height=3cm, text width=3cm, align=center] (z) {Latent\\ Representation ($\boldsymbol{h}$)};
			
			% Encoder trapezoid
			\node[trapezium, fill=MyAccent!20, minimum width=4cm, minimum height=4cm,
			trapezium left angle=86, trapezium right angle=86, shape border rotate=270,
			anchor=east, left=5mm of z, font=\sffamily\large, text=MyDarkBlue] (enc) {Encoder (LSTM)};
			
			
			% Gaussian Process circle
			\node[circle, fill=MyAccent!20, text=MyDarkBlue, font=\sffamily\small,
			below=2cm of z, minimum size=2cm, align=center] (gp) {Gaussian\\Process};
			
			% Arrows and labels
			\draw[<-, MyDarkBlue, thick] (enc.west) -- ++(-1.5,0) node[left, align=right, text=MyDarkBlue] {Input};
			\draw[->, MyAccent, thick] (z.south) -- (gp.north);
			\draw[->, MyDarkBlue, thick] (gp.east) -- ++(1.5,0) node[right, align=left, text=MyDarkBlue] {Predictive Distribution\\(Inference)};
		\end{tikzpicture}
	}
\end{frame}