\section[\empty]{Objective 1: Foundation Model}

\begin{frame}[fragile]{Objective 1: Foundation Model}
	
	\begin{block}{}
		We aims to train a general model using large amounts of data and tasks that can be fine-tuned easily in different downstream applications.
	\end{block}
	\centering
	\resizebox{\textwidth}{!}{
		\begin{tikzpicture}[font=\sffamily, >=stealth, node distance=3cm, thick]
			
			% Left Square (Training)
			\node[rectangle, draw=MyAccent, fill=MyLightGray, rounded corners, align=left, text=MyDarkBlue] (training) {
				\textbf{Data}\\[5pt]
				- Text\\
				- Images\\
				- Sensors\\
				- Speech
			};
			
			% Foundation Model (center)
			\node[rectangle, draw=MyAccent, fill=MyAccent!20, rounded corners, minimum width=3.5cm, minimum height=2cm, right=of training, text=MyDarkBlue] (foundation) {
				\textbf{Foundation Model}
			};
			
			% Right Square (Tasks)
			\node[rectangle, draw=MyAccent, fill=MyLightGray, rounded corners, align=left, right=of foundation, text=MyDarkBlue] (tasks) {
				\textbf{Task}\\[5pt]
				- Object Recognition\\
				- Text Generation\\
				- Sentiment Analysis\\
				- Image Captioning
			};
			
			% Arrows
			\draw[->, MyDarkBlue] (training.east) -- node[above, text=MyDarkBlue]{Pre-Training} (foundation.west);
			\draw[->, MyDarkBlue] (foundation.east) -- node[above, text=MyDarkBlue]{Fine-Tuning} (tasks.west);
			
		\end{tikzpicture}
	}
\end{frame}


\begin{frame}[fragile]{Architecture}
	\begin{block}{}
		We can leverage large amounts of unlabeled data to learn hidden‐state representations without committing to any specific downstream task.
	\end{block}
	
	\resizebox{\textwidth}{!}{
		\tikzset{trapezium stretches=true}
		\centering
		\begin{tikzpicture}[font=\sffamily, >=stealth, node distance=8mm]
			% Latent variable (z)
			\node[rectangle, fill=MyAccent, text=white, font=\sffamily\large, minimum height=3cm, text width=3cm, align=center] (z) {Latent\\ Representation ($\boldsymbol{f}$)};
			
			% Encoder trapezoid
			\node[trapezium, fill=MyAccent!20, minimum width=4cm, minimum height=4cm,
			trapezium left angle=86, trapezium right angle=86, shape border rotate=270,
			anchor=east, left=5mm of z, font=\sffamily\large, text=MyDarkBlue] (enc) {Encoder};
			
			% Decoder trapezoid
			\node[trapezium, fill=MyAccent!20, minimum width=4cm, minimum height=4cm,
			trapezium left angle=86, trapezium right angle=86, shape border rotate=90,
			anchor=west, right=5mm of z, font=\sffamily\large, text=MyDarkBlue] (dec) {Decoder};
			
		
			% Arrows and labels
			\draw[<-, MyDarkBlue, thick] (enc.west) -- ++(-1.5,0) node[left, align=right, text=MyDarkBlue] {Input};
			\draw[->, MyDarkBlue, thick] (dec.east) -- ++(1.5,0) node[right, align=left, text=MyDarkBlue] {Reconstruction};
%			\draw[->, MyAccent, thick] (z.south) -- (gp.north);
%			\draw[->, MyDarkBlue, thick] (gp.east) -- ++(1.5,0) node[right, align=left, text=MyDarkBlue] {Predictive Distribution\\(Inference)};
		\end{tikzpicture}
	}
\end{frame}

\begin{frame}{Transformer and Attention Mechanism}
	\begin{columns}[c] % [c] = vertically center
		
		% Left column
		\begin{column}{0.48\textwidth}
			\centering
			\scalebox{0.5}{
\begin{tikzpicture}[
	>=LaTeX, % Use default LaTeX arrows
	very thick,
	arrow/.style={
		-latex,
		very thick,
		rounded corners=0.2cm
	},
	block/.style={
		rectangle,
		fill=gray!10,
		rounded corners=3mm,
		draw,
		very thick
	},
	layer/.style={
		rectangle,
		fill=white!10,
		rounded corners=1mm,
		inner xsep=0em,
		inner ysep=0.25em,
		minimum height=1.4em,
		align=center,
		text width=2.5cm,
		draw,
		very thick
	},
	input/.style={ % Input or output node
		circle,
		minimum width=2.25em,
		draw,
		fill=gray!10,
		thick
	},
	do path picture/.style={%
		path picture={%
			\pgfpointdiff{\pgfpointanchor{path picture bounding box}{south west}}%
			{\pgfpointanchor{path picture bounding box}{north east}}%
			\pgfgetlastxy\x\y%
			\tikzset{x=\x/2,y=\y/2}%
			#1
		}
	},
	sin wave/.style={do path picture={    
			\draw [line cap=round] (-3/4,0)
			sin (-3/8,1/2) cos (0,0) sin (3/8,-1/2) cos (3/4,0);
		}
	}
	]
	
	\node[input] (iemb) at (1.25,-0.45) {$\mathbf{x}$};
	\node[input] (oemb) at (5.25,-0.45) {$\mathbf{y}$};
	
	% Sums
	\node[circle, draw, minimum size=0.25em, inner sep=0pt, above=1em of iemb] (sum1) {$\mathbf{+}$};
	\node[circle, draw, minimum size=0.25em, inner sep=0pt, above=1em of oemb] (sum2) {$\mathbf{+}$};
	% Positional Encoding
	\node [circle, draw, sin wave, minimum size=2em, left=0.8em of sum1] (pe1) {};
	\node [circle, draw, sin wave, minimum size=2em, right=0.8em of sum2] (pe2) {};
	
	% Encoder, 1st layer
	\node[layer] (add1) at (1.25,3.43) {Add \& Norm};
	\node[layer] (attn1) at (1.25,2.58) {Multi-Head \vspace{-0.05cm} \linebreak Attention};
	\draw[] (attn1) -- (add1);
	% Encoder 2nd sub-layer
	\node[layer] (add4) at (1.25,5.98) {Add \& Norm};
	\node[layer] (ff1) at (1.25,5.13) {Feed \vspace{-0.05cm} \linebreak Forward};
	\draw[] (ff1) -- (add4);
	
	% Decoder 1st sub-layer
	\node[layer] (add3) at (5.25,3.83) {Add \& Norm};
	\node[layer] (attn3) at (5.25,2.78) {Masked \vspace{-0.05cm} \linebreak Multi-Head \vspace{-0.05cm} \linebreak Attention};
	\draw[] (attn3) -- (add3);
	% Decoder, 2nd sub-layer
	\node[layer] (add2) at (5.25,6.38) {Add \& Norm};
	\node[layer] (attn2) at (5.25,5.53) {Multi-Head \vspace{-0.05cm} \linebreak Attention};
	\draw[] (attn2) -- (add2);
	% Decoder 3rd sub-layer
	\node[layer] (add5) at (5.25,8.93) {Add \& Norm};
	\node[layer] (ff2) at (5.25,8.08) {Feed \vspace{-0.05cm} \linebreak Forward};
	\draw[] (ff2) -- (add5);
	
	\coordinate (d1) at ($(attn3.south east) + (0.75,-0.7)$);
	\coordinate (d2) at ($(add5.north west) + (-0.15,0.05)$);
	\coordinate (e1) at ($(attn1.south east) + (0.15,-0.7)$);
	\coordinate (e2) at ($(add4.north west) + (-0.75,0.05)$);
	\begin{scope}[on background layer]
		\node[block, fit=(d1) (d2)] (decoder) {};
		\node[block, fit=(e1) (e2)] (encoder) {};
	\end{scope}
	
	% Classifier
	\node[layer, above=1.5em of add5] (linear) {Linear};
	\node[layer, above=1em of linear] (softmax) {Softmax};
	\node[input, above=1em of softmax] (probs) {$\hat{\mathbf{y}}$};
	
	\draw[arrow] (add1) -- (ff1);
	\draw[arrow] (add2) -- (ff2);
	\draw[arrow] (add5) -- (linear);
	\draw[arrow] (linear) -- (softmax);
	\draw[arrow] (iemb) -- (sum1);
	\draw[arrow] (sum1) -- (attn1);
	\draw[arrow] (oemb) -- (sum2);
	\draw[arrow] (sum2) -- (attn3);
	\draw[] (sum1) -- (pe1);
	\draw[] (sum2) -- (pe2);
	
	\draw[arrow] (ff1.south)++(0, -0.6) -| ($(add4.west) + (-0.6,-0.5)$) |- (add4.west);
	\draw[arrow] (attn1.south)++(0, -0.6) -| ($(add1.west) + (-0.6,-0.5)$) |- (add1.west);
	
	\draw[arrow] (attn3.south)++(0, -0.6) -| ($(add3.east) + (0.6,-0.5)$) |- (add3.east);
	\draw[arrow] (attn2.south)++(0, -0.6) -| ($(add2.east) + (0.6,-0.5)$) |- (add2.east);
	\draw[arrow] (ff2.south)++(0, -0.6) -| ($(add5.east) + (0.6,-0.5)$) |- (add5.east);
	
	\draw[arrow] (attn1.south)++(0, -0.4) -| ($(attn1.south) + (-1,0)$);
	\draw[arrow] (attn1.south)++(0, -0.4) -| ($(attn1.south) + (1,0)$);
	
	\draw[arrow] (attn3.south)++(0, -0.4) -| ($(attn3.south) + (-1,0)$);
	\draw[arrow] (attn3.south)++(0, -0.4) -| ($(attn3.south) + (1,0)$);
	
	\draw[arrow] (add4.north) |- ($(add4.north) + (1,1)$) -| ($(add4.north) + (2,-1)$) |- ($(attn2.south) + (-1,-0.4)$) -| ($(attn2.south) + (-1,0)$);
	\draw[arrow] (add4.north) |- ($(add4.north) + (1,1)$) -| ($(add4.north) + (2,-1)$) |- ($(attn2.south) + (0,-0.4)$) -| ($(attn2.south)$);
	\draw[arrow] (add3.north) |- ($(attn2.south) + (0.5,-0.4)$) -| ($(attn2.south) + (1,0)$);
	
%	\node[] at ($(oemb.east) + (1.4,0)$) {(shifted right)};
	\node[] at ($(pe1.west) + (-0.3,0)$) {$\mathbf{P}$};
	\node[] at ($(pe2.east) + (0.3,0)$) {$\mathbf{P}$};
	
	\draw[arrow] (iemb) -- (sum1);
	\draw[arrow] (oemb) -- (sum2);
	\draw[arrow] (softmax) -- (probs);
	
	\node[anchor=east] at ($(encoder.west) + (-0.2,0)$) {$N\times$};
	\node[anchor=west] at ($(decoder.east) + (0.2,0)$) {$N\times$};
\end{tikzpicture}
}
		\end{column}
		
		% Right column
		\begin{column}{0.48\textwidth}
			\centering
			\scalebox{0.7}{\input{figures/multihead_attention.tex}}
		\end{column}
%		\begin{itemize}
%			\item $N$: Number of transformer layers
%			\item $\boldsymbol{P}$: Positional embedding
%		\end{itemize}
	\end{columns}
\end{frame}

