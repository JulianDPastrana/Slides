\section{Objective 2}
\begin{frame}{Objetive 2: Stochastic Approach}
		\centering
		\begin{tikzpicture}
			\begin{groupplot}[
				group style={
					group name=my plots,
					group size=2 by 1,
					horizontal sep=1cm,
				},
				width=5cm,
				height=4cm,
				scale only axis,
				axis lines=left,
				ymin=0, ymax=1,
				xmin=0, xmax=5.5,
				xlabel style={text=MyDarkBlue},
				title style={text=MyDarkBlue},
				]
				% Left subplot: Point Estimate Prediction
				\nextgroupplot[
				title={Point Estimation},
				xticklabel=\empty,
				yticklabel=\empty,
				]
				\addplot+[smooth, thick, color=MyAccent!30, mark=*,
				mark options={fill=MyAccent, draw=MyAccent}
				] coordinates {(0,0.2) (1,0.3) (2,0.5) (3,0.7) (4,0.8)};
				\addplot[only marks, mark=*, MyDarkBlue] coordinates {(5,0.85)};
				
				% Right subplot: Distribution Estimation with Error Bars
				\nextgroupplot[
				title={Distribution Estimation},
				xticklabel=\empty,
				yticklabel=\empty,
				]
				\addplot+[smooth, thick, color=MyAccent!30, mark=*,
				mark options={fill=MyAccent, draw=MyAccent}
				] coordinates {(0,0.2) (1,0.3) (2,0.5) (3,0.7) (4,0.8)};
				
				% Predictive distribution at time=5
				\addplot[name path=right, domain=0.65:1.05, variable=\y, samples=100, draw=none]
				({5 + 0.2*exp(-((\y-0.85)^2)/(2*0.004))}, {\y});
				
				\addplot[name path=left, domain=0.65:1.05, variable=\y, samples=100, draw=none]
				({5 - 0.2*exp(-((\y-0.85)^2)/(2*0.004))}, {\y});
				
				\addplot[fill=MyAccent!50, fill opacity=0.4, draw=none]
				fill between[of=right and left];
				
				\addplot[only marks, mark=*, MyDarkBlue] coordinates {(5,0.85)};
				
			\end{groupplot}
		\end{tikzpicture}
		
		\begin{block}{}
			Sometimes, a single point prediction is not enough. How confident is the model in its prediction?
		\end{block}
\end{frame}

\begin{frame}{Likelihood Model}
	\begin{block}{}
		For this propose, consider a likelihood functions that rule the generation of recorded targets \(\boldsymbol{Y}\) from inputs \(\boldsymbol{X}\) through some set of parameters \(\boldsymbol{\theta} \subseteq \mathbb{R}^J\) 
		
	\end{block}
	
	\begin{equation*}
		p\left(\boldsymbol{Y} \mid \boldsymbol{\theta}(\boldsymbol{X})\right) = \prod_{n=1}^N p\left(\boldsymbol{y}_n \mid \boldsymbol{\theta}(\boldsymbol{x}_n)\right). 
	\end{equation*}
	
	Each element of \( \boldsymbol{\theta}(\boldsymbol{x}) \), denoted as \( \theta_{j}(\boldsymbol{x}) \), could be restricted to some subset of \( \mathbb{R} \). To handle that, we model \( \theta_{j}(\boldsymbol{x}) = h_{j}(f_{j}(\boldsymbol{x})) \) as a transformation of an unrestricted latent variable \( f_{j}(\boldsymbol{x}) \) via a link function \( h_{j} \). Our task boils down to finding the latent vector function \(\boldsymbol{f}(\boldsymbol{x}) = [f_1(\boldsymbol{x}), \cdots, f_J(\boldsymbol{x})]^\top \in \mathbb{R}^J\).
\end{frame}


\begin{frame}[fragile]{Architecture}
	\resizebox{\textwidth}{!}{
		\tikzset{trapezium stretches=true}
		\centering
		\begin{tikzpicture}[font=\sffamily, >=stealth, node distance=8mm]
			% Latent variable (z)
			\node[rectangle, fill=MyAccent, text=white, font=\sffamily\large, minimum height=3cm, text width=3cm, align=center] (z) {Latent\\ Representation ($\boldsymbol{f}$)};
			
			% Encoder trapezoid
			\node[trapezium, fill=MyAccent!20, minimum width=4cm, minimum height=4cm,
			trapezium left angle=86, trapezium right angle=86, shape border rotate=270,
			anchor=east, left=5mm of z, font=\sffamily\large, text=MyDarkBlue] (enc) {Encoder};
			
			% Decoder trapezoid
			\node[trapezium, fill=MyAccent!20, minimum width=4cm, minimum height=4cm,
			trapezium left angle=86, trapezium right angle=86, shape border rotate=90,
			anchor=west, right=5mm of z, font=\sffamily\large, text=MyDarkBlue] (dec) {Decoder};
			
			% Gaussian Process circle
			\node[circle, fill=MyAccent!20, text=MyDarkBlue, font=\sffamily\small,
			below=2cm of z, minimum size=2cm, align=center] (gp) {Likelihood\\Model};
			
			% Arrows and labels
			\draw[<-, MyDarkBlue, thick] (enc.west) -- ++(-1.5,0) node[left, align=right, text=MyDarkBlue] {Input};
			\draw[->, MyDarkBlue, thick] (dec.east) -- ++(1.5,0) node[right, align=left, text=MyDarkBlue] {Reconstruction};
			\draw[->, MyAccent, thick] (z.south) -- (gp.north);
			\draw[->, MyDarkBlue, thick] (gp.east) -- ++(1.5,0) node[right, align=left, text=MyDarkBlue] {Predictive Distribution\\(Inference)};
		\end{tikzpicture}
	}
\end{frame}

%\begin{frame}{Chained Correlated GP }
%	\centering
%	\begin{tikzpicture}[x=4.5cm,y=1.5cm]
%		% \message{^^JNeural network activation}
%		\def\NI{4} % number of nodes in input layer
%		\def\NM{4} % number of nodes in middle layer  
%		\def\NO{2} % number of nodes in output layer
%		\def\yshift{0.0} % shift last node for dots
%		
%		% INPUT LAYER
%		\foreach \i [evaluate={\c=int(\i==\NI); \y=\NI/2-\i-\c*\yshift; \index=(\i<\NI?int(\i):"Q");}]
%		in {1,...,\NI}{ % loop over nodes
%			\node[node in,outer sep=0.6] (NI-\i) at (0,\y) {$u_{\index}$};
%		}
%		
%		% MIDDLE LAYER
%		
%		% MIDDLE LAYER with specific labels
%		\node[node hidden] (NM-1) at (1,\NM/2-1) {$f_{1, J_1}$};
%		\node[node hidden] (NM-2) at (1,\NM/2-2) {$f_{2, J_1}$};
%		\node[node hidden] (NM-3) at (1,\NM/2-3) {$f_{1, J_D}$};
%		\node[node hidden] (NM-4) at (1,\NM/2-4) {$f_{D, J_D}$};
%		
%		\foreach \i [evaluate={\c=int(\i==\NM); \y=\NM/2-\i-\c*\yshift; \index=(\i<\NM?int(\i):"D");}]
%		in {\NM,...,1}{ % loop over nodes
%			\ifnum\i=1 % highlighted node
%			\foreach \j [evaluate={\index=(\j<\NI?int(\j):"Q");}] in {1,...,\NI}{ % loop over nodes in previous layer
%				\draw[connect,white,line width=1.2] (NI-\j) -- (NM-\i);
%				\draw[connect] (NI-\j) -- (NM-\i)
%				node[pos=0.50] {\contour{white}{$a_{1, 1,\index}$}};
%			}
%			\else % other light-colored nodes
%			\foreach \j in {1,...,\NI}{ % loop over nodes in previous layer
%				\draw[connect,white,line width=1.2] (NI-\j) -- (NM-\i);
%				\draw[connect,myblue!20] (NI-\j) -- (NM-\i);i
%			}
%			\fi
%		}
%		
%		
%		% OUTPUT LAYER
%		\foreach \i [evaluate={\c=int(\i==\NO); \y=\NO/2-\i-\c*\yshift; \index=(\i<\NO?int(\i):"D");}]
%		in {\NO,...,1}{ % loop over nodes
%			\ifnum\i=1 % highlighted node
%			\node[node out]
%			(NO-\i) at (2,\y) {$y_{\index}$};
%			\foreach \j [evaluate={\index=(\j<3?int(\j):"D");}] in {1,2}{ % loop over nodes 1 and 2 in middle layer
%				\draw[connect,white,line width=1.2] (NM-\j) -- (NO-\i);
%				\draw[connect, myred] (NM-\j) -- (NO-\i)
%				node[pos=0.50] {\contour{white}{$g_{\index,J_{1}}(\cdot)$}};
%			}
%			\else % other light-colored nodes
%			\node[node out]
%			(NO-\i) at (2,\y) {$y_{\index}$};
%			\foreach \j in {3,4}{ % loop over nodes 3 and 4 in middle layer
%				\draw[connect,white,line width=1.2] (NM-\j) -- (NO-\i);
%				\draw[connect,myred!20] (NM-\j) -- (NO-\i);
%			}
%			\fi
%		}
%		
%		
%		% DOTS
%		\path (NI-\NI) --++ (0,1+\yshift) node[midway,scale=1.2] {$\vdots$};
%		\path (NM-\NM) --++ (0,3+\yshift) node[midway,scale=1.2] {$\vdots$};
%		\path (NM-\NM) --++ (0,1+\yshift) node[midway,scale=1.2] {$\vdots$};
%		\path (NO-\NO) --++ (0,1+\yshift) node[midway,scale=1.2] {$\vdots$};
%		
%	\end{tikzpicture}
%	\scriptsize
%	\begin{columns}[T] % Top alignment
%		\begin{column}{0.33\textwidth}
%			\centering	
%			\textcolor{mygreen}{
%				\textbf{Independent Process}
%				\vspace{-1.3em}
%							\begin{equation*}
%									u_{q}(\mathbf{h}) \sim \mathcal{GP}(0, k_{q}(\mathbf{h}, \mathbf{h}'))
%								\end{equation*}
%			}
%		\end{column}
%		
%		\begin{column}{0.33\textwidth}
%			\centering
%			\textcolor{myblue}{
%				\textbf{Latent Process}
%				\vspace{-1.3em}
%							\begin{equation*}
%									f_{d,j}(\mathbf{h}) = \sum_{q=1}^Q a_{d,j,q} u_{q}(\mathbf{h})
%								\end{equation*}
%			}
%		\end{column}
%		
%		\begin{column}{0.33\textwidth}
%			\centering
%			\textcolor{myred}{
%				\textbf{Likelihood}
%				\vspace{-1.3em}
%							\begin{equation*}
%									\mathbf{y} \mid \mathbf{f} \sim \prod_{d=1}^D p(\theta_{d, 1}, \cdots, \theta_{d, J_d})
%								\end{equation*}
%							\vspace{-1.5em}
%							\begin{equation*}
%									\theta_{d, j} = g_{d, j}(f_{d, j})
%								\end{equation*}
%			}
%		\end{column}
%		
%	\end{columns}
%\end{frame}