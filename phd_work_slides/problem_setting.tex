\section{The Dataset}

\begin{frame}{Problem Setting}
	\begin{block}{}
		We define the multimodal input space as the Cartesian product of $P$ modality-specific input spaces:
		\[
		\mathcal{X} = \prod_{p=1}^{P} \mathcal{X}^{(p)},
		\]
		where each modality $\mathcal{X}^{(p)}$ may represent data from distinct sources.
	\end{block}
	
\end{frame}



\begin{frame}{Toadstool 2 Dataset}
		\begin{columns}[T] % Top alignment
		\begin{column}{0.4\textwidth}
			\includegraphics[width=\linewidth]{figures/toadstool.jpg}
			
		\end{column}
		\begin{column}{0.6\textwidth}
			\begin{block}{}
				The dataset consists of video, sensor, and demographic data collected from 10 participants playing a Super Mario Bros.
			\end{block}

		\vspace{0.5em}
		\centering
		\small
		\begin{tabular}{lcc}
		\toprule
		\textbf{Signal} & \textbf{Rate (Hz)} & \textbf{Channels} \\
		\midrule
		BVP  & 64 & 1      \\
		ACC  & 32 & 3 		\\
		EDA  & 4  & 1      \\
		HR   & 1  & 1      \\
		\bottomrule
		\end{tabular}


		\end{column}
	\end{columns}
\end{frame}


\begin{frame}{Toadstool 2 Dataset}
	\begin{columns}[T] % Top alignment
		\begin{column}{0.45\textwidth}
			\includegraphics[width=\linewidth]{figures/toadstool.jpg}
			\vspace{0.5cm}
			\begin{block}{Sequential Signals}
				\begin{itemize}
					\item \textbf{BVP:} 64\,Hz $\times$ 4\,s per sample
					\item \textbf{ACC:} 32\,Hz $\times$ 4\,s per sample (x,y,z)
					\item \textbf{EDA:} 4\,Hz $\times$ 4\,s per sample
					\item \textbf{HR:} 1\,Hz $\times$ 4\,s per sample
					\item \textbf{Samples per subject:} $\approx$210 (2,097 total)
					\item \textbf{Labels:} 7 classes (0:Anger, 1:Disgust, 2:Fear, 3:Happy, 4:Sad, 5:Surprised, 6:Neutral)
				\end{itemize}
			\end{block}
		\end{column}
		\begin{column}{0.55\textwidth}
			\begin{block}{}
				The dataset consists of video, sensor, and demographic data collected from ten participants playing Super Mario Bros.
			\end{block}
			
			\begin{block}{}
				Although volumetric measurements are recorded, they are reported in kilowatt-hours (kWh) by the hydroelectric power plants.
			\end{block}
		\end{column}
	\end{columns}
	\note{Emphasize the sequential signal sampling rates, total samples per subject, and label classes.}
\end{frame}

