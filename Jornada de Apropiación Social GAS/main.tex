\documentclass[aspectratio=169,12pt]{beamer}
\usepackage{graphicx}
\usepackage{subcaption}
\graphicspath{ {figures/} }





\begin{document}
\frame{\titlepage}

\begin{frame}{General Objective}
	\begin{block}{}
		Develop a tool for the optimal long-term planning of the operation of Colombia's natural gas transportation system that allows for the evaluation of multiple investment scenarios, taking into account existing infrastructure, uncertainty in supply and demand, and the effect of various climate scenarios, making efficient use of thermal and electrical energy.
	\end{block}
\end{frame}

\begin{frame}{Specific Objectives}
	\begin{block}{}
		\begin{enumerate}
			\item Develop a methodology for predicting the dispatch of thermoelectric power plants based on historical information provided by the grid operator, which allows establishing the average natural gas demand value for each plant for each of the periods to be considered within the planning horizon.
			
			\item Develop a methodology for predicting the availability of gas pipelines and compressors in the transmission network based on historical information on failures and the occurrence of various climatic phenomena such as El Niño and La Niña, which will make it possible to establish the transmission and compression capacity for each of the periods to be considered within the planning horizon.
		\end{enumerate}
	\end{block}
\end{frame}

\begin{frame}{Specific Objectives}
	\begin{block}{}
		\begin{enumerate}
			\setcounter{enumi}{3}
			\item Develop a stochastic and/or exact optimization strategy that allows for the identification of a set of alternatives and their respective execution periods for the long-term planning of Colombia's natural gas transportation network, in such a way as to minimize operating and investment costs, taking into account the current infrastructure of the system and the prediction models developed.
			
			\item Implement a tool for long-term planning of Colombia's natural gas transport network based on the optimization strategy and prediction models developed.
		\end{enumerate}
	\end{block}
\end{frame}
	
\end{document}
