\usepackage{pgfplots}
\usepgfplotslibrary{groupplots}
\usepgfplotslibrary{fillbetween}

\usepgfplotslibrary{external}
%\tikzexternalize
\pgfplotsset{compat=1.18}
\tikzsetexternalprefix{figures/}
\usetikzlibrary{positioning}
\usepackage{subcaption}
\usetheme{Copenhagen}
\usecolortheme{seahorse}

\definecolor{background_color}{RGB}{234,234,242}
\definecolor{lightgray204}{RGB}{204,204,204}
\definecolor{darkgray176}{RGB}{176,176,176}

\author[J. Pastrana]{%
	Julián David Pastrana Cortés
}

\AtBeginSection[]{
	\begin{frame}
		\frametitle{Table of Contents}
		\tableofcontents[currentsection]
	\end{frame}
}

\AtBeginDocument{%
	\title[Chained Correlated GPs]{Time Series Forecasting Based on Chained Correlated Gaussian Procesees}
	\subtitle{An Stochastic Approach}%
	\institute[UTP]{Universidad Tecnológica de Pereira}
	\date[2025]{\today}
	
}

\setbeamertemplate{page number in head/foot}[totalframenumber]




% For NN graph

\usepackage{listofitems} % for \readlist to create arrays
\usetikzlibrary{arrows.meta} % for arrow size
\usepackage[outline]{contour} % glow around text
\contourlength{1.4pt}

% COLORS
\usepackage{xcolor}
\colorlet{myred}{red!80!black}
\colorlet{myblue}{blue!80!black}
\colorlet{mygreen}{green!60!black}
\colorlet{myorange}{orange!70!red!60!black}
\colorlet{mydarkred}{red!30!black}
\colorlet{mydarkblue}{blue!40!black}
\colorlet{mydarkgreen}{green!30!black}

% STYLES
\tikzset{
	>=latex, % for default LaTeX arrow head
	node/.style={thick,circle,draw=myblue,minimum size=22,inner sep=0.5,outer sep=0.6},
	node in/.style={node,green!20!black,draw=mygreen!30!black,fill=mygreen!25},
	node hidden/.style={node,blue!20!black,draw=myblue!30!black,fill=myblue!20},
	node convol/.style={node,orange!20!black,draw=myorange!30!black,fill=myorange!20},
	node out/.style={node,red!20!black,draw=myred!30!black,fill=myred!20},
	connect/.style={thick,mydarkblue}, %,line cap=round
	connect arrow/.style={-{Latex[length=4,width=3.5]},thick,mydarkblue,shorten <=0.5,shorten >=1},
	node 1/.style={node in}, % node styles, numbered for easy mapping with \nstyle
	node 2/.style={node hidden},
	node 3/.style={node out}
}
\def\nstyle{int(\lay<\Nnodlen?min(2,\lay):3)} % map layer number onto 1, 2, or 3