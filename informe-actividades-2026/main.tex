\documentclass[10pt,xcdraw, spanish]{article}

\usepackage{graphicx}
\usepackage{subcaption}
\graphicspath{ {figures/} }
 % Carga el preámbulo
\begin{document}
%%%%%%%%%%%%%%%%%%%%%%%%%%%%%%%%%% PORTADA %%%%%%%%%%%%%%%%%%%%%%%%%%%%%%%%%%%%%%%%%%%%
\vspace*{0.5cm}
\begin{center}
	\newcommand{\HRule}{\rule{\linewidth}{0.5mm}}
	\vspace*{-1.5cm}
	% \textsc{\huge Universidad Tecnológica\\ \vspace{5px} de Pereira}\\[1.5cm]
	% \includegraphics[width=0.2\textwidth]{imagenes/Logo_UTP.png}
	% \vspace{1.5cm}
	\textsc{\LARGE Informe de actividades en el marco del programa: \\ 
		\vspace{0.4cm}
		XXXXXXXXXXX}\\
	\vspace{2cm}
	\HRule \\[0.4cm]
	{ \bfseries Nombre: Julián David Pastrana Cortés }\\[0.4cm]
	{ \bfseries Contrato de servicios: 5551 }\\[0.4cm]
	\vspace{1.5cm}
	\includegraphics[width=0.28\textwidth]{imagenes/LogoU.png}
	\includegraphics[width=0.35\textwidth]{imagenes/logo_automatica.png}
	\HRule \\[1.5cm]
	\vspace{1.5cm}
	\vspace*{0.5cm}
	\textsc{\textbf{\Large Grupo de investigación en Automática } }\\
	\vspace{3.5cm} 
	\begin{center}
		{\large \today}
	\end{center}
\end{center}
\newpage
   % Carga la portada y otros elementos de layout

\tableofcontents
\newpage

\section{Información general del contrato}
\begin{table}[H]
	\centering
	\begin{tabular}{>{\arraybackslash}m{8cm} >{\arraybackslash}m{6.5cm}}
		\toprule
		\textbf{Rol}                                   & \vspace{2mm}Contratista - Estudiante de Doctorado\vspace{2mm}                                                                                                                                                                                                                                                                                                 \\\hline
		\vspace{2mm}
		\textbf{Contrato de servicios No.}\vspace{2mm} & \vspace{2mm} 5533 de 2026\vspace{2mm}                                                                                                                                                                                                                                                                                                                         \\\hline
		\vspace{2mm}
		\textbf{Objeto del contrato} \vspace{2mm}      & \vspace{2mm} Prestación de servicios profesionales para el Desarrollo de metodología de gamificación para entrenamiento de niños con trastornos de impulsividad ALIANZA CIENTÍFICA CON ENFOQUE COMUNITARIO PARA MITIGAR BRECHAS DE ATENCIÓN Y MANEJO DE TRASTORNOS MENTALES RELACIONADOS CON IMPULSIVIDAD EN COLOMBIA - ACEMATE MINCIENCIAS CONTRATO 790-2023
		\vspace{2mm}                                                                                                                                                                                                                                                                                                                                                                                                   \\\hline
		\textbf{Período del informe}                   & \vspace{2mm} 19 de enero de 2026 al 31 de agosto de 2026\vspace{2mm}                                                                                                                                                                                                                                                                                          \\\hline
	\end{tabular}
\end{table}

\subsection{Descripción general de la vinculación}
Como resultado de la vinculación se contribuye al alcance de el objetivo 3 del programa de investigación: ALIANZA CIENTÍFICA CON ENFOQUE COMUNITARIO PARA MITIGAR BRECHAS DE ATENCIÓN Y MANEJO DE TRASTORNOS MENTALES RELACIONADOS CON IMPULSIVIDAD EN COLOMBIA

\subsection{Objetivo general}
Desarrollar una metodología de gamificación para entrenamiento de niños con trastornos de impulsividad ALIANZA CIENTÍFICA CON ENFOQUE COMUNITARIO PARA MITIGAR BRECHAS DE ATENCIÓN Y MANEJO DE TRASTORNOS MENTALES RELACIONADOS CON IMPULSIVIDAD EN COLOMBIA - ACEMATE MINCIENCIAS CONTRATO 790-2023.

%	\section{State of Art}

%%%%% Emotion detencton form wearable devices

Wearable devices allows to measure physiological biomarkers in a non-intrusive fashion. The collected data may correlate with stress levels, emotion states and biological responses. Those measurement often include Hear Rate Variability (HRV), Electrodermal Activity (ADA), Heart Rate (HR) and three axis acceleration (ACC) \cite{Vos2023}. Advance in machine learning techniques have allowed to construct models that predict sentimental state based on recorded biomarkers, understanding that human health extends to mental well-being.

Authors in \cite{Zhu2023} evaluate a set of traditional machine learning algorithms to predict people's stress based on EDA activity, including K-Nearest Neighbor, Support Vector Machine (SVM), Naive Bayes, Logistic Regression and Random Forest. Those model were trained fed by a feature vector containing statistic of EDA data, and raw data itself for comparing classification results. Researchers reported best performance for SVM model, and no consistency training with raw data or feature vector.

However, traditional machine learning models lack of expressiveness and capacity to generalize well \cite{Yang2023}. Moreover, feature often rely on statistics, forgetting sequential dependencies in the data. A closer overview dive us to a multi-modality scenario, that is, different measurement could be taken at its own frequency sample.

Deep learning techniques allow us to handle multi-modal measurement, leveraging data structure like time dependencies for sequential recordings, or spatial patters for images. That can be done through feature representation from multiple data entities. However, that arise the challenge of how to combine data structures from different sources \cite{Baltrusaitis2019,Liang2024}. 

The work developed by \cite{Venugopalan2021} integrate multiple data modalities using auto-encoders for genetic data and 3D convolutional neural network (CNN) for imaging data, demonstrating outperform over shallow models and single modality.




\section{Metodología}
La metodología propuesta para el desarrollo del proyecto se estructura en cuatro actividades principales, distribuidas a lo largo de siete meses, con entregables específicos para cada una:

\subsection{Actividad 1: Planteamiento de la metodología, técnicas y modelos a implementar (Meses 1-2)}
En esta etapa inicial se realizará la conceptualización y diseño de la metodología de trabajo. Se definirán las técnicas de análisis y los modelos computacionales que se implementarán para el procesamiento de señales EEG. Esta actividad incluye la revisión de fundamentos teóricos, selección de herramientas y establecimiento del marco de trabajo.

\textbf{Entregable:} Informe técnico que documenta la metodología propuesta, las técnicas seleccionadas y los modelos a desarrollar.

\subsection{Actividad 2: Revisión técnica del estado del arte (Meses 3-4)}
Se llevará a cabo una revisión exhaustiva y sistemática del estado del arte en análisis de señales EEG, identificando las metodologías existentes, técnicas avanzadas y tendencias actuales en el procesamiento de señales cerebrales. Esta revisión permitirá contextualizar el trabajo y validar las decisiones metodológicas.

\textbf{Entregable:} Informe técnico con el análisis del estado del arte y la fundamentación teórica del proyecto.

\subsection{Actividad 3: Desarrollo e implementación de modelos de análisis de EEG (Meses 5-6)}
Esta etapa constituye el núcleo del proyecto, donde se desarrollarán e implementarán los modelos de análisis de señales EEG. Se realizará el montaje y configuración de los modelos para diferentes tareas específicas, incluyendo pruebas, validación y optimización de los algoritmos desarrollados.

\textbf{Entregable:} Código fuente de los modelos implementados, documentado y funcional.

\subsection{Actividad 4: Documentación de resultados consolidados (Meses 6-7)}
En la fase final se consolidarán todos los resultados obtenidos durante el proyecto. Se documentarán los hallazgos, análisis de desempeño de los modelos, conclusiones y recomendaciones. Esta documentación servirá como soporte integral del proyecto.

\textbf{Entregable:} Informe técnico final con la documentación completa de resultados, análisis y conclusiones del proyecto.

\section{Resultados}

\subsection{Actividad 1}

Se desarrolló la conceptualización y el diseño de la metodología de trabajo para el procesamiento de señales EEG, con el objetivo de clasificar el trastorno por déficit de atención e hiperactividad (TDAH) en población infantil. Para ello, se realizó una revisión relacionada con el procesamiento de señales EEG, las arquitecturas de aprendizaje profundo y los métodos de validación cruzada aplicados a datos neurofisiológicos.

A partir de este análisis, se seleccionó como modelo principal una arquitectura Transformer linealizada basada en embeddings de Takens (LinearTakensFormer). Esta elección se fundamenta en su capacidad para capturar dependencias temporales complejas en señales no estacionarias, como las EEG, mientas mantieniendo la escalabilidad para grandes volúmenes de datos.

Se utilizó la base de datos pública EEG ADHD/Control Children, conformada por registros de electroencefalografía de 19 canales dispuestos según el sistema internacional 10-20, adquiridos a una frecuencia de muestreo de 128 Hz. El conjunto incluye sujetos en condición de reposo clasificados en dos grupos: niños diagnosticados con TDAH y niños de desarrollo típico empleados como grupo control. Los archivos de señal se encuentran en formato .mat y la etiqueta clínica de cada sujeto se determina a partir de la carpeta de origen dentro de la estructura del repositorio de datos.

En cuanto al procesamiento de las señales, se definió una cadena de preprocesamiento compuesta por varias etapas. En primer lugar, se aplicó un filtrado pasabanda entre 0.5 y 60 Hz, junto con la eliminación de la interferencia de la red eléctrica en el rango de 48 a 52 Hz. Posteriormente, las señales fueron segmentadas mediante ventanas deslizantes con un 50\% de solapamiento, para preservar la continuidad temporal y aumentar el número de muestras disponibles para el entrenamiento.

El marco metodológico adoptado integra un esquema de validación cruzada Leave-K-Subjects-Out (LKSO), combinado con optimización bayesiana de hiperparámetros mediante Optuna, garantizando una evaluación generalizable del modelo mientras se maximiza la metrica de acierto (accuracy).


% \includepdf[
% 	pages=3-last,
% 	fitpaper=true,
% 	pagecommand={},
% 	noautoscale=false,
% 	frame=false,
% 	trim=0 0 0 0,
% 	clip=true
% ]{../../RequiemMentis/codex/main.pdf}


\end{document}
