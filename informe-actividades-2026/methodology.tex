\section{Metodología}
La metodología propuesta para el desarrollo del proyecto se estructura en cuatro actividades principales, distribuidas a lo largo de siete meses, con entregables específicos para cada una:

\subsection{Actividad 1: Planteamiento de la metodología, técnicas y modelos a implementar (Meses 1-2)}
En esta etapa inicial se realizará la conceptualización y diseño de la metodología de trabajo. Se definirán las técnicas de análisis y los modelos computacionales que se implementarán para el procesamiento de señales EEG. Esta actividad incluye la revisión de fundamentos teóricos, selección de herramientas y establecimiento del marco de trabajo.

\textbf{Entregable:} Informe técnico que documenta la metodología propuesta, las técnicas seleccionadas y los modelos a desarrollar.

\subsection{Actividad 2: Revisión técnica del estado del arte (Meses 3-4)}
Se llevará a cabo una revisión exhaustiva y sistemática del estado del arte en análisis de señales EEG, identificando las metodologías existentes, técnicas avanzadas y tendencias actuales en el procesamiento de señales cerebrales. Esta revisión permitirá contextualizar el trabajo y validar las decisiones metodológicas.

\textbf{Entregable:} Informe técnico con el análisis del estado del arte y la fundamentación teórica del proyecto.

\subsection{Actividad 3: Desarrollo e implementación de modelos de análisis de EEG (Meses 5-6)}
Esta etapa constituye el núcleo del proyecto, donde se desarrollarán e implementarán los modelos de análisis de señales EEG. Se realizará el montaje y configuración de los modelos para diferentes tareas específicas, incluyendo pruebas, validación y optimización de los algoritmos desarrollados.

\textbf{Entregable:} Código fuente de los modelos implementados, documentado y funcional.

\subsection{Actividad 4: Documentación de resultados consolidados (Meses 6-7)}
En la fase final se consolidarán todos los resultados obtenidos durante el proyecto. Se documentarán los hallazgos, análisis de desempeño de los modelos, conclusiones y recomendaciones. Esta documentación servirá como soporte integral del proyecto.

\textbf{Entregable:} Informe técnico final con la documentación completa de resultados, análisis y conclusiones del proyecto.