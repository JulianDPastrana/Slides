\section*{Objective 3: Chained Correlated Gaussian Processes}

\subsection{Mathematical Framework}

\begin{frame}{Objective 3: Chained Correlated Gaussian Processes}
	\scriptsize
	\centering
	\setlength\figurewidth{0.35\textwidth} 
	\setlength\figureheight{0.15\textwidth}
	\begin{tikzpicture}[x=\figurewidth,y=\figureheight]
  % \message{^^JNeural network activation}
  \def\NI{4} % number of nodes in input layer
  \def\NM{4} % number of nodes in middle layer  
  \def\NO{2} % number of nodes in output layer
  \def\yshift{0.0} % shift last node for dots
  
  % INPUT LAYER
  \foreach \i [evaluate={\c=int(\i==\NI); \y=\NI/2-\i-\c*\yshift; \index=(\i<\NI?int(\i):"Q");}]
              in {1,...,\NI}{ % loop over nodes
    \node[node in,outer sep=0.6] (NI-\i) at (0,\y) {$u_{\index}$};
  }
  
  % MIDDLE LAYER

      % MIDDLE LAYER with specific labels
  \node[node hidden] (NM-1) at (1,\NM/2-1) {$f_{1, J_1}$};
  \node[node hidden] (NM-2) at (1,\NM/2-2) {$f_{2, J_1}$};
  \node[node hidden] (NM-3) at (1,\NM/2-3) {$f_{1, J_D}$};
  \node[node hidden] (NM-4) at (1,\NM/2-4) {$f_{D, J_D}$};

  \foreach \i [evaluate={\c=int(\i==\NM); \y=\NM/2-\i-\c*\yshift; \index=(\i<\NM?int(\i):"D");}]
    in {\NM,...,1}{ % loop over nodes
      \ifnum\i=1 % highlighted node
      \foreach \j [evaluate={\index=(\j<\NI?int(\j):"Q");}] in {1,...,\NI}{ % loop over nodes in previous layer
        \draw[connect,white,line width=1.2] (NI-\j) -- (NM-\i);
        \draw[connect] (NI-\j) -- (NM-\i)
          node[pos=0.50] {\contour{white}{$a_{1, 1,\index}$}};
      }
    \else % other light-colored nodes
      \foreach \j in {1,...,\NI}{ % loop over nodes in previous layer
        \draw[connect,white,line width=1.2] (NI-\j) -- (NM-\i);
        \draw[connect,myblue!20] (NI-\j) -- (NM-\i);i
      }
    \fi
  }
  

% OUTPUT LAYER
  \foreach \i [evaluate={\c=int(\i==\NO); \y=\NO/2-\i-\c*\yshift; \index=(\i<\NO?int(\i):"D");}]
    in {\NO,...,1}{ % loop over nodes
    \ifnum\i=1 % highlighted node
      \node[node out]
        (NO-\i) at (2,\y) {$y_{\index}$};
      \foreach \j [evaluate={\index=(\j<3?int(\j):"D");}] in {1,2}{ % loop over nodes 1 and 2 in middle layer
        \draw[connect,white,line width=1.2] (NM-\j) -- (NO-\i);
        \draw[connect, myred] (NM-\j) -- (NO-\i)
          node[pos=0.50] {\contour{white}{$g_{\index,J_{1}}(\cdot)$}};
      }
    \else % other light-colored nodes
      \node[node out]
        (NO-\i) at (2,\y) {$y_{\index}$};
      \foreach \j in {3,4}{ % loop over nodes 3 and 4 in middle layer
        \draw[connect,white,line width=1.2] (NM-\j) -- (NO-\i);
        \draw[connect,myred!20] (NM-\j) -- (NO-\i);
      }
    \fi
  }


  % DOTS
  \path (NI-\NI) --++ (0,1+\yshift) node[midway,scale=1.2] {$\vdots$};
  \path (NM-\NM) --++ (0,3+\yshift) node[midway,scale=1.2] {$\vdots$};
  \path (NM-\NM) --++ (0,1+\yshift) node[midway,scale=1.2] {$\vdots$};
  \path (NO-\NO) --++ (0,1+\yshift) node[midway,scale=1.2] {$\vdots$};
    
\end{tikzpicture}

	
	\vspace{-0.6em}
	\begin{columns}[T] % Top alignment
		\begin{column}{0.33\textwidth}
			\centering	
			\textcolor{mygreen}{
			\textbf{Independent Process (IGP)}
			\vspace{-1.3em}
			\begin{equation*}
				u_{q}(\mathbf{x}) \sim \mathcal{GP}(0, k_{q}(\mathbf{x}, \mathbf{x}'))
			\end{equation*}
			}
		\end{column}
		
		\begin{column}{0.33\textwidth}
			\centering
			\textcolor{myblue}{
				\textbf{Latent Process (LMCGP)}
				\vspace{-1.3em}
			\begin{equation*}
				f_{d,j}(\mathbf{x}) = \sum_{q=1}^Q a_{d,j,q} u_{q}(\mathbf{x})
			\end{equation*}
			}
		\end{column}
		
		\begin{column}{0.33\textwidth}
			\centering
			\textcolor{myred}{
				\textbf{Likelihood (Chd GP)}
				\vspace{-1.3em}
			\begin{equation*}
				\mathbf{y} \mid \mathbf{f} \sim \prod_{d=1}^D p(\theta_{d, 1}, \cdots, \theta_{d, J_d})
			\end{equation*}
			\vspace{-1.5em}
			\begin{equation*}
				\theta_{d, j} = g_{d, j}(f_{d, j})
			\end{equation*}
			}
		\end{column}
		
	\end{columns}
\end{frame}



\begin{frame}{Variational Inference, ELBO and Predictive Distribution}
	We extent our variational inference, providing the following ELBO:
	
\begin{equation*}
	\begin{split}
	\mathcal{L} &= \sum_{d=1}^D \sum_{n=1}^{N} \mathbb{E}_{\textcolor{myred}{q(f_{d,1,n}), \cdots, q(f_{d,J_d,n})}} 
	\left\{ \log \textcolor{myred}{p\left( y_{d,n} \mid f_{d,1,n}, \cdots, f_{d,J_d,n} \right)} \right\} \\
	&- \sum_{{\textcolor{mygreen}{q}}=1}^{\textcolor{mygreen}{Q}} \text{KL}\left\{ q(\mathbf{u}_{\textcolor{mygreen}{q}}) \parallel p(\mathbf{u}_{\textcolor{mygreen}{q}}) \right\}
	\end{split}
\end{equation*}

The approximated posterior over test points is given by:

\begin{equation*}
	p(\mathbf{f}_* \mid \mathbf{y}) \approx q(\mathbf{f}_*) = \int p(\mathbf{f}_* \mid \mathbf{u}) q(\mathbf{u}) d\mathbf{u}
\end{equation*}

And the predictive distribution for a new output $\mathbf{y}_*$:
\begin{equation*}
	p(\mathbf{y}_* \mid \mathbf{y}) \approx \int p(\mathbf{y}_* \mid \mathbf{f}_*) q(\mathbf{f}_*) \, d\mathbf{f}_*,
\end{equation*}

The expectation values can be approximated via Monte Carlo methods. The complexity is now \textcolor{mygreen}{$\mathcal{O}(JNQM^2)$} with $J=\sum_d^D J_d$.
	
\end{frame}

\begin{frame}{Model Setup}
	We again make use of squared exponential kernel to construct the covariance function and Adam + NG framework to train the models.
		
	\begin{block}{\textbf{Gaussian Likelihood: ChdGP Normal}}
		\begin{equation*}
			p(\mathbf{y} \mid \mathbf{f}) = \prod_{d=1}^{D} \mathcal{N}\left(y_{d}\mid g_{d,1}(f_{d,1}), g_{d,2}(f_{d,2}) \right)
		\end{equation*}
		In this formulation, \(g_{d,1}(\cdot) = \cdot\), while \(g_{d,2}(\cdot) = \ln(\exp(\cdot) + 1)\).
	\end{block} 
	
	\begin{block}{\textbf{Gamma Likelihood: ChdGP Gamma}}
		\begin{equation*}
		p(\mathbf{y} \mid \mathbf{f}) = \prod_{d=1}^{D} \mathcal{G}amma\left( y_{d} \mid g_{d,1}(f_{d,1}), g_{d,2}(f_{d,2}) \right)
		\end{equation*}
		
		In this formulation \(g_{d,1}(\cdot) = g_{d,2}(\cdot) = \ln(\exp(\cdot) + 1)\).
	\end{block}
		
\end{frame}

\subsection{Results and Discussions}

\begin{frame}{Performance metrics vs Q}
	\vspace{-0.8em}
	\begin{figure}[htbp]
		\tiny
		\centering
		\setlength\figurewidth{0.5\columnwidth} 
		\setlength\figureheight{0.28\columnwidth}
		\subfloat[MSE]{% This file was created with tikzplotlib v0.10.1.
%\begin{tikzpicture}

\definecolor{darkgray176}{RGB}{176,176,176}
\definecolor{steelblue31119180}{RGB}{31,119,180}

%\begin{axis}[
%font=\tiny,
%width=\figurewidth,
%height=\figureheight,
%axis background/.style={fill=background_color},
%axis line style={white},
%tick align=inside,
%tick pos=left,
%x grid style={white},
%y grid style={white},
%xmajorgrids,
%ymajorgrids,
%xmin=-1.25, xmax=48.25,
%xtick style={color=black},
%ymajorgrids,
%ymin=0.00432172839722227, ymax=0.0132074497620714,
%ytick style={color=black}
%]
\addplot [semithick, steelblue31119180, mark=*, mark size=1, mark options={solid}]
table {%
1 0.0128035533363964
2 0.00905686675869738
3 0.00827670369639854
4 0.00647312454537253
5 0.0059998944618926
6 0.00624419146120023
7 0.00579712581238592
8 0.00571172094861466
9 0.00533441391077178
10 0.00531633238994604
11 0.00532343324902325
12 0.0050867372520667
13 0.00526346840550523
14 0.00508599694700343
15 0.00518337894714425
16 0.00513812905102857
17 0.00682742467122673
18 0.00510528673403877
19 0.00563946517817211
20 0.00551913123809022
21 0.00494528640998287
22 0.00585845931386892
23 0.00547909049889633
24 0.00472562482289723
25 0.00615358787709803
26 0.00483592747529357
27 0.0075631880355845
28 0.0056618067323409
29 0.00482810431270197
30 0.00603131705475209
31 0.00655705532258234
32 0.00710549067844106
33 0.00569378426511534
34 0.00603826998078558
35 0.00708460267778661
36 0.00849317978089477
37 0.00740070646203432
38 0.00816300146172802
39 0.00638476700981259
40 0.00632862480530453
41 0.00700173319188677
42 0.00642937589842245
43 0.00635723590641261
44 0.00655464545991826
45 0.00670091040125378
46 0.00706706114782923
};
%\end{axis}

%\end{tikzpicture}
}
		\subfloat[MSLL]{% This file was created with tikzplotlib v0.10.1.
\begin{tikzpicture}

\definecolor{darkgray176}{RGB}{176,176,176}
\definecolor{steelblue31119180}{RGB}{31,119,180}

\begin{axis}[
font=\tiny,
width=\figurewidth,
height=\figureheight,
axis background/.style={fill=background_color},
axis line style={white},
tick align=inside,
tick pos=left,
x grid style={white},
y grid style={white},
xmajorgrids,
ymajorgrids,
xmin=-1.25, xmax=48.25,
xtick style={color=black},
ymin=-0.981574217818357, ymax=-0.254980175714052,
ytick style={color=black},
%xlabel={Number of Independent GPs (Q)}
]
\addplot [semithick, steelblue31119180, mark=*, mark size=1, mark options={solid}]
table {%
1 -0.288007177627884
2 -0.486433875918135
3 -0.593083030305358
4 -0.689721765965117
5 -0.707669865514075
6 -0.720422487612926
7 -0.737789583039415
8 -0.76977958126218
9 -0.790749444413204
10 -0.801765978552222
11 -0.819791223747015
12 -0.86272547623958
13 -0.857831565839939
14 -0.876741782841828
15 -0.906781806077751
16 -0.866509321979023
17 -0.567252287936316
18 -0.918594583955142
19 -0.756984658169378
20 -0.760631883882401
21 -0.89582623985992
22 -0.717113890413948
23 -0.80039183605464
24 -0.948547215904525
25 -0.636946438064492
26 -0.942583598637442
27 -0.571591759252111
28 -0.681619424160601
29 -0.912816354860318
30 -0.626429497657631
31 -0.646795260122204
32 -0.571941714185512
33 -0.706618646926849
34 -0.619101344506579
35 -0.456559364448254
36 -0.395685084563509
37 -0.427008525048951
38 -0.399195477809696
39 -0.503866482905641
40 -0.471543594639428
41 -0.501112946719914
42 -0.473864288674822
43 -0.496469966822398
44 -0.535960786269092
45 -0.402501473063068
46 -0.518758524526979
};
\end{axis}

\end{tikzpicture}
}\\
		\vspace{-1.0em}
		\subfloat[CRPS]{% This file was created with tikzplotlib v0.10.1.
\begin{tikzpicture}

\definecolor{darkgray176}{RGB}{176,176,176}
\definecolor{steelblue31119180}{RGB}{31,119,180}

\begin{axis}[
font=\tiny,
width=\figurewidth,
height=\figureheight,
axis background/.style={fill=background_color},
axis line style={white},
tick align=inside,
tick pos=left,
x grid style={white},
y grid style={white},
xmajorgrids,
ymajorgrids,
xmin=-1.25, xmax=48.25,
xtick style={color=black},
ymin=0.0348094266820055, ymax=0.0709032142609183,
ytick style={color=black}
]
\addplot [semithick, steelblue31119180, mark=*, mark size=1, mark options={solid}]
table {%
1 0.0692625875527859
2 0.0583418913384871
3 0.0531484180012032
4 0.047056356675341
5 0.0449029435290782
6 0.0445437236593147
7 0.0431980837800703
8 0.0430452280003688
9 0.0416328055777329
10 0.0409632413188986
11 0.0402963364803899
12 0.039384017699264
13 0.0395846642962409
14 0.0388320970910489
15 0.0387081337400137
16 0.039126234609027
17 0.0491941696382463
18 0.0379831549953404
19 0.0432462278461219
20 0.042695264181545
21 0.0372925949709695
22 0.0433018503234648
23 0.0407115097688033
24 0.0365903175603876
25 0.045801824400241
26 0.0364500533901379
27 0.0502896333830452
28 0.0443865163231249
29 0.0368671329512049
30 0.0448614141553844
31 0.0476395833048069
32 0.0493362816046532
33 0.0425521043485125
34 0.0457527571906415
35 0.052230329459415
36 0.0551579779906624
37 0.0480511174064211
38 0.0544011936142509
39 0.0471229221091566
40 0.0492771856166812
41 0.0507579544138762
42 0.0482187538122221
43 0.047794452895302
44 0.0491308454885213
45 0.0506342900377509
46 0.050702200878765
};
\end{axis}

\end{tikzpicture}
}
		\subfloat[NLPD]{% This file was created with tikzplotlib v0.10.1.
%\begin{tikzpicture}

\definecolor{darkgray176}{RGB}{176,176,176}
\definecolor{steelblue31119180}{RGB}{31,119,180}
%
%\begin{axis}[
%font=\tiny,
%width=\figurewidth,
%height=\figureheight,
%axis background/.style={fill=background_color},
%axis line style={white},
%tick align=inside,
%tick pos=left,
%x grid style={white},
%y grid style={white},
%xmajorgrids,
%ymajorgrids,
%xlabel={Number of Independent GPs (Q)},
%xmin=-1.25, xmax=48.25,
%xtick style={color=black},
%ymin=-45.6861666181673, ymax=-22.4781635869566,
%ytick style={color=black},
%legend columns=2, 
%legend style={
%	nodes={scale=0.8, transform shape},
%	fill opacity=1,
%	draw opacity=1,
%	text opacity=1,
%	at={(0.03,0.97)},
%	anchor=north west,
%	legend pos=north west,
%	draw=lightgray204,
%	fill=background_color
%},
%]


\addplot [semithick, steelblue31119180, mark=*, mark size=1, mark options={solid}]
table {%
1 -23.2603455429207
2 -27.8848249850358
3 -30.4369312439101
4 -32.6964903456607
5 -33.1232986099899
6 -33.4855328194626
7 -33.8686447547946
8 -34.569686095424
9 -35.0744292562848
10 -35.308029353193
11 -35.7810292144787
12 -36.8321087458006
13 -36.5933177130925
14 -37.1242868378589
15 -37.7921966600965
16 -36.8819101985195
17 -29.8066012717576
18 -38.0986443765348
19 -34.3334916420744
20 -34.3246742688512
21 -37.6975472734407
22 -33.3439962248266
23 -35.5867094338705
24 -38.8940739993621
25 -31.7179722825456
26 -38.9039846622032
27 -30.2604768016445
28 -32.5358172516084
29 -38.5464590119924
30 -31.3103746956277
31 -31.8564810551671
32 -30.1602721024852
33 -33.1776395942215
34 -31.1976293972342
35 -27.3598985157701
36 -25.9121672175452
37 -26.9506667846989
38 -25.9451559849429
39 -28.5723189298692
40 -27.8413368773994
41 -28.3534875298917
42 -27.9990546137006
43 -28.4826836100414
44 -29.2448436151267
45 -26.1478740943578
46 -28.9634391193978
};

\addlegendentry{Chd Normal}
\addlegendentry{Chd Gamma}
%\end{axis}

%\end{tikzpicture}
}\\
	\end{figure}
	\vspace{-1.5em}
\begin{itemize}
	\item The Gamma likelihood provides greater stability by offering a more precise description of the data.
	\item We select $Q=15$ for the Normal likelihood and $Q=26$ for the Gamma likelihood.
\end{itemize}

\end{frame}

\subsection{Performance Analysis}
\begin{frame}{LMCGP vs ChdGP across horizons}
	\begin{figure}[htbp]
		\centering
		\tiny
		\setlength\figurewidth{0.56\columnwidth}
		\setlength\figureheight{0.52\columnwidth}
		
		\subfloat[MSLL]{% This file was created with tikzplotlib v0.10.1.
\begin{tikzpicture}

\definecolor{green01270}{RGB}{0,127,0}
\definecolor{darkgray176}{RGB}{176,176,176}

\begin{axis}[
width=\figurewidth,%
height=\figureheight,%
axis background/.style={fill=background_color},
tick align=outside,
tick pos=left,
axis line style={white},
tick align=inside,
tick pos=left,
x grid style={white},
y grid style={white},
xmajorgrids,
ymajorgrids,
xlabel={Horizon},
xmin=-0.85, xmax=9.6,
xtick style={color=black},
xtick={0,1,2,3,4,5,6,7,8,9},
xticklabels={1,2,3,4,5,6,7,14,21,30},
ymin=-0.923060773977745, ymax=0,
ytick style={color=black},
legend columns=3, 
legend style={
  nodes={scale=0.8, transform shape},
  fill opacity=1,
  draw opacity=1,
  text opacity=1,
  at={(0.03,0.97)},
  anchor=north west,
  legend pos=north west,
  draw=lightgray204,
  fill=background_color
},
]
\draw[draw=none,fill=myNewColorB] (axis cs:-0.375,0) rectangle (axis cs:-0.125,-0.702288288967207);
\addlegendimage{ybar,ybar legend,draw=none,fill=myNewColorB}
\addlegendentry{LMCGP}

\draw[draw=none,fill=myNewColorB] (axis cs:0.625,0) rectangle (axis cs:0.875,-0.594224113937007);
\draw[draw=none,fill=myNewColorB] (axis cs:1.625,0) rectangle (axis cs:1.875,-0.411836682291104);
\draw[draw=none,fill=myNewColorB] (axis cs:2.625,0) rectangle (axis cs:2.875,-0.3535743617084);
\draw[draw=none,fill=myNewColorB] (axis cs:3.625,0) rectangle (axis cs:3.875,-0.40301909358068);
\draw[draw=none,fill=myNewColorB] (axis cs:4.625,0) rectangle (axis cs:4.875,-0.388074648574892);
\draw[draw=none,fill=myNewColorB] (axis cs:5.625,0) rectangle (axis cs:5.875,-0.342340526507888);
\draw[draw=none,fill=myNewColorB] (axis cs:6.625,0) rectangle (axis cs:6.875,-0.306616481316462);
\draw[draw=none,fill=myNewColorB] (axis cs:7.625,0) rectangle (axis cs:7.875,-0.344959033505315);
\draw[draw=none,fill=myNewColorB] (axis cs:8.625,0) rectangle (axis cs:8.875,-0.340675043899372);
\draw[draw=none,fill=myNewColorC] (axis cs:-0.125,0) rectangle (axis cs:0.125,-0.879105499026424);
\addlegendimage{ybar,ybar legend,draw=none,fill=myNewColorC}
\addlegendentry{ChdNormal}

\draw[draw=none,fill=myNewColorC] (axis cs:0.875,0) rectangle (axis cs:1.125,-0.681191498170826);
\draw[draw=none,fill=myNewColorC] (axis cs:1.875,0) rectangle (axis cs:2.125,-0.607558924593337);
\draw[draw=none,fill=myNewColorC] (axis cs:2.875,0) rectangle (axis cs:3.125,-0.57495648241054);
\draw[draw=none,fill=myNewColorC] (axis cs:3.875,0) rectangle (axis cs:4.125,-0.517959518063628);
\draw[draw=none,fill=myNewColorC] (axis cs:4.875,0) rectangle (axis cs:5.125,-0.550090004032213);
\draw[draw=none,fill=myNewColorC] (axis cs:5.875,0) rectangle (axis cs:6.125,-0.554156438489196);
\draw[draw=none,fill=myNewColorC] (axis cs:6.875,0) rectangle (axis cs:7.125,-0.509945837690314);
\draw[draw=none,fill=myNewColorC] (axis cs:7.875,0) rectangle (axis cs:8.125,-0.466212819953932);
\draw[draw=none,fill=myNewColorC] (axis cs:8.875,0) rectangle (axis cs:9.125,-0.395250514083511);

\addlegendimage{ybar,ybar legend,draw=none,fill=myNewColorD}
\addlegendentry{ChdGamma}

\end{axis}

\end{tikzpicture}
}
		\hfill
		\subfloat[NLPD]{% This file was created with tikzplotlib v0.10.1.
%\begin{tikzpicture}
	
	\definecolor{darkgray176}{RGB}{176,176,176}
	\definecolor{green01270}{RGB}{0,127,0}
	
%	\begin{axis}[
%		width=\figurewidth,%
%		height=\figureheight,%
%		axis background/.style={fill=background_color},
%		tick align=outside,
%		tick pos=left,
%		axis line style={white},
%		tick align=inside,
%		tick pos=left,
%		x grid style={white},
%		y grid style={white},
%		xmajorgrids,
%		ymajorgrids,
%		xlabel={Horizon},
%		xmin=-0.85, xmax=9.6,
%		xtick style={color=black},
%		xtick={0,1,2,3,4,5,6,7,8,9},
%		xticklabels={1,2,3,4,5,6,7,14,21,30},
%		ymin=-45.0575239116442, ymax=0,
%		ytick style={color=black}
%		]
		
		% Green bars
		\draw[draw=none,fill=myNewColorB] (axis cs:-0.375,0) rectangle (axis cs:-0.125,-32.745747598516);
		\draw[draw=none,fill=myNewColorB] (axis cs:0.625,0) rectangle (axis cs:0.875,-30.2613167702501);
		\draw[draw=none,fill=myNewColorB] (axis cs:1.625,0) rectangle (axis cs:1.875,-26.0674625485646);
		\draw[draw=none,fill=myNewColorB] (axis cs:2.625,0) rectangle (axis cs:2.875,-24.7284934760365);
		\draw[draw=none,fill=myNewColorB] (axis cs:3.625,0) rectangle (axis cs:3.875,-25.8667874240686);
		\draw[draw=none,fill=myNewColorB] (axis cs:4.625,0) rectangle (axis cs:4.875,-25.5241222520839);
		\draw[draw=none,fill=myNewColorB] (axis cs:5.625,0) rectangle (axis cs:5.875,-24.4733528784515);
		\draw[draw=none,fill=myNewColorB] (axis cs:6.625,0) rectangle (axis cs:6.875,-23.7189187161835);
		\draw[draw=none,fill=myNewColorB] (axis cs:7.625,0) rectangle (axis cs:7.875,-24.6262096068829);
		\draw[draw=none,fill=myNewColorB] (axis cs:8.625,0) rectangle (axis cs:8.875,-24.551868432483);
		
		% Blue bars
		\draw[draw=none,fill=myNewColorC] (axis cs:-0.125,0) rectangle (axis cs:0.125,-37.1976418206135);
		\draw[draw=none,fill=myNewColorC] (axis cs:0.875,0) rectangle (axis cs:1.125,-32.57009392034);
		\draw[draw=none,fill=myNewColorC] (axis cs:1.875,0) rectangle (axis cs:2.125,-31.1545299186111);
		\draw[draw=none,fill=myNewColorC] (axis cs:2.875,0) rectangle (axis cs:3.125,-30.7281603098309);
		\draw[draw=none,fill=myNewColorC] (axis cs:3.875,0) rectangle (axis cs:4.125,-28.7134159542646);
		\draw[draw=none,fill=myNewColorC] (axis cs:4.875,0) rectangle (axis cs:5.125,-29.5248659053096);
		\draw[draw=none,fill=myNewColorC] (axis cs:5.875,0) rectangle (axis cs:6.125,-29.6611175531187);
		\draw[draw=none,fill=myNewColorC] (axis cs:6.875,0) rectangle (axis cs:7.125,-28.9486441224176);
		\draw[draw=none,fill=myNewColorC] (axis cs:7.875,0) rectangle (axis cs:8.125,-27.8247484458317);
		\draw[draw=none,fill=myNewColorC] (axis cs:8.875,0) rectangle (axis cs:9.125,-26.5153492416896);
		
		% Red bars now positioned directly next to the myNewColorC bars
		\draw[draw=gray,fill=myNewColorD] (axis cs:0.125,0) rectangle (axis cs:0.375,-44.2987203321051);
		\draw[draw=gray,fill=myNewColorD] (axis cs:1.125,0) rectangle (axis cs:1.375,-40.5759976099418);
		\draw[draw=gray,fill=myNewColorD] (axis cs:2.125,0) rectangle (axis cs:2.375,-39.2559882886404);
		\draw[draw=gray,fill=myNewColorD] (axis cs:3.125,0) rectangle (axis cs:3.375,-38.2783040944686);
		\draw[draw=gray,fill=myNewColorD] (axis cs:4.125,0) rectangle (axis cs:4.375,-38.1712370167515);
		\draw[draw=gray,fill=myNewColorD] (axis cs:5.125,0) rectangle (axis cs:5.375,-37.2277238201093);
		\draw[draw=gray,fill=myNewColorD] (axis cs:6.125,0) rectangle (axis cs:6.375,-36.6801100288926);
		\draw[draw=gray,fill=myNewColorD] (axis cs:7.125,0) rectangle (axis cs:7.375,-35.2829648913755);
		\draw[draw=gray,fill=myNewColorD] (axis cs:8.125,0) rectangle (axis cs:8.375,-33.5852389481076);
		\draw[draw=gray,fill=myNewColorD] (axis cs:9.125,0) rectangle (axis cs:9.375,-32.2988855952842);
		
\addlegendimage{ybar,ybar legend,draw=none,fill=myNewColorB}
\addlegendentry{LMC GP}

\addlegendimage{ybar,ybar legend,draw=none,fill=myNewColorC}
\addlegendentry{Chd Normal}

\addlegendimage{ybar,ybar legend,draw=none,fill=myNewColorD}
\addlegendentry{Chd Gamma}
		
%	\end{axis}
%	
%\end{tikzpicture}
}
	\end{figure}
	\vspace{-2.0em}
\begin{itemize}
	\item The ChdGP model consistently outperforms the LMCGP model in all scenarios.
	\item The Gamma likelihood offers a superior explanation of the data, leading to enhanced performance.
\end{itemize}

\end{frame}

\begin{frame}{One-day-ahead Models Forecasting}
	\centering
	\begin{figure}[htbp]
		\centering
		\tiny
		\setlength\figurewidth{0.52\columnwidth} 
		\setlength\figureheight{0.5\columnwidth}
		% This file was created with tikzplotlib v0.10.1.
\begin{tikzpicture}

\definecolor{darkgray176}{RGB}{176,176,176}

\begin{axis}[
font=\tiny,
width=\figurewidth,
height=\figureheight,
axis background/.style={fill=background_color},
axis line style={white},
tick align=inside,
tick pos=left,
x grid style={white},
y grid style={white},
xmajorgrids,
ymajorgrids,
ylabel={Reservoir O Contributions (kWh)},
xlabel={Time (days)},
xmin=50, xmax=250,
xtick style={color=black},
%ymin=-45456751.9352056, 
ymin=-15081530.7548032,
ymax=127003887.236995,
ytick style={color=black},
ytick={-60000000,-40000000,-20000000,0,20000000,40000000,60000000,80000000,100000000,120000000,140000000},
yticklabels={
  \ensuremath{-}0.6,
  \ensuremath{-}0.4,
  \ensuremath{-}0.2,
  0.0,
  0.2,
  0.4,
  0.6,
  0.8,
  1.0,
  1.2,
  1.4
},
legend columns=1, 
legend style={
	nodes={scale=0.8, transform shape},
	fill opacity=0.8,
	draw opacity=1,
	text opacity=1,
	at={(0.03,0.97)},
	anchor=north west,
	legend pos=north west,
	draw=lightgray204,
	fill=background_color
}
]
\path [draw=blue, fill=blue, opacity=0.5]
(axis cs:0,34596132.2023549)
--(axis cs:0,-853642.916707363)
--(axis cs:1,-5586036.25237962)
--(axis cs:2,-7201242.86375978)
--(axis cs:3,-3885830.64629037)
--(axis cs:4,-7377707.89922689)
--(axis cs:5,-4314966.53566143)
--(axis cs:6,-11243113.7404949)
--(axis cs:7,-3793918.01846635)
--(axis cs:8,-684886.063293882)
--(axis cs:9,-9347966.12018801)
--(axis cs:10,-1564665.70153078)
--(axis cs:11,190357.861505184)
--(axis cs:12,1762132.89948877)
--(axis cs:13,2191433.82488414)
--(axis cs:14,3659683.99097321)
--(axis cs:15,-674782.950712027)
--(axis cs:16,-27866933.5375742)
--(axis cs:17,-2148625.55590721)
--(axis cs:18,613921.908777934)
--(axis cs:19,1558885.01446836)
--(axis cs:20,-1086393.11588616)
--(axis cs:21,-4306884.15800038)
--(axis cs:22,28994.6232421631)
--(axis cs:23,1204761.44111507)
--(axis cs:24,-173696.387492052)
--(axis cs:25,-481070.425738214)
--(axis cs:26,1217473.39596628)
--(axis cs:27,-596560.121096932)
--(axis cs:28,171650.835193245)
--(axis cs:29,846203.078132635)
--(axis cs:30,464231.727900502)
--(axis cs:31,236065.25375426)
--(axis cs:32,290918.383912574)
--(axis cs:33,-424303.088068643)
--(axis cs:34,628516.331419422)
--(axis cs:35,470042.406659556)
--(axis cs:36,-820118.549467575)
--(axis cs:37,609160.577481308)
--(axis cs:38,477743.660301628)
--(axis cs:39,-607485.55794722)
--(axis cs:40,1319550.85942407)
--(axis cs:41,243586.78781639)
--(axis cs:42,-110012.538194735)
--(axis cs:43,-1103614.26552132)
--(axis cs:44,-282314.073868448)
--(axis cs:45,-126521.755109994)
--(axis cs:46,-134323.705593717)
--(axis cs:47,-361646.644113813)
--(axis cs:48,160471.510602185)
--(axis cs:49,1331660.34573446)
--(axis cs:50,512572.36649817)
--(axis cs:51,414622.691098604)
--(axis cs:52,2267478.30853995)
--(axis cs:53,1027575.41865049)
--(axis cs:54,231837.897510699)
--(axis cs:55,-272316.036409954)
--(axis cs:56,-109653.086289829)
--(axis cs:57,288812.61673625)
--(axis cs:58,-68459.863652898)
--(axis cs:59,-539379.487332284)
--(axis cs:60,-6803558.77507209)
--(axis cs:61,-1729810.16535422)
--(axis cs:62,-464753.864409848)
--(axis cs:63,-568169.910615373)
--(axis cs:64,-206331.154378275)
--(axis cs:65,-570309.420299839)
--(axis cs:66,-558564.834619601)
--(axis cs:67,-538542.656631168)
--(axis cs:68,-542527.92990522)
--(axis cs:69,-595137.336279767)
--(axis cs:70,-758205.216451677)
--(axis cs:71,-557684.307408952)
--(axis cs:72,-200977.902371153)
--(axis cs:73,-869535.199248858)
--(axis cs:74,-1392286.03506421)
--(axis cs:75,-704572.291364986)
--(axis cs:76,-1225650.46133084)
--(axis cs:77,-829437.659462901)
--(axis cs:78,-1042373.46280107)
--(axis cs:79,-647938.468475775)
--(axis cs:80,-773639.801701888)
--(axis cs:81,-805116.083343962)
--(axis cs:82,-184414.114405604)
--(axis cs:83,-767421.672268401)
--(axis cs:84,-1068046.24629489)
--(axis cs:85,-1407405.78288135)
--(axis cs:86,-483198.789368061)
--(axis cs:87,-424546.206316665)
--(axis cs:88,-853761.812333852)
--(axis cs:89,-596468.171136236)
--(axis cs:90,-1805705.55720369)
--(axis cs:91,-1007180.18448149)
--(axis cs:92,-1316825.54189521)
--(axis cs:93,-1429170.63875181)
--(axis cs:94,-1077535.84273982)
--(axis cs:95,-1203733.54302157)
--(axis cs:96,-1169100.53416684)
--(axis cs:97,-1030966.70545409)
--(axis cs:98,-1058303.93408855)
--(axis cs:99,-1617434.32844333)
--(axis cs:100,-1415869.35788812)
--(axis cs:101,-1854019.50636267)
--(axis cs:102,-1106604.97898954)
--(axis cs:103,-2240138.87182825)
--(axis cs:104,-710066.274119868)
--(axis cs:105,-2140086.94166556)
--(axis cs:106,-1660174.11126362)
--(axis cs:107,-1175781.32887217)
--(axis cs:108,-1065619.25656241)
--(axis cs:109,-1699284.22529631)
--(axis cs:110,207859.782313252)
--(axis cs:111,818019.05281376)
--(axis cs:112,-124831.498242869)
--(axis cs:113,-879780.751741853)
--(axis cs:114,-911643.582673929)
--(axis cs:115,1630963.05897811)
--(axis cs:116,-8740977.95145991)
--(axis cs:117,-2419315.3938779)
--(axis cs:118,-1439566.82601015)
--(axis cs:119,-2862202.8589534)
--(axis cs:120,-4988137.2801126)
--(axis cs:121,-3925583.00666199)
--(axis cs:122,-2799873.61775819)
--(axis cs:123,1392181.96990653)
--(axis cs:124,-338864.442047734)
--(axis cs:125,-628346.180771267)
--(axis cs:126,-2846692.37689084)
--(axis cs:127,-3047992.50204139)
--(axis cs:128,-4969348.8512655)
--(axis cs:129,-2964134.92230303)
--(axis cs:130,-4930652.53577281)
--(axis cs:131,-2430420.8937225)
--(axis cs:132,-5845221.86019168)
--(axis cs:133,-2398781.37642603)
--(axis cs:134,-3647944.66621006)
--(axis cs:135,-1994228.63699626)
--(axis cs:136,-2587075.50240009)
--(axis cs:137,-2298255.99718467)
--(axis cs:138,-6277803.17601118)
--(axis cs:139,-4691758.45608638)
--(axis cs:140,-1248891.94616418)
--(axis cs:141,-537481.282271397)
--(axis cs:142,-5170727.09131346)
--(axis cs:143,-6584461.46439788)
--(axis cs:144,-7340797.43675084)
--(axis cs:145,-4299518.64451829)
--(axis cs:146,-3009225.55358306)
--(axis cs:147,-2562688.91656017)
--(axis cs:148,-5707295.67168825)
--(axis cs:149,3257510.95748859)
--(axis cs:150,-459361.312905714)
--(axis cs:151,-15950935.1949887)
--(axis cs:152,249285.476868764)
--(axis cs:153,7501147.55731863)
--(axis cs:154,-516062.126329729)
--(axis cs:155,-1544652.73099584)
--(axis cs:156,-7346406.29953524)
--(axis cs:157,-3236481.19578696)
--(axis cs:158,-2235712.43621855)
--(axis cs:159,33967.0054667145)
--(axis cs:160,-210784.235149051)
--(axis cs:161,937261.626378498)
--(axis cs:162,-2419272.72173398)
--(axis cs:163,40053.4310830645)
--(axis cs:164,477876.08476729)
--(axis cs:165,297736.088031914)
--(axis cs:166,-3995539.19998951)
--(axis cs:167,-4951122.98603767)
--(axis cs:168,-6418296.16330056)
--(axis cs:169,-17937888.6121986)
--(axis cs:170,-1368115.64586104)
--(axis cs:171,-6310591.56678044)
--(axis cs:172,-7028475.81544593)
--(axis cs:173,-6357787.22346673)
--(axis cs:174,-1299114.14505558)
--(axis cs:175,-4879303.09821001)
--(axis cs:176,-5658201.82072798)
--(axis cs:177,-11802686.2945046)
--(axis cs:178,-8051369.51883466)
--(axis cs:179,-37617631.9728328)
--(axis cs:180,-18872287.7034134)
--(axis cs:181,-8805290.36168908)
--(axis cs:182,-4608592.18042584)
--(axis cs:183,-2823024.56055954)
--(axis cs:184,-3781178.89289106)
--(axis cs:185,-11877467.9613899)
--(axis cs:186,-25458206.0650464)
--(axis cs:187,-12110949.3145174)
--(axis cs:188,-7013488.27665986)
--(axis cs:189,-16550.9050385244)
--(axis cs:190,1099512.84357836)
--(axis cs:191,-984671.059019106)
--(axis cs:192,-10449943.88004)
--(axis cs:193,-1707694.61154558)
--(axis cs:194,-4622028.01127557)
--(axis cs:195,-6858243.43671096)
--(axis cs:196,-16729520.5619626)
--(axis cs:197,-9593083.01624552)
--(axis cs:198,-2525981.05381178)
--(axis cs:199,-15481916.6640258)
--(axis cs:200,-17990463.7617433)
--(axis cs:201,-5989638.50137157)
--(axis cs:202,-2166329.21255841)
--(axis cs:203,3120866.43403987)
--(axis cs:204,3452727.11388417)
--(axis cs:205,-6692584.24044261)
--(axis cs:206,-40609.7036448084)
--(axis cs:207,-6037174.81304305)
--(axis cs:208,-7627182.62225763)
--(axis cs:209,-6927071.99617288)
--(axis cs:210,-1193876.82332588)
--(axis cs:211,-9235308.4598045)
--(axis cs:212,-13471027.6110434)
--(axis cs:213,-23949564.6179913)
--(axis cs:214,-18012850.536091)
--(axis cs:215,-20720042.4915089)
--(axis cs:216,-10892368.2456733)
--(axis cs:217,-12784215.3535881)
--(axis cs:218,-4940817.80323673)
--(axis cs:219,3200410.9540083)
--(axis cs:220,-7456041.50249811)
--(axis cs:221,-29361868.1890028)
--(axis cs:222,-6897295.21485596)
--(axis cs:223,-3593395.4624185)
--(axis cs:224,-7902181.95856141)
--(axis cs:225,-2288842.78907528)
--(axis cs:226,-2080775.17782328)
--(axis cs:227,-2655782.66776101)
--(axis cs:228,-6617255.11092236)
--(axis cs:229,-5491308.07912598)
--(axis cs:230,-4759706.53423403)
--(axis cs:231,-8942537.2900903)
--(axis cs:232,-3591606.39053635)
--(axis cs:233,1301511.51917957)
--(axis cs:234,2776048.11165733)
--(axis cs:235,2703962.37432899)
--(axis cs:236,3405068.74301388)
--(axis cs:237,2605294.57187009)
--(axis cs:238,-6799167.67768139)
--(axis cs:239,-19299435.4926114)
--(axis cs:240,-12422580.9897746)
--(axis cs:241,-1212646.66977061)
--(axis cs:242,-11516083.8576173)
--(axis cs:243,-21683567.1278713)
--(axis cs:244,-13551922.705739)
--(axis cs:245,-7809862.75403043)
--(axis cs:246,-12521904.8549863)
--(axis cs:247,-6462446.78323065)
--(axis cs:248,-20661204.3847918)
--(axis cs:249,-9110792.87831304)
--(axis cs:250,-916062.994226411)
--(axis cs:251,-15833761.5528611)
--(axis cs:252,-34769816.7650859)
--(axis cs:253,-23591363.5909791)
--(axis cs:254,-5130875.73352298)
--(axis cs:255,-7699366.7284573)
--(axis cs:256,-8470152.24426969)
--(axis cs:257,-18887724.8044429)
--(axis cs:258,-4368976.04062563)
--(axis cs:259,-3276301.71895052)
--(axis cs:260,-9958918.20861071)
--(axis cs:261,-3423470.09524097)
--(axis cs:262,2812893.6115282)
--(axis cs:263,-4715800.64234176)
--(axis cs:264,-17097920.1633138)
--(axis cs:265,1806012.79531715)
--(axis cs:266,5433184.91716761)
--(axis cs:267,4703849.75581855)
--(axis cs:268,4596694.54840808)
--(axis cs:269,2703806.00867197)
--(axis cs:270,4754954.48085195)
--(axis cs:271,-30870384.4808447)
--(axis cs:272,-6863843.72789572)
--(axis cs:273,-10013545.1505522)
--(axis cs:274,-20655814.2283266)
--(axis cs:275,-14057429.8717049)
--(axis cs:276,-13951720.8140713)
--(axis cs:277,-1306257.85578844)
--(axis cs:278,-18723206.2946305)
--(axis cs:279,-20041092.4976386)
--(axis cs:280,-18599703.2506623)
--(axis cs:281,-13315209.4836778)
--(axis cs:282,-6742112.09961487)
--(axis cs:283,-7970239.4041408)
--(axis cs:284,-7924368.96293962)
--(axis cs:285,-5173767.30196053)
--(axis cs:286,-6093568.45496081)
--(axis cs:287,-5881853.21032764)
--(axis cs:288,-2518062.11042509)
--(axis cs:289,-12055311.0115328)
--(axis cs:290,-965591.581666676)
--(axis cs:291,-6564890.88642826)
--(axis cs:292,-5347784.89103709)
--(axis cs:293,-9556019.60255712)
--(axis cs:294,-2970924.21358785)
--(axis cs:295,571100.551677326)
--(axis cs:296,-2093384.39350916)
--(axis cs:297,-4509191.82353545)
--(axis cs:298,-6858076.16006655)
--(axis cs:299,-8533354.80024281)
--(axis cs:300,-4444687.76326814)
--(axis cs:301,-4476269.08110779)
--(axis cs:302,-2058226.06816)
--(axis cs:303,-19010462.0906737)
--(axis cs:304,-13414250.1255867)
--(axis cs:305,-4427633.2281632)
--(axis cs:306,-2026620.33112856)
--(axis cs:307,-5212902.79147882)
--(axis cs:308,380172.531112641)
--(axis cs:309,642710.663414631)
--(axis cs:310,186821.082978457)
--(axis cs:311,-1758058.65414377)
--(axis cs:312,795336.477366805)
--(axis cs:313,-3167829.24338658)
--(axis cs:314,-7389146.45124001)
--(axis cs:315,-8811427.63416337)
--(axis cs:316,-9945030.41053307)
--(axis cs:317,-3454037.32238915)
--(axis cs:318,-4139022.16176669)
--(axis cs:319,-261132.244497405)
--(axis cs:320,-1831553.43137401)
--(axis cs:321,-2474960.95853128)
--(axis cs:322,-4442154.13775478)
--(axis cs:323,-7580810.91208872)
--(axis cs:324,-4586018.50697874)
--(axis cs:325,-2869125.47343885)
--(axis cs:326,-2397231.43278097)
--(axis cs:327,-3125568.52110313)
--(axis cs:328,-6837630.03945869)
--(axis cs:329,-14182974.0715743)
--(axis cs:330,-4582133.68107387)
--(axis cs:331,-2209570.53832247)
--(axis cs:332,-4990866.09866306)
--(axis cs:333,-7717790.77843551)
--(axis cs:334,1278422.19581366)
--(axis cs:335,-675591.919549271)
--(axis cs:336,-2892872.25751592)
--(axis cs:337,-3507794.14429825)
--(axis cs:338,-9435322.71605253)
--(axis cs:339,-7403766.3682121)
--(axis cs:340,-3827583.26813812)
--(axis cs:341,-3346103.04741603)
--(axis cs:342,-6136612.77170264)
--(axis cs:343,-5908406.97625974)
--(axis cs:344,-1748024.55161935)
--(axis cs:345,-544833.782020444)
--(axis cs:346,-853411.780319469)
--(axis cs:347,222081.033384022)
--(axis cs:348,-3574867.78022069)
--(axis cs:349,-9394961.6872776)
--(axis cs:350,413884.516955905)
--(axis cs:351,2616859.91990838)
--(axis cs:352,5822766.73131966)
--(axis cs:353,2276255.3772444)
--(axis cs:354,-2195092.34724252)
--(axis cs:355,389138.065302197)
--(axis cs:356,2755244.787623)
--(axis cs:357,-1165685.85112906)
--(axis cs:358,3353981.36092319)
--(axis cs:359,3083528.14384787)
--(axis cs:360,1007139.26241781)
--(axis cs:361,1942673.65483855)
--(axis cs:362,3939190.69453904)
--(axis cs:363,954938.190396296)
--(axis cs:364,2198515.53387038)
--(axis cs:365,3452431.74561573)
--(axis cs:366,305098.094649933)
--(axis cs:367,473075.248144118)
--(axis cs:368,1936124.79698392)
--(axis cs:369,1320455.99403821)
--(axis cs:370,1956989.95166613)
--(axis cs:371,2083437.00124845)
--(axis cs:372,1255213.77473085)
--(axis cs:373,3718361.94190649)
--(axis cs:374,1327183.64839101)
--(axis cs:375,1443026.71086686)
--(axis cs:376,-1221702.68597638)
--(axis cs:377,2685.80178994313)
--(axis cs:378,-11042822.5690393)
--(axis cs:379,-4931447.11756587)
--(axis cs:380,-7357913.83576206)
--(axis cs:381,55165.505209811)
--(axis cs:382,1166755.64562814)
--(axis cs:383,-765900.438946562)
--(axis cs:384,-2930105.19172323)
--(axis cs:385,-790544.061724236)
--(axis cs:386,-537943.787740132)
--(axis cs:387,-1369752.72683536)
--(axis cs:388,-145113.417432954)
--(axis cs:389,-265727.575619845)
--(axis cs:390,-148454.051977733)
--(axis cs:391,396553.536955224)
--(axis cs:392,863014.516595487)
--(axis cs:393,410875.09627723)
--(axis cs:394,371282.915107106)
--(axis cs:395,312829.393506904)
--(axis cs:396,755403.549889308)
--(axis cs:397,-4334.0181289711)
--(axis cs:398,262533.755178912)
--(axis cs:399,648980.602528025)
--(axis cs:400,140356.355737659)
--(axis cs:401,-360139.804558288)
--(axis cs:402,-281290.435829532)
--(axis cs:403,-630477.716884528)
--(axis cs:404,-294318.844487011)
--(axis cs:405,-130837.102044765)
--(axis cs:406,-82610.1828517285)
--(axis cs:407,-72522.9257967807)
--(axis cs:408,-49411.9622985022)
--(axis cs:409,-347291.200277215)
--(axis cs:410,-380536.755376732)
--(axis cs:411,-595079.991248671)
--(axis cs:412,-702188.859491176)
--(axis cs:413,-1176858.35788536)
--(axis cs:414,-455765.331001713)
--(axis cs:415,-725572.97874669)
--(axis cs:416,-491147.747432771)
--(axis cs:417,-910269.157597187)
--(axis cs:418,-926371.337693206)
--(axis cs:419,-989255.982537909)
--(axis cs:420,-1515106.127096)
--(axis cs:421,-1795275.11220244)
--(axis cs:422,-1179253.11970219)
--(axis cs:423,-1029335.74707505)
--(axis cs:424,-919666.904299572)
--(axis cs:425,-1097657.4954725)
--(axis cs:426,-813004.236800329)
--(axis cs:427,-1281579.21806386)
--(axis cs:428,-1434181.7248627)
--(axis cs:429,-1301086.55810716)
--(axis cs:430,-1125516.15048048)
--(axis cs:431,-727736.080871173)
--(axis cs:432,-882633.77279605)
--(axis cs:433,-1041871.91544223)
--(axis cs:434,-1022300.81587218)
--(axis cs:435,-1063357.49578363)
--(axis cs:436,-673629.409934112)
--(axis cs:437,-848179.695353922)
--(axis cs:438,-834509.083013375)
--(axis cs:439,-711374.95424074)
--(axis cs:440,-998024.375220916)
--(axis cs:440,2571782.70865133)
--(axis cs:440,2571782.70865133)
--(axis cs:439,3237640.62436406)
--(axis cs:438,2318343.12674239)
--(axis cs:437,2108608.82642856)
--(axis cs:436,2825113.50129583)
--(axis cs:435,2913200.13600152)
--(axis cs:434,2983313.38091609)
--(axis cs:433,2854764.61886411)
--(axis cs:432,3037063.48571)
--(axis cs:431,2514542.39210243)
--(axis cs:430,3050047.95882202)
--(axis cs:429,3992924.28392573)
--(axis cs:428,6359774.90190807)
--(axis cs:427,2335006.3117532)
--(axis cs:426,2878667.67996498)
--(axis cs:425,2718006.59847332)
--(axis cs:424,2640187.99588761)
--(axis cs:423,2925346.60885262)
--(axis cs:422,3289673.72179354)
--(axis cs:421,5206876.3912535)
--(axis cs:420,3518268.49217077)
--(axis cs:419,2763845.75916849)
--(axis cs:418,2983619.86405221)
--(axis cs:417,3211994.03075606)
--(axis cs:416,4315360.16169814)
--(axis cs:415,3111190.39415309)
--(axis cs:414,4365042.49108483)
--(axis cs:413,6083647.92476273)
--(axis cs:412,5640059.20662084)
--(axis cs:411,6339988.60589198)
--(axis cs:410,5624104.03326828)
--(axis cs:409,6103102.48072491)
--(axis cs:408,4537826.5754507)
--(axis cs:407,4844381.30499894)
--(axis cs:406,5535845.56416899)
--(axis cs:405,6011640.96408003)
--(axis cs:404,6064922.53418855)
--(axis cs:403,7340901.17555302)
--(axis cs:402,7014409.9032408)
--(axis cs:401,6516374.53415201)
--(axis cs:400,5528901.56455887)
--(axis cs:399,4924163.07824601)
--(axis cs:398,5190001.47935991)
--(axis cs:397,4805356.74053602)
--(axis cs:396,5869944.70990666)
--(axis cs:395,5678323.19483934)
--(axis cs:394,4983499.34417523)
--(axis cs:393,4882041.69767455)
--(axis cs:392,5439100.72109522)
--(axis cs:391,6010575.85913621)
--(axis cs:390,6352851.56118922)
--(axis cs:389,6515381.67113507)
--(axis cs:388,7237814.97161744)
--(axis cs:387,6787327.35167215)
--(axis cs:386,7190923.43482344)
--(axis cs:385,8020831.41964127)
--(axis cs:384,10557809.4617634)
--(axis cs:383,11438666.3936354)
--(axis cs:382,10953907.505505)
--(axis cs:381,13230383.6891546)
--(axis cs:380,23271367.2112783)
--(axis cs:379,25700925.0361117)
--(axis cs:378,38037699.6929755)
--(axis cs:377,20896935.121299)
--(axis cs:376,20590006.3297011)
--(axis cs:375,10653283.3410965)
--(axis cs:374,10385826.0948622)
--(axis cs:373,12655877.8350002)
--(axis cs:372,9719694.79816536)
--(axis cs:371,8957340.57198132)
--(axis cs:370,9873556.74570077)
--(axis cs:369,11436693.5755289)
--(axis cs:368,13130786.1060234)
--(axis cs:367,15202284.1600738)
--(axis cs:366,17806859.78707)
--(axis cs:365,13561272.3705847)
--(axis cs:364,16942798.6058113)
--(axis cs:363,19159813.5398804)
--(axis cs:362,24040316.8798133)
--(axis cs:361,20721806.5123708)
--(axis cs:360,25073561.1276108)
--(axis cs:359,26208809.1598008)
--(axis cs:358,33238230.7881453)
--(axis cs:357,67903603.4946563)
--(axis cs:356,46345677.9145936)
--(axis cs:355,29708355.1032785)
--(axis cs:354,30257718.6469852)
--(axis cs:353,35096726.4418381)
--(axis cs:352,51955962.010936)
--(axis cs:351,39715042.5412721)
--(axis cs:350,42805406.342788)
--(axis cs:349,48499739.4662735)
--(axis cs:348,40828860.7763733)
--(axis cs:347,15845795.2491297)
--(axis cs:346,18127629.2391969)
--(axis cs:345,17530308.8543933)
--(axis cs:344,20643038.0272883)
--(axis cs:343,32267352.0645135)
--(axis cs:342,24789297.3867803)
--(axis cs:341,20874827.7393471)
--(axis cs:340,24132824.5616206)
--(axis cs:339,40266204.7004735)
--(axis cs:338,42523110.024757)
--(axis cs:337,20318515.0367585)
--(axis cs:336,20781939.3344781)
--(axis cs:335,18647322.7820928)
--(axis cs:334,19472228.5936075)
--(axis cs:333,31075149.4384974)
--(axis cs:332,22338946.4792924)
--(axis cs:331,22661455.0203699)
--(axis cs:330,27976911.2815015)
--(axis cs:329,45157266.5080733)
--(axis cs:328,47183935.2339755)
--(axis cs:327,24956245.7229208)
--(axis cs:326,20224518.2625506)
--(axis cs:325,19729520.5125292)
--(axis cs:324,22174546.2943155)
--(axis cs:323,30076539.2026673)
--(axis cs:322,26978691.2077231)
--(axis cs:321,31748456.277789)
--(axis cs:320,24150235.056846)
--(axis cs:319,22513727.9011122)
--(axis cs:318,33929014.4528167)
--(axis cs:317,48366621.5199747)
--(axis cs:316,50156936.4689183)
--(axis cs:315,64935823.0833381)
--(axis cs:314,41863742.8261092)
--(axis cs:313,23095345.9840894)
--(axis cs:312,20847340.2261038)
--(axis cs:311,21633877.4652743)
--(axis cs:310,20498875.9910094)
--(axis cs:309,23966357.144416)
--(axis cs:308,24977132.1657175)
--(axis cs:307,29696425.0592774)
--(axis cs:306,26360639.7242414)
--(axis cs:305,36197065.2146962)
--(axis cs:304,87857031.7404048)
--(axis cs:303,87781526.0402347)
--(axis cs:302,27546656.7052501)
--(axis cs:301,22582927.4658816)
--(axis cs:300,23400635.7373775)
--(axis cs:299,27192618.1515564)
--(axis cs:298,26695965.8550909)
--(axis cs:297,23768982.2067522)
--(axis cs:296,24115172.906188)
--(axis cs:295,21205675.5406685)
--(axis cs:294,26670749.3123028)
--(axis cs:293,39145844.7244768)
--(axis cs:292,38884793.6456095)
--(axis cs:291,42507480.2175165)
--(axis cs:290,32778813.2749817)
--(axis cs:289,77127365.8038309)
--(axis cs:288,27352905.9246508)
--(axis cs:287,32202745.0055116)
--(axis cs:286,33408314.2256078)
--(axis cs:285,37114542.5334878)
--(axis cs:284,43720707.4068279)
--(axis cs:283,60521622.4220209)
--(axis cs:282,48199045.0139863)
--(axis cs:281,59271702.843023)
--(axis cs:280,67282070.4589634)
--(axis cs:279,70225754.121881)
--(axis cs:278,89352662.5514869)
--(axis cs:277,49681351.359428)
--(axis cs:276,57637101.7968957)
--(axis cs:275,60754794.607339)
--(axis cs:274,82155396.6502254)
--(axis cs:273,71027631.4567525)
--(axis cs:272,58919325.389185)
--(axis cs:271,104183352.530914)
--(axis cs:270,36221606.8023792)
--(axis cs:269,28465558.7263576)
--(axis cs:268,32498367.5381783)
--(axis cs:267,43374079.4625834)
--(axis cs:266,44407026.4376042)
--(axis cs:265,56592745.0781638)
--(axis cs:264,86944077.604442)
--(axis cs:263,72876823.1133718)
--(axis cs:262,38241137.0610778)
--(axis cs:261,56851121.435855)
--(axis cs:260,78315260.591191)
--(axis cs:259,48587157.4810924)
--(axis cs:258,52410747.3788668)
--(axis cs:257,70229847.3961562)
--(axis cs:256,70202419.237281)
--(axis cs:255,74725451.3343306)
--(axis cs:254,70268996.5589349)
--(axis cs:253,116021454.897523)
--(axis cs:252,119164767.274623)
--(axis cs:251,93584287.6032926)
--(axis cs:250,40084613.1418811)
--(axis cs:249,55853849.8665631)
--(axis cs:248,67723933.1485922)
--(axis cs:247,39412617.1718968)
--(axis cs:246,58206322.6332334)
--(axis cs:245,49453943.9706928)
--(axis cs:244,63860264.2941953)
--(axis cs:243,80876608.4178955)
--(axis cs:242,65461441.0102734)
--(axis cs:241,52559201.9141961)
--(axis cs:240,66114471.8122473)
--(axis cs:239,95440103.0541213)
--(axis cs:238,63139250.1418308)
--(axis cs:237,32309306.3149586)
--(axis cs:236,30654897.6117544)
--(axis cs:235,28285913.3087394)
--(axis cs:234,26810304.6850482)
--(axis cs:233,27197980.7592009)
--(axis cs:232,42495069.9192829)
--(axis cs:231,55872447.7714156)
--(axis cs:230,57930489.2093634)
--(axis cs:229,48170451.1535554)
--(axis cs:228,34855392.4483797)
--(axis cs:227,30422323.097323)
--(axis cs:226,35648756.0744436)
--(axis cs:225,39076845.3001248)
--(axis cs:224,49830497.4985461)
--(axis cs:223,51179385.9447976)
--(axis cs:222,66944046.5843529)
--(axis cs:221,99955763.5672249)
--(axis cs:220,94759358.1124249)
--(axis cs:219,45931079.6132399)
--(axis cs:218,52077389.5343025)
--(axis cs:217,72276375.1798604)
--(axis cs:216,60376728.5632365)
--(axis cs:215,82126168.6131145)
--(axis cs:214,93663219.4745855)
--(axis cs:213,92510683.1180308)
--(axis cs:212,77779071.7042248)
--(axis cs:211,56411875.2332252)
--(axis cs:210,63023844.3395253)
--(axis cs:209,57173112.3152059)
--(axis cs:208,69081730.9161585)
--(axis cs:207,50591111.9600118)
--(axis cs:206,52131851.3112886)
--(axis cs:205,44285627.0626978)
--(axis cs:204,43613619.7709586)
--(axis cs:203,48153580.9958027)
--(axis cs:202,44024126.1892149)
--(axis cs:201,74117162.0091726)
--(axis cs:200,61694296.764301)
--(axis cs:199,62161447.1906941)
--(axis cs:198,43426689.8925851)
--(axis cs:197,49656582.1472698)
--(axis cs:196,66025541.5232728)
--(axis cs:195,49566179.8648047)
--(axis cs:194,51508511.8679749)
--(axis cs:193,48784273.0706519)
--(axis cs:192,33070649.1421254)
--(axis cs:191,22368009.1521464)
--(axis cs:190,21528385.699878)
--(axis cs:189,24518520.2416253)
--(axis cs:188,37508198.2559846)
--(axis cs:187,63462471.6250062)
--(axis cs:186,74468198.8752429)
--(axis cs:185,49769489.2018885)
--(axis cs:184,31655744.1698706)
--(axis cs:183,39547391.4687049)
--(axis cs:182,47231645.3126254)
--(axis cs:181,57370938.2841126)
--(axis cs:180,75457677.6732184)
--(axis cs:179,118031070.195462)
--(axis cs:178,56776189.7018831)
--(axis cs:177,57160566.4144318)
--(axis cs:176,55450140.3532165)
--(axis cs:175,53314602.8763941)
--(axis cs:174,60096789.352107)
--(axis cs:173,57163069.6110679)
--(axis cs:172,62471131.777267)
--(axis cs:171,59149612.0769755)
--(axis cs:170,45384094.8687679)
--(axis cs:169,97609601.9761474)
--(axis cs:168,37640561.1201824)
--(axis cs:167,26902858.3671081)
--(axis cs:166,24353136.8396797)
--(axis cs:165,11774059.9830391)
--(axis cs:164,11445750.6386295)
--(axis cs:163,13643778.7939727)
--(axis cs:162,14525364.8974329)
--(axis cs:161,12561116.048744)
--(axis cs:160,16306247.2945684)
--(axis cs:159,17760046.3384036)
--(axis cs:158,24242531.3003955)
--(axis cs:157,34693366.5490744)
--(axis cs:156,56770427.9439681)
--(axis cs:155,25493823.7758946)
--(axis cs:154,32511453.78086)
--(axis cs:153,33611785.1524437)
--(axis cs:152,35870399.0655219)
--(axis cs:151,93816306.2765162)
--(axis cs:150,20787762.8339549)
--(axis cs:149,26909611.7385088)
--(axis cs:148,42376822.7387426)
--(axis cs:147,25026388.4199369)
--(axis cs:146,23270821.9796784)
--(axis cs:145,24052369.5168922)
--(axis cs:144,29135983.0940603)
--(axis cs:143,21438139.6640784)
--(axis cs:142,19199606.527845)
--(axis cs:141,16609619.9328199)
--(axis cs:140,23620537.2347816)
--(axis cs:139,31608483.6846315)
--(axis cs:138,34459162.1505142)
--(axis cs:137,21164149.9010958)
--(axis cs:136,17769463.6600841)
--(axis cs:135,31237723.4937156)
--(axis cs:134,18706849.0208458)
--(axis cs:133,9799030.74306726)
--(axis cs:132,13146073.3065937)
--(axis cs:131,9843626.5128962)
--(axis cs:130,14179313.1434025)
--(axis cs:129,15232912.8019655)
--(axis cs:128,17464615.6940063)
--(axis cs:127,16270310.1943255)
--(axis cs:126,23995202.9441827)
--(axis cs:125,15488660.473101)
--(axis cs:124,28501770.8109702)
--(axis cs:123,13210841.8922435)
--(axis cs:122,10116965.4611028)
--(axis cs:121,13697978.8466512)
--(axis cs:120,11845449.7018084)
--(axis cs:119,9213911.99373606)
--(axis cs:118,9443442.0866228)
--(axis cs:117,11997761.6698075)
--(axis cs:116,26772040.7709316)
--(axis cs:115,13977161.9122069)
--(axis cs:114,7430613.09768144)
--(axis cs:113,8020363.37765927)
--(axis cs:112,7367699.20669164)
--(axis cs:111,9794695.38743918)
--(axis cs:110,11354411.3898994)
--(axis cs:109,8503747.52173967)
--(axis cs:108,9458772.05887763)
--(axis cs:107,10612808.4767264)
--(axis cs:106,11885181.6524068)
--(axis cs:105,13735356.8072376)
--(axis cs:104,4529846.83260842)
--(axis cs:103,5580128.95966158)
--(axis cs:102,5873012.82342121)
--(axis cs:101,4389707.39177331)
--(axis cs:100,4668309.12321229)
--(axis cs:99,2908756.43743959)
--(axis cs:98,2632564.72252357)
--(axis cs:97,2979787.50104724)
--(axis cs:96,3401115.38017222)
--(axis cs:95,5184269.66843069)
--(axis cs:94,4688367.7393749)
--(axis cs:93,5133923.88583514)
--(axis cs:92,6110159.70656019)
--(axis cs:91,5952220.72834351)
--(axis cs:90,5924883.87087149)
--(axis cs:89,4754438.57287731)
--(axis cs:88,3370499.65485787)
--(axis cs:87,3386864.38794918)
--(axis cs:86,3828940.21962903)
--(axis cs:85,4198254.70063352)
--(axis cs:84,3010342.71444515)
--(axis cs:83,4148436.54312056)
--(axis cs:82,3905561.1097291)
--(axis cs:81,3643029.67323505)
--(axis cs:80,2827161.30343423)
--(axis cs:79,3233386.44754158)
--(axis cs:78,3626549.15885457)
--(axis cs:77,3483275.76085126)
--(axis cs:76,4781937.83095133)
--(axis cs:75,6838473.07255014)
--(axis cs:74,7576348.19621543)
--(axis cs:73,5778599.22230404)
--(axis cs:72,4788112.7347225)
--(axis cs:71,3669174.88802808)
--(axis cs:70,3125817.30048843)
--(axis cs:69,3384620.35601429)
--(axis cs:68,3520897.33473701)
--(axis cs:67,3824697.70923198)
--(axis cs:66,3615623.16509058)
--(axis cs:65,4097231.33880218)
--(axis cs:64,4535887.41038848)
--(axis cs:63,5432181.52887327)
--(axis cs:62,7461826.90131818)
--(axis cs:61,13259719.3827644)
--(axis cs:60,20395394.4570106)
--(axis cs:59,12646090.8497341)
--(axis cs:58,8352857.68941383)
--(axis cs:57,5897032.66433034)
--(axis cs:56,5094603.89364182)
--(axis cs:55,6003841.31865959)
--(axis cs:54,6144359.78675367)
--(axis cs:53,5481712.45450961)
--(axis cs:52,6724765.73391629)
--(axis cs:51,6428134.90459145)
--(axis cs:50,5195697.42983763)
--(axis cs:49,7126025.31306164)
--(axis cs:48,4667037.34957992)
--(axis cs:47,4640867.5395035)
--(axis cs:46,5295535.78309924)
--(axis cs:45,5485972.42275806)
--(axis cs:44,6754845.99208397)
--(axis cs:43,6592498.94560791)
--(axis cs:42,5023979.54573547)
--(axis cs:41,5905970.54094092)
--(axis cs:40,8292377.62891233)
--(axis cs:39,6763915.69361597)
--(axis cs:38,6960376.60096706)
--(axis cs:37,5477160.96907715)
--(axis cs:36,8160682.07246871)
--(axis cs:35,5899321.06023989)
--(axis cs:34,5946233.04786544)
--(axis cs:33,5700767.60226444)
--(axis cs:32,6024751.99844728)
--(axis cs:31,7830457.56760958)
--(axis cs:30,7544038.71249236)
--(axis cs:29,9071106.61237436)
--(axis cs:28,10095951.0770328)
--(axis cs:27,12719501.3828668)
--(axis cs:26,11680586.0914555)
--(axis cs:25,12209470.4704433)
--(axis cs:24,13652241.7777301)
--(axis cs:23,12413590.9104601)
--(axis cs:22,14957468.2175066)
--(axis cs:21,24711182.4053503)
--(axis cs:20,26710048.3645034)
--(axis cs:19,20483897.2085983)
--(axis cs:18,24263347.4122052)
--(axis cs:17,30904999.0004691)
--(axis cs:16,50968021.7871738)
--(axis cs:15,17494555.6331959)
--(axis cs:14,12714359.4714595)
--(axis cs:13,19113065.6260549)
--(axis cs:12,20882435.2687863)
--(axis cs:11,18377263.9092672)
--(axis cs:10,20990022.8720978)
--(axis cs:9,37075648.6821711)
--(axis cs:8,31716017.5590973)
--(axis cs:7,44395598.2663442)
--(axis cs:6,55436744.6354327)
--(axis cs:5,43815771.8228967)
--(axis cs:4,44285315.8986164)
--(axis cs:3,41174449.9453694)
--(axis cs:2,47649626.7846114)
--(axis cs:1,40522031.3386555)
--(axis cs:0,34596132.2023549)
--cycle;

\addplot [only marks, red, mark=asterisk, mark size=1.2]
table {%
0 17920100
1 20248400
2 18087300
3 18316900
4 18044700
5 22639100
6 19789300
7 14686600
8 11502400
9 9352700
10 8582100
11 11649900
12 9782300
13 7409700
14 7937700
15 11951600
16 13791300
17 10920300
18 10713700
19 12009000
20 11282600
21 7367100
22 6603000
23 6539000
24 5942200
25 5765100
26 5302700
27 4925600
28 4325500
29 3851600
30 3472800
31 3169500
32 3046500
33 3474500
34 3267900
35 3308900
36 3322000
37 3338400
38 2776000
39 2633300
40 2776000
41 2375900
42 2215200
43 2999000
44 2917000
45 2402100
46 2274200
47 2433300
48 4332000
49 2994100
50 2869400
51 5627400
52 4043500
53 3120300
54 2528400
55 2233200
56 2666100
57 3199000
58 4791100
59 4153300
60 4274600
61 3363000
62 2495600
63 1993900
64 1765900
65 1702000
66 1592100
67 1464200
68 1372400
69 1347800
70 1311700
71 1508500
72 1659400
73 1702000
74 2341500
75 1420000
76 1308500
77 1223200
78 1192000
79 1136300
80 1149400
81 1190400
82 1129700
83 1101900
84 1167500
85 1208400
86 1551100
87 1272400
88 1116600
89 1388800
90 1728200
91 1726600
92 1623300
93 1160900
94 1170700
95 1015000
96 1167500
97 1231400
98 1115000
99 1129700
100 1228100
101 1549500
102 1251100
103 1665900
104 1821700
105 2193900
106 2202100
107 2903900
108 2428400
109 4323800
110 4814100
111 2985900
112 2297200
113 2305400
114 6327500
115 4494400
116 2805500
117 2077500
118 2008600
119 1970900
120 2710400
121 2228300
122 6494800
123 14245500
124 5363400
125 9349500
126 6568600
127 6753800
128 5697900
129 4345200
130 3863100
131 3307200
132 3369500
133 5522400
134 15231000
135 5953700
136 6871900
137 8970700
138 8341100
139 8483700
140 6193100
141 4950200
142 4327100
143 6491500
144 8265600
145 8141000
146 8800200
147 19151500
148 16346000
149 10290600
150 40800200
151 17308500
152 24067200
153 15927900
154 11067800
155 25603600
156 15375300
157 9887300
158 8095100
159 7209700
160 6063500
161 5289600
162 4866600
163 4815700
164 5242100
165 8669000
166 9542900
167 14517700
168 52177900
169 24288600
170 29417500
171 28182800
172 25541300
173 30627600
174 25752800
175 26618600
176 21983200
177 25367500
178 47919600
179 28571400
180 22909600
181 20286100
182 16623100
183 13602800
184 15458900
185 24282000
186 27756500
187 15190000
188 11868000
189 10492300
190 9096900
191 9602000
192 23829500
193 26753000
194 21184700
195 21588000
196 19899200
197 22309500
198 24147600
199 23190000
200 49352700
201 27041600
202 30930900
203 27174400
204 18992400
205 29878200
206 22434100
207 31185100
208 24457500
209 37904500
210 25323200
211 37802800
212 39396600
213 44325500
214 33485500
215 24906700
216 31806500
217 25085500
218 25892200
219 51576100
220 38050400
221 29583100
222 23290000
223 21069900
224 18662800
225 17557700
226 13837300
227 13066600
228 22163600
229 29327300
230 25428200
231 19318700
232 14771900
233 15132600
234 16177100
235 19100600
236 19381000
237 27615500
238 35612200
239 26056200
240 27774500
241 26820200
242 31006300
243 21214200
244 18223400
245 20350100
246 15621200
247 20760000
248 22301300
249 18144700
250 39506400
251 46898100
252 64052500
253 37488000
254 39360500
255 33231400
256 27777800
257 25434700
258 24719800
259 41657700
260 29101000
261 22374000
262 43966500
263 44027000
264 36616400
265 30216200
266 28524100
267 20749000
268 16987300
269 23598500
270 36276400
271 27510500
272 35321700
273 29521400
274 22944500
275 22181100
276 26176500
277 35362500
278 25040400
279 25037100
280 22156600
281 19184500
282 27206500
283 17706600
284 15772600
285 11930800
286 11368500
287 10760300
288 32362700
289 14795000
290 18355600
291 17165500
292 13585300
293 10915600
294 10258400
295 9548900
296 8646500
297 8092300
298 7856900
299 7505400
300 7456400
301 9874200
302 37072500
303 43216100
304 14688700
305 10892700
306 10770100
307 12025600
308 11775500
309 9648600
310 8914600
311 9890600
312 8917900
313 12215300
314 35035600
315 21501000
316 23971200
317 14410800
318 10387600
319 9969100
320 14520400
321 12476900
322 10735800
323 8744600
324 7547900
325 7343600
326 8913000
327 17425400
328 13748700
329 10690000
330 9058500
331 7732600
332 10789700
333 9107500
334 7734300
335 7060700
336 7121200
337 12414700
338 16140500
339 8811600
340 7513600
341 7546300
342 11554800
343 9539100
344 8471600
345 8579500
346 7785000
347 18571400
348 21973500
349 23778300
350 23215900
351 34620300
352 21178900
353 13477400
354 15167700
355 29351300
356 44963800
357 20335400
358 14605400
359 12252900
360 10199600
361 15777500
362 9835000
363 9583200
364 8497700
365 7502100
366 7086900
367 7266700
368 6712500
369 6231900
370 5965400
371 5161100
372 9223600
373 6310400
374 5527300
375 8223100
376 10583700
377 12437600
378 10085100
379 7753900
380 6228600
381 5417800
382 5015600
383 3457600
384 3439600
385 3336600
386 3197700
387 3408600
388 3338300
389 3375900
390 3305600
391 3357900
392 3335000
393 2476700
394 3181300
395 3163400
396 2633700
397 2883800
398 2865800
399 2744800
400 2661500
401 2594400
402 2538900
403 2507800
404 2679500
405 2543800
406 2460400
407 2529000
408 2468600
409 2350900
410 2243000
411 1978100
412 1816300
413 1790100
414 1427200
415 1401000
416 1334000
417 1271900
418 1185200
419 1185200
420 1101900
421 1062600
422 1029900
423 1052800
424 1054500
425 1070800
426 1041400
427 1798300
428 1412500
429 1257200
430 1255500
431 1193400
432 1093700
433 1039700
434 951500
435 904000
436 874600
437 954700
438 994000
439 966200
440 948200
};
\addplot [blue]
table {%
0 16871244.6428238
1 17467997.543138
2 20224191.9604258
3 18644309.6495395
4 18453803.9996947
5 19750402.6436176
6 22096815.4474689
7 20300840.1239389
8 15515565.7479017
9 13863841.2809915
10 9712678.5852835
11 9283810.88538617
12 11322284.0841375
13 10652249.7254695
14 8187021.73121635
15 8409886.34124191
16 11550544.1247998
17 14378186.722281
18 12438634.6604916
19 11021391.1115333
20 12811827.6243086
21 10202149.123675
22 7493231.42037438
23 6809176.17578758
24 6739272.69511903
25 5864200.02235252
26 6449029.7437109
27 6061470.63088494
28 5133800.95611302
29 4958654.8452535
30 4004135.22019643
31 4033261.41068192
32 3157835.19117992
33 2638232.2570979
34 3287374.68964243
35 3184681.73344972
36 3670281.76150057
37 3043160.77327923
38 3719060.13063435
39 3078215.06783437
40 4805964.2441682
41 3074778.66437866
42 2456983.50377037
43 2744442.3400433
44 3236265.95910776
45 2679725.33382403
46 2580606.03875276
47 2139610.44769484
48 2413754.43009105
49 4228842.82939805
50 2854134.8981679
51 3421378.79784503
52 4496122.02122812
53 3254643.93658005
54 3188098.84213218
55 2865762.64112482
56 2492475.403676
57 3092922.6405333
58 4142198.91288047
59 6053355.68120089
60 6795917.84096927
61 5764954.60870509
62 3498536.51845417
63 2432005.80912895
64 2164778.1280051
65 1763460.95925117
66 1528529.16523549
67 1643077.5263004
68 1489184.70241589
69 1394741.50986726
70 1183806.04201838
71 1555745.29030956
72 2293567.41617567
73 2454532.01152759
74 3092031.08057561
75 3066950.39059258
76 1778143.68481024
77 1326919.05069418
78 1292087.84802675
79 1292723.9895329
80 1026760.75086617
81 1418956.79494554
82 1860573.49766175
83 1690507.43542608
84 971148.234075129
85 1395424.45887608
86 1672870.71513048
87 1481159.09081626
88 1258368.92126201
89 2078985.20087054
90 2059589.1568339
91 2472520.27193101
92 2396667.08233249
93 1852376.62354167
94 1805415.94831754
95 1990268.06270456
96 1116007.42300269
97 974410.397796576
98 787130.394217513
99 645661.05449813
100 1626219.88266208
101 1267843.94270532
102 2383203.92221584
103 1669995.04391666
104 1909890.27924428
105 5797634.93278604
106 5112503.77057158
107 4718513.57392712
108 4196576.40115761
109 3402231.64822168
110 5781135.58610631
111 5306357.22012647
112 3621433.85422438
113 3570291.31295871
114 3259484.75750376
115 7804062.4855925
116 9015531.40973584
117 4789223.13796478
118 4001937.63030632
119 3175854.56739133
120 3428656.21084789
121 4886197.91999461
122 3658545.92167231
123 7301511.93107503
124 14081453.1844612
125 7430157.14616488
126 10574255.2836459
127 6611158.84614208
128 6247633.42137042
129 6134388.93983126
130 4624330.30381487
131 3706602.80958685
132 3650425.723201
133 3700124.68332062
134 7529452.17731789
135 14621747.4283597
136 7591194.07884203
137 9432946.95195556
138 14090679.4872515
139 13458362.6142725
140 11185822.6443087
141 8036069.32527427
142 7014439.71826576
143 7426839.09984023
144 10897592.8286547
145 9876425.43618695
146 10130798.2130477
147 11231849.7516884
148 18334763.5335272
149 15083561.3479987
150 10164200.7605246
151 38932685.5407638
152 18059842.2711953
153 20556466.3548812
154 15997695.8272651
155 11974585.5224494
156 24712010.8222164
157 15728442.6766437
158 11003409.4320885
159 8897006.67193518
160 8047731.52970967
161 6749188.83756126
162 6053046.08784944
163 6841916.11252787
164 5961813.36169841
165 6035898.03553551
166 10178798.8198451
167 10975867.6905352
168 15611132.4784409
169 39835856.6819744
170 22007989.6114534
171 26419510.2550975
172 27721327.9809105
173 25402641.1938006
174 29398837.6035257
175 24217649.8890921
176 24895969.2662443
177 22678940.0599636
178 24362410.0915242
179 40206719.1113144
180 28292694.9849025
181 24282823.9612117
182 21311526.5660998
183 18362183.4540727
184 13937282.6384898
185 18946010.6202493
186 24504996.4050982
187 25675761.1552444
188 15247354.9896624
189 12250984.6682934
190 11313949.2717282
191 10691669.0465636
192 11310352.6310427
193 23538289.2295532
194 23443241.9283497
195 21353968.2140469
196 24648010.4806551
197 20031749.5655121
198 20450354.4193867
199 23339765.2633341
200 21851916.5012788
201 34063761.7539005
202 20928898.4883283
203 25637223.7149213
204 23533173.4424214
205 18796521.4111276
206 26045620.8038219
207 22276968.5734844
208 30727274.1469505
209 25123020.1595165
210 30914983.7580997
211 23588283.3867103
212 32154022.0465907
213 34280559.2500197
214 37825184.4692472
215 30703063.0608028
216 24742180.1587816
217 29746079.9131362
218 23568285.8655329
219 24565745.2836241
220 43651658.3049634
221 35296947.689111
222 30023375.6847485
223 23792995.2411896
224 20964157.7699923
225 18394001.2555247
226 16783990.4483101
227 13883270.214781
228 14119068.6687287
229 21339571.5372147
230 26585391.3375647
231 23464955.2406627
232 19451731.7643733
233 14249746.1391902
234 14793176.3983528
235 15494937.8415342
236 17029983.1773841
237 17457300.4434143
238 28170041.2320747
239 38070333.7807549
240 26845945.4112363
241 25673277.6222127
242 26972678.5763281
243 29596520.6450121
244 25154170.7942282
245 20822040.6083312
246 22842208.8891236
247 16475085.1943331
248 23531364.3819002
249 23371528.494125
250 19584275.0738273
251 38875263.0252158
252 42197475.2547684
253 46215045.6532721
254 32569060.412706
255 33513042.3029367
256 30866133.4965057
257 25671061.2958566
258 24020885.6691206
259 22655427.8810709
260 34178171.1912902
261 26713825.670307
262 20527015.336303
263 34080511.235515
264 34923078.7205641
265 29199378.9367405
266 24920105.6773859
267 24038964.609201
268 18547531.0432932
269 15584682.3675148
270 20488280.6416156
271 36656484.0250347
272 26027740.8306447
273 30507043.1531002
274 30749791.2109494
275 23348682.367817
276 21842690.4914122
277 24187546.7518198
278 35314728.1284282
279 25092330.8121212
280 24341183.6041506
281 22978246.6796726
282 20728466.4571857
283 26275691.5089401
284 17898169.2219441
285 15970387.6157636
286 13657372.8853235
287 13160445.897592
288 12417421.9071128
289 32536027.3961491
290 15906610.8466575
291 17971294.6655441
292 16768504.3772862
293 14794912.5609598
294 11849912.5493575
295 10888388.0461729
296 11010894.2563394
297 9629895.19160837
298 9918944.84751217
299 9329631.67565682
300 9477973.98705467
301 9053329.19238693
302 12744215.3185451
303 34385531.9747805
304 37221390.807409
305 15884715.9932665
306 12167009.6965564
307 12241761.1338993
308 12678652.3484151
309 12304533.9039153
310 10342848.5369939
311 9937909.40556525
312 10821338.3517353
313 9963758.37035139
314 17237298.1874346
315 28062197.7245874
316 20105953.0291926
317 22456292.0987928
318 14894996.145525
319 11126297.8283074
320 11159340.812736
321 14636747.6596289
322 11268268.5349842
323 11247864.1452893
324 8794263.89366839
325 8430197.5195452
326 8913643.41488482
327 10915338.6009089
328 20173152.5972584
329 15487146.2182495
330 11697388.8002138
331 10225942.2410237
332 8674040.19031468
333 11678679.330031
334 10375325.3947106
335 8985865.43127174
336 8944533.53848109
337 8405360.44623012
338 16543893.6543522
339 16431219.1661307
340 10152620.6467412
341 8764362.34596551
342 9326342.30753885
343 13179472.5441269
344 9447506.7378345
345 8492737.53618643
346 8637108.72943872
347 8033938.14125687
348 18626996.4980763
349 19552388.8894979
350 21609645.429872
351 21165951.2305902
352 28889364.3711278
353 18686490.9095412
354 14031313.1498714
355 15048746.5842903
356 24550461.3511083
357 33368958.8217636
358 18296106.0745342
359 14646168.6518243
360 13040350.1950143
361 11332240.0836047
362 13989753.7871762
363 10057375.8651383
364 9570657.06984086
365 8506852.0581002
366 9055978.94085997
367 7837679.70410895
368 7533455.45150364
369 6378574.78478355
370 5915273.34868345
371 5520388.78661489
372 5487454.2864481
373 8187119.88845336
374 5856504.87162659
375 6048155.0259817
376 9684151.82186234
377 10449810.4615445
378 13497438.5619681
379 10384738.9592729
380 7956726.68775811
381 6642774.59718219
382 6060331.57556659
383 5336382.97734443
384 3813852.13502007
385 3615143.67895852
386 3326489.82354165
387 2708787.3124184
388 3546350.77709224
389 3124827.04775761
390 3102198.75460574
391 3203564.69804572
392 3151057.61884535
393 2646458.39697589
394 2677391.12964117
395 2995576.29417312
396 3312674.12989798
397 2400511.36120352
398 2726267.61726941
399 2786571.84038702
400 2834628.96014827
401 3078117.36479686
402 3366559.73370563
403 3355211.72933425
404 2885301.84485077
405 2940401.93101763
406 2726617.69065863
407 2385929.18960108
408 2244207.3065761
409 2877905.64022385
410 2621783.63894577
411 2872454.30732165
412 2468935.17356483
413 2453394.78343869
414 1954638.58004156
415 1192808.7077032
416 1912106.20713269
417 1150862.43657943
418 1028624.2631795
419 887294.888315292
420 1001581.18253739
421 1705800.63952553
422 1055210.30104567
423 948005.430888788
424 860260.54579402
425 810174.55150041
426 1032831.72158232
427 526713.546844671
428 2462796.58852268
429 1345918.86290928
430 962265.904170771
431 893403.155615626
432 1077214.85645697
433 906446.351710942
434 980506.282521958
435 924921.320108942
436 1075742.04568086
437 630214.565537319
438 741917.021864509
439 1263132.83506166
440 786879.166715209
};
\addlegendentry{Test data}
\addlegendentry{ChdGP Normal mean}
\addlegendimage{line width=5pt,draw=blue,opacity=0.5}
\addlegendentry{$95\%$ ChdGP Normal CI}
\end{axis}

\end{tikzpicture}

		% This file was created with tikzplotlib v0.10.1.
\begin{tikzpicture}


\begin{axis}[
	axis lines=left,
	font=\small,
	width=15cm,
	height=7cm,
%	axis background/.style={fill=background_color},
%	axis line style={white},
%	tick align=inside,
%	tick pos=left,
%	x grid style={white},
%	y grid style={white},
%	xmajorgrids,
%	ymajorgrids,
	title={Reservoir Contributions (kWh)},
	xlabel={Time (days)},
	xmin=50, xmax=250,
%	xtick style={color=black},
	ymin=-10000000, % Adjusted to -0.1*10^8
	ymax=80000000,  % Adjusted to 0.8*10^8
%	ytick style={color=black},
%	ytick={-10000000,0,20000000,40000000,60000000,80000000}, % Updated ticks for new range
%	yticklabels={
%		\ensuremath{-}0.1,
%		0.0,
%		0.2,
%		0.4,
%		0.6,
%		0.8
%	},
	legend columns=1, 
	legend style={
		nodes={scale=0.8, transform shape},
		fill opacity=0.8,
		draw opacity=1,
		text opacity=1,
		at={(0.03,0.97)},
		anchor=north west,
		legend pos=north west,
%		draw=lightgray204,
%		fill=background_color
	}
	]

\path [draw=blue, fill=blue, opacity=0.5]
(axis cs:0,30907406.4514597)
--(axis cs:0,5885922.78993782)
--(axis cs:1,5674543.05807476)
--(axis cs:2,7301785.31760189)
--(axis cs:3,6956792.82687586)
--(axis cs:4,8183185.13398946)
--(axis cs:5,8195896.04875933)
--(axis cs:6,8977533.45208891)
--(axis cs:7,9019784.38992647)
--(axis cs:8,6563541.59216855)
--(axis cs:9,5024031.82018058)
--(axis cs:10,3951316.81423389)
--(axis cs:11,3403244.46128388)
--(axis cs:12,3782391.59382283)
--(axis cs:13,3504576.48729532)
--(axis cs:14,2707612.09794435)
--(axis cs:15,3163042.44660711)
--(axis cs:16,3974093.56616021)
--(axis cs:17,5852836.73089923)
--(axis cs:18,4263134.95795006)
--(axis cs:19,3959661.0661793)
--(axis cs:20,3964063.41803515)
--(axis cs:21,3525253.3595841)
--(axis cs:22,1978248.24136687)
--(axis cs:23,1930215.12969607)
--(axis cs:24,1795470.67961426)
--(axis cs:25,1521195.59313902)
--(axis cs:26,1741044.87681918)
--(axis cs:27,1501987.83486739)
--(axis cs:28,1175185.73026311)
--(axis cs:29,1396483.47880615)
--(axis cs:30,949149.462315896)
--(axis cs:31,1123228.24390233)
--(axis cs:32,818526.291392528)
--(axis cs:33,937686.908303233)
--(axis cs:34,737592.724089867)
--(axis cs:35,813494.274872942)
--(axis cs:36,960671.555528153)
--(axis cs:37,860559.099717059)
--(axis cs:38,1066637.65883783)
--(axis cs:39,880174.293292029)
--(axis cs:40,1203749.6207275)
--(axis cs:41,799299.435247251)
--(axis cs:42,1041410.8596224)
--(axis cs:43,1122440.68145136)
--(axis cs:44,1185025.34164734)
--(axis cs:45,1068031.32659201)
--(axis cs:46,976186.989842073)
--(axis cs:47,625172.089805166)
--(axis cs:48,670027.717485929)
--(axis cs:49,1184285.79550688)
--(axis cs:50,883777.483314366)
--(axis cs:51,1066163.18486263)
--(axis cs:52,1140114.9025892)
--(axis cs:53,575074.852245501)
--(axis cs:54,626863.146294142)
--(axis cs:55,632801.920292059)
--(axis cs:56,578851.716113477)
--(axis cs:57,1196437.80523812)
--(axis cs:58,989773.072905666)
--(axis cs:59,1797140.29591263)
--(axis cs:60,1798154.77964696)
--(axis cs:61,1304855.40957626)
--(axis cs:62,779553.20094473)
--(axis cs:63,569662.222496089)
--(axis cs:64,578825.174967114)
--(axis cs:65,500652.438646471)
--(axis cs:66,448378.423212281)
--(axis cs:67,420966.139404685)
--(axis cs:68,416336.675522104)
--(axis cs:69,388884.452952956)
--(axis cs:70,351795.021472162)
--(axis cs:71,416945.249087078)
--(axis cs:72,621184.216491492)
--(axis cs:73,437569.654576349)
--(axis cs:74,291952.967227485)
--(axis cs:75,406831.279136508)
--(axis cs:76,286448.731488487)
--(axis cs:77,307267.713257018)
--(axis cs:78,310281.944410744)
--(axis cs:79,291076.089038305)
--(axis cs:80,284069.499330887)
--(axis cs:81,298451.884961761)
--(axis cs:82,377695.916852841)
--(axis cs:83,354547.143998929)
--(axis cs:84,321760.601013329)
--(axis cs:85,468192.91339697)
--(axis cs:86,393200.776244804)
--(axis cs:87,312057.91216927)
--(axis cs:88,295412.833710447)
--(axis cs:89,554706.917688585)
--(axis cs:90,516360.598094657)
--(axis cs:91,494111.100014269)
--(axis cs:92,370889.854193524)
--(axis cs:93,358099.934615687)
--(axis cs:94,358539.275531733)
--(axis cs:95,468705.953740391)
--(axis cs:96,295965.310266312)
--(axis cs:97,197864.488907552)
--(axis cs:98,356521.415093029)
--(axis cs:99,276908.054028984)
--(axis cs:100,253172.4422219)
--(axis cs:101,350615.41508964)
--(axis cs:102,266061.615309853)
--(axis cs:103,449415.0290544)
--(axis cs:104,322254.859010478)
--(axis cs:105,524458.209057414)
--(axis cs:106,1334293.54515099)
--(axis cs:107,896890.548558071)
--(axis cs:108,948184.657988197)
--(axis cs:109,484743.186544185)
--(axis cs:110,1018441.13138522)
--(axis cs:111,1031671.3502079)
--(axis cs:112,597977.644558914)
--(axis cs:113,569193.262254381)
--(axis cs:114,579148.38560621)
--(axis cs:115,1062122.20096183)
--(axis cs:116,2465459.77493679)
--(axis cs:117,1234431.38923625)
--(axis cs:118,1383804.85495062)
--(axis cs:119,712382.348651562)
--(axis cs:120,985623.702975463)
--(axis cs:121,1065269.3816668)
--(axis cs:122,954909.247549378)
--(axis cs:123,1391683.66143444)
--(axis cs:124,4187814.01334828)
--(axis cs:125,2217462.29546424)
--(axis cs:126,2921683.35272755)
--(axis cs:127,2092471.97076557)
--(axis cs:128,2013094.92666548)
--(axis cs:129,2347429.64240725)
--(axis cs:130,1678117.56757269)
--(axis cs:131,1319292.24769294)
--(axis cs:132,1293322.03081131)
--(axis cs:133,1116452.77034514)
--(axis cs:134,2206196.11739621)
--(axis cs:135,4494548.99962802)
--(axis cs:136,1770540.6675187)
--(axis cs:137,2105862.53864624)
--(axis cs:138,1837669.25303643)
--(axis cs:139,2179670.96744299)
--(axis cs:140,3074966.30594941)
--(axis cs:141,2234087.37280266)
--(axis cs:142,2161899.1783829)
--(axis cs:143,2490254.16821019)
--(axis cs:144,3409825.17436242)
--(axis cs:145,3119550.55482692)
--(axis cs:146,3153374.39827717)
--(axis cs:147,3712193.23046244)
--(axis cs:148,5194753.34239912)
--(axis cs:149,4470694.92589561)
--(axis cs:150,2897807.47054268)
--(axis cs:151,7859520.64788529)
--(axis cs:152,5807236.26266111)
--(axis cs:153,6537234.69029279)
--(axis cs:154,5060620.91715702)
--(axis cs:155,4654802.77415325)
--(axis cs:156,6807040.69104126)
--(axis cs:157,5846368.87004621)
--(axis cs:158,3049321.67459329)
--(axis cs:159,2407199.20059873)
--(axis cs:160,2423266.93220722)
--(axis cs:161,2304686.08890137)
--(axis cs:162,2056799.04374939)
--(axis cs:163,2710417.98343284)
--(axis cs:164,1950640.59611637)
--(axis cs:165,2108626.80029966)
--(axis cs:166,3355919.31777448)
--(axis cs:167,3554253.21428481)
--(axis cs:168,4825345.21652113)
--(axis cs:169,8583443.04109458)
--(axis cs:170,7352386.69048913)
--(axis cs:171,8438919.30352168)
--(axis cs:172,7959027.84575208)
--(axis cs:173,9323736.16401268)
--(axis cs:174,12226641.1956416)
--(axis cs:175,9168070.65343566)
--(axis cs:176,9646519.74676877)
--(axis cs:177,7853798.91381352)
--(axis cs:178,9462950.47762435)
--(axis cs:179,13748834.5005414)
--(axis cs:180,11334287.0340279)
--(axis cs:181,9365122.73363676)
--(axis cs:182,8500298.65753706)
--(axis cs:183,6974809.83608806)
--(axis cs:184,4093380.09949642)
--(axis cs:185,4137784.01428194)
--(axis cs:186,4428680.53816348)
--(axis cs:187,6991351.65173873)
--(axis cs:188,5315324.82770793)
--(axis cs:189,4472481.41795101)
--(axis cs:190,3868805.84633733)
--(axis cs:191,3487904.20338225)
--(axis cs:192,3269921.01495045)
--(axis cs:193,5610934.16841571)
--(axis cs:194,6409566.46787521)
--(axis cs:195,5683457.85231325)
--(axis cs:196,6197642.06216503)
--(axis cs:197,5168836.20709938)
--(axis cs:198,7081783.41341364)
--(axis cs:199,8222176.50188293)
--(axis cs:200,6672683.07126275)
--(axis cs:201,7789608.95212143)
--(axis cs:202,8400243.36157936)
--(axis cs:203,9101659.25258377)
--(axis cs:204,8895883.87374998)
--(axis cs:205,6233890.2294476)
--(axis cs:206,7398549.90511274)
--(axis cs:207,6430356.26271581)
--(axis cs:208,7940391.41339114)
--(axis cs:209,6901018.81867239)
--(axis cs:210,11895836.1328784)
--(axis cs:211,6356548.59057485)
--(axis cs:212,10338172.6082507)
--(axis cs:213,10861931.1816734)
--(axis cs:214,11872467.3086382)
--(axis cs:215,9594892.87895865)
--(axis cs:216,9642674.81646281)
--(axis cs:217,13395779.6000364)
--(axis cs:218,8513769.7207618)
--(axis cs:219,8369877.68060045)
--(axis cs:220,12742191.1041494)
--(axis cs:221,17844527.5247142)
--(axis cs:222,12309350.2711541)
--(axis cs:223,8961720.65478013)
--(axis cs:224,8095881.73521033)
--(axis cs:225,7246445.98931923)
--(axis cs:226,5600278.62800738)
--(axis cs:227,5078266.29217994)
--(axis cs:228,6061725.77955029)
--(axis cs:229,8427322.37777764)
--(axis cs:230,9633530.8166702)
--(axis cs:231,9647537.97854844)
--(axis cs:232,8791207.46929713)
--(axis cs:233,5594834.67091531)
--(axis cs:234,6054270.84717083)
--(axis cs:235,6330371.29844263)
--(axis cs:236,6401054.58601598)
--(axis cs:237,7212764.8196859)
--(axis cs:238,9809815.68165749)
--(axis cs:239,15428721.5136098)
--(axis cs:240,10279396.2328681)
--(axis cs:241,7452596.29537626)
--(axis cs:242,9960130.84129409)
--(axis cs:243,11638915.7063169)
--(axis cs:244,10789041.8465765)
--(axis cs:245,9759680.11014523)
--(axis cs:246,9381321.64545867)
--(axis cs:247,8131430.18191267)
--(axis cs:248,9502108.30393242)
--(axis cs:249,11816373.1473228)
--(axis cs:250,10810543.4266018)
--(axis cs:251,18182887.3666676)
--(axis cs:252,25668170.5000103)
--(axis cs:253,23414250.6982734)
--(axis cs:254,12249167.6070946)
--(axis cs:255,13146536.9276894)
--(axis cs:256,13211775.666662)
--(axis cs:257,10956399.719752)
--(axis cs:258,13905908.41429)
--(axis cs:259,10229376.6760242)
--(axis cs:260,13749834.7425519)
--(axis cs:261,13502351.9181865)
--(axis cs:262,10952763.5736677)
--(axis cs:263,16911381.3101086)
--(axis cs:264,18048853.6073122)
--(axis cs:265,13540809.8481116)
--(axis cs:266,11088934.1988711)
--(axis cs:267,11440076.0192202)
--(axis cs:268,8479460.79407775)
--(axis cs:269,6496459.88255137)
--(axis cs:270,7511496.97898794)
--(axis cs:271,13871495.3357715)
--(axis cs:272,13218869.7271439)
--(axis cs:273,11306477.0735967)
--(axis cs:274,9073863.72767941)
--(axis cs:275,7243336.22443831)
--(axis cs:276,9016953.15442407)
--(axis cs:277,8415278.38614876)
--(axis cs:278,10947933.3322801)
--(axis cs:279,6998521.75621858)
--(axis cs:280,6909559.10969476)
--(axis cs:281,7379816.54111927)
--(axis cs:282,8080349.87329687)
--(axis cs:283,9636050.27412812)
--(axis cs:284,6674818.28170529)
--(axis cs:285,5643060.0877621)
--(axis cs:286,5190403.02260596)
--(axis cs:287,5007120.39754427)
--(axis cs:288,4631625.70947617)
--(axis cs:289,6368279.23901656)
--(axis cs:290,5961320.00721586)
--(axis cs:291,6676578.92018345)
--(axis cs:292,5439848.93986802)
--(axis cs:293,4599629.60502644)
--(axis cs:294,4618721.54446433)
--(axis cs:295,3571849.92165438)
--(axis cs:296,4048967.16713747)
--(axis cs:297,3441618.84813054)
--(axis cs:298,3219848.44612806)
--(axis cs:299,3696846.96266786)
--(axis cs:300,3247579.25179106)
--(axis cs:301,3283952.03138324)
--(axis cs:302,4642174.53896253)
--(axis cs:303,10102445.1414483)
--(axis cs:304,10170619.0929285)
--(axis cs:305,6666880.56787654)
--(axis cs:306,5067312.83377437)
--(axis cs:307,4400712.87731744)
--(axis cs:308,5042738.62250831)
--(axis cs:309,5013968.83767498)
--(axis cs:310,4029528.35661837)
--(axis cs:311,4385369.73331181)
--(axis cs:312,4596221.72229076)
--(axis cs:313,3508063.15146899)
--(axis cs:314,5431127.44258737)
--(axis cs:315,5351588.39306578)
--(axis cs:316,7362632.67197411)
--(axis cs:317,9420654.0854381)
--(axis cs:318,5682107.72658379)
--(axis cs:319,4256823.8750039)
--(axis cs:320,4697740.15697164)
--(axis cs:321,5707905.18248479)
--(axis cs:322,3883883.97735439)
--(axis cs:323,3918063.66695527)
--(axis cs:324,3335132.94007076)
--(axis cs:325,3143126.46797185)
--(axis cs:326,3245422.54779617)
--(axis cs:327,4002944.84741737)
--(axis cs:328,5611325.11617583)
--(axis cs:329,6662831.56306687)
--(axis cs:330,4380736.90980662)
--(axis cs:331,3640388.39934131)
--(axis cs:332,3298331.89462565)
--(axis cs:333,3450353.22629363)
--(axis cs:334,3756787.12673499)
--(axis cs:335,3632144.96065723)
--(axis cs:336,3173995.28178716)
--(axis cs:337,2936465.30958241)
--(axis cs:338,5688568.66478927)
--(axis cs:339,6354918.27897763)
--(axis cs:340,3612129.24388249)
--(axis cs:341,3655025.41338249)
--(axis cs:342,3946968.45189771)
--(axis cs:343,5373970.37347272)
--(axis cs:344,3384818.16340067)
--(axis cs:345,2699665.40736224)
--(axis cs:346,2821564.70720534)
--(axis cs:347,2727266.92077257)
--(axis cs:348,5758722.19064767)
--(axis cs:349,7596014.62233575)
--(axis cs:350,7443577.35666882)
--(axis cs:351,8356835.84293042)
--(axis cs:352,9627219.46011807)
--(axis cs:353,6845695.51505982)
--(axis cs:354,6065686.37170465)
--(axis cs:355,5708725.78605777)
--(axis cs:356,9203669.97008046)
--(axis cs:357,10651545.9824798)
--(axis cs:358,7059257.49950107)
--(axis cs:359,5729878.43461214)
--(axis cs:360,5346773.13236028)
--(axis cs:361,4986651.28279049)
--(axis cs:362,4858615.43991823)
--(axis cs:363,3635066.5917467)
--(axis cs:364,3261658.97302984)
--(axis cs:365,2729343.78025953)
--(axis cs:366,2393662.33304993)
--(axis cs:367,2397017.93032576)
--(axis cs:368,2183968.85073964)
--(axis cs:369,1561968.27568165)
--(axis cs:370,1504705.38734753)
--(axis cs:371,1115345.94915297)
--(axis cs:372,1548269.14966113)
--(axis cs:373,1725084.35071295)
--(axis cs:374,1637833.16178425)
--(axis cs:375,2253411.33341079)
--(axis cs:376,4063904.38212003)
--(axis cs:377,3893908.94085638)
--(axis cs:378,5503964.55824355)
--(axis cs:379,3655975.40685349)
--(axis cs:380,2837472.92147622)
--(axis cs:381,1975589.69351533)
--(axis cs:382,2074520.1697708)
--(axis cs:383,1569034.66048513)
--(axis cs:384,1593125.1907466)
--(axis cs:385,1188698.67094589)
--(axis cs:386,1258446.03761076)
--(axis cs:387,1129419.84001063)
--(axis cs:388,997809.681282995)
--(axis cs:389,1047942.25675551)
--(axis cs:390,1034396.71414504)
--(axis cs:391,834479.349800151)
--(axis cs:392,923140.32875804)
--(axis cs:393,782335.914563685)
--(axis cs:394,1019495.0995653)
--(axis cs:395,859245.071988825)
--(axis cs:396,929741.013921958)
--(axis cs:397,582710.68067314)
--(axis cs:398,564277.115197866)
--(axis cs:399,765149.908633839)
--(axis cs:400,662383.531390484)
--(axis cs:401,790968.714943065)
--(axis cs:402,656868.530271685)
--(axis cs:403,712265.826661303)
--(axis cs:404,956314.743010091)
--(axis cs:405,982663.792073087)
--(axis cs:406,776901.887679556)
--(axis cs:407,640899.799539382)
--(axis cs:408,623205.05472591)
--(axis cs:409,968760.52056564)
--(axis cs:410,972558.038259843)
--(axis cs:411,928068.87001041)
--(axis cs:412,1023035.3434562)
--(axis cs:413,873814.437637977)
--(axis cs:414,653636.052045234)
--(axis cs:415,784966.294567908)
--(axis cs:416,785282.572579655)
--(axis cs:417,666827.845717447)
--(axis cs:418,518345.578987251)
--(axis cs:419,505802.962987898)
--(axis cs:420,537978.068505467)
--(axis cs:421,639022.101410031)
--(axis cs:422,552775.651046558)
--(axis cs:423,507323.168394768)
--(axis cs:424,484162.927752758)
--(axis cs:425,404238.958893168)
--(axis cs:426,488064.294943239)
--(axis cs:427,449202.257629827)
--(axis cs:428,658603.676186485)
--(axis cs:429,469330.385086818)
--(axis cs:430,371638.602575499)
--(axis cs:431,512666.39473428)
--(axis cs:432,454228.701608211)
--(axis cs:433,374808.909598031)
--(axis cs:434,345175.884406033)
--(axis cs:435,397675.743865078)
--(axis cs:436,562026.310520272)
--(axis cs:437,514053.464560066)
--(axis cs:438,331948.934847177)
--(axis cs:439,771993.608710197)
--(axis cs:440,425095.600541194)
--(axis cs:440,4199289.41172533)
--(axis cs:440,4199289.41172533)
--(axis cs:439,5263544.99962118)
--(axis cs:438,3818942.60412088)
--(axis cs:437,4618188.69990215)
--(axis cs:436,4728135.30626294)
--(axis cs:435,4052103.13101315)
--(axis cs:434,4133688.57304111)
--(axis cs:433,3986905.48424963)
--(axis cs:432,4266202.43738488)
--(axis cs:431,4639610.26834708)
--(axis cs:430,4086673.67146971)
--(axis cs:429,4630651.88498612)
--(axis cs:428,5909738.08595537)
--(axis cs:427,4291683.26714343)
--(axis cs:426,4430276.03212385)
--(axis cs:425,4146270.05640581)
--(axis cs:424,4379158.57181653)
--(axis cs:423,4468609.3331454)
--(axis cs:422,4658124.13311879)
--(axis cs:421,4824686.54011011)
--(axis cs:420,4510879.42990366)
--(axis cs:419,4553589.50305825)
--(axis cs:418,4422728.93501595)
--(axis cs:417,4851426.62973183)
--(axis cs:416,5614774.7346897)
--(axis cs:415,5526964.00691462)
--(axis cs:414,5470393.57455906)
--(axis cs:413,5926840.43958532)
--(axis cs:412,6208355.04309339)
--(axis cs:411,5968121.02158074)
--(axis cs:410,5884214.47079459)
--(axis cs:409,6025731.40357646)
--(axis cs:408,4987703.05070686)
--(axis cs:407,5308124.66512419)
--(axis cs:406,5767149.89908403)
--(axis cs:405,6474017.77728772)
--(axis cs:404,6406287.84291345)
--(axis cs:403,5877550.19988039)
--(axis cs:402,5655336.47440097)
--(axis cs:401,6099774.67390184)
--(axis cs:400,5291037.08656023)
--(axis cs:399,6130470.43969601)
--(axis cs:398,5258439.13700402)
--(axis cs:397,5250051.64528962)
--(axis cs:396,8007819.0213936)
--(axis cs:395,6382625.3894454)
--(axis cs:394,6810077.26067491)
--(axis cs:393,5738505.43750242)
--(axis cs:392,6509519.83214562)
--(axis cs:391,6256760.80901626)
--(axis cs:390,7268595.62931472)
--(axis cs:389,7176817.46126866)
--(axis cs:388,7420154.99863558)
--(axis cs:387,8203979.65226962)
--(axis cs:386,8208120.54934509)
--(axis cs:385,8527250.74451589)
--(axis cs:384,10114216.0523928)
--(axis cs:383,10077226.1671876)
--(axis cs:382,12515729.3942162)
--(axis cs:381,12029744.16748)
--(axis cs:380,18382389.4497451)
--(axis cs:379,21931659.4898066)
--(axis cs:378,32119867.070031)
--(axis cs:377,20820209.8705683)
--(axis cs:376,24948150.4232784)
--(axis cs:375,13294629.1234139)
--(axis cs:374,10965545.7905627)
--(axis cs:373,11314382.9071257)
--(axis cs:372,10593742.0637399)
--(axis cs:371,8576076.23809106)
--(axis cs:370,9069698.93993837)
--(axis cs:369,9485139.24966531)
--(axis cs:368,12852205.4554322)
--(axis cs:367,13762894.4832561)
--(axis cs:366,13687612.4645395)
--(axis cs:365,15566675.848878)
--(axis cs:364,17061337.2891881)
--(axis cs:363,19664924.6194977)
--(axis cs:362,25144748.6747629)
--(axis cs:361,24232926.5112259)
--(axis cs:360,25016658.7708773)
--(axis cs:359,26422032.2018139)
--(axis cs:358,32996183.5984118)
--(axis cs:357,47398028.8214451)
--(axis cs:356,40219331.6917484)
--(axis cs:355,28905967.3930454)
--(axis cs:354,32007322.0913327)
--(axis cs:353,31985076.1367082)
--(axis cs:352,43111394.2949487)
--(axis cs:351,37219593.2611589)
--(axis cs:350,36268949.2467646)
--(axis cs:349,36098827.4871247)
--(axis cs:348,32821909.2721048)
--(axis cs:347,15787497.5914623)
--(axis cs:346,16445861.5089191)
--(axis cs:345,14812096.2438514)
--(axis cs:344,18756905.9335852)
--(axis cs:343,30971165.7016194)
--(axis cs:342,22302483.5983266)
--(axis cs:341,20993698.1816097)
--(axis cs:340,20348169.9786559)
--(axis cs:339,33284668.8577015)
--(axis cs:338,33237255.123715)
--(axis cs:337,17936012.1471995)
--(axis cs:336,18016255.2876366)
--(axis cs:335,18959121.8845204)
--(axis cs:334,21275356.8717298)
--(axis cs:333,22358248.7255727)
--(axis cs:332,19026992.0625653)
--(axis cs:331,20314532.2896306)
--(axis cs:330,24161823.2796418)
--(axis cs:329,37377306.3708081)
--(axis cs:328,34282242.611517)
--(axis cs:327,24366583.5055708)
--(axis cs:326,18149451.0588443)
--(axis cs:325,18191804.0407082)
--(axis cs:324,19769761.1426609)
--(axis cs:323,22918496.4287368)
--(axis cs:322,22929127.0902387)
--(axis cs:321,29626099.0200389)
--(axis cs:320,23163977.3193191)
--(axis cs:319,21877404.9032236)
--(axis cs:318,28570111.2474229)
--(axis cs:317,42804253.2748846)
--(axis cs:316,36196246.8076043)
--(axis cs:315,36674359.4645165)
--(axis cs:314,31421425.1906672)
--(axis cs:313,21217535.1550024)
--(axis cs:312,22553204.4858835)
--(axis cs:311,22101177.0678996)
--(axis cs:310,20993247.0917678)
--(axis cs:309,25858639.8454585)
--(axis cs:308,25264769.7295539)
--(axis cs:307,23365468.291973)
--(axis cs:306,26392697.1962685)
--(axis cs:305,35255657.2822883)
--(axis cs:304,45904336.2582476)
--(axis cs:303,49064582.6554454)
--(axis cs:302,27899613.7706029)
--(axis cs:301,19496350.7354926)
--(axis cs:300,18904851.617763)
--(axis cs:299,20863260.8189016)
--(axis cs:298,19989921.1833747)
--(axis cs:297,20731501.2251396)
--(axis cs:296,21875505.3174502)
--(axis cs:295,20617078.4280775)
--(axis cs:294,25773286.5491508)
--(axis cs:293,29997732.4868623)
--(axis cs:292,30877926.3490141)
--(axis cs:291,32257968.4820293)
--(axis cs:290,32282313.1228062)
--(axis cs:289,37502244.6044886)
--(axis cs:288,27348258.8395536)
--(axis cs:287,27813690.4435484)
--(axis cs:286,27002598.9074423)
--(axis cs:285,29326763.3269021)
--(axis cs:284,33061960.5431371)
--(axis cs:283,44810607.6673006)
--(axis cs:282,38839697.0114324)
--(axis cs:281,39548602.7998612)
--(axis cs:280,36969658.6218567)
--(axis cs:279,38966399.2340311)
--(axis cs:278,59325482.663408)
--(axis cs:277,39858016.07084)
--(axis cs:276,42014756.5951939)
--(axis cs:275,38124287.3146453)
--(axis cs:274,46059938.7169577)
--(axis cs:273,49881722.1286587)
--(axis cs:272,56914531.8272636)
--(axis cs:271,73079825.3809388)
--(axis cs:270,32375885.3533778)
--(axis cs:269,28433886.9913605)
--(axis cs:268,34480946.230896)
--(axis cs:267,47154270.6506619)
--(axis cs:266,44160137.7425085)
--(axis cs:265,58089025.7558108)
--(axis cs:264,78333884.5980151)
--(axis cs:263,68816427.0143886)
--(axis cs:262,47470360.2480301)
--(axis cs:261,56747625.9222415)
--(axis cs:260,63250019.5038976)
--(axis cs:259,48845449.5107154)
--(axis cs:258,58811523.6211435)
--(axis cs:257,48893925.3812921)
--(axis cs:256,57910547.9343189)
--(axis cs:255,56681689.1139916)
--(axis cs:254,51247492.3237838)
--(axis cs:253,119579645.037083)
--(axis cs:252,305785769.363122)
--(axis cs:251,93327319.9002433)
--(axis cs:250,50756456.9957249)
--(axis cs:249,54073495.033407)
--(axis cs:248,47756510.6623956)
--(axis cs:247,41158152.994058)
--(axis cs:246,45924689.6402783)
--(axis cs:245,45185173.0286492)
--(axis cs:244,53157935.394833)
--(axis cs:243,57040752.9040771)
--(axis cs:242,49593942.0901391)
--(axis cs:241,40671142.563746)
--(axis cs:240,54778407.4952251)
--(axis cs:239,86206805.1205929)
--(axis cs:238,51149787.9960311)
--(axis cs:237,31982658.7044678)
--(axis cs:236,29061661.9785873)
--(axis cs:235,29983347.3115835)
--(axis cs:234,29605645.9352351)
--(axis cs:233,27147369.9770555)
--(axis cs:232,41730796.2947655)
--(axis cs:231,45727997.4309129)
--(axis cs:230,45993485.0198137)
--(axis cs:229,41338497.6753431)
--(axis cs:228,32287820.9056374)
--(axis cs:227,26211792.5515864)
--(axis cs:226,29856739.813329)
--(axis cs:225,36558914.4728581)
--(axis cs:224,39561108.8255466)
--(axis cs:223,41430871.6979982)
--(axis cs:222,56331651.5976199)
--(axis cs:221,94489742.6225434)
--(axis cs:220,238728692.658533)
--(axis cs:219,43860154.7920334)
--(axis cs:218,40256613.7793458)
--(axis cs:217,61529186.099065)
--(axis cs:216,49180430.2715794)
--(axis cs:215,51696071.3398773)
--(axis cs:214,60579010.4360006)
--(axis cs:213,54738534.6118452)
--(axis cs:212,52057275.3123109)
--(axis cs:211,37252920.7073407)
--(axis cs:210,55323960.7667376)
--(axis cs:209,38957899.1281953)
--(axis cs:208,44643183.5954432)
--(axis cs:207,35501894.8048815)
--(axis cs:206,37247947.5079995)
--(axis cs:205,33347775.6830233)
--(axis cs:204,41750122.8307517)
--(axis cs:203,42883287.5820727)
--(axis cs:202,38415374.5349056)
--(axis cs:201,41729699.734216)
--(axis cs:200,37240659.001287)
--(axis cs:199,46862055.7254133)
--(axis cs:198,36458114.0304558)
--(axis cs:197,33249517.6092754)
--(axis cs:196,39595135.8804486)
--(axis cs:195,32508123.0477923)
--(axis cs:194,36862176.2543281)
--(axis cs:193,32365151.0246501)
--(axis cs:192,25105088.2294362)
--(axis cs:191,22903797.8731198)
--(axis cs:190,24826961.7395492)
--(axis cs:189,26177037.6405506)
--(axis cs:188,32606151.7361513)
--(axis cs:187,42149932.4141385)
--(axis cs:186,35605431.2187158)
--(axis cs:185,30071075.7194126)
--(axis cs:184,26808768.2281718)
--(axis cs:183,37379003.3201648)
--(axis cs:182,43715898.5868172)
--(axis cs:181,47041899.3997055)
--(axis cs:180,56451508.3282476)
--(axis cs:179,74059373.3070983)
--(axis cs:178,45741177.7066333)
--(axis cs:177,42628409.3226226)
--(axis cs:176,45925586.2285622)
--(axis cs:175,45635323.8654739)
--(axis cs:174,58526888.4029096)
--(axis cs:173,43252798.7037189)
--(axis cs:172,46219241.4597439)
--(axis cs:171,44248732.8907601)
--(axis cs:170,38167804.8096629)
--(axis cs:169,48360630.5680163)
--(axis cs:168,30380202.9341655)
--(axis cs:167,22075636.8426656)
--(axis cs:166,21417404.9796468)
--(axis cs:165,15449271.2575766)
--(axis cs:164,14193505.2428978)
--(axis cs:163,16410203.2887796)
--(axis cs:162,15293227.4690749)
--(axis cs:161,15988802.8175019)
--(axis cs:160,16410006.6267682)
--(axis cs:159,17006613.5068585)
--(axis cs:158,22148215.8249944)
--(axis cs:157,30412818.7667384)
--(axis cs:156,37451413.0768894)
--(axis cs:155,25907320.4800542)
--(axis cs:154,29796321.8137097)
--(axis cs:153,33675433.1359814)
--(axis cs:152,33651699.566187)
--(axis cs:151,48399966.3322372)
--(axis cs:150,21043173.7399156)
--(axis cs:149,26437965.7828873)
--(axis cs:148,30659116.8284679)
--(axis cs:147,26370942.4893997)
--(axis cs:146,23154414.7396531)
--(axis cs:145,23407901.3196011)
--(axis cs:144,24662706.2701574)
--(axis cs:143,21932521.3781466)
--(axis cs:142,19513936.2353838)
--(axis cs:141,20461046.5476423)
--(axis cs:140,25103300.2639423)
--(axis cs:139,21912393.5078485)
--(axis cs:138,19231517.9150977)
--(axis cs:137,20288123.8723443)
--(axis cs:136,17659691.892833)
--(axis cs:135,28992565.606704)
--(axis cs:134,18909537.0122337)
--(axis cs:133,9846382.82989278)
--(axis cs:132,11443441.0379957)
--(axis cs:131,10800821.2126233)
--(axis cs:130,13530505.3159609)
--(axis cs:129,19360701.8583425)
--(axis cs:128,17021778.2061125)
--(axis cs:127,16766165.6684989)
--(axis cs:126,25152410.3523359)
--(axis cs:125,16988109.2911192)
--(axis cs:124,27920254.6201206)
--(axis cs:123,17385837.2854033)
--(axis cs:122,11276396.7988257)
--(axis cs:121,12630869.7752584)
--(axis cs:120,8901250.0818607)
--(axis cs:119,7668918.75883771)
--(axis cs:118,12616812.0524451)
--(axis cs:117,11817036.3463745)
--(axis cs:116,24852128.2608868)
--(axis cs:115,15463589.1890112)
--(axis cs:114,6297129.69147975)
--(axis cs:113,6293631.04049408)
--(axis cs:112,6439658.64207188)
--(axis cs:111,11755927.0095615)
--(axis cs:110,12269578.218331)
--(axis cs:109,7293472.90140121)
--(axis cs:108,10327909.2357345)
--(axis cs:107,10222168.2334478)
--(axis cs:106,12969254.8018652)
--(axis cs:105,13244789.1176147)
--(axis cs:104,5044976.20097081)
--(axis cs:103,5710747.26996729)
--(axis cs:102,6728089.9427976)
--(axis cs:101,4368750.58573883)
--(axis cs:100,4415523.74963993)
--(axis cs:99,4000092.37141886)
--(axis cs:98,4284010.08869339)
--(axis cs:97,3490376.60940341)
--(axis cs:96,3865187.75572023)
--(axis cs:95,4519671.4742157)
--(axis cs:94,4613062.24608463)
--(axis cs:93,4177861.27667571)
--(axis cs:92,5033785.94701024)
--(axis cs:91,5707540.58931229)
--(axis cs:90,5599455.31266141)
--(axis cs:89,4747961.61216315)
--(axis cs:88,3850863.80318259)
--(axis cs:87,4103481.71630969)
--(axis cs:86,4675993.77923644)
--(axis cs:85,4940873.40835879)
--(axis cs:84,3791606.47415043)
--(axis cs:83,4792824.65342116)
--(axis cs:82,4428362.75821313)
--(axis cs:81,4242658.82889773)
--(axis cs:80,3657509.10868021)
--(axis cs:79,4054534.82920661)
--(axis cs:78,4037710.36183407)
--(axis cs:77,3898448.61515546)
--(axis cs:76,4516804.16665704)
--(axis cs:75,5729245.3165944)
--(axis cs:74,8263451.89912157)
--(axis cs:73,5516158.77842074)
--(axis cs:72,7603119.96047562)
--(axis cs:71,4462300.92007251)
--(axis cs:70,4086342.9563068)
--(axis cs:69,4251436.98869144)
--(axis cs:68,4239491.38409234)
--(axis cs:67,4503542.3388013)
--(axis cs:66,4338614.14249858)
--(axis cs:65,4714606.74619845)
--(axis cs:64,5358833.73768554)
--(axis cs:63,5627121.03027346)
--(axis cs:62,6838349.51114454)
--(axis cs:61,13254043.2033068)
--(axis cs:60,19111471.0362781)
--(axis cs:59,14279685.6789195)
--(axis cs:58,10268089.8490413)
--(axis cs:57,8097279.19854255)
--(axis cs:56,4998346.9356354)
--(axis cs:55,5275477.89839847)
--(axis cs:54,5393959.72502817)
--(axis cs:53,5457645.64223258)
--(axis cs:52,9206202.37065737)
--(axis cs:51,9731647.30718806)
--(axis cs:50,7086408.94073138)
--(axis cs:49,9981135.63327471)
--(axis cs:48,5411792.86035991)
--(axis cs:47,5147326.35877483)
--(axis cs:46,6261454.67355194)
--(axis cs:45,6985651.31913097)
--(axis cs:44,7160892.75926486)
--(axis cs:43,6363552.28692704)
--(axis cs:42,6541472.34047879)
--(axis cs:41,5918323.0243611)
--(axis cs:40,6747156.21716429)
--(axis cs:39,6177516.19461915)
--(axis cs:38,7267928.26480801)
--(axis cs:37,5963307.52297306)
--(axis cs:36,6539234.16054271)
--(axis cs:35,5762756.94620261)
--(axis cs:34,5481795.56676344)
--(axis cs:33,6685318.96357387)
--(axis cs:32,5617334.43244526)
--(axis cs:31,8421154.50384803)
--(axis cs:30,6396627.25635895)
--(axis cs:29,9362920.18032182)
--(axis cs:28,7393880.28051603)
--(axis cs:27,9801266.61087961)
--(axis cs:26,10542239.2729267)
--(axis cs:25,9100827.61774091)
--(axis cs:24,10613262.965633)
--(axis cs:23,11461534.9194487)
--(axis cs:22,11383551.213304)
--(axis cs:21,18294344.5244007)
--(axis cs:20,20287612.7119578)
--(axis cs:19,19027300.8296004)
--(axis cs:18,20189562.5128942)
--(axis cs:17,28261813.569889)
--(axis cs:16,22987212.2921109)
--(axis cs:15,17012807.2491663)
--(axis cs:14,15427795.4761646)
--(axis cs:13,18200869.7162744)
--(axis cs:12,19240157.355112)
--(axis cs:11,17737071.6739917)
--(axis cs:10,20061172.1473353)
--(axis cs:9,26305564.9902142)
--(axis cs:8,29539023.8489169)
--(axis cs:7,43610232.6175159)
--(axis cs:6,44239662.2620842)
--(axis cs:5,40578806.3473961)
--(axis cs:4,44812877.4860693)
--(axis cs:3,35931535.608027)
--(axis cs:2,37271950.6111963)
--(axis cs:1,33355288.1255068)
--(axis cs:0,30907406.4514597)
--cycle;

\addplot [only marks, red, mark=asterisk, mark size=1.2]
table {%
0 17920100
1 20248400
2 18087300
3 18316900
4 18044700
5 22639100
6 19789300
7 14686600
8 11502400
9 9352700
10 8582100
11 11649900
12 9782300
13 7409700
14 7937700
15 11951600
16 13791300
17 10920300
18 10713700
19 12009000
20 11282600
21 7367100
22 6603000
23 6539000
24 5942200
25 5765100
26 5302700
27 4925600
28 4325500
29 3851600
30 3472800
31 3169500
32 3046500
33 3474500
34 3267900
35 3308900
36 3322000
37 3338400
38 2776000
39 2633300
40 2776000
41 2375900
42 2215200
43 2999000
44 2917000
45 2402100
46 2274200
47 2433300
48 4332000
49 2994100
50 2869400
51 5627400
52 4043500
53 3120300
54 2528400
55 2233200
56 2666100
57 3199000
58 4791100
59 4153300
60 4274600
61 3363000
62 2495600
63 1993900
64 1765900
65 1702000
66 1592100
67 1464200
68 1372400
69 1347800
70 1311700
71 1508500
72 1659400
73 1702000
74 2341500
75 1420000
76 1308500
77 1223200
78 1192000
79 1136300
80 1149400
81 1190400
82 1129700
83 1101900
84 1167500
85 1208400
86 1551100
87 1272400
88 1116600
89 1388800
90 1728200
91 1726600
92 1623300
93 1160900
94 1170700
95 1015000
96 1167500
97 1231400
98 1115000
99 1129700
100 1228100
101 1549500
102 1251100
103 1665900
104 1821700
105 2193900
106 2202100
107 2903900
108 2428400
109 4323800
110 4814100
111 2985900
112 2297200
113 2305400
114 6327500
115 4494400
116 2805500
117 2077500
118 2008600
119 1970900
120 2710400
121 2228300
122 6494800
123 14245500
124 5363400
125 9349500
126 6568600
127 6753800
128 5697900
129 4345200
130 3863100
131 3307200
132 3369500
133 5522400
134 15231000
135 5953700
136 6871900
137 8970700
138 8341100
139 8483700
140 6193100
141 4950200
142 4327100
143 6491500
144 8265600
145 8141000
146 8800200
147 19151500
148 16346000
149 10290600
150 40800200
151 17308500
152 24067200
153 15927900
154 11067800
155 25603600
156 15375300
157 9887300
158 8095100
159 7209700
160 6063500
161 5289600
162 4866600
163 4815700
164 5242100
165 8669000
166 9542900
167 14517700
168 52177900
169 24288600
170 29417500
171 28182800
172 25541300
173 30627600
174 25752800
175 26618600
176 21983200
177 25367500
178 47919600
179 28571400
180 22909600
181 20286100
182 16623100
183 13602800
184 15458900
185 24282000
186 27756500
187 15190000
188 11868000
189 10492300
190 9096900
191 9602000
192 23829500
193 26753000
194 21184700
195 21588000
196 19899200
197 22309500
198 24147600
199 23190000
200 49352700
201 27041600
202 30930900
203 27174400
204 18992400
205 29878200
206 22434100
207 31185100
208 24457500
209 37904500
210 25323200
211 37802800
212 39396600
213 44325500
214 33485500
215 24906700
216 31806500
217 25085500
218 25892200
219 51576100
220 38050400
221 29583100
222 23290000
223 21069900
224 18662800
225 17557700
226 13837300
227 13066600
228 22163600
229 29327300
230 25428200
231 19318700
232 14771900
233 15132600
234 16177100
235 19100600
236 19381000
237 27615500
238 35612200
239 26056200
240 27774500
241 26820200
242 31006300
243 21214200
244 18223400
245 20350100
246 15621200
247 20760000
248 22301300
249 18144700
250 39506400
251 46898100
252 64052500
253 37488000
254 39360500
255 33231400
256 27777800
257 25434700
258 24719800
259 41657700
260 29101000
261 22374000
262 43966500
263 44027000
264 36616400
265 30216200
266 28524100
267 20749000
268 16987300
269 23598500
270 36276400
271 27510500
272 35321700
273 29521400
274 22944500
275 22181100
276 26176500
277 35362500
278 25040400
279 25037100
280 22156600
281 19184500
282 27206500
283 17706600
284 15772600
285 11930800
286 11368500
287 10760300
288 32362700
289 14795000
290 18355600
291 17165500
292 13585300
293 10915600
294 10258400
295 9548900
296 8646500
297 8092300
298 7856900
299 7505400
300 7456400
301 9874200
302 37072500
303 43216100
304 14688700
305 10892700
306 10770100
307 12025600
308 11775500
309 9648600
310 8914600
311 9890600
312 8917900
313 12215300
314 35035600
315 21501000
316 23971200
317 14410800
318 10387600
319 9969100
320 14520400
321 12476900
322 10735800
323 8744600
324 7547900
325 7343600
326 8913000
327 17425400
328 13748700
329 10690000
330 9058500
331 7732600
332 10789700
333 9107500
334 7734300
335 7060700
336 7121200
337 12414700
338 16140500
339 8811600
340 7513600
341 7546300
342 11554800
343 9539100
344 8471600
345 8579500
346 7785000
347 18571400
348 21973500
349 23778300
350 23215900
351 34620300
352 21178900
353 13477400
354 15167700
355 29351300
356 44963800
357 20335400
358 14605400
359 12252900
360 10199600
361 15777500
362 9835000
363 9583200
364 8497700
365 7502100
366 7086900
367 7266700
368 6712500
369 6231900
370 5965400
371 5161100
372 9223600
373 6310400
374 5527300
375 8223100
376 10583700
377 12437600
378 10085100
379 7753900
380 6228600
381 5417800
382 5015600
383 3457600
384 3439600
385 3336600
386 3197700
387 3408600
388 3338300
389 3375900
390 3305600
391 3357900
392 3335000
393 2476700
394 3181300
395 3163400
396 2633700
397 2883800
398 2865800
399 2744800
400 2661500
401 2594400
402 2538900
403 2507800
404 2679500
405 2543800
406 2460400
407 2529000
408 2468600
409 2350900
410 2243000
411 1978100
412 1816300
413 1790100
414 1427200
415 1401000
416 1334000
417 1271900
418 1185200
419 1185200
420 1101900
421 1062600
422 1029900
423 1052800
424 1054500
425 1070800
426 1041400
427 1798300
428 1412500
429 1257200
430 1255500
431 1193400
432 1093700
433 1039700
434 951500
435 904000
436 874600
437 954700
438 994000
439 966200
440 948200
};
\addplot [blue]
table {%
0 15817166.9029387
1 17145190.5225306
2 19703475.5105783
3 18366736.5824923
4 22129341.9521366
5 21306174.908615
6 23187246.6687183
7 23371020.6348475
8 15896690.0053414
9 13741863.5486684
10 10424577.0164218
11 9080490.87441059
12 10013378.7443896
13 9182104.06261942
14 7846247.88242018
15 8595205.26428845
16 11511290.9300549
17 14899460.0106026
18 10670458.049483
19 9979722.79245217
20 10794865.9503605
21 9709820.95568015
22 5640538.97860684
23 5616153.9244689
24 5197156.93713589
25 4548266.10054598
26 5314543.00332894
27 4721815.68558829
28 3573737.71570562
29 4519775.75500272
30 3100719.75848485
31 3870869.16124664
32 2694471.66639465
33 3200701.2371607
34 2616177.07934864
35 2755737.97811234
36 3201327.49819082
37 2809819.71292162
38 3490717.52570412
39 2904829.74159846
40 3400483.45573234
41 2730737.02965754
42 3170535.39629684
43 3220589.57931373
44 3582313.62486965
45 3311635.31864139
46 3037045.60066886
47 2352686.80081
48 2499687.28697661
49 4538686.53775737
50 3224267.9289801
51 4285912.76483627
52 4238386.82476412
53 2391420.73346951
54 2417153.56286799
55 2391550.3385881
56 2223965.5403143
57 3943611.29506132
58 4484577.14456321
59 6602806.22063108
60 7862122.88719539
61 5732097.37389192
62 3052571.74047431
63 2451709.22492641
64 2364887.0271427
65 2007080.50751847
66 1908264.74471907
67 1941827.7473877
68 1813393.26031246
69 1806817.21795047
70 1689273.10304811
71 1898369.86212898
72 3109582.06755396
73 2258302.59010804
74 2874504.26571522
75 2317164.25553454
76 1761716.56760604
77 1565062.8292228
78 1614692.05972512
79 1612752.99904145
80 1475797.31956482
81 1694741.26618903
82 1847645.141592
83 1942621.24493914
84 1575096.52578418
85 2114560.76283538
86 1951151.33717167
87 1618964.81480327
88 1542587.96726445
89 2158072.71553427
90 2355808.13646589
91 2375383.76644832
92 2064138.4550911
93 1705341.59540626
94 1875553.09285395
95 1996577.57431946
96 1616251.69613577
97 1329501.32287569
98 1764291.54413542
99 1498404.07424741
100 1706750.81492763
101 1828069.79457661
102 2335417.30557071
103 2312260.3853618
104 1910045.63439667
105 4689766.24085449
106 5681508.23436574
107 4221029.90303453
108 4375039.47411462
109 2835057.11608126
110 5026304.10091156
111 4944164.04608513
112 2688040.45641937
113 2640485.54634326
114 2685431.78192499
115 6268336.11662248
116 10810267.7751976
117 5131443.81071823
118 5591279.25522077
119 3229836.56317685
120 3948910.39269931
121 5281301.48351834
122 4521541.23626272
123 7150497.90137576
124 13304132.8565417
125 7831691.15026142
126 11305386.2794738
127 7760746.66352563
128 7786368.01499926
129 8859583.31097712
130 6158088.54079236
131 4914631.07092061
132 5077495.31563268
133 4392648.03245461
134 8443942.0368202
135 14395306.8687438
136 7644108.8624477
137 8769869.23282104
138 7970041.60712091
139 9460650.0986257
140 11364259.721617
141 9141525.56233264
142 8729581.71874131
143 9502865.7819968
144 11889987.4793034
145 10914131.8706718
146 11087038.9307394
147 12520983.3460805
148 14958314.0298709
149 12990465.8431474
150 9781807.80972213
151 23117339.2727531
152 16691438.6513235
153 17690181.3809365
154 14947696.774802
155 13052704.4048142
156 18481656.5983629
157 15670104.684711
158 10114210.8777156
159 8118929.95797702
160 7869362.10966852
161 7575244.81212652
162 7178454.91680303
163 8019116.8496968
164 6772678.8687646
165 7111980.83840174
166 10513614.3091575
167 10751949.1640092
168 14902944.9211813
169 24369524.2102699
170 19842279.4558989
171 22469725.3368968
172 22684036.6355168
173 23185777.8241614
174 31098812.2636249
175 24130611.45815
176 24119749.950085
177 21744844.1856101
178 23781403.5908344
179 35846457.8776203
180 29300834.7852873
181 24429275.8890735
182 22947298.1274158
183 18822381.8122583
184 12850334.2635998
185 14190204.1007543
186 15717137.5875855
187 20493636.7845707
188 15969208.4300027
189 12965556.2345471
190 12059048.9434909
191 11115882.943589
192 11670584.380552
193 15945009.0818837
194 18883807.6422594
195 16031881.0225333
196 19278213.1580589
197 16035549.9606281
198 18984525.9336109
199 23109339.4407823
200 19203536.1529147
201 21460367.0913395
202 20813820.024191
203 22579939.7274051
204 22720952.2997906
205 17112860.0100969
206 19764576.7848033
207 17805341.6972037
208 22263459.3083684
209 19749110.7406204
210 29986885.1709166
211 18589813.6964119
212 26858300.2111651
213 28712323.1470952
214 31076024.7571954
215 26104543.7860655
216 25260201.6447637
217 32972642.9163986
218 21549993.9082119
219 22845599.0259692
220 61310619.5845119
221 46429670.0344703
222 30353582.8232426
223 22253140.9758421
224 21008577.2396638
225 19162323.4044949
226 15410261.0997889
227 13493714.5859948
228 16640599.2950091
229 21812942.3092414
230 24344877.2890107
231 24395789.1426938
232 22230274.5852992
233 14505771.4005165
234 15578556.29157
235 15699371.558797
236 15608658.4668774
237 16972435.9899214
238 26397340.9133122
239 40463018.6028649
240 27693970.2287068
241 20374161.1848208
242 25642702.3162933
243 30207771.8296936
244 27912000.6090829
245 24282004.4819338
246 24218980.0735596
247 21813125.0750976
248 24251126.6147131
249 28757133.215754
250 26763331.5055557
251 47589853.6306894
252 85876621.7765533
253 59640960.0422263
254 28236662.3118587
255 30941156.7329272
256 31712491.2340658
257 26116171.5182185
258 32278329.3513308
259 25948446.4489517
260 34185380.1474541
261 31476736.7663939
262 25867181.1734402
263 38561932.7340806
264 42004420.431799
265 32089234.4417422
266 25070680.1878136
267 26090450.0579419
268 19340712.3791421
269 15382610.9820694
270 17908802.8710566
271 35853662.7616761
272 31031921.0115907
273 27239953.6316347
274 24039254.1785751
275 19728196.3418067
276 22516160.4085836
277 21549256.3199576
278 28517092.6667087
279 19985147.0645098
280 18780448.392693
281 20361337.2777619
282 20593379.2603232
283 24294555.8596071
284 17305416.1559645
285 15022872.467961
286 13953284.3598747
287 14018742.5673007
288 13742789.7099644
289 18142349.1759614
290 16558625.4635079
291 17126433.2227616
292 15670226.8263745
293 14489078.3597249
294 13195014.9273231
295 10221924.3231966
296 11005332.6689434
297 10025015.5445261
298 10076816.4759409
299 10525537.0395549
300 9511758.69481277
301 9636988.69669458
302 13843218.6769109
303 25492439.4068606
304 24575228.9280661
305 18335494.3954573
306 13382913.3809106
307 12115748.53758
308 13327074.7777563
309 13180924.9403318
310 10702158.1141293
311 11515697.5966926
312 11924140.0358933
313 10553377.0174844
314 15338368.1638863
315 17237039.4147474
316 19186021.9218797
317 23136697.873801
318 15082110.3103782
319 11316361.910599
320 12206432.9626541
321 15055667.1144521
322 11739311.5697793
323 11516879.8466384
324 9864045.41846232
325 9002723.62918769
326 9132162.68355163
327 12131969.7256949
328 16812339.1203243
329 19035766.7389835
330 12285851.4132363
331 10366553.3670918
332 9591379.30356058
333 11032456.4520627
334 10879575.1739264
335 9786896.16415578
336 9165904.66111255
337 8781190.46220768
338 16280379.0704675
339 17344299.4952445
340 10375784.1723856
341 10465232.8667995
342 11346649.7962284
343 15613148.4189859
344 9611537.47116247
345 7618522.80730687
346 8193692.00952102
347 7989624.65768473
348 16520311.6308929
349 19097798.7951571
350 19298130.6792488
351 20167112.8390764
352 22680905.8693014
353 17388777.8578242
354 16336914.7986623
355 15003465.6896234
356 22230434.5939455
357 25575774.0824056
358 17784424.7100276
359 14138349.4487893
360 13572671.572507
361 12767175.4381347
362 12930872.824674
363 9843073.09357965
364 8738997.80090356
365 7901298.03283077
366 6923214.0744638
367 6856832.46367288
368 6282491.50254044
369 4697144.4203149
370 4436390.54578279
371 3920976.32494381
372 5086547.26526382
373 5592935.36769166
374 5143068.56560185
375 6508300.35611906
376 12388450.0824522
377 10676372.5851138
378 16128647.3037612
379 11080481.6862442
380 9169247.30820924
381 5856733.92241032
382 6243287.77184244
383 4806598.0195022
384 4998467.84292711
385 4044365.85973294
386 3927530.78502294
387 3766024.795146
388 3474533.68850079
389 3403666.64329599
390 3397706.48159642
391 2885670.90664532
392 2999857.754135
393 2630684.39242121
394 3335288.79249748
395 2965618.12093371
396 3515697.99065885
397 2374499.1779725
398 2369302.89321954
399 2802952.79158214
400 2434885.63361194
401 2857970.54188138
402 2518604.35254323
403 2680708.61104026
404 3011243.95615938
405 3065257.53455731
406 2722669.28019223
407 2454667.83664112
408 2240404.92492015
409 2970434.37890756
410 2905033.21132058
411 2892472.64514179
412 3088245.72557693
413 2816044.52759822
414 2517857.43940275
415 2633785.67434188
416 2616730.79755236
417 2212522.49731027
418 1948999.30887365
419 2023939.67282425
420 1961149.00377695
421 2226555.29859337
422 2093701.50693544
423 1970964.37156584
424 1936975.79495566
425 1829219.61860749
426 1957949.99227134
427 1847840.65904331
428 2592558.61059873
429 2043245.09087693
430 1702394.71530586
431 2044106.11330845
432 1824513.8661629
433 1676446.45022735
434 1721964.45290067
435 1746150.46709673
436 2113311.35151209
437 2025922.64921875
438 1614084.80221801
439 2573838.69135064
440 1812585.62323777
};
\addplot [black]
table {%
0 14892701.5549125
1 16105592.4363421
2 18266448.1806344
3 17504638.8336834
4 21400453.7368507
5 20197528.8711394
6 21933579.5864444
7 22306413.5272016
8 15161858.0266189
9 12732650.5528128
10 9852728.79590429
11 8711352.15680785
12 9487736.38601163
13 8715691.20762302
14 7470644.70052106
15 8137756.20158925
16 10749444.8108311
17 14249938.6522305
18 10245973.9018169
19 9495904.18325791
20 10245028.155478
21 9107149.49607602
22 5298490.9264577
23 5271424.11227119
24 4933130.71642757
25 4223993.71173469
26 4979192.33891741
27 4391027.2671512
28 3349503.90639006
29 4241459.01136462
30 2836111.09340364
31 3536230.93088559
32 2533624.60962837
33 2946222.76300503
34 2431064.41473476
35 2560682.52508425
36 2950159.44312263
37 2618580.52901552
38 3250866.46071164
39 2684336.16544333
40 3225205.66368502
41 2537848.66499306
42 2961451.1565054
43 3042510.69768472
44 3307986.45856947
45 3153004.587146
46 2795617.85515411
47 2136250.81385285
48 2287282.98990377
49 4167849.03199317
50 2971301.03140765
51 3921446.38514843
52 3782382.33049124
53 2170190.19030649
54 2206193.65302136
55 2158720.99863844
56 2040275.0560086
57 3644375.00720007
58 4011493.42088385
59 6096953.50194743
60 7064601.82303547
61 5144218.68767642
62 2767207.76146467
63 2234457.356348
64 2170290.75241437
65 1820249.21669804
66 1742083.73024891
67 1731752.61328364
68 1630910.2490352
69 1618756.81924232
70 1472845.02410038
71 1704947.383727
72 2803437.56455603
73 1973958.28363113
74 2330894.741921
75 2009199.68417525
76 1549890.57707243
77 1384106.19810937
78 1436964.51029405
79 1391649.56410332
80 1284995.43578754
81 1485210.42808339
82 1643147.74731137
83 1752580.84604355
84 1374996.04872073
85 1884917.12516795
86 1746080.49596359
87 1432146.91938598
88 1369912.81550229
89 1988106.6690747
90 2136416.3484932
91 2148372.40429351
92 1814696.15511319
93 1527141.27705308
94 1666052.64323002
95 1805798.5592818
96 1416536.73995068
97 1157550.12248053
98 1590462.72223883
99 1307117.51546335
100 1466083.72646004
101 1633900.31644726
102 1943594.41359681
103 2005827.28669708
104 1651817.19071959
105 3958118.46912934
106 5136255.01569268
107 3815979.02812778
108 3842673.83270983
109 2491667.70794794
110 4460530.77571751
111 4434575.6246253
112 2371831.16739132
113 2411385.52189616
114 2376848.10909736
115 5418366.67464636
116 9649965.56107295
117 4635634.65068756
118 5136031.38952198
119 2880608.00140485
120 3494165.93817339
121 4714874.9787833
122 4079398.24475713
123 6401915.05277008
124 12339053.1417716
125 7305034.99377738
126 10319842.8394841
127 7126237.5171012
128 6996577.57826913
129 8208296.65977568
130 5584902.73210995
131 4445154.63284783
132 4546088.84091157
133 4033928.2713101
134 7765392.61659981
135 13290473.5468912
136 6842693.72237758
137 7889128.20066047
138 7186240.2434083
139 8409819.68348522
140 10355423.2343793
141 8252386.77082822
142 7853715.07013867
143 8528314.14932212
144 10926916.6812138
145 10058718.0275423
146 10214879.948713
147 11743707.531887
148 14196491.8794839
149 12197634.625411
150 9256023.64847716
151 21610346.9993928
152 15712395.6831525
153 16869908.8899271
154 14019281.8485531
155 12272950.4462683
156 17459629.0984731
157 14976765.554702
158 9447583.31738906
159 7595020.11168568
160 7244850.10933867
161 7025626.01532164
162 6618734.04034122
163 7507924.0417813
164 6302388.79630156
165 6584564.95651218
166 9760720.46563324
167 10094599.7857546
168 14098168.2901059
169 23104678.9887904
170 18859079.6122961
171 21350501.1040146
172 21678961.4123353
173 21935071.1835384
174 29406406.6882252
175 22634113.2619574
176 22704898.9741285
177 20578518.4684799
178 22932293.3980214
179 33879443.335289
180 27502650.5605419
181 23012559.3408345
182 21724465.3927037
183 17969943.2769561
184 12049259.1800632
185 13022981.5158371
186 13927726.0554185
187 19180715.1948025
188 14954180.9622705
189 12371123.7221246
190 11395289.7094494
191 10371062.5531947
192 10837926.4992647
193 15266289.8351919
194 17588826.508124
195 15329037.2274309
196 17852272.0517497
197 15079100.3533849
198 18008072.1243269
199 21807011.3605691
200 17607605.0564492
201 19863833.4118281
202 20002965.0324689
203 21612643.2320905
204 21561211.4557563
205 16311532.5058479
206 18342450.8485463
207 16865824.2053881
208 20772594.1868633
209 18675499.5556948
210 28316255.8861119
211 17508132.1843205
212 25458873.3414471
213 26926893.8285132
214 28911927.3121327
215 24511000.8431349
216 23986525.0752304
217 31627378.7846685
218 20748996.0987056
219 21718881.1586193
220 35604175.4100394
221 42655110.1562589
222 29082075.7567452
223 21431750.3932708
224 19934197.0331458
225 18127844.9138457
226 14443264.3959176
227 12839419.4532014
228 15733550.9008249
229 20704138.7381817
230 23198780.8625109
231 23212532.3954736
232 20873170.2830999
233 13749350.3957616
234 14695946.0307301
235 14915535.7620647
236 14965353.8325148
237 16364069.5737344
238 25000041.9612734
239 37700772.578519
240 26254951.354144
241 19224465.6801539
242 24234438.1538962
243 28295755.2027793
244 26481032.0550915
245 22904420.80896
246 22948573.3227791
247 20414469.7216052
248 23136966.1552481
249 27190934.9314836
250 25532399.3302991
251 43947906.7470062
252 66767222.1307156
253 54635435.4502686
254 27210122.648439
255 29884669.8700173
256 30209828.1952439
257 24835300.7352778
258 31010501.7941556
259 24605246.2279667
260 32333191.0109222
261 30120780.2907996
262 24644278.6532865
263 37094318.5037047
264 40203136.8319307
265 30759755.9478505
266 23880010.4896208
267 25372818.7254138
268 18407516.1977322
269 14817284.3115973
270 17071487.726248
271 32939627.8004205
272 29913551.6845149
273 26045078.9028003
274 22232660.9620587
275 18580850.4179655
276 21268204.3278441
277 20412785.5964816
278 27248634.8304861
279 18658430.9368129
280 17909454.8465971
281 19160321.3607153
282 19373276.1740477
283 22978056.7436543
284 16545932.105681
285 14411187.8646251
286 13158783.807441
287 13165822.5165997
288 12895682.9502299
289 17424622.3575661
290 15453265.7706959
291 16395604.8810162
292 14597068.1111017
293 13397197.7744849
294 12348463.86287
295 9542576.57762315
296 10449441.45127
297 9375039.72707347
298 9370128.58162165
299 9950854.84494464
300 8829046.41081527
301 9044288.06905944
302 12976185.0680466
303 24320324.0034909
304 23490274.5603481
305 17517517.5080175
306 12454951.5772011
307 11239046.9734587
308 12490104.8978153
309 12581273.8406782
310 10168647.6732363
311 10900569.8435776
312 11229646.9542005
313 9901158.88159031
314 14499439.7469843
315 16080900.5855952
316 18133804.6799823
317 21978182.7211381
318 14256483.1367079
319 10726133.8940739
320 11462781.3887298
321 14261101.2702967
322 11013643.9805025
323 10911327.6595703
324 9148996.2914564
325 8440220.70886438
326 8608913.16840721
327 11366133.1590038
328 15792385.6039002
329 17958581.6254609
330 11669686.9490569
331 9657166.77658437
332 9030583.85481668
333 10482960.2814032
334 10200640.5399405
335 9304149.31024398
336 8608966.68211998
337 8342295.42721414
338 15461819.6287262
339 16183805.946172
340 9959221.17877958
341 9862344.63798509
342 10672169.3682607
343 14493235.8577101
344 8989597.00709746
345 7130656.87893723
346 7701670.69103186
347 7606712.53644355
348 15492279.0737609
349 17906967.1340456
350 18283248.0579907
351 19104772.1744042
352 21932117.4916859
353 16485208.0063793
354 15401832.2990553
355 14296767.7418352
356 21279424.0463911
357 24410793.7851165
358 16787302.1117872
359 13401312.8271559
360 12848201.5166925
361 11986417.7005987
362 12361185.5523782
363 9270781.12218172
364 8186290.73779104
365 7456075.23648666
366 6608857.72557701
367 6353937.30758582
368 5819086.55135636
369 4409864.5992719
370 4177808.24906268
371 3671825.48218606
372 4748482.89888408
373 5107663.65536728
374 4748172.72948324
375 6040003.09607141
376 11646338.1951229
377 10087409.7725359
378 15125433.2033337
379 10350397.2867323
380 8533435.51486708
381 5510883.08612065
382 5782323.52905837
383 4536404.51064012
384 4706043.77911151
385 3745112.25545179
386 3710607.97515886
387 3518527.37086626
388 3249731.06599588
389 3202323.26701014
390 3158972.04397173
391 2650725.46466819
392 2801758.85354954
393 2399380.81359136
394 3072032.23641454
395 2739373.60959914
396 3212076.94336684
397 2147750.66743433
398 2153353.59065654
399 2595524.19832103
400 2260078.75834455
401 2649259.65865974
402 2315614.86396452
403 2412562.38681832
404 2814204.30974039
405 2874105.08628668
406 2490250.31890561
407 2258460.4126195
408 2067826.8410591
409 2780769.0874866
410 2678979.87938628
411 2657700.88819395
412 2859977.59763358
413 2599068.83675615
414 2348091.23576721
415 2444429.14962575
416 2446622.21113604
417 2074785.25112385
418 1762580.55606247
419 1862162.87622667
420 1784952.86244942
421 2058947.71190814
422 1892510.37292223
423 1828710.66211916
424 1785631.23256978
425 1631599.59165952
426 1775645.56377612
427 1641542.84688509
428 2399552.2487258
429 1832891.61257615
430 1517379.41010488
431 1858803.33354891
432 1672913.7647959
433 1518716.21056749
434 1534931.1238448
435 1540388.50403076
436 1919667.44575147
437 1856968.49570591
438 1454597.48933966
439 2396299.68286054
440 1606874.44519124
};
\addlegendentry{Real Observations}
\addlegendentry{Prediction mean}
\addlegendentry{Prediction median}
\addlegendimage{line width=5pt,draw=blue,opacity=0.5}
\addlegendentry{$95\%$ CI}
\end{axis}

\end{tikzpicture}

	\end{figure}
	\vspace{-1.5em}
\begin{itemize}
	\item The predictive variance dynamically adapts to the complexity of uncertainty.
	\item For the ChdGP Gamma setting, the predictive distribution does not accommodate negative values.
	\item The presence of peaks results in a discrepancy between the mean and median.
\end{itemize}

\end{frame}

%\begin{frame}{Tuning Q for Chd Gamma}
%	\begin{figure}[htbp]
%		\tiny
%		\centering
%		\setlength\figurewidth{\columnwidth} 
%		\setlength\figureheight{0.55\columnwidth}
%		% This file was created with tikzplotlib v0.10.1.
\begin{tikzpicture}

\definecolor{darkgray176}{RGB}{176,176,176}
\definecolor{steelblue31119180}{RGB}{31,119,180}

\begin{axis}[
width=\figurewidth,
height=\figureheight,
axis background/.style={fill=background_color},
axis line style={white},
tick align=inside,
tick pos=left,
x grid style={white},
y grid style={white},
xmajorgrids,
ymajorgrids,
xlabel={Number of Independent GPs (Q)},
xmin=-1.25, xmax=48.25,
xtick style={color=black},
ylabel={NLPD Value},
ymin=-45.0867325502887, ymax=-27.8931730372172,
ytick style={color=black}
]
\addplot [semithick, steelblue31119180, mark=*, mark size=3, mark options={solid}]
table {%
1 -28.6746984696296
2 -31.5889426660672
3 -35.3989366424205
4 -39.6304035471465
5 -40.0310986157881
6 -40.4071082856604
7 -41.3861955215952
8 -41.6396034090683
9 -41.8829049685177
10 -42.1333303384224
11 -42.3247710795398
12 -42.6582399844266
13 -42.9183897815454
14 -43.3099777906844
15 -42.0125382432448
16 -42.687482466638
17 -43.2491593304651
18 -43.6833759873654
19 -43.5565869259225
20 -43.8869765181748
21 -43.6589353753988
22 -43.8472080760075
23 -43.615007488357
24 -43.9875811235743
25 -43.4078655642594
26 -44.3052071178764
27 -43.1673148197223
28 -43.6882103663044
29 -43.8654896325373
30 -43.5720578619609
31 -43.5841539634015
32 -42.5809560073391
33 -43.9628588215701
34 -44.1906731214213
35 -43.8584591730126
36 -43.9204600977339
37 -44.1078089721578
38 -43.8371183156964
39 -43.9069257783752
40 -44.0969023557631
41 -43.9413303854096
42 -43.4136441335025
43 -44.0187249225636
44 -43.4461289003536
45 -43.4625600301139
46 -43.6619351462037
};
\end{axis}

\end{tikzpicture}

%	\end{figure}
%	\vspace{-1.0em}
%	\begin{block}{}
%		NLPD metric for ChdGP Gamma models as a function of the number of independent GPs. The tuning step is more stable as $Q$ increases. We select $Q = 26$ as the optimal value.
%	\end{block}
%\end{frame}

\subsection{Performance Analysis}

%\begin{frame}{ChdnGP Gamma vs ChdGP Normal}
%	\begin{figure}[htbp]
%		\tiny
%		\centering
%		\setlength\figurewidth{\columnwidth} 
%		\setlength\figureheight{0.55\columnwidth}
%		% This file was created with tikzplotlib v0.10.1.
\begin{tikzpicture}

\definecolor{darkgray176}{RGB}{176,176,176}

\begin{axis}[
width=\figurewidth,%
height=\figureheight,%
axis background/.style={fill=background_color},
tick align=outside,
tick pos=left,
axis line style={white},
tick align=inside,
tick pos=left,
x grid style={white},
y grid style={white},
xmajorgrids,
ymajorgrids,
xlabel={Horizon},
xmin=-0.85, xmax=9.6,
xtick style={color=black},
xtick={0,1,2,3,4,5,6,7,8,9},
xticklabels={1,2,3,4,5,6,7,14,21,30},
ylabel={NLPD Value},
ymin=-46.5136563487104, ymax=0,
ytick style={color=black},
legend columns=2, 
legend style={
  nodes={scale=0.8, transform shape},
  fill opacity=1,
  draw opacity=1,
  text opacity=1,
  at={(0.03,0.97)},
  anchor=north west,
  legend pos=north west,
  draw=lightgray204,
  fill=background_color
}
]
\draw[draw=none,fill=blue] (axis cs:-0.375,0) rectangle (axis cs:-0.125,-37.1976418206135);
\addlegendimage{ybar,ybar legend,draw=none,fill=blue}
\addlegendentry{ChdNormal}

\draw[draw=none,fill=blue] (axis cs:0.625,0) rectangle (axis cs:0.875,-32.57009392034);
\draw[draw=none,fill=blue] (axis cs:1.625,0) rectangle (axis cs:1.875,-31.1545299186111);
\draw[draw=none,fill=blue] (axis cs:2.625,0) rectangle (axis cs:2.875,-30.7281603098309);
\draw[draw=none,fill=blue] (axis cs:3.625,0) rectangle (axis cs:3.875,-28.7134159542646);
\draw[draw=none,fill=blue] (axis cs:4.625,0) rectangle (axis cs:4.875,-29.5248659053096);
\draw[draw=none,fill=blue] (axis cs:5.625,0) rectangle (axis cs:5.875,-29.6611175531187);
\draw[draw=none,fill=blue] (axis cs:6.625,0) rectangle (axis cs:6.875,-28.9486441224176);
\draw[draw=none,fill=blue] (axis cs:7.625,0) rectangle (axis cs:7.875,-27.8247484458317);
\draw[draw=none,fill=blue] (axis cs:8.625,0) rectangle (axis cs:8.875,-26.5153492416896);
\draw[draw=none,fill=red] (axis cs:-0.125,0) rectangle (axis cs:0.125,-44.2987203321051);
\addlegendimage{ybar,ybar legend,draw=none,fill=red}
\addlegendentry{ChdGamma}

\draw[draw=none,fill=red] (axis cs:0.875,0) rectangle (axis cs:1.125,-40.5759976099418);
\draw[draw=none,fill=red] (axis cs:1.875,0) rectangle (axis cs:2.125,-39.2559882886404);
\draw[draw=none,fill=red] (axis cs:2.875,0) rectangle (axis cs:3.125,-38.2783040944686);
\draw[draw=none,fill=red] (axis cs:3.875,0) rectangle (axis cs:4.125,-38.1712370167515);
\draw[draw=none,fill=red] (axis cs:4.875,0) rectangle (axis cs:5.125,-37.2277238201093);
\draw[draw=none,fill=red] (axis cs:5.875,0) rectangle (axis cs:6.125,-36.6801100288926);
\draw[draw=none,fill=red] (axis cs:6.875,0) rectangle (axis cs:7.125,-35.2829648913755);
\draw[draw=none,fill=red] (axis cs:7.875,0) rectangle (axis cs:8.125,-33.5852389481076);
\draw[draw=none,fill=red] (axis cs:8.875,0) rectangle (axis cs:9.125,-32.2988855952842);
\end{axis}

\end{tikzpicture}

%	\end{figure}
%	\vspace{-1.0em}
%	\begin{block}{}
%		Comparison of NLPD metric across different prediction horizons for the ChdGP Normal, and ChdGP Gamma models. Gamma likelihood better explains data, providing better performance.
%	\end{block}
%\end{frame}

%\begin{frame}{ChdGP Gamma Forecasting}
%	\centering
%	\begin{figure}[htbp]
%		\tiny
%		\setlength\figurewidth{0.55\columnwidth} 
%		\setlength\figureheight{0.5\columnwidth}
%		
%%		\subfloat[$T.$]{% This file was created with tikzplotlib v0.10.1.
\begin{tikzpicture}

\definecolor{darkgray176}{RGB}{176,176,176}

\begin{axis}[
width=\figurewidth,
height=\figureheight,
axis background/.style={fill=background_color},
axis line style={white},
tick align=inside,
tick pos=left,
x grid style={white},
y grid style={white},
xmajorgrids,
ymajorgrids,
ylabel={Contributions (kWh)},
xmin=100, xmax=250,
xtick style={color=black},
ymin=17652.106574535, ymax=10568877.0323536,
ytick style={color=black},
ytick={0,2000000,4000000,6000000,8000000,10000000,12000000},
yticklabels={0.0,0.2,0.4,0.6,0.8,1.0,1.2}
]
\path [draw=blue, fill=blue, opacity=0.5]
(axis cs:0,2353569.05305084)
--(axis cs:0,802043.088262047)
--(axis cs:1,1133896.86041405)
--(axis cs:2,1962006.49977423)
--(axis cs:3,1101585.20987692)
--(axis cs:4,1016107.06009701)
--(axis cs:5,1465351.09565576)
--(axis cs:6,1643519.15951113)
--(axis cs:7,1439086.54801468)
--(axis cs:8,1997642.2329274)
--(axis cs:9,2441702.42880781)
--(axis cs:10,2017004.47747904)
--(axis cs:11,1565649.43703521)
--(axis cs:12,1511926.88323874)
--(axis cs:13,1849211.31226925)
--(axis cs:14,3253119.32879309)
--(axis cs:15,2787272.49256317)
--(axis cs:16,2538220.31759326)
--(axis cs:17,2138683.23437327)
--(axis cs:18,2173301.13009361)
--(axis cs:19,1971377.46340673)
--(axis cs:20,4007710.79889275)
--(axis cs:21,2968762.00403191)
--(axis cs:22,2926382.01305151)
--(axis cs:23,2253142.22881515)
--(axis cs:24,2212006.83335931)
--(axis cs:25,2040265.72836993)
--(axis cs:26,2616865.56374616)
--(axis cs:27,2464688.31651494)
--(axis cs:28,2504296.65962207)
--(axis cs:29,2449749.35845524)
--(axis cs:30,2434896.26870693)
--(axis cs:31,2044456.86106993)
--(axis cs:32,2005615.36672225)
--(axis cs:33,2425286.26990595)
--(axis cs:34,2238202.56266918)
--(axis cs:35,2242170.38501584)
--(axis cs:36,2333820.53732151)
--(axis cs:37,3005422.55977868)
--(axis cs:38,3099428.88390591)
--(axis cs:39,2760598.22334915)
--(axis cs:40,2259296.55168866)
--(axis cs:41,2526523.39414991)
--(axis cs:42,3016927.21116361)
--(axis cs:43,3046769.22359206)
--(axis cs:44,4834635.23288425)
--(axis cs:45,4860157.99314783)
--(axis cs:46,4399661.02566162)
--(axis cs:47,4061050.74151983)
--(axis cs:48,4638627.73074679)
--(axis cs:49,4154745.38208088)
--(axis cs:50,4415981.45454836)
--(axis cs:51,3761789.08658812)
--(axis cs:52,4584163.98983951)
--(axis cs:53,4814911.3079452)
--(axis cs:54,4411627.30805063)
--(axis cs:55,4285397.32620291)
--(axis cs:56,4853337.39092129)
--(axis cs:57,4552500.30745793)
--(axis cs:58,3894902.03137256)
--(axis cs:59,3562214.32854569)
--(axis cs:60,3756917.72424492)
--(axis cs:61,3832030.6806842)
--(axis cs:62,2840197.21555738)
--(axis cs:63,2628594.02153074)
--(axis cs:64,2277638.09035959)
--(axis cs:65,2246272.11911496)
--(axis cs:66,2269720.26702364)
--(axis cs:67,1848968.3962837)
--(axis cs:68,1836350.20240557)
--(axis cs:69,1817839.59038901)
--(axis cs:70,1736487.04958208)
--(axis cs:71,1890708.86070975)
--(axis cs:72,1855475.26858533)
--(axis cs:73,1814264.09789649)
--(axis cs:74,1814269.92911417)
--(axis cs:75,1764142.38860382)
--(axis cs:76,1749210.19911271)
--(axis cs:77,1709146.00736366)
--(axis cs:78,1597653.7581863)
--(axis cs:79,1569907.71484365)
--(axis cs:80,1384743.60428506)
--(axis cs:81,1295963.91133409)
--(axis cs:82,1319400.05982832)
--(axis cs:83,1347300.95912086)
--(axis cs:84,1248679.15403309)
--(axis cs:85,1136773.94320884)
--(axis cs:86,1075470.37749359)
--(axis cs:87,1149391.9774025)
--(axis cs:88,1178569.17836443)
--(axis cs:89,1580883.13682119)
--(axis cs:90,2649771.07758134)
--(axis cs:91,1865941.58674619)
--(axis cs:92,1746273.66874807)
--(axis cs:93,1995606.69810567)
--(axis cs:94,1815168.04740455)
--(axis cs:95,2462613.65330711)
--(axis cs:96,2169652.59763446)
--(axis cs:97,2063136.80917023)
--(axis cs:98,2914554.86108096)
--(axis cs:99,3038740.40855001)
--(axis cs:100,2685574.62418431)
--(axis cs:101,2977426.29484135)
--(axis cs:102,3593137.75712743)
--(axis cs:103,3061200.81431605)
--(axis cs:104,2388835.2382192)
--(axis cs:105,2385974.43163139)
--(axis cs:106,2261693.72197625)
--(axis cs:107,2326530.85173039)
--(axis cs:108,2546512.16588046)
--(axis cs:109,4035600.39074749)
--(axis cs:110,4259783.04389859)
--(axis cs:111,3689406.43398128)
--(axis cs:112,3787217.7943292)
--(axis cs:113,3557354.6909016)
--(axis cs:114,3354787.99926935)
--(axis cs:115,3169351.15738788)
--(axis cs:116,3151214.41701289)
--(axis cs:117,2949219.80185172)
--(axis cs:118,3883028.07328012)
--(axis cs:119,3485745.00061569)
--(axis cs:120,3786447.00365266)
--(axis cs:121,3354427.36079137)
--(axis cs:122,3187944.57408428)
--(axis cs:123,3784559.82301468)
--(axis cs:124,3321177.24779833)
--(axis cs:125,3814167.9311921)
--(axis cs:126,4040253.27782502)
--(axis cs:127,4609688.49256991)
--(axis cs:128,3772904.40496943)
--(axis cs:129,3264032.83278041)
--(axis cs:130,4094504.85808034)
--(axis cs:131,4064484.94411786)
--(axis cs:132,3403368.90855542)
--(axis cs:133,3246706.2574608)
--(axis cs:134,3094811.47064851)
--(axis cs:135,3007283.80891847)
--(axis cs:136,2687621.98740057)
--(axis cs:137,2429436.93824351)
--(axis cs:138,2188393.9763332)
--(axis cs:139,2237013.83555784)
--(axis cs:140,1913024.80639519)
--(axis cs:141,2054184.33792889)
--(axis cs:142,2123037.65882294)
--(axis cs:143,2085244.91752095)
--(axis cs:144,2023602.15770191)
--(axis cs:145,1931474.54354634)
--(axis cs:146,1960234.30369739)
--(axis cs:147,1967763.41434677)
--(axis cs:148,2072125.76851022)
--(axis cs:149,1879003.79125326)
--(axis cs:150,1909783.61468378)
--(axis cs:151,1987274.55261)
--(axis cs:152,1634743.44411326)
--(axis cs:153,1406530.81535513)
--(axis cs:154,1810003.13138395)
--(axis cs:155,1747910.68034691)
--(axis cs:156,1668970.82751433)
--(axis cs:157,1588633.33426124)
--(axis cs:158,1832687.86370499)
--(axis cs:159,1731317.41939635)
--(axis cs:160,2159080.19152239)
--(axis cs:161,1935145.36015397)
--(axis cs:162,1824127.63555045)
--(axis cs:163,1873944.87688525)
--(axis cs:164,1669503.10458749)
--(axis cs:165,1964778.0419391)
--(axis cs:166,2472681.21011206)
--(axis cs:167,3551629.4330704)
--(axis cs:168,3508001.97750412)
--(axis cs:169,4131777.37916604)
--(axis cs:170,3722715.82703588)
--(axis cs:171,3888318.6818357)
--(axis cs:172,3801876.42019978)
--(axis cs:173,2803780.62656314)
--(axis cs:174,2749461.06711268)
--(axis cs:175,2887385.12465962)
--(axis cs:176,2864328.90938647)
--(axis cs:177,3334279.17644125)
--(axis cs:178,2772014.74875635)
--(axis cs:179,2620727.42688429)
--(axis cs:180,2801178.20249558)
--(axis cs:181,2831390.90004925)
--(axis cs:182,2556440.00427169)
--(axis cs:183,2328692.44272313)
--(axis cs:184,2760578.41973652)
--(axis cs:185,3563355.49932809)
--(axis cs:186,3162651.97465747)
--(axis cs:187,2805944.59804594)
--(axis cs:188,2769256.23991523)
--(axis cs:189,2467737.08304162)
--(axis cs:190,2141797.49693043)
--(axis cs:191,2269010.57028926)
--(axis cs:192,2013894.74314419)
--(axis cs:193,2102439.63452594)
--(axis cs:194,1965305.49127119)
--(axis cs:195,2119642.45847018)
--(axis cs:196,2247104.01202614)
--(axis cs:197,2463920.61586014)
--(axis cs:198,2282792.3920562)
--(axis cs:199,2572256.08090466)
--(axis cs:200,2672331.47481145)
--(axis cs:201,2474665.70672965)
--(axis cs:202,2096789.46295124)
--(axis cs:203,2846065.65207018)
--(axis cs:204,2151006.31437323)
--(axis cs:205,2249071.01341853)
--(axis cs:206,2079829.74628361)
--(axis cs:207,2538572.72295561)
--(axis cs:208,3516291.61006041)
--(axis cs:209,4378858.02834717)
--(axis cs:210,3545327.59574542)
--(axis cs:211,3457796.64588893)
--(axis cs:212,3118406.76889159)
--(axis cs:213,3180970.08664108)
--(axis cs:214,3779266.85341372)
--(axis cs:215,4723435.19475007)
--(axis cs:216,3570783.07570567)
--(axis cs:217,2848569.57109)
--(axis cs:218,2563817.87820374)
--(axis cs:219,2583903.72425228)
--(axis cs:220,2682832.6523229)
--(axis cs:221,2446059.84960023)
--(axis cs:222,2113309.04281349)
--(axis cs:223,2102025.81948313)
--(axis cs:224,2127560.25022746)
--(axis cs:225,2388649.26321192)
--(axis cs:226,2034736.07463745)
--(axis cs:227,2087789.94033544)
--(axis cs:228,2090409.83536592)
--(axis cs:229,1890407.64659241)
--(axis cs:230,1893308.35273995)
--(axis cs:231,2360738.13442432)
--(axis cs:232,2043217.34600996)
--(axis cs:233,1538137.52119804)
--(axis cs:234,1388827.89215214)
--(axis cs:235,1135183.15325172)
--(axis cs:236,1100354.16891299)
--(axis cs:237,1268212.23199053)
--(axis cs:238,1267001.30545753)
--(axis cs:239,1365923.54199546)
--(axis cs:240,1472349.35139913)
--(axis cs:241,1400602.82857305)
--(axis cs:242,1720519.67940361)
--(axis cs:243,1614842.08344564)
--(axis cs:244,1811817.01728318)
--(axis cs:245,1287655.18535878)
--(axis cs:246,1169133.90455361)
--(axis cs:247,905634.126628061)
--(axis cs:248,1386866.22809605)
--(axis cs:249,1949888.31878804)
--(axis cs:250,1839684.97860627)
--(axis cs:251,1747820.54502123)
--(axis cs:252,2710071.10257556)
--(axis cs:253,2735467.38262596)
--(axis cs:254,2123679.90582571)
--(axis cs:255,1570594.40099274)
--(axis cs:256,1441654.05074003)
--(axis cs:257,1544943.93977522)
--(axis cs:258,1195822.21971833)
--(axis cs:259,1183187.60886348)
--(axis cs:260,839014.52767637)
--(axis cs:261,825031.395000137)
--(axis cs:262,641031.641633696)
--(axis cs:263,823714.541214444)
--(axis cs:264,1768376.95708905)
--(axis cs:265,1120515.63229709)
--(axis cs:266,815448.864167471)
--(axis cs:267,666521.394821534)
--(axis cs:268,583571.03861766)
--(axis cs:269,529988.23577069)
--(axis cs:270,497253.239564491)
--(axis cs:271,1078762.41614193)
--(axis cs:272,610996.42441486)
--(axis cs:273,983066.103076079)
--(axis cs:274,1283407.97219672)
--(axis cs:275,1117418.44515237)
--(axis cs:276,1042549.1780657)
--(axis cs:277,1096825.8490109)
--(axis cs:278,1040231.74550268)
--(axis cs:279,1308762.45973364)
--(axis cs:280,1681279.74032962)
--(axis cs:281,1247452.29490531)
--(axis cs:282,1259373.36619673)
--(axis cs:283,920755.024089371)
--(axis cs:284,965434.999947746)
--(axis cs:285,1100594.20659604)
--(axis cs:286,2274393.04771928)
--(axis cs:287,2218956.75940168)
--(axis cs:288,1444026.55365564)
--(axis cs:289,1647659.83773432)
--(axis cs:290,1087941.65157041)
--(axis cs:291,1042788.46754848)
--(axis cs:292,1198618.15445029)
--(axis cs:293,1109664.14275757)
--(axis cs:294,903257.205415106)
--(axis cs:295,984493.43878852)
--(axis cs:296,876811.344542451)
--(axis cs:297,669316.780811685)
--(axis cs:298,1303172.68386289)
--(axis cs:299,1339679.20920302)
--(axis cs:300,1220900.1143654)
--(axis cs:301,804203.613608989)
--(axis cs:302,858613.32335773)
--(axis cs:303,1004534.47023435)
--(axis cs:304,1183128.58550608)
--(axis cs:305,951121.90979442)
--(axis cs:306,842063.507166397)
--(axis cs:307,760869.125421047)
--(axis cs:308,759912.63795092)
--(axis cs:309,636515.513857088)
--(axis cs:310,733058.354646214)
--(axis cs:311,620444.37847061)
--(axis cs:312,642226.747625207)
--(axis cs:313,648368.481691717)
--(axis cs:314,821611.986102081)
--(axis cs:315,793407.682649741)
--(axis cs:316,888868.95905414)
--(axis cs:317,624807.990702815)
--(axis cs:318,577291.75781622)
--(axis cs:319,560411.202202899)
--(axis cs:320,629498.990509696)
--(axis cs:321,647847.672978747)
--(axis cs:322,1404756.43700515)
--(axis cs:323,1041639.72001688)
--(axis cs:324,786842.484301609)
--(axis cs:325,675688.933210033)
--(axis cs:326,841323.688286196)
--(axis cs:327,1169782.49009885)
--(axis cs:328,1269341.94254173)
--(axis cs:329,1440966.54083372)
--(axis cs:330,1657348.8889454)
--(axis cs:331,2328892.65193496)
--(axis cs:332,2180028.54235894)
--(axis cs:333,2665483.33333377)
--(axis cs:334,2295269.09903882)
--(axis cs:335,1927415.4863318)
--(axis cs:336,1643550.00766032)
--(axis cs:337,1712321.87291868)
--(axis cs:338,2110304.54444737)
--(axis cs:339,2250620.86493541)
--(axis cs:340,2032812.56574935)
--(axis cs:341,1766545.03945982)
--(axis cs:342,1811590.98876062)
--(axis cs:343,2802753.91071709)
--(axis cs:344,2524502.97844534)
--(axis cs:345,2689467.88517556)
--(axis cs:346,2565600.9738698)
--(axis cs:347,2202877.08284116)
--(axis cs:348,2236784.65258837)
--(axis cs:349,2780925.37897009)
--(axis cs:350,2483494.25703635)
--(axis cs:351,2485760.16972899)
--(axis cs:352,2491101.33702162)
--(axis cs:353,3247658.93305827)
--(axis cs:354,2339940.4911763)
--(axis cs:355,2861902.38974711)
--(axis cs:356,3099071.20581477)
--(axis cs:357,3486641.65644107)
--(axis cs:358,2911757.40669711)
--(axis cs:359,2561820.10436079)
--(axis cs:360,2326986.76484087)
--(axis cs:361,2230313.93411565)
--(axis cs:362,3678051.66617926)
--(axis cs:363,3082536.96487204)
--(axis cs:364,4927272.25285533)
--(axis cs:365,3543715.78721319)
--(axis cs:366,5037243.38265505)
--(axis cs:367,4359110.70917192)
--(axis cs:368,4247411.3877068)
--(axis cs:369,4021776.68141925)
--(axis cs:370,3264162.25588344)
--(axis cs:371,3342373.82617603)
--(axis cs:372,4506985.81216198)
--(axis cs:373,4440010.99046578)
--(axis cs:374,4324139.45996662)
--(axis cs:375,4166385.60442633)
--(axis cs:376,4554539.80873066)
--(axis cs:377,5072343.41007712)
--(axis cs:378,4999689.22195949)
--(axis cs:379,4728126.91030765)
--(axis cs:380,4535718.00335929)
--(axis cs:381,4006918.30032763)
--(axis cs:382,4652422.05396134)
--(axis cs:383,4496703.9939513)
--(axis cs:384,4047744.7749847)
--(axis cs:385,3570150.48616532)
--(axis cs:386,3638177.08134121)
--(axis cs:387,3639228.13316702)
--(axis cs:388,3274434.50439693)
--(axis cs:389,3284331.35511005)
--(axis cs:390,2788193.5113516)
--(axis cs:391,2666851.59949514)
--(axis cs:392,2758818.69310713)
--(axis cs:393,2797218.38548483)
--(axis cs:394,4037058.96322589)
--(axis cs:395,4701314.79552618)
--(axis cs:396,5622005.89079453)
--(axis cs:397,4352520.65884295)
--(axis cs:398,3918827.07457713)
--(axis cs:399,3375444.65685013)
--(axis cs:400,3088395.68222025)
--(axis cs:401,3340292.54919902)
--(axis cs:402,3157245.23929569)
--(axis cs:403,3273631.9713812)
--(axis cs:404,3671427.01797963)
--(axis cs:405,3934490.68088973)
--(axis cs:406,3514238.65462793)
--(axis cs:407,2923764.14683351)
--(axis cs:408,2753780.08147005)
--(axis cs:409,2959805.16940487)
--(axis cs:410,3212389.92639351)
--(axis cs:411,2894294.37099766)
--(axis cs:412,3171160.19671119)
--(axis cs:413,3302699.07907083)
--(axis cs:414,3595123.52253549)
--(axis cs:415,4129884.24863144)
--(axis cs:416,4112717.38585515)
--(axis cs:417,4512401.12613898)
--(axis cs:418,3740349.53989663)
--(axis cs:419,4635756.73440996)
--(axis cs:420,5342513.76497685)
--(axis cs:421,3188725.75644313)
--(axis cs:422,3881790.41756512)
--(axis cs:423,3886625.45540775)
--(axis cs:424,3995666.4055166)
--(axis cs:425,4767765.56477407)
--(axis cs:426,3853054.26268945)
--(axis cs:427,3891122.11476311)
--(axis cs:428,3794642.3215538)
--(axis cs:429,3168130.44002361)
--(axis cs:430,2905263.98977809)
--(axis cs:431,2803895.30411218)
--(axis cs:432,2567846.25288374)
--(axis cs:433,2273909.13087018)
--(axis cs:434,2098089.89234832)
--(axis cs:435,2190101.17111126)
--(axis cs:436,2358958.94875246)
--(axis cs:437,2532644.23753598)
--(axis cs:438,2439619.69003994)
--(axis cs:439,3274624.39949802)
--(axis cs:440,2280838.85079139)
--(axis cs:440,4721532.49983522)
--(axis cs:440,4721532.49983522)
--(axis cs:439,6469768.96779518)
--(axis cs:438,4917198.17653463)
--(axis cs:437,5078585.42004206)
--(axis cs:436,4820835.38905284)
--(axis cs:435,4441349.70807369)
--(axis cs:434,4307305.97831638)
--(axis cs:433,4546455.10364137)
--(axis cs:432,5048800.13241089)
--(axis cs:431,5479164.60301938)
--(axis cs:430,5602111.22545459)
--(axis cs:429,6060104.12654445)
--(axis cs:428,7210826.25106682)
--(axis cs:427,7215670.02492537)
--(axis cs:426,7138750.5417699)
--(axis cs:425,8607452.70391671)
--(axis cs:424,7421769.69189588)
--(axis cs:423,7262929.00658989)
--(axis cs:422,7382244.21450458)
--(axis cs:421,6368141.41277575)
--(axis cs:420,9735347.85483628)
--(axis cs:419,8594544.36826793)
--(axis cs:418,7065423.03245759)
--(axis cs:417,8325984.51434524)
--(axis cs:416,7677401.21387859)
--(axis cs:415,7765403.21792828)
--(axis cs:414,6919420.57093015)
--(axis cs:413,6439267.01912799)
--(axis cs:412,6137170.56098779)
--(axis cs:411,5736464.45688248)
--(axis cs:410,6147965.23167093)
--(axis cs:409,5888872.4935641)
--(axis cs:408,5453437.67493608)
--(axis cs:407,5765766.42856694)
--(axis cs:406,6745621.70237147)
--(axis cs:405,7512198.53319003)
--(axis cs:404,7125590.09682175)
--(axis cs:403,6448485.82303085)
--(axis cs:402,6237143.93109138)
--(axis cs:401,6543436.45188883)
--(axis cs:400,6002333.49273761)
--(axis cs:399,6440321.12283934)
--(axis cs:398,7247956.39381407)
--(axis cs:397,7984279.53756827)
--(axis cs:396,10089275.8993636)
--(axis cs:395,8857957.91527688)
--(axis cs:394,7650060.60023192)
--(axis cs:393,5608410.36887043)
--(axis cs:392,5457408.87531996)
--(axis cs:391,5320348.69074556)
--(axis cs:390,5507809.94398785)
--(axis cs:389,6334434.40656165)
--(axis cs:388,6325635.47086405)
--(axis cs:387,6888405.56526865)
--(axis cs:386,6994809.99939713)
--(axis cs:385,6880651.23201479)
--(axis cs:384,7771910.58891213)
--(axis cs:383,8571330.0675784)
--(axis cs:382,8551059.90661898)
--(axis cs:381,7561881.57601506)
--(axis cs:380,8728330.43676457)
--(axis cs:379,9203271.50277873)
--(axis cs:378,9551772.0831042)
--(axis cs:377,9754345.11769478)
--(axis cs:376,8653215.59057293)
--(axis cs:375,7994448.38612307)
--(axis cs:374,7974255.07183612)
--(axis cs:373,8358430.00490134)
--(axis cs:372,8400602.36582442)
--(axis cs:371,6507451.50226679)
--(axis cs:370,6312562.29690468)
--(axis cs:369,7533121.83244217)
--(axis cs:368,7924957.21671065)
--(axis cs:367,8290264.63098716)
--(axis cs:366,9398749.17107679)
--(axis cs:365,6808111.40317626)
--(axis cs:364,9162740.90490084)
--(axis cs:363,6163277.98468353)
--(axis cs:362,7221081.8671303)
--(axis cs:361,4703694.84868515)
--(axis cs:360,4902576.98098366)
--(axis cs:359,5327518.4190104)
--(axis cs:358,5920833.71664415)
--(axis cs:357,7145787.03169146)
--(axis cs:356,6250828.89083569)
--(axis cs:355,5896530.01174216)
--(axis cs:354,4994700.82006355)
--(axis cs:353,6582143.78715915)
--(axis cs:352,5180629.93857587)
--(axis cs:351,5131728.64617878)
--(axis cs:350,5200464.57550408)
--(axis cs:349,5873424.07439452)
--(axis cs:348,4746285.1907409)
--(axis cs:347,4584758.16488278)
--(axis cs:346,5220813.54347881)
--(axis cs:345,5514976.25709282)
--(axis cs:344,5137660.11969795)
--(axis cs:343,5724770.2028508)
--(axis cs:342,3974054.7976794)
--(axis cs:341,3962238.2342457)
--(axis cs:340,4420252.59217983)
--(axis cs:339,4748370.72578171)
--(axis cs:338,4653792.57961268)
--(axis cs:337,3869954.77768971)
--(axis cs:336,3833854.08718432)
--(axis cs:335,4173820.14485743)
--(axis cs:334,4807362.38767075)
--(axis cs:333,5457980.56158424)
--(axis cs:332,4599603.65471981)
--(axis cs:331,4888922.65292292)
--(axis cs:330,3762208.25684704)
--(axis cs:329,3526110.5640661)
--(axis cs:328,3128999.00521965)
--(axis cs:327,2910699.03391259)
--(axis cs:326,2280017.48948558)
--(axis cs:325,2038263.49338565)
--(axis cs:324,2249544.93768561)
--(axis cs:323,2736327.91337973)
--(axis cs:322,3452643.60245685)
--(axis cs:321,1924767.9559989)
--(axis cs:320,1907982.6763931)
--(axis cs:319,1736567.04235201)
--(axis cs:318,1776802.10885314)
--(axis cs:317,1895734.63942994)
--(axis cs:316,2464529.74176046)
--(axis cs:315,2487764.86179707)
--(axis cs:314,2295918.82873322)
--(axis cs:313,1962533.82102881)
--(axis cs:312,1895274.45421676)
--(axis cs:311,1850572.14088681)
--(axis cs:310,2004312.89719113)
--(axis cs:309,1901939.98343032)
--(axis cs:308,2164792.91310385)
--(axis cs:307,2242504.08282279)
--(axis cs:306,2365862.61031319)
--(axis cs:305,2500345.24595026)
--(axis cs:304,2990148.96003798)
--(axis cs:303,2712975.43870404)
--(axis cs:302,2386764.82584722)
--(axis cs:301,2224649.88039686)
--(axis cs:300,3106317.35346975)
--(axis cs:299,3299034.00231156)
--(axis cs:298,3355582.80896308)
--(axis cs:297,2116218.41907739)
--(axis cs:296,2412357.06137472)
--(axis cs:295,2567436.87598354)
--(axis cs:294,2443212.9174454)
--(axis cs:293,2797623.90045108)
--(axis cs:292,3076055.03529743)
--(axis cs:291,2808368.38485195)
--(axis cs:290,2747854.11549359)
--(axis cs:289,3798387.276901)
--(axis cs:288,3334740.13671064)
--(axis cs:287,4755432.80848493)
--(axis cs:286,4910098.38130225)
--(axis cs:285,2880734.54185321)
--(axis cs:284,2653462.90449764)
--(axis cs:283,2534980.88441341)
--(axis cs:282,3179909.41984954)
--(axis cs:281,3228498.84066556)
--(axis cs:280,3914846.13366101)
--(axis cs:279,3346349.38233261)
--(axis cs:278,2921375.97224467)
--(axis cs:277,2893185.28296916)
--(axis cs:276,2782164.2672126)
--(axis cs:275,2906033.56287157)
--(axis cs:274,3169444.22400672)
--(axis cs:273,2715023.92854212)
--(axis cs:272,1938186.92549137)
--(axis cs:271,2829240.54886726)
--(axis cs:270,1668113.99835645)
--(axis cs:269,1689576.90472608)
--(axis cs:268,1729677.15356638)
--(axis cs:267,1856487.85337257)
--(axis cs:266,2185325.55035245)
--(axis cs:265,2732975.33758294)
--(axis cs:264,3859006.35494156)
--(axis cs:263,2272209.63993429)
--(axis cs:262,1863096.33968845)
--(axis cs:261,2228279.70774939)
--(axis cs:260,2414276.50676864)
--(axis cs:259,2901899.53978166)
--(axis cs:258,2944330.08641001)
--(axis cs:257,3675631.52244107)
--(axis cs:256,3394152.65361584)
--(axis cs:255,3721711.80532736)
--(axis cs:254,4532142.84713873)
--(axis cs:253,5567896.14244991)
--(axis cs:252,5504433.99786901)
--(axis cs:251,3949122.79409767)
--(axis cs:250,3954630.37452876)
--(axis cs:249,4208464.34492358)
--(axis cs:248,3377430.36203119)
--(axis cs:247,2521655.10797837)
--(axis cs:246,2879720.3928449)
--(axis cs:245,3046904.89975177)
--(axis cs:244,3943095.11821189)
--(axis cs:243,3655871.6364163)
--(axis cs:242,3878902.93310427)
--(axis cs:241,3388867.78901135)
--(axis cs:240,3462951.27952187)
--(axis cs:239,3305119.53736986)
--(axis cs:238,3032497.435772)
--(axis cs:237,2941463.70158391)
--(axis cs:236,2652799.02470213)
--(axis cs:235,2687702.45046701)
--(axis cs:234,3198501.16558062)
--(axis cs:233,3486701.69914117)
--(axis cs:232,4413513.59339889)
--(axis cs:231,4947441.41848227)
--(axis cs:230,4109422.04921139)
--(axis cs:229,4149415.8942038)
--(axis cs:228,4442917.82076862)
--(axis cs:227,4513554.20833071)
--(axis cs:226,4497768.98209429)
--(axis cs:225,5124788.94809548)
--(axis cs:224,4712107.84923328)
--(axis cs:223,4576983.48619975)
--(axis cs:222,4718260.70567237)
--(axis cs:221,5483875.54021506)
--(axis cs:220,5791381.65534405)
--(axis cs:219,5518941.28871856)
--(axis cs:218,5345069.56245635)
--(axis cs:217,5935411.88722538)
--(axis cs:216,7154303.86864352)
--(axis cs:215,9478763.78317965)
--(axis cs:214,7678006.78181859)
--(axis cs:213,6680670.59850976)
--(axis cs:212,6446787.90475036)
--(axis cs:211,7056508.1571989)
--(axis cs:210,7070842.74366483)
--(axis cs:209,8667506.07639599)
--(axis cs:208,7100045.82235287)
--(axis cs:207,5369783.90778863)
--(axis cs:206,4715320.16827989)
--(axis cs:205,4875980.32845978)
--(axis cs:204,4774177.07921422)
--(axis cs:203,5851644.88820973)
--(axis cs:202,4681032.63821359)
--(axis cs:201,5614171.38311486)
--(axis cs:200,5700585.84233408)
--(axis cs:199,5599140.07202938)
--(axis cs:198,5109421.08319286)
--(axis cs:197,5246716.89082092)
--(axis cs:196,4832985.8131211)
--(axis cs:195,4690176.86339468)
--(axis cs:194,4474089.81119789)
--(axis cs:193,4544427.71784651)
--(axis cs:192,4444608.34384933)
--(axis cs:191,4781837.79156695)
--(axis cs:190,4581032.76975749)
--(axis cs:189,5149267.70339466)
--(axis cs:188,5806001.86209107)
--(axis cs:187,6298849.85445199)
--(axis cs:186,6544867.03746751)
--(axis cs:185,7169139.66427219)
--(axis cs:184,5751476.93802838)
--(axis cs:183,4965941.89954777)
--(axis cs:182,5319385.09268938)
--(axis cs:181,5812221.41630311)
--(axis cs:180,5962470.17684019)
--(axis cs:179,5707150.90915602)
--(axis cs:178,5907274.83286934)
--(axis cs:177,6702028.42944723)
--(axis cs:176,6042696.60782553)
--(axis cs:175,6106214.22023668)
--(axis cs:174,5767178.32627578)
--(axis cs:173,5997157.22852137)
--(axis cs:172,7769987.52162569)
--(axis cs:171,7750856.06228707)
--(axis cs:170,7216639.62184227)
--(axis cs:169,8297657.06764388)
--(axis cs:168,6968730.33286792)
--(axis cs:167,6952932.63168069)
--(axis cs:166,5109855.85842132)
--(axis cs:165,4158584.85346934)
--(axis cs:164,3680279.48793279)
--(axis cs:163,4084293.44702815)
--(axis cs:162,3926754.78835785)
--(axis cs:161,4140249.90856273)
--(axis cs:160,4600632.37194636)
--(axis cs:159,3801457.49810952)
--(axis cs:158,4050984.73192783)
--(axis cs:157,3702391.01162548)
--(axis cs:156,3865467.47363249)
--(axis cs:155,3840903.38269876)
--(axis cs:154,4059155.69685524)
--(axis cs:153,3356862.45771135)
--(axis cs:152,3597513.87212097)
--(axis cs:151,4392258.11564812)
--(axis cs:150,4112010.22385932)
--(axis cs:149,4045760.84391165)
--(axis cs:148,4574684.67622223)
--(axis cs:147,4199680.3155646)
--(axis cs:146,4284251.3581496)
--(axis cs:145,4386694.45746828)
--(axis cs:144,4320396.15702833)
--(axis cs:143,4518371.19438259)
--(axis cs:142,4546123.00213402)
--(axis cs:141,4271029.81089474)
--(axis cs:140,4054161.31181862)
--(axis cs:139,4561065.79312143)
--(axis cs:138,4638227.85244871)
--(axis cs:137,5003283.22721421)
--(axis cs:136,5464990.6726175)
--(axis cs:135,6053120.6516378)
--(axis cs:134,6205790.92296024)
--(axis cs:133,6366672.10232275)
--(axis cs:132,6896633.44799393)
--(axis cs:131,7793397.29046425)
--(axis cs:130,7961483.30186187)
--(axis cs:129,6386976.74992499)
--(axis cs:128,7430772.11592079)
--(axis cs:127,8604243.10916697)
--(axis cs:126,7765222.5282066)
--(axis cs:125,7322950.02458913)
--(axis cs:124,6497068.09692747)
--(axis cs:123,7210874.39519347)
--(axis cs:122,6230899.3345021)
--(axis cs:121,6622347.05031303)
--(axis cs:120,7342683.94699944)
--(axis cs:119,6660126.8874167)
--(axis cs:118,7312079.89316449)
--(axis cs:117,5823102.29304593)
--(axis cs:116,6295193.60072063)
--(axis cs:115,6042483.75519023)
--(axis cs:114,6414456.0114401)
--(axis cs:113,6951379.29955976)
--(axis cs:112,7187635.43534543)
--(axis cs:111,6947960.49391277)
--(axis cs:110,8009769.80951851)
--(axis cs:109,7526874.45807585)
--(axis cs:108,5072552.93289783)
--(axis cs:107,4768803.559878)
--(axis cs:106,4630129.61268953)
--(axis cs:105,4714939.65537848)
--(axis cs:104,4791222.36545748)
--(axis cs:103,5977991.15770567)
--(axis cs:102,6845903.49795855)
--(axis cs:101,5859690.99849803)
--(axis cs:100,5301418.90821475)
--(axis cs:99,5912166.455344)
--(axis cs:98,5674336.61287228)
--(axis cs:97,4341258.64640443)
--(axis cs:96,4464700.63888254)
--(axis cs:95,5091303.58131237)
--(axis cs:94,3912485.96521721)
--(axis cs:93,4254404.96854981)
--(axis cs:92,3776023.87760462)
--(axis cs:91,4019544.05884373)
--(axis cs:90,5341655.21997314)
--(axis cs:89,3619357.01892143)
--(axis cs:88,2834406.14364184)
--(axis cs:87,2752080.72303383)
--(axis cs:86,2647346.5953103)
--(axis cs:85,2784546.23018075)
--(axis cs:84,2927964.07562611)
--(axis cs:83,3092322.8037059)
--(axis cs:82,3026663.51777074)
--(axis cs:81,2958829.17094197)
--(axis cs:80,3085090.99495267)
--(axis cs:79,3427625.50386699)
--(axis cs:78,3438177.00407353)
--(axis cs:77,3596671.45329963)
--(axis cs:76,3698036.25534472)
--(axis cs:75,3793135.33900571)
--(axis cs:74,3809032.48602852)
--(axis cs:73,3918426.85403934)
--(axis cs:72,3964047.68844598)
--(axis cs:71,3966550.98945894)
--(axis cs:70,3672296.66650472)
--(axis cs:69,3803782.04965756)
--(axis cs:68,3850945.76073055)
--(axis cs:67,3886552.04899312)
--(axis cs:66,4617641.37947845)
--(axis cs:65,4600644.45581955)
--(axis cs:64,4619967.15418625)
--(axis cs:63,5180671.72860464)
--(axis cs:62,5559155.84156092)
--(axis cs:61,7270109.17966684)
--(axis cs:60,7310691.65947754)
--(axis cs:59,6791781.54359451)
--(axis cs:58,7299903.58091096)
--(axis cs:57,8395060.34276156)
--(axis cs:56,9083272.16629534)
--(axis cs:55,7974214.00471074)
--(axis cs:54,8160835.64307596)
--(axis cs:53,8728081.7683562)
--(axis cs:52,8312543.21948993)
--(axis cs:51,7135466.18832207)
--(axis cs:50,8171675.9451676)
--(axis cs:49,7567798.97124066)
--(axis cs:48,8515494.48133226)
--(axis cs:47,7605830.18211758)
--(axis cs:46,8254266.19522328)
--(axis cs:45,8953465.91405965)
--(axis cs:44,9409952.41018935)
--(axis cs:43,6118251.26687844)
--(axis cs:42,5941102.53552173)
--(axis cs:41,5087335.89759495)
--(axis cs:40,4774047.44945541)
--(axis cs:39,5437576.5346946)
--(axis cs:38,6001050.46787631)
--(axis cs:37,5962013.15505577)
--(axis cs:36,4751699.86711551)
--(axis cs:35,4573619.03256725)
--(axis cs:34,4537847.49175563)
--(axis cs:33,4871890.01373864)
--(axis cs:32,4286710.6558835)
--(axis cs:31,4229272.11830228)
--(axis cs:30,4916356.27286879)
--(axis cs:29,4898022.63091916)
--(axis cs:28,5024260.93975688)
--(axis cs:27,4941504.34881509)
--(axis cs:26,5331166.18640597)
--(axis cs:25,4388913.20543393)
--(axis cs:24,4603170.58812725)
--(axis cs:23,4710694.86181495)
--(axis cs:22,5933609.09100154)
--(axis cs:21,6110631.61210926)
--(axis cs:20,7743977.48525121)
--(axis cs:19,4328879.61099963)
--(axis cs:18,4836843.95284928)
--(axis cs:17,4755557.13903489)
--(axis cs:16,5539451.45028963)
--(axis cs:15,5609115.43708124)
--(axis cs:14,6234312.70907845)
--(axis cs:13,4169979.45995265)
--(axis cs:12,3518107.64306156)
--(axis cs:11,3643351.42811772)
--(axis cs:10,4384175.51717533)
--(axis cs:9,5327688.90918997)
--(axis cs:8,4454760.62215455)
--(axis cs:7,3624041.95015458)
--(axis cs:6,4019970.30353744)
--(axis cs:5,3623261.12455218)
--(axis cs:4,2936735.99414653)
--(axis cs:3,2939971.2151676)
--(axis cs:2,4514022.65666823)
--(axis cs:1,3015156.03936156)
--(axis cs:0,2353569.05305084)
--cycle;

\addplot [only marks, red, mark=asterisk, mark size=1.2]
table {%
0 1811900
1 2950200
2 1817300
3 1816900
4 2271100
5 2510500
6 2288800
7 3174300
8 3646500
9 3002900
10 2336100
11 2254700
12 2661500
13 4982300
14 4045900
15 3433400
16 3294700
17 3166200
18 3009200
19 6084000
20 4332200
21 4207900
22 3374700
23 3295100
24 3073300
25 4142000
26 3655200
27 3744100
28 3744100
29 3641900
30 3253400
31 3031800
32 3558000
33 3319200
34 3328800
35 3451900
36 4412400
37 4396000
38 3925600
39 3467300
40 3754500
41 4058500
42 4092200
43 7297200
44 7012300
45 6235800
46 5705800
47 7073900
48 6717800
49 6939200
50 6193200
51 7487800
52 8148200
53 6703700
54 6683500
55 7868100
56 6811200
57 5505000
58 4985200
59 5481700
60 6089200
61 4231100
62 3903800
63 3434500
64 3285400
65 3375500
66 2838300
67 2825600
68 2842800
69 2717600
70 2797300
71 2643500
72 2659400
73 2669600
74 2752700
75 2646200
76 2619200
77 2538600
78 2454500
79 2186800
80 2033500
81 2029000
82 2041600
83 1891300
84 1766300
85 1760600
86 1875700
87 1820500
88 2204200
89 3592500
90 2735200
91 2598200
92 2856700
93 2593200
94 3478900
95 3199200
96 3019100
97 4171900
98 4427600
99 4213500
100 4365900
101 5690400
102 4459800
103 3720400
104 3635100
105 3411600
106 3340600
107 3688500
108 6044300
109 6861500
110 6231400
111 6058400
112 5697500
113 5012600
114 4644800
115 4554200
116 4556500
117 5831000
118 5064100
119 5524500
120 5076700
121 4941600
122 6442400
123 5240200
124 5782200
125 6059500
126 6629700
127 5507700
128 4775000
129 5671900
130 5773300
131 4902900
132 4809100
133 4293600
134 4435000
135 4055400
136 3638800
137 3305700
138 3474600
139 2917400
140 3147700
141 3083300
142 2956800
143 2946200
144 2864800
145 2884100
146 2908400
147 2944200
148 2762200
149 2850900
150 2479600
151 2637600
152 2255300
153 2770100
154 2642000
155 2306400
156 2395300
157 2684400
158 2534800
159 3081400
160 2917600
161 2672300
162 2742700
163 2510800
164 2874400
165 3502600
166 5254700
167 5111400
168 6195900
169 5648700
170 5800300
171 5547500
172 4379000
173 4311100
174 4581900
175 4392900
176 4709000
177 4125600
178 3863700
179 4489700
180 4446700
181 4004700
182 3373200
183 3897200
184 5274200
185 4474300
186 4310400
187 4085500
188 3586100
189 3133600
190 3207100
191 2884800
192 3267300
193 3172800
194 3172400
195 3248800
196 3604100
197 3358500
198 3837600
199 3962600
200 4147200
201 3334000
202 4282800
203 3251900
204 3390500
205 3135000
206 3631800
207 5172900
208 6400000
209 5456400
210 5464100
211 5115400
212 5065500
213 5812500
214 6945000
215 5561700
216 4547200
217 4139500
218 4247100
219 4097600
220 4039400
221 3659800
222 3320100
223 3539800
224 3605800
225 3208300
226 3151200
227 3226800
228 2968500
229 3008100
230 3908200
231 3058200
232 2550400
233 2345200
234 1914100
235 1868200
236 2182400
237 2060100
238 1896000
239 2427900
240 2419000
241 2551000
242 2413800
243 2982200
244 2174900
245 1916600
246 1694000
247 1910600
248 2973400
249 2991500
250 2385300
251 4807600
252 5394900
253 3829100
254 2475700
255 2190300
256 2382000
257 1975400
258 1972900
259 1562100
260 1395600
261 1208000
262 1366300
263 3296700
264 1998800
265 1506000
266 1217300
267 1125000
268 1065000
269 974300
270 1621800
271 1085100
272 1563400
273 1855000
274 1754600
275 1705500
276 1725600
277 1559600
278 2031400
279 2515800
280 1955600
281 1988800
282 1627900
283 1727200
284 1817700
285 3388400
286 3130700
287 2240500
288 2280000
289 1737400
290 1703900
291 1898000
292 1732700
293 1540100
294 1595600
295 1481700
296 1251700
297 2006700
298 2034500
299 1855400
300 1320300
301 1380900
302 1487200
303 1633800
304 1553700
305 1452900
306 1291700
307 1335100
308 1180000
309 1294500
310 1145300
311 1197200
312 1196800
313 1351300
314 1527900
315 1541900
316 1141000
317 1096800
318 1060700
319 1125400
320 1064000
321 2156300
322 1732100
323 1386600
324 1198600
325 1379500
326 1809800
327 1940200
328 2223800
329 2445300
330 3258400
331 3025800
332 3715900
333 3465000
334 2927100
335 2545400
336 2406300
337 3339500
338 3323700
339 3033700
340 2808200
341 2855500
342 4484000
343 3796400
344 3992500
345 4025700
346 3529700
347 3506600
348 4210800
349 4138000
350 3929200
351 3971800
352 4943700
353 3703700
354 4286300
355 4712500
356 5405600
357 4591900
358 3988600
359 3633700
360 3445000
361 5647700
362 4572400
363 7786400
364 5313100
365 7995100
366 6747200
367 6405700
368 5925900
369 4952600
370 4883300
371 7171600
372 6939800
373 6449700
374 6123600
375 6851600
376 7203000
377 7068600
378 6852800
379 6643100
380 5865500
381 7082600
382 6590000
383 5693500
384 5029500
385 5025600
386 5159400
387 4680900
388 4698500
389 3975400
390 3958600
391 4126500
392 4172900
393 6345300
394 7193900
395 10066000
396 7074000
397 6074800
398 5300600
399 4628000
400 4796300
401 4861800
402 4713500
403 5040400
404 5581800
405 5028200
406 4294400
407 4095700
408 4346300
409 4683300
410 4195600
411 4379800
412 4570400
413 5224900
414 5962800
415 5969900
416 6426400
417 5434800
418 7391400
419 8254800
420 4183500
421 5586700
422 5466200
423 5908900
424 7937300
425 5826200
426 5692100
427 5829200
428 4836800
429 4460300
430 4207400
431 3905100
432 3502300
433 3373200
434 3400400
435 3284400
436 3720900
437 3594000
438 4594100
439 3260600
440 5988400
};
\addplot [blue]
table {%
0 1462135.86593289
1 1938511.8148966
2 3129768.01306228
3 1896787.64802399
4 1833619.14220277
5 2426070.62192331
6 2728022.48666487
7 2406127.37639993
8 3137274.53325256
9 3742496.16392957
10 3096012.25081109
11 2477274.07492796
12 2393328.01471257
13 2882182.17821151
14 4616817.91341928
15 4044090.2103727
16 3916120.51853946
17 3322407.8455472
18 3362181.99906438
19 3067258.38262694
20 5749721.12793496
21 4398873.60478371
22 4285767.73720924
23 3359842.62539026
24 3284679.44436048
25 3099343.30982722
26 3886513.82877723
27 3590108.70351649
28 3651275.0728572
29 3587107.03284684
30 3553450.36918851
31 3080731.64834609
32 3043010.30780752
33 3566429.09752234
34 3284473.9534889
35 3317911.88363134
36 3453702.77575825
37 4342600.47466988
38 4422985.05255141
39 3975290.71519108
40 3387430.88102514
41 3683569.5945263
42 4324653.31588663
43 4436677.42855287
44 6983498.72125475
45 6760888.78202654
46 6224643.66437056
47 5690256.78936597
48 6412224.60587029
49 5700482.0793317
50 6139853.94281035
51 5339470.83294711
52 6332728.14745723
53 6658115.3964246
54 6104654.19973453
55 5990063.09779574
56 6773303.5630333
57 6350465.51259211
58 5476685.4028768
59 5006900.76859086
60 5385227.96675262
61 5418819.21472842
62 4106033.95995253
63 3796165.48046701
64 3313248.99266563
65 3332115.25724393
66 3314492.42986399
67 2785345.60336217
68 2728929.1885676
69 2717393.43451141
70 2634308.55918358
71 2832011.68224913
72 2803226.82569937
73 2771644.19945675
74 2722579.70334751
75 2690135.21989721
76 2619913.69287608
77 2568152.56356483
78 2423542.62599582
79 2385888.88769629
80 2146922.34534775
81 2063785.28990703
82 2075567.28884869
83 2135231.2050334
84 2005753.98653596
85 1868473.89989129
86 1788818.89011803
87 1886275.42646594
88 1916984.28973967
89 2472161.84791251
90 3908537.1672615
91 2863372.90549691
92 2681932.26632027
93 2989389.73540457
94 2771250.67848499
95 3652960.82285212
96 3218313.19626546
97 3094805.45023021
98 4206902.67962397
99 4372548.49429367
100 3890597.83385119
101 4319343.79192381
102 5100002.3380042
103 4418408.92795404
104 3502089.83295438
105 3465122.60544848
106 3369345.36513709
107 3447289.18538883
108 3711876.1245853
109 5629833.56732387
110 6031574.60750859
111 5190378.87585053
112 5372924.39488605
113 5068791.34895854
114 4748710.89477047
115 4482349.16246214
116 4612979.177487
117 4260920.77125856
118 5466198.73087529
119 4961106.68194236
120 5425150.92865086
121 4839979.91036075
122 4591487.16402696
123 5435705.83924116
124 4788225.30629432
125 5429424.42546376
126 5731746.31751236
127 6475519.76687442
128 5404228.59903712
129 4699451.8071892
130 5833009.6915664
131 5755797.94380749
132 5057863.37358015
133 4696751.6844907
134 4536157.1721531
135 4414675.72393879
136 3974563.96895562
137 3592481.48433634
138 3317949.08209729
139 3289928.71163807
140 2886321.39609649
141 3071588.07145746
142 3225908.02922434
143 3176830.73916074
144 3058297.11646267
145 3071595.43109382
146 2993944.91690855
147 2991375.26639302
148 3209988.75408052
149 2848470.03953492
150 2915089.14542886
151 3090897.10070566
152 2541922.02563597
153 2261887.29902047
154 2821843.4569764
155 2701735.85494977
156 2631156.61541227
157 2556193.6355629
158 2854687.14846429
159 2647610.68036833
160 3263129.56695761
161 2959921.20473107
162 2749638.16230406
163 2870958.51975891
164 2564171.06014871
165 2953789.91588351
166 3684259.07958077
167 5142067.34095933
168 5131895.76551996
169 6023273.02955572
170 5329654.43169488
171 5643792.05311382
172 5605128.94362187
173 4266074.32714581
174 4180181.30180074
175 4363482.72754264
176 4264696.64526148
177 4874266.58454935
178 4199877.47466797
179 4014198.03246193
180 4248760.18710551
181 4179794.89388872
182 3791387.99622353
183 3507965.88716058
184 4134232.21279056
185 5208019.21129674
186 4720096.78633103
187 4389007.07936023
188 4223366.45075295
189 3684607.77682529
190 3229664.4872116
191 3396381.89402141
192 3102849.94285916
193 3231700.25712914
194 3083256.03941832
195 3269060.54291899
196 3447268.73604042
197 3703450.82778157
198 3568136.61530524
199 3951197.02542985
200 4099262.6983644
201 3898749.51807042
202 3273144.09686704
203 4221247.40880388
204 3340791.9192709
205 3421660.64584514
206 3294000.81806164
207 3814649.0515336
208 5162652.73983007
209 6338776.69806211
210 5193143.29319401
211 5050382.75272385
212 4638458.58073672
213 4790745.84150957
214 5563669.8189883
215 6911947.58674975
216 5205095.02814319
217 4266946.4003387
218 3800156.7746789
219 3939009.2453174
220 4074094.46766016
221 3845015.71139605
222 3296230.45947407
223 3232031.62110744
224 3318968.12967145
225 3608462.4939189
226 3140067.86351134
227 3195408.05721725
228 3167377.09673234
229 2902623.15369377
230 2905353.50966905
231 3531835.38123773
232 3086417.41868245
233 2455854.60712695
234 2197348.2277048
235 1840659.52873543
236 1797222.13332868
237 2030603.82221129
238 2064403.67556248
239 2207548.88075966
240 2362800.7395542
241 2307445.57196979
242 2685227.50606179
243 2532325.17452653
244 2795229.40058176
245 2086911.55668557
246 1919518.42363
247 1597705.41714467
248 2294111.15858858
249 2955745.151742
250 2785336.76694567
251 2731138.48390455
252 3999322.61160112
253 4042393.58546619
254 3210805.98547471
255 2528349.20268517
256 2321642.90262972
257 2492766.25820834
258 1962264.52089421
259 1963614.36365296
260 1540485.03463305
261 1450968.03713696
262 1175363.50200136
263 1455846.83499061
264 2709852.30720419
265 1834612.42802914
266 1416859.08685723
267 1175849.88669882
268 1074973.36005215
269 1043981.66171576
270 993455.993292201
271 1840564.71609766
272 1182926.47308273
273 1743327.4626151
274 2144158.92930571
275 1893823.33295698
276 1801408.98151534
277 1885732.78377495
278 1842394.41336861
279 2220440.74900336
280 2686737.09299843
281 2111684.1005346
282 2102745.45099086
283 1633596.73258603
284 1707277.93379554
285 1907246.71781063
286 3506226.38333855
287 3385819.44920729
288 2315982.99700482
289 2623047.21289581
290 1838620.53285569
291 1831259.04211815
292 2029139.77547386
293 1845604.03877298
294 1575580.52251537
295 1698675.91648827
296 1557997.03716371
297 1289225.23980837
298 2212672.20971495
299 2196205.83552975
300 2047439.13022076
301 1439305.933234
302 1536832.14367322
303 1759242.72991996
304 2007786.39876572
305 1628504.27907448
306 1496405.92407713
307 1388589.19168769
308 1390104.2678193
309 1191870.56756337
310 1304269.10265702
311 1158772.35662334
312 1200718.1182093
313 1237588.6779538
314 1491696.41167719
315 1519999.00343735
316 1579221.09477803
317 1169349.23998559
318 1090851.46744949
319 1078055.99749466
320 1194192.44916662
321 1197653.17753763
322 2342473.03353656
323 1777896.33711028
324 1420646.26988061
325 1259851.68174528
326 1472575.30389952
327 1940911.30776311
328 2104951.25723906
329 2378360.9931377
330 2595902.12693948
331 3499000.95427817
332 3276786.91062475
333 3969784.79247879
334 3463766.5119183
335 2947777.1974211
336 2646583.21990294
337 2674688.26124368
338 3272529.98279275
339 3365818.44693263
340 3108168.64612451
341 2746594.84873168
342 2800819.55162646
343 4127321.75589979
344 3680279.41985507
345 3955031.06872591
346 3787739.27066447
347 3318913.44243585
348 3383273.31532517
349 4199824.73760587
350 3754544.90667551
351 3694334.56650735
352 3730320.74788683
353 4785817.77691371
354 3554994.83142975
355 4252790.97653587
356 4535470.54342089
357 5149496.65534605
358 4274784.17371753
359 3828108.58159107
360 3475694.05751279
361 3366564.5635534
362 5333550.99873536
363 4463289.918847
364 6938603.98971047
365 5063477.10354437
366 7096026.01132357
367 6102719.93079769
368 5951641.76098186
369 5630901.56507241
370 4658777.61931368
371 4796949.68064404
372 6339901.30614166
373 6293480.94335
374 6050057.12866883
375 5956804.01749757
376 6501625.50581968
377 7277396.70947572
378 7102166.23672851
379 6829148.7179428
380 6504010.16841774
381 5677226.49739388
382 6469362.17841811
383 6323737.9603247
384 5778955.38306191
385 5085776.85498561
386 5132707.49777604
387 5078818.73778702
388 4635355.01808383
389 4667319.83243661
390 4020935.38548384
391 3912433.00061373
392 3993316.24444888
393 4083227.35976654
394 5690172.70899869
395 6595631.86646852
396 7699554.44046669
397 6062470.3215946
398 5427778.16756684
399 4791046.89638499
400 4420356.64498011
401 4784310.53820931
402 4606375.4605575
403 4717678.24605059
404 5240220.86423504
405 5593639.88862577
406 5009969.78741477
407 4208193.74531216
408 3985437.95269849
409 4319393.58501612
410 4580750.88197801
411 4185250.45053124
412 4547825.50141695
413 4720344.81090795
414 5108455.43551468
415 5804476.13848239
416 5754448.29130795
417 6256590.30066917
418 5253019.50408166
419 6479550.98345198
420 7356277.74956042
421 4651887.02041661
422 5488674.41811311
423 5437007.38731773
424 5598964.7932186
425 6540145.63857006
426 5394482.86407862
427 5423245.00449079
428 5369666.21957054
429 4506507.43755157
430 4152198.68811431
431 4019632.68458065
432 3708921.35126123
433 3323436.98782686
434 3124742.96950784
435 3211119.88790732
436 3465167.35243929
437 3678560.62011338
438 3557727.90331498
439 4750049.99482331
440 3376602.23445196
};
\addplot [black]
table {%
0 1441453.63195456
1 1911560.30106987
2 3062790.60794601
3 1857083.33443758
4 1801969.45850766
5 2374876.66042994
6 2667449.28041616
7 2364218.83252325
8 3077211.41962674
9 3684752.33774717
10 3055916.63411059
11 2437218.47654638
12 2364026.43735976
13 2874036.18940032
14 4585515.69457208
15 3981540.00363485
16 3854513.02831124
17 3287618.39347378
18 3308611.58058412
19 3019175.65423504
20 5687697.86673283
21 4329359.20232975
22 4239404.45278563
23 3321990.85351918
24 3238931.82926521
25 3059400.50229209
26 3874591.0039267
27 3566834.07336132
28 3593313.1489985
29 3546289.87110975
30 3516306.01146526
31 3022180.81693601
32 3003932.63083573
33 3525721.74778153
34 3239015.9841806
35 3283633.93941727
36 3420007.57964906
37 4323931.23539984
38 4376089.62641919
39 3934495.32717402
40 3361977.91902704
41 3657762.21158954
42 4287204.18273685
43 4391884.09151287
44 6877318.68017818
45 6700049.84431574
46 6176250.34106152
47 5634190.2975533
48 6378703.54656959
49 5655958.18424965
50 6064251.93460714
51 5273973.53794533
52 6279627.01341352
53 6560855.06254966
54 6024799.09523991
55 5935459.80989408
56 6718982.40052882
57 6306507.0058611
58 5418098.05457111
59 4974989.59036331
60 5366953.91004248
61 5367219.49502067
62 4078722.28826553
63 3788855.96405084
64 3278153.52657172
65 3289571.21186752
66 3262948.12561663
67 2763614.23424922
68 2702754.58883637
69 2675156.48829685
70 2584875.5493433
71 2787898.12679223
72 2774200.34027668
73 2737727.53847742
74 2687534.2830653
75 2665165.28174336
76 2592551.42247413
77 2534497.92963834
78 2402016.1122724
79 2360251.92181185
80 2117114.39548966
81 2042846.95279042
82 2028479.31745694
83 2102950.13660619
84 1974360.38906886
85 1833295.10972758
86 1759279.14647773
87 1858384.89619082
88 1894474.58371398
89 2424731.5125636
90 3835490.89240385
91 2831323.36863466
92 2640789.09465947
93 2941642.42993145
94 2731274.32737885
95 3614802.84045339
96 3166436.43789982
97 3060928.00282409
98 4147919.2475659
99 4329101.38928196
100 3841798.37767097
101 4276472.28683932
102 5043546.05521709
103 4382033.21481076
104 3466341.80705514
105 3417895.74567328
106 3317305.49110913
107 3407244.43556279
108 3666836.49853935
109 5568590.93362628
110 5980686.93868618
111 5142150.73419056
112 5313668.72388494
113 5027915.23908363
114 4693900.12772644
115 4456499.43097376
116 4521992.2411395
117 4215538.35137786
118 5400096.61555599
119 4906890.26213913
120 5408913.36873804
121 4813498.62415764
122 4540118.22307596
123 5354024.77015756
124 4746215.44739664
125 5393927.46855075
126 5683878.25276525
127 6410646.42649628
128 5354333.63934239
129 4625014.25719064
130 5791675.07709316
131 5711686.20639524
132 4934581.51662081
133 4650629.99813159
134 4494894.09379636
135 4369631.53631271
136 3930312.4675894
137 3541004.22778514
138 3281749.71585862
139 3265075.72703564
140 2855891.74967869
141 3042918.84842942
142 3183693.63291075
143 3131577.4962397
144 3048454.65617085
145 3017507.14330404
146 2961493.54044505
147 2950353.59371997
148 3169283.16425664
149 2819509.63465006
150 2866447.05578212
151 3041260.7679455
152 2500541.74694525
153 2236972.74406285
154 2793387.70720624
155 2642871.38815943
156 2592093.39894366
157 2507011.40746297
158 2814630.84104933
159 2611574.52059865
160 3238202.32618359
161 2924136.32010069
162 2717075.80577284
163 2839642.34097853
164 2527542.53780323
165 2927119.29665522
166 3640440.15546002
167 5081207.84686782
168 5081202.80505165
169 5968685.47452519
170 5266894.63370642
171 5614724.5575968
172 5509453.93813517
173 4206933.00830316
174 4108372.95852374
175 4307829.38570479
176 4221409.88248236
177 4826183.98088112
178 4146688.8401975
179 3949119.09109453
180 4185178.39797737
181 4138252.30326561
182 3753343.23782772
183 3477776.36653394
184 4065511.16696726
185 5115063.6948357
186 4628530.98257685
187 4318986.29623704
188 4152423.73769847
189 3629531.75733023
190 3194791.95102852
191 3343668.76959444
192 3057578.38772108
193 3181236.98641261
194 3046626.00174167
195 3248128.1229297
196 3388338.73498998
197 3650748.96826422
198 3514842.69272801
199 3900660.75106598
200 4037865.71574289
201 3857574.09281461
202 3220033.12379244
203 4153096.92783692
204 3296890.48193521
205 3379667.65874557
206 3240207.83494998
207 3783639.71486188
208 5112978.94819814
209 6283702.92715688
210 5152026.44787511
211 5020500.28130343
212 4620708.08250189
213 4727645.70300402
214 5431528.97660791
215 6797264.47753083
216 5159952.51685581
217 4244782.79576379
218 3768127.48539642
219 3883504.06126252
220 4015675.4904919
221 3755915.7314291
222 3265825.19758939
223 3177289.10568666
224 3246948.55012411
225 3547942.12702696
226 3084345.32838332
227 3153889.50125813
228 3132529.86865491
229 2862918.19644544
230 2874113.56029397
231 3499293.67718627
232 3071290.12450255
233 2410222.45412604
234 2171347.6708484
235 1808814.11816238
236 1774442.78754794
237 1984869.7430292
238 2039719.6833699
239 2181334.63878444
240 2327203.68539294
241 2285925.63877381
242 2648688.82516586
243 2502930.70066735
244 2749530.48028409
245 2058509.12835142
246 1890492.47064424
247 1566325.02677051
248 2262371.92013773
249 2904526.94709344
250 2738757.95737212
251 2711243.34839158
252 3930530.37549197
253 3984240.27284786
254 3165286.72734464
255 2493941.75858431
256 2278913.52809467
257 2453261.15143815
258 1942316.63458332
259 1923651.31336532
260 1503909.52067686
261 1422957.36982642
262 1143381.74984793
263 1424830.96664251
264 2671043.19377365
265 1817944.15046001
266 1388964.18187439
267 1158946.57757779
268 1042603.6494251
269 1013016.97445075
270 977890.629273511
271 1808951.45174778
272 1154041.13258959
273 1707371.8485241
274 2105922.18844386
275 1856424.51329072
276 1774458.31267613
277 1847598.22686981
278 1826123.48524329
279 2176888.87695037
280 2623460.65679568
281 2069380.70354355
282 2089680.20788718
283 1598325.26183667
284 1675409.11209816
285 1871370.91625191
286 3450077.23569582
287 3334023.25801317
288 2272136.56321727
289 2572677.15684919
290 1798891.55109549
291 1796326.6586741
292 2005217.24477809
293 1829049.36845003
294 1550510.26370735
295 1657031.51021155
296 1526381.74751152
297 1255563.58217298
298 2165565.25681158
299 2178921.88506165
300 2016216.26815936
301 1406801.94336911
302 1509349.90699874
303 1721766.31865126
304 1963672.02556129
305 1592643.95022916
306 1461644.09635001
307 1353360.55734702
308 1355252.40310361
309 1171226.38792407
310 1265016.28187921
311 1121867.96901021
312 1173523.31077471
313 1205890.91084552
314 1439799.32859418
315 1496723.30547004
316 1539785.99801367
317 1133940.75872007
318 1062216.03492104
319 1036565.77929723
320 1163093.13505979
321 1165465.81492882
322 2290745.09946063
323 1741220.90492938
324 1394214.63041419
325 1230396.60425389
326 1439926.04139177
327 1903012.67269705
328 2079377.97951297
329 2352694.27929465
330 2557014.16438257
331 3470024.58174553
332 3250104.80343316
333 3919640.81353825
334 3417839.35768453
335 2912049.4749917
336 2599204.75749808
337 2642165.24475989
338 3243793.25999084
339 3333105.2814996
340 3090318.05912718
341 2685097.50125151
342 2771623.60298769
343 4100365.4451343
344 3666691.07932089
345 3902832.69243132
346 3756547.02902793
347 3273760.21851284
348 3340335.43883756
349 4132800.46195186
350 3704805.49131591
351 3624834.52442536
352 3710909.06685475
353 4734989.30290219
354 3519517.396924
355 4229701.91962505
356 4511376.71626075
357 5109819.38428043
358 4244662.39695995
359 3765066.00124383
360 3429758.59638245
361 3318326.82281375
362 5246661.1256963
363 4404436.89921726
364 6870678.14275422
365 4991328.36019906
366 7013741.46985823
367 6075779.11335332
368 5932449.02294199
369 5587486.41612901
370 4599193.69531721
371 4752271.06663399
372 6252482.82795362
373 6204257.60426235
374 5980030.51613663
375 5851820.78979152
376 6418494.97850383
377 7185536.97610787
378 7010267.31331407
379 6707122.20821218
380 6419246.75257936
381 5630952.27163638
382 6413058.99993775
383 6292894.73982221
384 5704018.51634535
385 5036885.2700465
386 5095628.48417071
387 5050599.61848191
388 4585204.69842355
389 4622242.64346516
390 3985290.78390006
391 3855603.33193333
392 3936520.96134273
393 4040051.23387516
394 5654880.67567299
395 6548314.13247147
396 7628941.50305783
397 6000633.34391808
398 5404272.53504986
399 4776081.4501279
400 4396336.88902684
401 4754414.08084371
402 4543595.84357263
403 4712002.40419664
404 5220379.08283744
405 5549100.07215801
406 4960644.04833284
407 4153256.11242927
408 3956391.15666779
409 4278619.31805446
410 4533685.78303048
411 4143894.43083199
412 4495424.51316192
413 4685657.59790972
414 5049979.71067471
415 5735717.81965559
416 5724622.58580255
417 6205049.00147737
418 5201913.33643046
419 6424681.82779331
420 7311841.65510402
421 4598371.39541464
422 5464519.85855247
423 5394667.1004189
424 5553076.31771157
425 6494815.5058689
426 5380203.57992155
427 5369206.19571706
428 5310342.94913299
429 4486610.17581884
430 4111337.53322079
431 3957967.67957308
432 3675525.93127992
433 3273897.97033207
434 3098631.40926193
435 3196990.07030709
436 3439869.59673364
437 3655441.42701713
438 3533297.03829673
439 4681243.62113768
440 3352189.08377276
};
\end{axis}

\end{tikzpicture}
}\hfill
%%		\subfloat[$A.$]{% This file was created with tikzplotlib v0.10.1.
\begin{tikzpicture}

\definecolor{darkgray176}{RGB}{176,176,176}

\begin{axis}[
width=\figurewidth,
height=\figureheight,
axis background/.style={fill=background_color},
axis line style={white},
tick align=inside,
tick pos=left,
x grid style={white},
y grid style={white},
xmajorgrids,
ymajorgrids,
%ylabel={Contributions (kWh)},
xmin=50, xmax=200,
xtick style={color=black},
ymin=-212629.992846074, ymax=11354433.3118096,
ytick style={color=black},
ytick={-2000000,0,2000000,4000000,6000000,8000000,10000000,12000000},
yticklabels={\ensuremath{-}0.2,0.0,0.2,0.4,0.6,0.8,1.0,1.2}
]
\path [draw=blue, fill=blue, opacity=0.5]
(axis cs:0,7711796.89815992)
--(axis cs:0,2705345.66975293)
--(axis cs:1,3378626.50595514)
--(axis cs:2,3191843.04124707)
--(axis cs:3,2903000.34329433)
--(axis cs:4,3168421.43101453)
--(axis cs:5,3302721.63579624)
--(axis cs:6,3117893.1429199)
--(axis cs:7,3035021.8384453)
--(axis cs:8,2627724.2018352)
--(axis cs:9,2938373.26748152)
--(axis cs:10,2519588.07168455)
--(axis cs:11,2287564.41397439)
--(axis cs:12,1814873.76828572)
--(axis cs:13,3776028.80931266)
--(axis cs:14,3573201.68004368)
--(axis cs:15,3012860.62408603)
--(axis cs:16,3405155.88039684)
--(axis cs:17,2988371.90341279)
--(axis cs:18,3034074.4156766)
--(axis cs:19,3014123.54472012)
--(axis cs:20,3775851.0995684)
--(axis cs:21,3705702.43628425)
--(axis cs:22,3725253.02202492)
--(axis cs:23,2866922.66356293)
--(axis cs:24,2912554.15146347)
--(axis cs:25,2365021.17901728)
--(axis cs:26,2391152.76628737)
--(axis cs:27,2182046.4383084)
--(axis cs:28,2212508.7646494)
--(axis cs:29,2825189.68267933)
--(axis cs:30,2188312.31839416)
--(axis cs:31,1967217.36188941)
--(axis cs:32,2155960.00032745)
--(axis cs:33,2575166.86432231)
--(axis cs:34,1908800.59946669)
--(axis cs:35,2063304.47505139)
--(axis cs:36,2244017.93928917)
--(axis cs:37,3001727.42266113)
--(axis cs:38,3135324.17085454)
--(axis cs:39,2653344.45596589)
--(axis cs:40,3439571.10324812)
--(axis cs:41,2379457.51529501)
--(axis cs:42,3578722.44751919)
--(axis cs:43,3361969.64143614)
--(axis cs:44,3316707.1515354)
--(axis cs:45,3450472.54465092)
--(axis cs:46,3478292.56638944)
--(axis cs:47,2973452.14984918)
--(axis cs:48,2768365.63223762)
--(axis cs:49,2539422.33006709)
--(axis cs:50,3471856.29457737)
--(axis cs:51,2674376.40526259)
--(axis cs:52,3524252.43945091)
--(axis cs:53,2789592.01317079)
--(axis cs:54,2620586.54170768)
--(axis cs:55,2606304.60699472)
--(axis cs:56,3011661.22187069)
--(axis cs:57,3482997.13829144)
--(axis cs:58,2858491.92874986)
--(axis cs:59,2594965.21554956)
--(axis cs:60,3219014.75684987)
--(axis cs:61,2607394.88741136)
--(axis cs:62,2495148.13685283)
--(axis cs:63,2420159.7727381)
--(axis cs:64,2658984.81379742)
--(axis cs:65,2300400.60461416)
--(axis cs:66,1960875.13066926)
--(axis cs:67,1605267.80951336)
--(axis cs:68,1398488.81385747)
--(axis cs:69,1323782.910459)
--(axis cs:70,1246391.24751547)
--(axis cs:71,1560156.7308916)
--(axis cs:72,3164635.26263227)
--(axis cs:73,2518527.17982634)
--(axis cs:74,2293996.59970759)
--(axis cs:75,2006486.4269339)
--(axis cs:76,1728538.66802994)
--(axis cs:77,1873597.26283576)
--(axis cs:78,1528985.34270291)
--(axis cs:79,1539599.16924827)
--(axis cs:80,1212311.97452083)
--(axis cs:81,1286621.50352517)
--(axis cs:82,2477217.90901059)
--(axis cs:83,1832450.45872709)
--(axis cs:84,1758293.32089789)
--(axis cs:85,2298224.57358047)
--(axis cs:86,2570582.64397243)
--(axis cs:87,1706704.05445089)
--(axis cs:88,1927503.59001838)
--(axis cs:89,2655676.35937528)
--(axis cs:90,2419392.94071748)
--(axis cs:91,2055893.19531133)
--(axis cs:92,1722797.27652642)
--(axis cs:93,1535098.22974384)
--(axis cs:94,1541066.80255936)
--(axis cs:95,1560029.71829929)
--(axis cs:96,1336901.98695129)
--(axis cs:97,1917572.26545102)
--(axis cs:98,2420516.79234591)
--(axis cs:99,2442744.32020066)
--(axis cs:100,1999879.76507172)
--(axis cs:101,1961561.73375269)
--(axis cs:102,1982226.15417393)
--(axis cs:103,1864528.63737786)
--(axis cs:104,1641573.19151929)
--(axis cs:105,1604733.07337567)
--(axis cs:106,1783969.34903992)
--(axis cs:107,2214132.60635561)
--(axis cs:108,2076761.04922573)
--(axis cs:109,2112695.17241453)
--(axis cs:110,2365555.07940351)
--(axis cs:111,1933880.36137843)
--(axis cs:112,1949709.05206518)
--(axis cs:113,1841351.3164714)
--(axis cs:114,1947748.24996327)
--(axis cs:115,2870279.11885729)
--(axis cs:116,3452660.95775308)
--(axis cs:117,2484023.96310546)
--(axis cs:118,2585323.11862255)
--(axis cs:119,2361884.59827127)
--(axis cs:120,2983664.54937774)
--(axis cs:121,2520865.05413682)
--(axis cs:122,2284586.94041283)
--(axis cs:123,2472811.71335933)
--(axis cs:124,2684203.94047405)
--(axis cs:125,2926916.47361885)
--(axis cs:126,2750042.54586679)
--(axis cs:127,3060685.72178713)
--(axis cs:128,2677558.71628327)
--(axis cs:129,2964919.70289489)
--(axis cs:130,3328227.17695684)
--(axis cs:131,2921866.25833598)
--(axis cs:132,2998858.60271756)
--(axis cs:133,2626328.43950407)
--(axis cs:134,2578919.94882663)
--(axis cs:135,2176999.51758305)
--(axis cs:136,2082894.64897437)
--(axis cs:137,1957760.98273046)
--(axis cs:138,1832656.71499928)
--(axis cs:139,1581502.05298148)
--(axis cs:140,1510330.74661907)
--(axis cs:141,1689561.44736412)
--(axis cs:142,2499444.73543272)
--(axis cs:143,2676327.54945432)
--(axis cs:144,2130771.2837957)
--(axis cs:145,2912329.79520681)
--(axis cs:146,2127125.0690203)
--(axis cs:147,1844021.69323307)
--(axis cs:148,2974683.48645776)
--(axis cs:149,2087577.38879899)
--(axis cs:150,2251059.35839899)
--(axis cs:151,1346382.06036548)
--(axis cs:152,1375607.79820876)
--(axis cs:153,1206277.38293013)
--(axis cs:154,1490971.88372617)
--(axis cs:155,1642953.66410913)
--(axis cs:156,2079367.65955157)
--(axis cs:157,1688642.99039614)
--(axis cs:158,2606251.68311852)
--(axis cs:159,1898602.19876107)
--(axis cs:160,1994156.3745179)
--(axis cs:161,1804612.2230935)
--(axis cs:162,1770870.83178379)
--(axis cs:163,1957178.92375914)
--(axis cs:164,1865429.89161447)
--(axis cs:165,1761201.66893512)
--(axis cs:166,2173627.20787503)
--(axis cs:167,2573036.80378769)
--(axis cs:168,2751114.10019677)
--(axis cs:169,3368432.53439754)
--(axis cs:170,2822991.69449518)
--(axis cs:171,2827880.58162718)
--(axis cs:172,3156146.97892718)
--(axis cs:173,2868487.92713356)
--(axis cs:174,2481814.03543174)
--(axis cs:175,2802103.10096557)
--(axis cs:176,3029783.95331893)
--(axis cs:177,3070859.3991271)
--(axis cs:178,2838448.90628099)
--(axis cs:179,2634145.4392334)
--(axis cs:180,2314398.10511675)
--(axis cs:181,2191986.20783152)
--(axis cs:182,2141446.8204371)
--(axis cs:183,2086454.76239819)
--(axis cs:184,2556867.35161557)
--(axis cs:185,2749143.25609097)
--(axis cs:186,2625234.48328144)
--(axis cs:187,2761216.29833959)
--(axis cs:188,3172414.78039509)
--(axis cs:189,2653178.88705761)
--(axis cs:190,2465102.90752127)
--(axis cs:191,2914993.48650884)
--(axis cs:192,2527100.17346149)
--(axis cs:193,1884612.46308245)
--(axis cs:194,1858141.66177703)
--(axis cs:195,2095798.13939424)
--(axis cs:196,2297021.52268211)
--(axis cs:197,2187667.24956207)
--(axis cs:198,3215140.64306278)
--(axis cs:199,3019118.373724)
--(axis cs:200,2655526.05982641)
--(axis cs:201,3283815.41181449)
--(axis cs:202,3156699.54029868)
--(axis cs:203,3218577.13149694)
--(axis cs:204,3228865.03720597)
--(axis cs:205,2672366.31812637)
--(axis cs:206,2477598.48426895)
--(axis cs:207,2562523.9534744)
--(axis cs:208,3305965.53734865)
--(axis cs:209,3303271.23242392)
--(axis cs:210,2887539.29517577)
--(axis cs:211,2698683.36296736)
--(axis cs:212,2279574.05320773)
--(axis cs:213,2145969.97729225)
--(axis cs:214,2393042.28475496)
--(axis cs:215,2988113.08631118)
--(axis cs:216,2796657.03278942)
--(axis cs:217,2009113.95496144)
--(axis cs:218,2029813.95601279)
--(axis cs:219,1820744.17166001)
--(axis cs:220,1348134.7315407)
--(axis cs:221,1419515.02435165)
--(axis cs:222,1591890.64046046)
--(axis cs:223,2089530.67400725)
--(axis cs:224,2071613.11872243)
--(axis cs:225,2763569.93739902)
--(axis cs:226,2311816.07289185)
--(axis cs:227,1977670.47110779)
--(axis cs:228,1851755.61803422)
--(axis cs:229,1457080.9703045)
--(axis cs:230,1184194.34728321)
--(axis cs:231,1194871.34412283)
--(axis cs:232,1121144.81032184)
--(axis cs:233,1103668.1901052)
--(axis cs:234,1144794.13493466)
--(axis cs:235,984647.668656982)
--(axis cs:236,840140.739685988)
--(axis cs:237,810924.716194153)
--(axis cs:238,720744.569094463)
--(axis cs:239,736518.345443213)
--(axis cs:240,1148700.87509862)
--(axis cs:241,1745527.09154187)
--(axis cs:242,915280.57871803)
--(axis cs:243,696177.949145666)
--(axis cs:244,717597.004021776)
--(axis cs:245,685518.331151039)
--(axis cs:246,818727.12914509)
--(axis cs:247,830960.969075284)
--(axis cs:248,600030.966959707)
--(axis cs:249,620197.040933382)
--(axis cs:250,606839.365976791)
--(axis cs:251,525186.767373963)
--(axis cs:252,514037.262578017)
--(axis cs:253,549764.126209874)
--(axis cs:254,569333.81578033)
--(axis cs:255,943384.497564329)
--(axis cs:256,770352.764155718)
--(axis cs:257,985898.318832785)
--(axis cs:258,682867.100219618)
--(axis cs:259,593220.850675466)
--(axis cs:260,370061.891694097)
--(axis cs:261,482313.630388557)
--(axis cs:262,429109.781861856)
--(axis cs:263,411647.003890416)
--(axis cs:264,414922.444534143)
--(axis cs:265,407265.096487976)
--(axis cs:266,371387.553393005)
--(axis cs:267,348644.486828841)
--(axis cs:268,313145.611911004)
--(axis cs:269,556539.251396435)
--(axis cs:270,886955.391464816)
--(axis cs:271,617561.3888677)
--(axis cs:272,507554.304947483)
--(axis cs:273,1662368.50933833)
--(axis cs:274,1585426.90730958)
--(axis cs:275,2719206.20252392)
--(axis cs:276,2390238.76604041)
--(axis cs:277,3075450.29141305)
--(axis cs:278,2290262.23203147)
--(axis cs:279,2361080.89733861)
--(axis cs:280,2746299.66528372)
--(axis cs:281,3218735.45555006)
--(axis cs:282,2442430.05355714)
--(axis cs:283,1665756.5290219)
--(axis cs:284,2137434.56990875)
--(axis cs:285,2822493.28421123)
--(axis cs:286,2407720.57647722)
--(axis cs:287,2972619.82189113)
--(axis cs:288,2292487.88069672)
--(axis cs:289,2337325.55852533)
--(axis cs:290,2182479.29231682)
--(axis cs:291,2121032.59791687)
--(axis cs:292,3060212.49460028)
--(axis cs:293,2412697.17679024)
--(axis cs:294,2248387.88442882)
--(axis cs:295,1841667.57083758)
--(axis cs:296,1761474.3159224)
--(axis cs:297,1713398.15603066)
--(axis cs:298,2163811.99319971)
--(axis cs:299,1897571.0667683)
--(axis cs:300,2783029.41833862)
--(axis cs:301,1935585.09966761)
--(axis cs:302,1629266.49647438)
--(axis cs:303,1310877.81148846)
--(axis cs:304,1271999.45155765)
--(axis cs:305,1263628.55342375)
--(axis cs:306,1560936.09036787)
--(axis cs:307,1878675.76272327)
--(axis cs:308,1273916.58033497)
--(axis cs:309,1042896.16024937)
--(axis cs:310,1003112.87874642)
--(axis cs:311,1024060.69510002)
--(axis cs:312,994107.763113682)
--(axis cs:313,2253548.17433246)
--(axis cs:314,1593122.12855212)
--(axis cs:315,2440088.94976755)
--(axis cs:316,1697543.51611334)
--(axis cs:317,1216145.22585435)
--(axis cs:318,1095664.62322794)
--(axis cs:319,1091600.17391703)
--(axis cs:320,1094717.78700986)
--(axis cs:321,1146718.21310199)
--(axis cs:322,1821298.51982704)
--(axis cs:323,1493747.99054957)
--(axis cs:324,2151524.2624527)
--(axis cs:325,1968006.23695298)
--(axis cs:326,1796863.25758039)
--(axis cs:327,2252745.69041667)
--(axis cs:328,1511222.89962751)
--(axis cs:329,1397833.73995367)
--(axis cs:330,1540983.06082926)
--(axis cs:331,2099521.82491372)
--(axis cs:332,1947944.54689149)
--(axis cs:333,2729170.79901309)
--(axis cs:334,1972979.52149956)
--(axis cs:335,2185613.57949197)
--(axis cs:336,2291045.694585)
--(axis cs:337,2075443.36858802)
--(axis cs:338,1790116.55986099)
--(axis cs:339,2033992.23280244)
--(axis cs:340,2037126.28150423)
--(axis cs:341,2062164.52565607)
--(axis cs:342,1836317.89156801)
--(axis cs:343,1654682.41588968)
--(axis cs:344,2145264.720924)
--(axis cs:345,2050245.55226243)
--(axis cs:346,1895231.83856055)
--(axis cs:347,1743046.61753171)
--(axis cs:348,2121865.12130509)
--(axis cs:349,2986226.82215484)
--(axis cs:350,2139975.67475675)
--(axis cs:351,2169497.53575225)
--(axis cs:352,1851286.25994985)
--(axis cs:353,2403223.09424456)
--(axis cs:354,2137424.8792569)
--(axis cs:355,2382319.88725926)
--(axis cs:356,2495867.10100523)
--(axis cs:357,3413255.81718825)
--(axis cs:358,2599094.41835204)
--(axis cs:359,2200222.46233254)
--(axis cs:360,2262359.88224933)
--(axis cs:361,2272846.60378728)
--(axis cs:362,2486539.17574427)
--(axis cs:363,3366857.8939616)
--(axis cs:364,2865547.41801762)
--(axis cs:365,3233655.56125793)
--(axis cs:366,2735343.80164705)
--(axis cs:367,2581700.19092771)
--(axis cs:368,2888407.12326097)
--(axis cs:369,2341832.49109073)
--(axis cs:370,2790391.2767774)
--(axis cs:371,2838173.54169188)
--(axis cs:372,3313178.83158976)
--(axis cs:373,3189439.5701019)
--(axis cs:374,3092488.18440201)
--(axis cs:375,3698937.85106678)
--(axis cs:376,3065178.73256512)
--(axis cs:377,3927790.92243627)
--(axis cs:378,4070303.78571709)
--(axis cs:379,3706873.2602438)
--(axis cs:380,3009714.962002)
--(axis cs:381,3220571.04696197)
--(axis cs:382,3506549.82864025)
--(axis cs:383,3689186.60443217)
--(axis cs:384,3568252.55065855)
--(axis cs:385,3578747.8279217)
--(axis cs:386,3536902.70055913)
--(axis cs:387,3170151.2617337)
--(axis cs:388,3213719.27895551)
--(axis cs:389,3272133.06995976)
--(axis cs:390,2997556.4076502)
--(axis cs:391,2595308.29887673)
--(axis cs:392,3373508.48564691)
--(axis cs:393,3170828.07913894)
--(axis cs:394,3733201.91587987)
--(axis cs:395,3508249.39248702)
--(axis cs:396,3161858.28817031)
--(axis cs:397,2662018.994608)
--(axis cs:398,2264916.73665154)
--(axis cs:399,3022826.21408672)
--(axis cs:400,2842092.54745723)
--(axis cs:401,3320314.31380722)
--(axis cs:402,2578986.39670928)
--(axis cs:403,3005920.49716353)
--(axis cs:404,3676165.28301154)
--(axis cs:405,3818206.04736791)
--(axis cs:406,3310407.84612767)
--(axis cs:407,2752359.67012297)
--(axis cs:408,2698998.24304435)
--(axis cs:409,3563700.90100011)
--(axis cs:410,3035504.18370063)
--(axis cs:411,2788432.73707372)
--(axis cs:412,3257919.07961026)
--(axis cs:413,3155496.89219732)
--(axis cs:414,3036066.44266157)
--(axis cs:415,3360725.96391549)
--(axis cs:416,2739124.81299356)
--(axis cs:417,2658257.86643557)
--(axis cs:418,2419134.43556275)
--(axis cs:419,2356321.72492862)
--(axis cs:420,2339521.11008097)
--(axis cs:421,2294757.51699122)
--(axis cs:422,2048074.32644771)
--(axis cs:423,2428345.50017435)
--(axis cs:424,1995858.68177048)
--(axis cs:425,1779226.0101748)
--(axis cs:426,1735646.58952932)
--(axis cs:427,1975808.94175792)
--(axis cs:428,1924837.19105206)
--(axis cs:429,1516162.95898925)
--(axis cs:430,1459930.82610268)
--(axis cs:431,2954710.86048089)
--(axis cs:432,2044415.83827573)
--(axis cs:433,1579138.30478953)
--(axis cs:434,1453750.8511597)
--(axis cs:435,1384077.62606045)
--(axis cs:436,3106347.58976288)
--(axis cs:437,3128653.99886517)
--(axis cs:438,2470238.85655025)
--(axis cs:439,3312616.27815764)
--(axis cs:440,2725107.57127343)
--(axis cs:440,8468514.27120957)
--(axis cs:440,8468514.27120957)
--(axis cs:439,10039706.3081262)
--(axis cs:438,7765190.9113143)
--(axis cs:437,9146894.80509136)
--(axis cs:436,9055335.12458971)
--(axis cs:435,5133147.92786061)
--(axis cs:434,5071361.26168898)
--(axis cs:433,5679570.95855285)
--(axis cs:432,6649658.55006268)
--(axis cs:431,8517644.49998509)
--(axis cs:430,5403254.37968982)
--(axis cs:429,5547473.93975994)
--(axis cs:428,6433339.45360801)
--(axis cs:427,6647739.80718546)
--(axis cs:426,6122404.15342234)
--(axis cs:425,6228245.42310328)
--(axis cs:424,6753041.93870359)
--(axis cs:423,7643312.12582019)
--(axis cs:422,7186220.43863723)
--(axis cs:421,8020733.35170246)
--(axis cs:420,7581468.58885245)
--(axis cs:419,7458463.33880294)
--(axis cs:418,7910648.02920936)
--(axis cs:417,8518228.61111155)
--(axis cs:416,8261614.45317976)
--(axis cs:415,9775659.44904574)
--(axis cs:414,8968196.20123142)
--(axis cs:413,9163303.12509006)
--(axis cs:412,9732463.43774885)
--(axis cs:411,8193626.02950105)
--(axis cs:410,9113206.32088215)
--(axis cs:409,10353207.6919024)
--(axis cs:408,8290954.1839017)
--(axis cs:407,8126778.72052045)
--(axis cs:406,9675238.93633759)
--(axis cs:405,10419349.8424964)
--(axis cs:404,10417371.7657651)
--(axis cs:403,8671244.58621632)
--(axis cs:402,7818866.80623262)
--(axis cs:401,9428300.74661058)
--(axis cs:400,8332421.35856462)
--(axis cs:399,8357531.30695444)
--(axis cs:398,7069238.67769449)
--(axis cs:397,7883840.96205972)
--(axis cs:396,8358316.39071284)
--(axis cs:395,9966675.95498577)
--(axis cs:394,9942863.51818993)
--(axis cs:393,9189972.68078653)
--(axis cs:392,9301780.87843597)
--(axis cs:391,7668343.34716738)
--(axis cs:390,8626157.96022992)
--(axis cs:389,9288371.1963261)
--(axis cs:388,8971812.77799433)
--(axis cs:387,9158438.36185903)
--(axis cs:386,9831451.28403679)
--(axis cs:385,9770821.68385688)
--(axis cs:384,9816994.26857478)
--(axis cs:383,9881054.19176195)
--(axis cs:382,9317612.50419281)
--(axis cs:381,8807748.31837518)
--(axis cs:380,8656281.49919692)
--(axis cs:379,10335155.0867468)
--(axis cs:378,10736474.8581027)
--(axis cs:377,10828657.7070526)
--(axis cs:376,8385184.33085694)
--(axis cs:375,9706424.73947867)
--(axis cs:374,8534451.76519211)
--(axis cs:373,8578101.49633635)
--(axis cs:372,8823743.7357739)
--(axis cs:371,8011606.72057305)
--(axis cs:370,7890097.78020723)
--(axis cs:369,7123811.83609458)
--(axis cs:368,8290935.74646348)
--(axis cs:367,7378202.6562818)
--(axis cs:366,7980559.34989096)
--(axis cs:365,8892299.3373945)
--(axis cs:364,8207371.05373828)
--(axis cs:363,8978010.85550978)
--(axis cs:362,7496170.49605406)
--(axis cs:361,6738317.84815271)
--(axis cs:360,6968946.33144553)
--(axis cs:359,6806538.16658519)
--(axis cs:358,7341981.88419128)
--(axis cs:357,9533466.52104834)
--(axis cs:356,7310909.68629614)
--(axis cs:355,6988550.25105212)
--(axis cs:354,6488560.82949682)
--(axis cs:353,7462962.79012022)
--(axis cs:352,6044622.31752254)
--(axis cs:351,6570285.16596493)
--(axis cs:350,6438456.93125432)
--(axis cs:349,8207535.38224713)
--(axis cs:348,6311129.6980059)
--(axis cs:347,5649600.34101731)
--(axis cs:346,5892656.54154904)
--(axis cs:345,6412536.94917041)
--(axis cs:344,6411704.26798966)
--(axis cs:343,5358311.99966102)
--(axis cs:342,6038208.83926813)
--(axis cs:341,6412181.08646009)
--(axis cs:340,6270445.9749389)
--(axis cs:339,6338295.16796469)
--(axis cs:338,5726200.5728375)
--(axis cs:337,6552640.91414492)
--(axis cs:336,7232933.57336446)
--(axis cs:335,6687037.36675948)
--(axis cs:334,6096794.17665751)
--(axis cs:333,7751678.6074329)
--(axis cs:332,6192394.90292075)
--(axis cs:331,6401626.394748)
--(axis cs:330,5341824.21517767)
--(axis cs:329,4919931.47314437)
--(axis cs:328,5123943.49334585)
--(axis cs:327,6654612.26613565)
--(axis cs:326,5911219.49928004)
--(axis cs:325,6471321.03051873)
--(axis cs:324,6781269.52695804)
--(axis cs:323,5320655.92100647)
--(axis cs:322,5972083.20301024)
--(axis cs:321,4238877.88040813)
--(axis cs:320,4272375.97072825)
--(axis cs:319,4198811.70924511)
--(axis cs:318,4002066.96568522)
--(axis cs:317,4339567.02395677)
--(axis cs:316,5629616.48658981)
--(axis cs:315,7573798.02511815)
--(axis cs:314,5310455.76127302)
--(axis cs:313,6844339.50052388)
--(axis cs:312,3970127.85418914)
--(axis cs:311,3946938.8719905)
--(axis cs:310,3870960.7586072)
--(axis cs:309,3972914.71243537)
--(axis cs:308,4507840.19986769)
--(axis cs:307,6066404.62170527)
--(axis cs:306,5258118.6084897)
--(axis cs:305,4498243.37143651)
--(axis cs:304,4587921.47179554)
--(axis cs:303,4649117.30882883)
--(axis cs:302,5389623.73340166)
--(axis cs:301,6146607.44980771)
--(axis cs:300,7839656.33139697)
--(axis cs:299,6310588.24483788)
--(axis cs:298,6884421.20323664)
--(axis cs:297,5844577.02735591)
--(axis cs:296,5800125.06458357)
--(axis cs:295,5900426.2005661)
--(axis cs:294,6742015.14010158)
--(axis cs:293,7099470.21600052)
--(axis cs:292,8538705.27926245)
--(axis cs:291,6619332.35513846)
--(axis cs:290,6223417.04532504)
--(axis cs:289,6612102.24793813)
--(axis cs:288,6612638.49065374)
--(axis cs:287,8332950.49833962)
--(axis cs:286,7202380.9175399)
--(axis cs:285,8104385.82035977)
--(axis cs:284,6676834.69170673)
--(axis cs:283,5662647.76345485)
--(axis cs:282,7101608.06505745)
--(axis cs:281,8544638.60368691)
--(axis cs:280,7859903.86057512)
--(axis cs:279,7087965.18542966)
--(axis cs:278,6909233.38151602)
--(axis cs:277,8552893.16030367)
--(axis cs:276,7069200.20847812)
--(axis cs:275,7621420.76080745)
--(axis cs:274,5269299.6573687)
--(axis cs:273,5526746.31787464)
--(axis cs:272,2869473.33569037)
--(axis cs:271,3246491.09437364)
--(axis cs:270,3693879.81318761)
--(axis cs:269,2836840.99488535)
--(axis cs:268,2091642.56294546)
--(axis cs:267,2173994.75061496)
--(axis cs:266,2176256.18102822)
--(axis cs:265,2182909.76401104)
--(axis cs:264,2326765.71994536)
--(axis cs:263,2306938.77771633)
--(axis cs:262,2343015.68046183)
--(axis cs:261,2498903.02737267)
--(axis cs:260,2320905.45415033)
--(axis cs:259,2984337.88196705)
--(axis cs:258,3071218.05804045)
--(axis cs:257,4100496.61019609)
--(axis cs:256,3397329.96798883)
--(axis cs:255,3957880.22278485)
--(axis cs:254,2791217.42487295)
--(axis cs:253,2770739.69160339)
--(axis cs:252,2596707.74793632)
--(axis cs:251,2514468.85054113)
--(axis cs:250,2746899.53748757)
--(axis cs:249,2925585.66546177)
--(axis cs:248,2880836.10956409)
--(axis cs:247,3620720.66135203)
--(axis cs:246,3362402.86558263)
--(axis cs:245,3057271.7080279)
--(axis cs:244,3077379.87203075)
--(axis cs:243,3189668.1602532)
--(axis cs:242,3729216.18824906)
--(axis cs:241,5464442.17419071)
--(axis cs:240,4261556.11878098)
--(axis cs:239,3226969.89857928)
--(axis cs:238,3158382.09288054)
--(axis cs:237,3357724.04460681)
--(axis cs:236,3387775.8159397)
--(axis cs:235,3676020.63106808)
--(axis cs:234,4055532.99660537)
--(axis cs:233,4014295.22729327)
--(axis cs:232,4059283.63347341)
--(axis cs:231,4333629.02638414)
--(axis cs:230,4289782.02216236)
--(axis cs:229,4986057.01388822)
--(axis cs:228,5775924.9123867)
--(axis cs:227,6534561.95733977)
--(axis cs:226,7113945.35809606)
--(axis cs:225,8002582.96058965)
--(axis cs:224,6311322.02053029)
--(axis cs:223,6290875.86173677)
--(axis cs:222,5400680.08812096)
--(axis cs:221,5272346.41538211)
--(axis cs:220,4938738.18368794)
--(axis cs:219,5783116.35337549)
--(axis cs:218,6263830.13380149)
--(axis cs:217,6383451.50075142)
--(axis cs:216,8003959.19074472)
--(axis cs:215,8365581.96594845)
--(axis cs:214,7445289.55614172)
--(axis cs:213,6829620.7567365)
--(axis cs:212,6984016.31812982)
--(axis cs:211,7674727.71673714)
--(axis cs:210,7972227.16634941)
--(axis cs:209,9018955.41064145)
--(axis cs:208,9008524.94514608)
--(axis cs:207,7628717.63074927)
--(axis cs:206,7254188.78771829)
--(axis cs:205,7582936.56186526)
--(axis cs:204,8943514.61085846)
--(axis cs:203,8948496.58963924)
--(axis cs:202,9071309.00531138)
--(axis cs:201,9435658.67895392)
--(axis cs:200,8054075.34622409)
--(axis cs:199,8455798.33663977)
--(axis cs:198,9066296.30073623)
--(axis cs:197,6763669.5120331)
--(axis cs:196,6753206.83492994)
--(axis cs:195,6427541.98210287)
--(axis cs:194,5932810.93848901)
--(axis cs:193,5952687.57597198)
--(axis cs:192,7258549.1404329)
--(axis cs:191,7978673.2113333)
--(axis cs:190,6959899.88294984)
--(axis cs:189,7482988.96569898)
--(axis cs:188,8685553.83082091)
--(axis cs:187,8524973.67017006)
--(axis cs:186,7915534.1873493)
--(axis cs:185,7927517.65351043)
--(axis cs:184,7681940.16078519)
--(axis cs:183,6400496.73878111)
--(axis cs:182,6384675.91640681)
--(axis cs:181,6617848.05089659)
--(axis cs:180,6782634.71512371)
--(axis cs:179,7875091.39261074)
--(axis cs:178,8213319.9518703)
--(axis cs:177,8838959.06633769)
--(axis cs:176,8497778.64058368)
--(axis cs:175,8208571.05617017)
--(axis cs:174,7322974.06607232)
--(axis cs:173,8281239.99297942)
--(axis cs:172,8871250.30660448)
--(axis cs:171,8179265.18125336)
--(axis cs:170,7766326.29233545)
--(axis cs:169,9221806.78884793)
--(axis cs:168,7855379.54434907)
--(axis cs:167,7443792.05972836)
--(axis cs:166,6400461.34692745)
--(axis cs:165,5635043.90655367)
--(axis cs:164,5643660.93903909)
--(axis cs:163,6135085.44923656)
--(axis cs:162,5779121.24916304)
--(axis cs:161,5660297.35203936)
--(axis cs:160,6142637.53995577)
--(axis cs:159,5703796.59014026)
--(axis cs:158,7326385.99196898)
--(axis cs:157,5610858.57659351)
--(axis cs:156,6232845.73876787)
--(axis cs:155,5353585.70575983)
--(axis cs:154,5226406.60190592)
--(axis cs:153,4312630.22937169)
--(axis cs:152,4500976.93176377)
--(axis cs:151,4694135.39289036)
--(axis cs:150,6513411.53693706)
--(axis cs:149,6321101.67539866)
--(axis cs:148,8370726.03238143)
--(axis cs:147,5773912.98534697)
--(axis cs:146,6339569.37245012)
--(axis cs:145,8258192.13021907)
--(axis cs:144,6331329.5604855)
--(axis cs:143,7331295.64981313)
--(axis cs:142,6998648.60456457)
--(axis cs:141,5205540.31115581)
--(axis cs:140,4826360.56417766)
--(axis cs:139,4816916.9482833)
--(axis cs:138,5593368.61322351)
--(axis cs:137,5860718.6706369)
--(axis cs:136,6210774.61561455)
--(axis cs:135,6508831.8478338)
--(axis cs:134,7459225.8167795)
--(axis cs:133,7479187.3113884)
--(axis cs:132,8735578.45755573)
--(axis cs:131,8336490.30325338)
--(axis cs:130,9077080.39058312)
--(axis cs:129,8017029.85068207)
--(axis cs:128,7499942.0377433)
--(axis cs:127,8332196.00178005)
--(axis cs:126,7649019.55371867)
--(axis cs:125,8017350.70603796)
--(axis cs:124,7491789.44881309)
--(axis cs:123,6729971.33538378)
--(axis cs:122,6537146.08071662)
--(axis cs:121,7109136.969358)
--(axis cs:120,8196790.11299205)
--(axis cs:119,6997392.81232232)
--(axis cs:118,7029020.96941806)
--(axis cs:117,6791105.90847692)
--(axis cs:116,8741153.43434293)
--(axis cs:115,7566054.95570782)
--(axis cs:114,5967620.963923)
--(axis cs:113,6031205.70148454)
--(axis cs:112,6133431.209637)
--(axis cs:111,5755120.3703706)
--(axis cs:110,6710403.59240197)
--(axis cs:109,6306593.06139742)
--(axis cs:108,6349567.92526045)
--(axis cs:107,6551898.57223601)
--(axis cs:106,5421044.16677557)
--(axis cs:105,5052625.24676424)
--(axis cs:104,5409272.47565483)
--(axis cs:103,6011745.16799244)
--(axis cs:102,5938729.41939989)
--(axis cs:101,6783492.55271044)
--(axis cs:100,6179630.00223165)
--(axis cs:99,7366257.51314702)
--(axis cs:98,7509815.92115595)
--(axis cs:97,6609887.4412718)
--(axis cs:96,5219675.07908717)
--(axis cs:95,5832490.17872798)
--(axis cs:94,5520242.52674437)
--(axis cs:93,5642881.66943901)
--(axis cs:92,5713682.43069275)
--(axis cs:91,6500308.8340077)
--(axis cs:90,7426973.10582046)
--(axis cs:89,8699885.38801785)
--(axis cs:88,6542555.51714007)
--(axis cs:87,5899204.16894006)
--(axis cs:86,7880203.07811734)
--(axis cs:85,7193840.98040057)
--(axis cs:84,6224990.86017604)
--(axis cs:83,6237669.54335465)
--(axis cs:82,7577712.07325957)
--(axis cs:81,4807906.03004025)
--(axis cs:80,4816420.76734778)
--(axis cs:79,5455985.75013368)
--(axis cs:78,5466114.43389414)
--(axis cs:77,6181207.78517278)
--(axis cs:76,5779272.01819913)
--(axis cs:75,6046319.74160825)
--(axis cs:74,6391408.43432596)
--(axis cs:73,7444507.51685993)
--(axis cs:72,8440921.87966305)
--(axis cs:71,5544916.96557676)
--(axis cs:70,4816478.32517866)
--(axis cs:69,4840560.65746931)
--(axis cs:68,5171476.84358372)
--(axis cs:67,5608256.11408654)
--(axis cs:66,6584035.404225)
--(axis cs:65,7186888.45475693)
--(axis cs:64,7568677.97496108)
--(axis cs:63,7110890.93873187)
--(axis cs:62,7146492.09907181)
--(axis cs:61,7073534.42710721)
--(axis cs:60,8247814.54644512)
--(axis cs:59,7165287.75263597)
--(axis cs:58,7774726.78606478)
--(axis cs:57,9467660.38044876)
--(axis cs:56,8799018.20247963)
--(axis cs:55,7901728.85668527)
--(axis cs:54,7958796.39922973)
--(axis cs:53,7962101.79268181)
--(axis cs:52,9181847.49741363)
--(axis cs:51,7454520.1205949)
--(axis cs:50,9365969.03800652)
--(axis cs:49,7010885.87009495)
--(axis cs:48,8209824.97106166)
--(axis cs:47,8979482.60823884)
--(axis cs:46,9769518.67703887)
--(axis cs:45,9825905.24072632)
--(axis cs:44,9468819.8939323)
--(axis cs:43,10262833.4877008)
--(axis cs:42,10023056.9511619)
--(axis cs:41,7278118.4367674)
--(axis cs:40,10061931.1683958)
--(axis cs:39,7927451.10226852)
--(axis cs:38,9047482.64077716)
--(axis cs:37,9059309.9480774)
--(axis cs:36,7054222.18987487)
--(axis cs:35,6744508.50867894)
--(axis cs:34,6541034.11427252)
--(axis cs:33,7822811.552867)
--(axis cs:32,6819272.0793942)
--(axis cs:31,6006224.24225151)
--(axis cs:30,6935223.28957539)
--(axis cs:29,8106433.51393717)
--(axis cs:28,6893545.24906342)
--(axis cs:27,6694930.11073675)
--(axis cs:26,7049568.34414898)
--(axis cs:25,7246235.0342383)
--(axis cs:24,8199682.16442428)
--(axis cs:23,8045006.94969434)
--(axis cs:22,10006727.7269345)
--(axis cs:21,10054106.5697591)
--(axis cs:20,10023365.9642084)
--(axis cs:19,8533953.20579573)
--(axis cs:18,8751370.44077332)
--(axis cs:17,8616303.22997001)
--(axis cs:16,9616468.1184713)
--(axis cs:15,8382732.71506628)
--(axis cs:14,9212554.81444745)
--(axis cs:13,9793815.07535442)
--(axis cs:12,5998922.09063701)
--(axis cs:11,6913109.76681409)
--(axis cs:10,7561386.65486947)
--(axis cs:9,8520074.34283876)
--(axis cs:8,7781254.52748094)
--(axis cs:7,8917934.74322633)
--(axis cs:6,9069106.33780628)
--(axis cs:5,9185333.35737421)
--(axis cs:4,9392911.57923683)
--(axis cs:3,8860267.94221521)
--(axis cs:2,9134110.40517918)
--(axis cs:1,9629747.98715215)
--(axis cs:0,7711796.89815992)
--cycle;

\addplot [only marks, red, mark=asterisk, mark size=1.2]
table {%
0 7149100
1 5658700
2 5278900
3 6046300
4 5661300
5 5928400
6 5221600
7 4428300
8 4796900
9 4144900
10 3767800
11 3453700
12 7709600
13 7709600
14 5534200
15 5959400
16 4790300
17 4820800
18 4843600
19 7379500
20 7029500
21 6528800
22 5612900
23 5680700
24 4828500
25 4393800
26 4046200
27 4235000
28 5846400
29 4380600
30 3798100
31 4207000
32 4958500
33 3948900
34 4204500
35 4362400
36 6925600
37 6458900
38 5364400
39 5995100
40 4883700
41 8473900
42 7491200
43 7058500
44 6415100
45 6576600
46 5782900
47 5479600
48 4524700
49 7920800
50 5019300
51 8852200
52 5798500
53 5956900
54 5249100
55 6044900
56 8281700
57 5647300
58 4850400
59 7208100
60 4797400
61 5228600
62 5174700
63 6327900
64 4725400
65 3995200
66 3364900
67 3007100
68 2773200
69 2709100
70 2954100
71 10084800
72 5496600
73 4964200
74 4087900
75 3579900
76 4194200
77 3393200
78 3388600
79 2876100
80 2652000
81 6194500
82 3557200
83 3872300
84 5395800
85 6194500
86 3897200
87 4466600
88 5014200
89 4057300
90 3780500
91 3121200
92 3035300
93 2766900
94 2847700
95 2480800
96 4124200
97 4759100
98 4759100
99 3726000
100 3217600
101 3130400
102 3074100
103 2881200
104 2418100
105 2747900
106 4268200
107 3733900
108 3835400
109 3862200
110 3172400
111 3248600
112 2834700
113 3209700
114 6202700
115 7724000
116 4160700
117 3946100
118 4168600
119 4587900
120 3794300
121 3508300
122 4960300
123 5194600
124 4757200
125 5037300
126 4577300
127 4343500
128 4838900
129 5547100
130 4331800
131 4591600
132 3884200
133 4230800
134 3919900
135 3447000
136 3375500
137 3653600
138 2934400
139 2704200
140 2984100
141 4224400
142 4231700
143 3532700
144 5728500
145 3739000
146 3302000
147 6888800
148 4376700
149 4427800
150 3371200
151 3079700
152 3038200
153 3045500
154 2783900
155 4552900
156 3226900
157 5103700
158 3608000
159 3633800
160 3034400
161 3512000
162 2962500
163 3088900
164 2882200
165 4026900
166 4617400
167 5124600
168 7773700
169 5641200
170 6349000
171 6525700
172 5305400
173 4846800
174 5394400
175 5407800
176 5351500
177 5015200
178 5513000
179 4517400
180 4271000
181 4220700
182 4023200
183 5952700
184 6192100
185 6109600
186 6124700
187 6094300
188 4790000
189 4472500
190 5605800
191 5186300
192 4291800
193 4173200
194 4022700
195 4806200
196 4205000
197 6726900
198 5977600
199 5216300
200 7415200
201 6250300
202 6979200
203 7303600
204 5190000
205 5815200
206 5235700
207 9781600
208 6692300
209 6353100
210 5431300
211 5204300
212 4729600
213 5068200
214 6122000
215 5059400
216 3967400
217 3688700
218 3742700
219 3118400
220 2941300
221 3205200
222 3930500
223 4250700
224 5111600
225 4421400
226 3775000
227 3451100
228 2984700
229 2712000
230 2500800
231 2312500
232 2314000
233 2447400
234 2275900
235 2090500
236 2015900
237 1912100
238 1875000
239 2692700
240 4866700
241 2094200
242 1805300
243 1607100
244 1474500
245 1870800
246 1663200
247 1451800
248 1333600
249 1245100
250 1207500
251 1180600
252 1154200
253 1123100
254 2155000
255 1463400
256 1873100
257 1340100
258 1340100
259 1230700
260 1160200
261 1104100
262 1055500
263 1022100
264 989200
265 959500
266 932600
267 905300
268 1649700
269 2359000
270 1346500
271 1139800
272 3925300
273 3750000
274 6550700
275 4871300
276 7055400
277 5808600
278 5128100
279 6022300
280 6574800
281 4805000
282 3766700
283 4804600
284 5674600
285 4063700
286 5243600
287 4373900
288 6620200
289 4486100
290 4234800
291 7840200
292 4896400
293 4088400
294 3589600
295 3241900
296 3224300
297 3727800
298 3196100
299 4726700
300 3278200
301 3079800
302 3213200
303 3031100
304 2591200
305 3070900
306 3489500
307 2517000
308 2212500
309 1990900
310 1849000
311 1947800
312 4983000
313 3274900
314 7221400
315 3673500
316 2890600
317 2486900
318 2228700
319 2038600
320 2631000
321 3451500
322 2727900
323 4411500
324 3607800
325 3217900
326 4213100
327 3042600
328 2697800
329 2699600
330 3654500
331 3325900
332 5257800
333 3309200
334 3438000
335 3834900
336 3367600
337 3426900
338 3866300
339 3354100
340 3635000
341 3162800
342 3024600
343 3957700
344 3484400
345 3246600
346 3211900
347 4457900
348 5736300
349 4331200
350 4098200
351 3952600
352 3992500
353 3673100
354 3894600
355 4672000
356 7140300
357 4707200
358 3695700
359 3964600
360 3656800
361 4355000
362 5632900
363 4689600
364 5576800
365 4433700
366 4036500
367 5008900
368 3930400
369 5319100
370 5797900
371 6756400
372 6488500
373 5668600
374 6442200
375 5849400
376 7113400
377 8568000
378 6312900
379 5103100
380 5300900
381 5946300
382 6725000
383 5804400
384 7214000
385 6788500
386 5741800
387 5898600
388 6116900
389 5599500
390 5021900
391 8005800
392 7747600
393 10203800
394 10338700
395 6655400
396 5403000
397 4469400
398 6673500
399 6047800
400 7027700
401 5099400
402 6563300
403 9234600
404 10016500
405 7341900
406 5782100
407 5582800
408 8059100
409 5688600
410 4980300
411 7363700
412 6216500
413 6770000
414 6974300
415 5061300
416 5165700
417 4762300
418 4461900
419 4335500
420 3995200
421 3494100
422 4353000
423 3605400
424 3149300
425 2922100
426 3169300
427 2875400
428 2509600
429 2772800
430 7051400
431 4182000
432 3320200
433 2970400
434 2779800
435 8009900
436 8343200
437 5654700
438 10685900
439 6363000
440 5326000
};
\addplot [blue]
table {%
0 4919764.428102
1 6125060.73259371
2 5759561.55352668
3 5489497.38000235
4 5787686.21061255
5 5938797.61917652
6 5734651.05603946
7 5581542.73638677
8 4897426.40563394
9 5368076.17997059
10 4690973.52100726
11 4265500.89727752
12 3637039.09692329
13 6420544.07500609
14 6062797.41861798
15 5357392.19852095
16 6219091.61136573
17 5463314.01214733
18 5538426.27372787
19 5377172.15081124
20 6505068.20733921
21 6486892.84948822
22 6480759.28870205
23 5156803.98782467
24 5169889.35502024
25 4509974.56137509
26 4420245.69073135
27 4195355.55987206
28 4226459.5735334
29 5125183.59148284
30 4242298.38328417
31 3716441.21343067
32 4146906.69859466
33 4822342.76063243
34 3891054.03555904
35 4117306.11707976
36 4254455.75257706
37 5624969.11749767
38 5741713.70670842
39 4940064.76234034
40 6287108.6668868
41 4479224.55481638
42 6371983.9884622
43 6372891.45440546
44 6039235.1691666
45 6204259.44738888
46 6248688.69612969
47 5495767.51826967
48 5113451.89445518
49 4459659.38209673
50 6114605.18232186
51 4795431.00832778
52 6067559.70974823
53 5084033.95432306
54 4955394.97916213
55 4842684.70513546
56 5480611.606925
57 6105930.03621569
58 4987174.43655527
59 4603123.08476982
60 5447492.75341678
61 4623272.5678363
62 4553057.06903027
63 4427905.36263189
64 4832436.69585365
65 4381378.83462913
66 3876334.47801746
67 3306212.63998362
68 3051266.76336012
69 2850265.30968202
70 2721665.14992146
71 3214392.86699295
72 5486962.03539621
73 4592951.54048708
74 4070115.18254397
75 3708387.92749697
76 3451230.98478887
77 3719216.74820482
78 3189000.30836196
79 3218672.22203675
80 2721672.87386838
81 2769129.19573084
82 4727533.10988073
83 3682683.56118713
84 3673140.60720199
85 4431740.78987933
86 4836187.97090091
87 3481584.58931247
88 3902260.31980423
89 5153852.21948571
90 4564720.02043795
91 3961108.90657581
92 3408787.53245561
93 3250024.9040322
94 3198297.3920748
95 3327541.86570861
96 2922170.7955967
97 3908985.51325215
98 4581784.37844762
99 4542119.6386054
100 3776891.6651951
101 3950123.36744544
102 3713609.32611098
103 3671697.04728814
104 3193060.94704022
105 3132460.90645311
106 3335843.82489692
107 4110774.53190123
108 3937284.2117534
109 3950665.92416943
110 4257274.62334635
111 3579020.28215718
112 3719341.70555143
113 3642909.59892881
114 3727187.37666299
115 4936741.18723733
116 5838171.87713149
117 4362664.54581487
118 4562502.30728618
119 4372329.30946886
120 5272072.96107484
121 4551801.17655961
122 4190585.43960491
123 4351075.91560901
124 4787862.71390196
125 5148222.13224188
126 4916069.64500811
127 5351939.60572352
128 4785187.60555435
129 5195032.58348336
130 5850710.29834149
131 5219153.85339782
132 5572219.50431659
133 4789836.77988068
134 4660872.42889136
135 4023382.60621011
136 3866305.42194935
137 3610162.3797514
138 3445456.27257628
139 2970644.70244934
140 2926266.79848865
141 3242346.2137148
142 4479014.79991043
143 4669710.78173811
144 4003717.19625852
145 5259329.71131909
146 3920862.87203822
147 3539730.19729703
148 5297387.91966794
149 3919128.43122096
150 4087904.19266838
151 2736634.58606853
152 2687185.04463815
153 2564283.95857691
154 3066894.16132126
155 3216386.01780615
156 3866374.67696856
157 3331780.44775342
158 4587592.6390658
159 3575740.12440628
160 3774149.02556037
161 3498297.90960693
162 3475420.23274369
163 3730512.8067097
164 3495314.26813603
165 3406110.45182382
166 4038018.59222274
167 4676280.46540998
168 4972284.93152858
169 5883125.53342344
170 4939575.39961879
171 5205562.83335026
172 5633438.60862127
173 5161725.84970467
174 4640513.66910302
175 5184194.83628205
176 5478886.01588467
177 5602393.02651795
178 5165866.46094523
179 4892233.67993705
180 4290297.74483562
181 4115987.57277452
182 3951244.38883249
183 3964949.01735234
184 4800109.58331969
185 5076491.90990459
186 4919956.383049
187 5243065.96094688
188 5517996.73989949
189 4797438.57633322
190 4441886.19835629
191 5187396.30708756
192 4590323.72695013
193 3629692.18495805
194 3602462.75503938
195 4024919.53214893
196 4259770.52625593
197 4182641.39075782
198 5809098.23724756
199 5427491.65531024
200 4949511.58166822
201 5877242.85814294
202 5746312.49830316
203 5670049.5198904
204 5706625.02274876
205 4747341.63616926
206 4514201.23193507
207 4697192.4661843
208 5868908.96877339
209 5822277.06166226
210 5144846.43795229
211 4873177.63195685
212 4319409.41271518
213 4144916.54224919
214 4501361.79153398
215 5284396.31522001
216 5103044.15063604
217 3888714.50062148
218 3837500.69193299
219 3566612.88226331
220 2833369.53536684
221 3018549.74874569
222 3222063.96936224
223 3839733.06946668
224 3875195.72412598
225 5074291.72873089
226 4379983.56001465
227 3916978.08121607
228 3541377.24265211
229 2957616.95140378
230 2519651.14409342
231 2523168.91140957
232 2355178.92094309
233 2370540.01796605
234 2378472.67206983
235 2099450.57997318
236 1890514.58402575
237 1875423.82787838
238 1735329.8654662
239 1774660.20722346
240 2462674.43706849
241 3371751.3598706
242 2052276.39678108
243 1711476.41395624
244 1693840.02894286
245 1668198.82945584
246 1877025.54145431
247 1991174.35520224
248 1547990.27666438
249 1545139.14832456
250 1503527.70345358
251 1346128.19789291
252 1407759.96379815
253 1450107.22089516
254 1492439.48688263
255 2187233.12670278
256 1839064.87444567
257 2301822.19614017
258 1680073.97180103
259 1555493.61912232
260 1171220.14285161
261 1291565.40087798
262 1186518.62025972
263 1168992.49466934
264 1195393.63951938
265 1151151.29320062
266 1078005.96640554
267 1043255.04506711
268 1027216.2541936
269 1478932.7235209
270 2044926.73079276
271 1641173.29753358
272 1443178.73768303
273 3348572.18964479
274 3168546.90808022
275 4847583.88440196
276 4403118.1050015
277 5555075.41785045
278 4278534.85901323
279 4368337.60142065
280 4990689.75478492
281 5559365.27200497
282 4484299.67461918
283 3377240.03223699
284 4119745.78722505
285 5082663.62902666
286 4498818.32442994
287 5317183.15339047
288 4179830.2473217
289 4243720.14062726
290 3950664.15433381
291 4106714.76358832
292 5504890.83479012
293 4408856.93442382
294 4183266.22685524
295 3585319.73968126
296 3500576.07128364
297 3473145.5427009
298 4227742.85292863
299 3820408.69915586
300 4961550.4102966
301 3732120.71322628
302 3261875.58000654
303 2757955.00586753
304 2666822.31370274
305 2622432.25218633
306 3142659.35648085
307 3666644.60549997
308 2645637.57356488
309 2278481.66630453
310 2194493.28273873
311 2258542.22278217
312 2199042.85049266
313 4229975.14770691
314 3154077.16013951
315 4672545.86891673
316 3342553.49739187
317 2568476.48883925
318 2353165.97993772
319 2411769.01364675
320 2423905.19132923
321 2467505.67908853
322 3623091.26170552
323 3083539.63350731
324 4123000.90395837
325 3865568.03026206
326 3571455.12846871
327 4186662.87566579
328 3067742.72053822
329 2888488.94433896
330 3130939.18287465
331 3915936.20574976
332 3760678.94991572
333 4974034.5162746
334 3776108.0006409
335 4119000.25875928
336 4426428.00275216
337 3986801.23293071
338 3501267.44857689
339 3875004.18262302
340 3869221.84628576
341 3965530.69535237
342 3676733.05944294
343 3253238.88176857
344 4011558.80429051
345 3974870.70862715
346 3598252.0596007
347 3439324.77361369
348 3981006.21102353
349 5318700.22769916
350 3963427.26037744
351 4075863.19907861
352 3620170.63055412
353 4595702.05569695
354 4015798.4164411
355 4404359.82212804
356 4586572.97137793
357 6041798.33782005
358 4680349.95159527
359 4240631.16711554
360 4303147.33301799
361 4194801.14087826
362 4598392.47221604
363 5805424.83614636
364 5255463.6373158
365 5746086.40504556
366 4974370.37422457
367 4703310.44947894
368 5301040.10464411
369 4380617.78381116
370 5058013.88262837
371 5216574.51420071
372 5725540.18767451
373 5553615.37116305
374 5494733.08746594
375 6384092.10028074
376 5331890.87183506
377 6966982.68107537
378 7034557.2158274
379 6626448.28683747
380 5474969.94597484
381 5720852.27839126
382 6074323.21729934
383 6385766.9485538
384 6316425.50795462
385 6304411.90819368
386 6355489.16515558
387 5858701.6497416
388 5675126.9643174
389 5820792.14025364
390 5400782.45067951
391 4818453.36485468
392 5905440.9440018
393 5788865.71802419
394 6454782.85538817
395 6373656.36391387
396 5458537.86869027
397 4996407.68652095
398 4336305.93036156
399 5331891.86031243
400 5211792.98479848
401 6064225.63355918
402 4851342.62639399
403 5453875.50434056
404 6591762.58506303
405 6786152.21696127
406 6088267.10425292
407 5124821.01923853
408 5039978.32661487
409 6517727.2166861
410 5636531.50161731
411 5172523.89164109
412 6144061.45516244
413 5862023.40021555
414 5647254.25314422
415 6182127.87009436
416 5097976.44422481
417 5193050.21106026
418 4692609.90752641
419 4596742.05849908
420 4472940.97484864
421 4727689.38792972
422 4204245.29753802
423 4639562.41534176
424 4039167.52142552
425 3597384.81727008
426 3593171.77095641
427 3954588.65155048
428 3846682.09033602
429 3211376.85720327
430 3143562.08113629
431 5425607.26634302
432 4038768.67131601
433 3340928.42585884
434 3033102.37792716
435 2955806.49810289
436 5745112.745178
437 5710514.82859754
438 4719739.86862021
439 6169356.47052346
440 5231176.89819252
};
\addplot [black]
table {%
0 4835911.2156095
1 5944008.79418586
2 5607045.04248772
3 5321791.7281927
4 5691121.48129974
5 5814683.25325133
6 5588997.15680226
7 5455743.51961329
8 4794897.66465935
9 5246235.56662669
10 4581649.8481951
11 4182972.61323834
12 3520490.52811605
13 6310772.95234442
14 5960896.73876227
15 5252586.87528954
16 5999569.71164551
17 5332885.33525275
18 5419711.20753048
19 5245281.57523773
20 6399334.78895733
21 6306126.57321085
22 6347570.40326408
23 5035290.2387905
24 5114006.01786984
25 4386077.01053393
26 4316103.56223678
27 4072954.93536123
28 4132022.23458535
29 5007027.0451894
30 4110239.0603541
31 3625937.58739359
32 4030150.28643827
33 4699292.72497174
34 3775310.83817193
35 3998950.80840032
36 4112610.08565637
37 5465410.05820729
38 5606835.15259438
39 4839344.0224905
40 6145281.61049759
41 4370947.08884568
42 6202029.47986324
43 6145539.91741338
44 5839998.92612554
45 6073662.94925755
46 6160819.74958553
47 5387304.88144665
48 4989905.11767697
49 4325511.00339334
50 5954451.59739837
51 4685218.36242211
52 5932361.35184649
53 4949551.0743691
54 4842490.24789449
55 4747041.8912354
56 5372116.546118
57 5965785.90886282
58 4910349.55505439
59 4516219.73134423
60 5374484.14618319
61 4516021.33415945
62 4453201.04357828
63 4312336.32566985
64 4720906.15251393
65 4287202.90152268
66 3738305.76832128
67 3168076.07931163
68 2914019.13366941
69 2732696.49500576
70 2629589.75027926
71 3101612.9476191
72 5412705.26954802
73 4484893.35909729
74 3959217.74993652
75 3614890.16130636
76 3342031.90333365
77 3598504.72510144
78 3066362.76066014
79 3123594.44407321
80 2602353.60955311
81 2633454.85554031
82 4591876.88777833
83 3566667.86196036
84 3583822.93489605
85 4284323.62500505
86 4700288.38485953
87 3364945.00828675
88 3749403.33851073
89 4987688.0820437
90 4462886.83142346
91 3856043.76821272
92 3274265.85448274
93 3143355.10711371
94 3066462.03884957
95 3231599.2726996
96 2821710.63461521
97 3786276.18359751
98 4473732.30650379
99 4407589.93415045
100 3665943.08946407
101 3853833.44124274
102 3608524.06287824
103 3560544.67084314
104 3090966.31846401
105 3033296.31954546
106 3245787.23701991
107 4015559.77192537
108 3825270.04668762
109 3850317.4173588
110 4158134.20619444
111 3462583.81533279
112 3600113.69192413
113 3548954.48657016
114 3627845.40032151
115 4883275.91416586
116 5711903.34798163
117 4281666.70105241
118 4454516.13527499
119 4274637.29923908
120 5153976.16934041
121 4443873.89287932
122 4085434.77789254
123 4285144.76296845
124 4663534.75614367
125 5051618.9729068
126 4783480.26105482
127 5258179.26464291
128 4701674.36049546
129 5091085.54005021
130 5725353.45549906
131 5056482.52633876
132 5413749.08781032
133 4689643.34073019
134 4554641.05603123
135 3937004.77982186
136 3770776.57171701
137 3515040.42922297
138 3349485.23745464
139 2918764.81984797
140 2824960.16571276
141 3120970.47281839
142 4369115.98220075
143 4574616.36495635
144 3908122.00429253
145 5088841.0601702
146 3831888.19407287
147 3411222.08857382
148 5201908.80493862
149 3816945.32637854
150 4012450.86121678
151 2673895.08321695
152 2596965.32395807
153 2477904.14581726
154 3008279.3904064
155 3138572.91398871
156 3789898.24895129
157 3246671.71211938
158 4511583.20800241
159 3512930.08765427
160 3676000.85538625
161 3403063.44938144
162 3374484.48372763
163 3644777.53204786
164 3403252.0388641
165 3282047.63107073
166 3949165.16686088
167 4583237.18828198
168 4832329.45863506
169 5858598.09398041
170 4888108.96248067
171 5098565.52753247
172 5535125.20890326
173 5054974.15974946
174 4496644.39103733
175 5091684.3356245
176 5355677.16598319
177 5463852.85516708
178 5075989.10489332
179 4774389.45646512
180 4159812.98986983
181 4043689.19237772
182 3875759.37213264
183 3831870.18563202
184 4638003.45325097
185 4965743.23824003
186 4828875.0262752
187 5149129.52539074
188 5426764.95437778
189 4700529.62264205
190 4359493.399568
191 5051955.53481734
192 4508518.04119596
193 3572077.77923822
194 3544537.21299506
195 3912759.61336738
196 4158359.44010169
197 4055860.62178732
198 5679785.32316868
199 5312980.24988761
200 4873697.03966183
201 5740574.39177131
202 5646645.03259591
203 5540676.53978171
204 5595418.18762173
205 4666282.95277116
206 4438700.69243057
207 4563674.01833223
208 5754526.56641786
209 5702734.64484178
210 5018571.08037612
211 4716029.3444679
212 4225503.71188057
213 4067358.35054383
214 4379211.63181886
215 5183455.91791364
216 4928359.36999108
217 3775626.06544533
218 3748741.24560245
219 3429571.65201324
220 2759963.2444644
221 2925900.23885688
222 3106423.73322527
223 3766199.97551245
224 3817781.05316686
225 4946701.77356577
226 4293095.03344171
227 3825137.42685089
228 3460717.94131383
229 2872195.96036111
230 2430881.83477292
231 2419604.0189647
232 2277282.03419563
233 2296051.73684042
234 2288042.93330848
235 1997980.83581599
236 1827827.44379379
237 1808342.39198927
238 1635796.02305636
239 1687651.32151691
240 2336038.65740175
241 3266597.50549099
242 1949091.84056252
243 1640982.82246947
244 1603802.33111955
245 1604357.90364933
246 1797942.16557439
247 1890766.81939697
248 1475770.93806716
249 1482232.53689783
250 1418596.95658609
251 1285707.06219898
252 1309408.37742217
253 1381232.81772003
254 1396053.31068346
255 2096343.93066536
256 1741849.95817342
257 2194259.91494206
258 1612739.76664911
259 1480488.11105489
260 1086265.08198656
261 1218085.28787371
262 1115329.53952774
263 1080828.98188669
264 1126813.14150858
265 1097029.51024673
266 1029034.2929604
267 979525.499206003
268 949628.329632053
269 1415047.79844909
270 1971116.17032514
271 1578330.96994743
272 1367792.45453242
273 3223698.03253809
274 3065675.95703317
275 4747386.91259336
276 4294739.25923917
277 5386688.47909865
278 4162391.11973108
279 4232020.37445625
280 4884029.52885984
281 5426550.67484279
282 4326330.59913507
283 3282845.88533177
284 4007293.35364608
285 4953659.34731351
286 4377271.54349405
287 5196246.69579319
288 4081249.49484599
289 4124314.87024712
290 3858971.71821369
291 3987214.0341179
292 5410964.79152279
293 4318516.55395257
294 4059792.13946061
295 3510142.67208185
296 3395116.05993808
297 3336100.70760326
298 4116880.02227186
299 3695508.65186318
300 4836348.13581374
301 3604905.51027559
302 3170986.85552059
303 2679217.13607899
304 2591286.17623439
305 2540354.89696753
306 3023925.11407158
307 3536114.54959553
308 2555514.09917504
309 2189990.70521528
310 2136893.64852708
311 2153180.96801308
312 2114157.43719987
313 4114937.15812987
314 3041504.12835373
315 4568071.82701908
316 3301466.05965331
317 2472116.82187458
318 2251353.3115223
319 2303825.12321243
320 2355428.29491341
321 2385281.32251201
322 3524463.11201292
323 3015763.39176419
324 4003954.95437839
325 3746410.97733326
326 3500915.4357796
327 4083754.88922408
328 2991501.12120656
329 2777971.84795682
330 3013280.61178948
331 3809873.2618234
332 3651371.67328356
333 4847356.03763562
334 3666274.94089454
335 3980998.10662977
336 4263726.19323596
337 3869253.18272351
338 3405085.47569825
339 3797231.77461797
340 3777189.78359576
341 3876058.57059773
342 3537419.01515418
343 3190531.36220575
344 3895293.35536238
345 3863364.46459654
346 3502052.48686348
347 3332099.2312585
348 3862724.79498153
349 5162506.42182775
350 3919559.03363217
351 3978497.54414648
352 3514207.95988086
353 4459825.73247449
354 3898544.04074782
355 4286055.34822428
356 4480901.63319489
357 5887785.23217393
358 4545032.37444324
359 4137867.22782977
360 4180741.57794403
361 4063614.46485438
362 4494096.72274627
363 5700631.58756219
364 5127470.01023706
365 5564026.74060934
366 4854473.73190613
367 4613618.40577014
368 5195525.04449922
369 4274146.74072172
370 4960413.75606414
371 5083307.18662975
372 5653879.48449018
373 5464936.4070681
374 5359371.41389331
375 6215092.02112864
376 5246306.08564855
377 6780494.43067325
378 6847249.14281081
379 6492125.69888424
380 5364008.38818536
381 5590083.63148466
382 5956388.02042202
383 6296510.69836586
384 6171424.02994134
385 6129910.70773394
386 6195338.08515175
387 5713663.32903482
388 5555173.775043
389 5739934.05937624
390 5294702.15338816
391 4714119.76389163
392 5766477.75662777
393 5698846.82143855
394 6337357.06646072
395 6230591.59352199
396 5340132.04362465
397 4869237.11486263
398 4236945.05643654
399 5163060.22463039
400 5128534.69354084
401 5958758.94554566
402 4725755.10504311
403 5365616.87103594
404 6448409.0915257
405 6599957.2085304
406 5946306.02012513
407 4993600.93647082
408 4926127.76783729
409 6373259.30958846
410 5491077.42491748
411 5076429.45998601
412 5956909.15510085
413 5733866.98135753
414 5480089.03554201
415 6056172.79728987
416 4980210.2659973
417 5050458.92343247
418 4601700.71125866
419 4473357.36264515
420 4337768.62327097
421 4570451.49449955
422 4077214.34338302
423 4535393.57165022
424 3945688.47872084
425 3521809.58446646
426 3465836.85794299
427 3808043.73472462
428 3701160.07699498
429 3095138.99771513
430 3023388.37067944
431 5275151.20876345
432 3891936.19330346
433 3250500.6688422
434 2929325.66077214
435 2865240.77087409
436 5578976.79902366
437 5626738.34395249
438 4629877.22909679
439 6003580.17603832
440 5148540.51609567
};
\end{axis}

\end{tikzpicture}
}\\[-0.5cm]
%		\subfloat[$I.$]{% This file was created with tikzplotlib v0.10.1.
\begin{tikzpicture}

\definecolor{darkgray176}{RGB}{176,176,176}

\begin{axis}[
width=\figurewidth,
height=\figureheight,
axis background/.style={fill=background_color},
axis line style={white},
tick align=inside,
tick pos=left,
x grid style={white},
y grid style={white},
xmajorgrids,
ymajorgrids,
ylabel={Contributions (kWh)},
xlabel={Time (days)},
xmin=150, xmax=300,
xtick style={color=black},
ymin=-4862207.72562858, ymax=118641609.891697,
ytick style={color=black},
ytick={-20000000,0,20000000,40000000,60000000,80000000,100000000,120000000},
yticklabels={\ensuremath{-}0.2,0.0,0.2,0.4,0.6,0.8,1.0,1.2},
]
\path [draw=blue, fill=blue, opacity=0.5]
(axis cs:0,35859165.7589368)
--(axis cs:0,3553893.70494481)
--(axis cs:1,2155947.33941062)
--(axis cs:2,4085970.39450474)
--(axis cs:3,1969118.4580092)
--(axis cs:4,1773362.49777085)
--(axis cs:5,2876673.02539644)
--(axis cs:6,2937432.31824604)
--(axis cs:7,2887158.59550819)
--(axis cs:8,2973793.53304275)
--(axis cs:9,2882829.36158654)
--(axis cs:10,4005299.35075504)
--(axis cs:11,3552214.67165973)
--(axis cs:12,3072930.26547241)
--(axis cs:13,2825587.50864711)
--(axis cs:14,2449426.29865274)
--(axis cs:15,3170170.06793467)
--(axis cs:16,4547580.79360294)
--(axis cs:17,3572698.78673868)
--(axis cs:18,3170007.13744458)
--(axis cs:19,3488935.81624904)
--(axis cs:20,2906645.52474923)
--(axis cs:21,3798077.3244383)
--(axis cs:22,3136214.22166048)
--(axis cs:23,2545095.78645657)
--(axis cs:24,3115974.72030543)
--(axis cs:25,2928790.18256873)
--(axis cs:26,2578101.58526744)
--(axis cs:27,2553566.22217579)
--(axis cs:28,2427708.66762385)
--(axis cs:29,2305952.61370368)
--(axis cs:30,2187168.92214099)
--(axis cs:31,2102304.02423322)
--(axis cs:32,2258785.86678255)
--(axis cs:33,2666009.01345685)
--(axis cs:34,2128654.08875769)
--(axis cs:35,2156771.66590961)
--(axis cs:36,2269051.03782374)
--(axis cs:37,1932605.62163696)
--(axis cs:38,2198748.15203689)
--(axis cs:39,2132032.4653566)
--(axis cs:40,1871932.84260065)
--(axis cs:41,1919764.04288173)
--(axis cs:42,2247112.23777891)
--(axis cs:43,2511876.11490701)
--(axis cs:44,2243106.22311745)
--(axis cs:45,2613394.10827709)
--(axis cs:46,2448238.01828638)
--(axis cs:47,2032498.44924541)
--(axis cs:48,1966163.98053946)
--(axis cs:49,2183751.41793814)
--(axis cs:50,1863401.56603063)
--(axis cs:51,2421761.42161975)
--(axis cs:52,2086905.401965)
--(axis cs:53,1686990.19183208)
--(axis cs:54,1584019.07744823)
--(axis cs:55,1822666.37084476)
--(axis cs:56,1656671.95378624)
--(axis cs:57,2393763.53559387)
--(axis cs:58,2089973.97202516)
--(axis cs:59,2973413.13851826)
--(axis cs:60,3289629.28852478)
--(axis cs:61,2676942.71715549)
--(axis cs:62,2182667.42211943)
--(axis cs:63,1906074.5712237)
--(axis cs:64,1679679.57011833)
--(axis cs:65,1739713.90247863)
--(axis cs:66,1843041.06310478)
--(axis cs:67,1633171.19064074)
--(axis cs:68,1583270.63359328)
--(axis cs:69,1566347.69042484)
--(axis cs:70,1513565.70403238)
--(axis cs:71,1662181.46876801)
--(axis cs:72,1520879.79764837)
--(axis cs:73,1885825.1040117)
--(axis cs:74,1764195.88394318)
--(axis cs:75,1606601.31968719)
--(axis cs:76,1535576.78100495)
--(axis cs:77,1528082.15115121)
--(axis cs:78,1499612.77924794)
--(axis cs:79,1552962.84043839)
--(axis cs:80,1547697.42017189)
--(axis cs:81,1538559.00288095)
--(axis cs:82,1328967.11442509)
--(axis cs:83,1729915.22213764)
--(axis cs:84,1986889.61063626)
--(axis cs:85,1814209.5064828)
--(axis cs:86,1719386.11880274)
--(axis cs:87,1534296.75614)
--(axis cs:88,1518048.73069466)
--(axis cs:89,1800991.74268914)
--(axis cs:90,2389228.55659462)
--(axis cs:91,1644920.15625012)
--(axis cs:92,1557000.56146545)
--(axis cs:93,1531982.66701145)
--(axis cs:94,1512427.12679926)
--(axis cs:95,1532261.56549658)
--(axis cs:96,1483627.01829393)
--(axis cs:97,1288808.68312396)
--(axis cs:98,2012527.81740861)
--(axis cs:99,2013081.86302949)
--(axis cs:100,1666573.63092936)
--(axis cs:101,2103094.17991063)
--(axis cs:102,1798702.79162849)
--(axis cs:103,2061504.17448422)
--(axis cs:104,1604070.68515927)
--(axis cs:105,1193655.25574663)
--(axis cs:106,2550328.20265092)
--(axis cs:107,2135923.80988378)
--(axis cs:108,2248493.20629585)
--(axis cs:109,1957641.20103766)
--(axis cs:110,2365293.78634708)
--(axis cs:111,1682592.14650709)
--(axis cs:112,1960377.29650364)
--(axis cs:113,2192953.73858902)
--(axis cs:114,1743710.00775208)
--(axis cs:115,2426988.9158627)
--(axis cs:116,3490257.47467487)
--(axis cs:117,2438400.41997051)
--(axis cs:118,2909367.40155691)
--(axis cs:119,2092141.28184522)
--(axis cs:120,2629436.3064068)
--(axis cs:121,2987569.44213864)
--(axis cs:122,2695286.30889381)
--(axis cs:123,1943749.87537317)
--(axis cs:124,3171903.1109792)
--(axis cs:125,3010827.6650574)
--(axis cs:126,4311873.2048312)
--(axis cs:127,5204619.01335694)
--(axis cs:128,3472180.79662485)
--(axis cs:129,5306581.74580818)
--(axis cs:130,4434706.84751731)
--(axis cs:131,3349733.75551722)
--(axis cs:132,3291734.28884113)
--(axis cs:133,3053026.27224344)
--(axis cs:134,3203766.2390161)
--(axis cs:135,3646157.33989213)
--(axis cs:136,2286345.21670863)
--(axis cs:137,2707511.17395681)
--(axis cs:138,2191256.10421062)
--(axis cs:139,1709536.13309018)
--(axis cs:140,2603672.7184838)
--(axis cs:141,2291797.40154646)
--(axis cs:142,3191115.76027192)
--(axis cs:143,3372748.30623277)
--(axis cs:144,3742240.46931852)
--(axis cs:145,2574579.97654074)
--(axis cs:146,2484032.46824675)
--(axis cs:147,2905508.77498731)
--(axis cs:148,3690779.25961676)
--(axis cs:149,3339558.3350266)
--(axis cs:150,3386923.63299474)
--(axis cs:151,2829224.73851378)
--(axis cs:152,2008961.4497362)
--(axis cs:153,2312388.53008354)
--(axis cs:154,2233061.51809277)
--(axis cs:155,3910655.90751646)
--(axis cs:156,1879044.93116112)
--(axis cs:157,2866431.39712954)
--(axis cs:158,2052877.52286709)
--(axis cs:159,2086251.30451473)
--(axis cs:160,2698692.97282734)
--(axis cs:161,2955842.25192428)
--(axis cs:162,2497109.91398023)
--(axis cs:163,3619972.44517203)
--(axis cs:164,2590940.95673582)
--(axis cs:165,3041853.12159778)
--(axis cs:166,3564613.07990012)
--(axis cs:167,2923435.0659026)
--(axis cs:168,2641639.61445652)
--(axis cs:169,1734359.37533542)
--(axis cs:170,2924820.43426583)
--(axis cs:171,1808750.76517402)
--(axis cs:172,1703939.13816049)
--(axis cs:173,3125897.66148227)
--(axis cs:174,2560009.2415609)
--(axis cs:175,2541310.18178925)
--(axis cs:176,3889948.9800454)
--(axis cs:177,5124872.03568553)
--(axis cs:178,3942028.98958748)
--(axis cs:179,2182350.29416907)
--(axis cs:180,2418512.32479255)
--(axis cs:181,2218874.55939661)
--(axis cs:182,2283814.66118281)
--(axis cs:183,3122759.36789842)
--(axis cs:184,2171343.35908197)
--(axis cs:185,2285439.68669798)
--(axis cs:186,2885963.07181877)
--(axis cs:187,1232016.78612615)
--(axis cs:188,3699274.66700205)
--(axis cs:189,4005387.15930218)
--(axis cs:190,3302643.693821)
--(axis cs:191,4100135.92448667)
--(axis cs:192,3744719.40430777)
--(axis cs:193,1766668.38958006)
--(axis cs:194,2780802.56120352)
--(axis cs:195,3071570.3256803)
--(axis cs:196,2224273.545714)
--(axis cs:197,3251866.44424986)
--(axis cs:198,3458537.44342806)
--(axis cs:199,2865380.27027366)
--(axis cs:200,3567728.15752463)
--(axis cs:201,2604637.31919066)
--(axis cs:202,4744515.55832826)
--(axis cs:203,5915410.26409454)
--(axis cs:204,3394097.98462816)
--(axis cs:205,1926857.19809506)
--(axis cs:206,1726700.78327555)
--(axis cs:207,3328505.51123488)
--(axis cs:208,2805658.50740453)
--(axis cs:209,5332476.83477676)
--(axis cs:210,4083364.02509394)
--(axis cs:211,3433188.51816405)
--(axis cs:212,2666988.66323565)
--(axis cs:213,2433368.47794615)
--(axis cs:214,3181645.43915767)
--(axis cs:215,2833951.30497732)
--(axis cs:216,3213802.90846072)
--(axis cs:217,2925506.19821959)
--(axis cs:218,3604931.89196541)
--(axis cs:219,2097669.95142425)
--(axis cs:220,2431296.44014698)
--(axis cs:221,3712051.17366684)
--(axis cs:222,2863974.62350122)
--(axis cs:223,3417006.9194279)
--(axis cs:224,2576753.5009356)
--(axis cs:225,5334664.94548246)
--(axis cs:226,2927365.75660638)
--(axis cs:227,3909965.8049607)
--(axis cs:228,4895207.1986296)
--(axis cs:229,4343563.58393919)
--(axis cs:230,3073472.1935371)
--(axis cs:231,2951347.96312167)
--(axis cs:232,2898319.09729124)
--(axis cs:233,2966308.38926266)
--(axis cs:234,2738562.07989824)
--(axis cs:235,2560828.36098139)
--(axis cs:236,2616138.68121505)
--(axis cs:237,2773844.23651997)
--(axis cs:238,1047166.65705257)
--(axis cs:239,1615521.37515773)
--(axis cs:240,1309641.45322053)
--(axis cs:241,1132671.11683443)
--(axis cs:242,1416276.56706625)
--(axis cs:243,1978683.75824286)
--(axis cs:244,1442137.32428267)
--(axis cs:245,2026581.26447517)
--(axis cs:246,3030067.71644337)
--(axis cs:247,3933682.97405688)
--(axis cs:248,2412804.73696993)
--(axis cs:249,2596389.38138643)
--(axis cs:250,2197124.48184456)
--(axis cs:251,1796673.63003364)
--(axis cs:252,1610339.50705913)
--(axis cs:253,1615675.16381486)
--(axis cs:254,2203439.43849295)
--(axis cs:255,1719520.36821183)
--(axis cs:256,2116858.9536823)
--(axis cs:257,2935986.52659708)
--(axis cs:258,3183705.17386098)
--(axis cs:259,3106292.881857)
--(axis cs:260,751602.166068015)
--(axis cs:261,1843160.8169606)
--(axis cs:262,1842209.80170299)
--(axis cs:263,1879866.19779867)
--(axis cs:264,1526098.46336537)
--(axis cs:265,1786675.38067589)
--(axis cs:266,1826575.88474795)
--(axis cs:267,1825903.13056424)
--(axis cs:268,2105633.30872706)
--(axis cs:269,2252286.76961512)
--(axis cs:270,2295090.97020634)
--(axis cs:271,1527315.93167003)
--(axis cs:272,2263921.15614503)
--(axis cs:273,2604829.41072404)
--(axis cs:274,3051751.4964096)
--(axis cs:275,3465843.68349859)
--(axis cs:276,5297461.79520057)
--(axis cs:277,3698805.55835516)
--(axis cs:278,1753172.06819881)
--(axis cs:279,2723776.70007546)
--(axis cs:280,2729136.00919936)
--(axis cs:281,3634791.26317789)
--(axis cs:282,2707939.41593939)
--(axis cs:283,2253370.74535589)
--(axis cs:284,2283310.64781375)
--(axis cs:285,3880097.79867456)
--(axis cs:286,5230475.41888061)
--(axis cs:287,4170949.90777754)
--(axis cs:288,3174459.13661849)
--(axis cs:289,2157978.36264106)
--(axis cs:290,2856143.36588883)
--(axis cs:291,3691289.40691329)
--(axis cs:292,4277573.84205886)
--(axis cs:293,4080021.11883845)
--(axis cs:294,3857983.50752014)
--(axis cs:295,3201517.90567537)
--(axis cs:296,3864282.79758938)
--(axis cs:297,2609088.88167185)
--(axis cs:298,2994651.58725961)
--(axis cs:299,3797273.63778131)
--(axis cs:300,4809116.58762339)
--(axis cs:301,3775583.24583267)
--(axis cs:302,3762696.37278571)
--(axis cs:303,3352430.34847056)
--(axis cs:304,3131915.45287295)
--(axis cs:305,3046707.61028988)
--(axis cs:306,4308921.09605062)
--(axis cs:307,5546454.12365471)
--(axis cs:308,3444192.40482614)
--(axis cs:309,2876570.80556149)
--(axis cs:310,2843470.37989391)
--(axis cs:311,3837343.42665396)
--(axis cs:312,2895705.01760967)
--(axis cs:313,2959258.68051947)
--(axis cs:314,2784262.35102389)
--(axis cs:315,3008291.01247321)
--(axis cs:316,5366166.91433841)
--(axis cs:317,3628392.20544043)
--(axis cs:318,2891388.69631483)
--(axis cs:319,3006213.15939757)
--(axis cs:320,3415255.16026995)
--(axis cs:321,3778600.66003269)
--(axis cs:322,4241502.26375356)
--(axis cs:323,3382652.523094)
--(axis cs:324,3475865.88775188)
--(axis cs:325,4292261.14474488)
--(axis cs:326,3782252.53185512)
--(axis cs:327,4667999.37897223)
--(axis cs:328,1064868.35737484)
--(axis cs:329,3253639.03006884)
--(axis cs:330,3839238.066191)
--(axis cs:331,3630335.36752705)
--(axis cs:332,3969503.45905109)
--(axis cs:333,3505042.38425805)
--(axis cs:334,2898192.25421442)
--(axis cs:335,3791018.99399874)
--(axis cs:336,2265784.07265426)
--(axis cs:337,3333229.38209744)
--(axis cs:338,3361107.70087058)
--(axis cs:339,4261242.99433224)
--(axis cs:340,3553545.00496753)
--(axis cs:341,3344379.18931616)
--(axis cs:342,3849825.90122163)
--(axis cs:343,3392058.6608147)
--(axis cs:344,3522318.32354625)
--(axis cs:345,3017286.04764113)
--(axis cs:346,2918082.63315466)
--(axis cs:347,2797383.88828758)
--(axis cs:348,2832909.02697086)
--(axis cs:349,4866241.27840742)
--(axis cs:350,2625123.72890589)
--(axis cs:351,3846028.33223644)
--(axis cs:352,2962795.82304369)
--(axis cs:353,4259783.42558174)
--(axis cs:354,3910805.06600046)
--(axis cs:355,3940956.11713777)
--(axis cs:356,4346570.30274029)
--(axis cs:357,3040608.01525805)
--(axis cs:358,2955240.87374208)
--(axis cs:359,3383000.58428321)
--(axis cs:360,3805421.58050146)
--(axis cs:361,3881576.57419798)
--(axis cs:362,3400490.62776227)
--(axis cs:363,3420179.33732794)
--(axis cs:364,3438743.38518957)
--(axis cs:365,3020301.46189193)
--(axis cs:366,2864218.56386902)
--(axis cs:367,3270286.93625551)
--(axis cs:368,2938863.44249961)
--(axis cs:369,2522516.06249488)
--(axis cs:370,2274874.98060223)
--(axis cs:371,2086425.54251089)
--(axis cs:372,2380336.77231322)
--(axis cs:373,2526341.93445851)
--(axis cs:374,2718752.0265203)
--(axis cs:375,3133144.67601602)
--(axis cs:376,3292701.26960354)
--(axis cs:377,3811765.85848667)
--(axis cs:378,5432716.93633453)
--(axis cs:379,4304809.72011968)
--(axis cs:380,3542071.75232893)
--(axis cs:381,3122976.77626515)
--(axis cs:382,2698874.10611806)
--(axis cs:383,2857595.45114066)
--(axis cs:384,3455901.39596541)
--(axis cs:385,2678748.46938783)
--(axis cs:386,2882354.67729084)
--(axis cs:387,2800361.3892466)
--(axis cs:388,2794283.70888254)
--(axis cs:389,2767300.50118011)
--(axis cs:390,2804801.51094927)
--(axis cs:391,2287351.25038508)
--(axis cs:392,2125439.19353021)
--(axis cs:393,1924582.48728944)
--(axis cs:394,1967823.69290272)
--(axis cs:395,1758352.26987668)
--(axis cs:396,1889349.11392637)
--(axis cs:397,1850196.0071785)
--(axis cs:398,1872127.28326662)
--(axis cs:399,1764293.78394357)
--(axis cs:400,2040122.48091691)
--(axis cs:401,2024714.83838056)
--(axis cs:402,1848772.39919176)
--(axis cs:403,1860903.86438567)
--(axis cs:404,1933309.67243144)
--(axis cs:405,1977399.81562825)
--(axis cs:406,1836911.84811826)
--(axis cs:407,1810486.51224259)
--(axis cs:408,1819245.28854829)
--(axis cs:409,2153494.21949536)
--(axis cs:410,2250967.01744562)
--(axis cs:411,2350818.58320863)
--(axis cs:412,2135260.78701452)
--(axis cs:413,2199805.90303897)
--(axis cs:414,1772631.59039593)
--(axis cs:415,2299281.47215945)
--(axis cs:416,2244055.20514145)
--(axis cs:417,2100680.3664737)
--(axis cs:418,1767252.75098275)
--(axis cs:419,1973104.04765474)
--(axis cs:420,1947908.84173039)
--(axis cs:421,2398739.35683742)
--(axis cs:422,2281647.90472704)
--(axis cs:423,2165370.65535405)
--(axis cs:424,2062798.06474022)
--(axis cs:425,1767735.8109708)
--(axis cs:426,1983580.39014036)
--(axis cs:427,2240233.55591979)
--(axis cs:428,2708159.10558422)
--(axis cs:429,2129267.22093942)
--(axis cs:430,1869912.40410384)
--(axis cs:431,2032276.82460161)
--(axis cs:432,1809014.21155909)
--(axis cs:433,1734560.24255048)
--(axis cs:434,1866107.98012695)
--(axis cs:435,1787492.9590302)
--(axis cs:436,1715006.59040838)
--(axis cs:437,1659704.52939418)
--(axis cs:438,1608545.90898179)
--(axis cs:439,1656840.50037019)
--(axis cs:440,1646233.84050051)
--(axis cs:440,7651991.62590523)
--(axis cs:440,7651991.62590523)
--(axis cs:439,7801542.38404112)
--(axis cs:438,7457811.85338341)
--(axis cs:437,7709494.210343)
--(axis cs:436,8117661.74109768)
--(axis cs:435,8116276.14402208)
--(axis cs:434,8103457.24621787)
--(axis cs:433,7919407.61939255)
--(axis cs:432,8137852.34632031)
--(axis cs:431,8671218.40032341)
--(axis cs:430,8339226.51787528)
--(axis cs:429,9350522.5522836)
--(axis cs:428,11690249.8265482)
--(axis cs:427,9595267.03080069)
--(axis cs:426,8697801.37131713)
--(axis cs:425,8078196.00936622)
--(axis cs:424,8461561.81880758)
--(axis cs:423,8863384.02231069)
--(axis cs:422,9249687.56468723)
--(axis cs:421,10144703.1104043)
--(axis cs:420,8231716.43726398)
--(axis cs:419,8567641.71551588)
--(axis cs:418,8026284.1919633)
--(axis cs:417,8790803.76228685)
--(axis cs:416,10079081.7118315)
--(axis cs:415,9925600.70390714)
--(axis cs:414,8322202.90207692)
--(axis cs:413,9804424.08516356)
--(axis cs:412,9657519.93380325)
--(axis cs:411,10242569.3728148)
--(axis cs:410,10234088.4007415)
--(axis cs:409,9282796.3288278)
--(axis cs:408,8441138.7868738)
--(axis cs:407,8866766.95997329)
--(axis cs:406,9167596.94256771)
--(axis cs:405,9216999.8115161)
--(axis cs:404,9449986.00969264)
--(axis cs:403,8483067.01880096)
--(axis cs:402,8952678.09464881)
--(axis cs:401,9525397.90002132)
--(axis cs:400,8799793.43958284)
--(axis cs:399,9028918.93314164)
--(axis cs:398,8579997.1316717)
--(axis cs:397,8613811.75283869)
--(axis cs:396,9691030.24693497)
--(axis cs:395,8890795.2843084)
--(axis cs:394,9454976.72223164)
--(axis cs:393,9116365.5683537)
--(axis cs:392,9791829.66048331)
--(axis cs:391,10261952.4473167)
--(axis cs:390,12988817.7307589)
--(axis cs:389,12351341.8564027)
--(axis cs:388,12665734.4038599)
--(axis cs:387,14219039.8443138)
--(axis cs:386,13422687.404595)
--(axis cs:385,13130055.4912236)
--(axis cs:384,17121188.6526728)
--(axis cs:383,14468710.5260984)
--(axis cs:382,14879459.1777237)
--(axis cs:381,15336185.0998933)
--(axis cs:380,24838413.1522527)
--(axis cs:379,29204671.2044365)
--(axis cs:378,46483274.9004182)
--(axis cs:377,25951307.8921422)
--(axis cs:376,25914683.2240261)
--(axis cs:375,16934075.7153662)
--(axis cs:374,14304270.4488419)
--(axis cs:373,13434148.503197)
--(axis cs:372,12230279.8168609)
--(axis cs:371,10722507.8946973)
--(axis cs:370,11600098.9425351)
--(axis cs:369,13046767.3303237)
--(axis cs:368,14908909.5819731)
--(axis cs:367,15878370.5507312)
--(axis cs:366,15855058.8202935)
--(axis cs:365,15947124.3567218)
--(axis cs:364,18452817.8634286)
--(axis cs:363,20349207.5652473)
--(axis cs:362,23428012.1889393)
--(axis cs:361,21409625.1604244)
--(axis cs:360,21644815.1886901)
--(axis cs:359,18832474.7163303)
--(axis cs:358,19272568.0300902)
--(axis cs:357,30644167.5968751)
--(axis cs:356,31234933.0964912)
--(axis cs:355,26443252.3009522)
--(axis cs:354,28899465.7181627)
--(axis cs:353,36579079.4885015)
--(axis cs:352,32836431.3554587)
--(axis cs:351,27756797.109862)
--(axis cs:350,23581784.9533036)
--(axis cs:349,32967996.2709964)
--(axis cs:348,28554992.7414484)
--(axis cs:347,15577066.4256682)
--(axis cs:346,16044990.0272435)
--(axis cs:345,16786161.7352186)
--(axis cs:344,18799383.0743915)
--(axis cs:343,21718683.8566458)
--(axis cs:342,25262010.3906843)
--(axis cs:341,22269197.642286)
--(axis cs:340,20889951.4171307)
--(axis cs:339,32765493.3877064)
--(axis cs:338,27520212.2836277)
--(axis cs:337,24366500.2067236)
--(axis cs:336,19888689.5852491)
--(axis cs:335,22117593.2812193)
--(axis cs:334,17456546.5200453)
--(axis cs:333,23237886.9045611)
--(axis cs:332,25075333.5818206)
--(axis cs:331,24074158.8805189)
--(axis cs:330,23615743.7564951)
--(axis cs:329,32172843.7526267)
--(axis cs:328,26181042.1538946)
--(axis cs:327,40298167.0508026)
--(axis cs:326,22148697.1295002)
--(axis cs:325,31066810.8937788)
--(axis cs:324,30207785.8639175)
--(axis cs:323,26383861.4307123)
--(axis cs:322,36846190.5046605)
--(axis cs:321,30842712.9165655)
--(axis cs:320,23800895.6455572)
--(axis cs:319,17427819.6756084)
--(axis cs:318,19849532.5401241)
--(axis cs:317,28276069.5202355)
--(axis cs:316,51667638.4968964)
--(axis cs:315,102549763.369219)
--(axis cs:314,31299820.753604)
--(axis cs:313,26052104.642931)
--(axis cs:312,19004924.4560174)
--(axis cs:311,23329424.9009146)
--(axis cs:310,17091837.1641637)
--(axis cs:309,18212082.7631987)
--(axis cs:308,21794555.4054667)
--(axis cs:307,40330791.4457965)
--(axis cs:306,26469001.2573285)
--(axis cs:305,22065795.1755089)
--(axis cs:304,33670392.7942412)
--(axis cs:303,42762274.4050007)
--(axis cs:302,29351660.9662433)
--(axis cs:301,25924308.6487058)
--(axis cs:300,31275211.8797229)
--(axis cs:299,29502303.695444)
--(axis cs:298,30123349.5425141)
--(axis cs:297,27417151.7601922)
--(axis cs:296,24244458.8612855)
--(axis cs:295,21534081.7170691)
--(axis cs:294,26482253.0486315)
--(axis cs:293,32849596.0713835)
--(axis cs:292,35501827.6341542)
--(axis cs:291,32817374.1555512)
--(axis cs:290,19671510.6489548)
--(axis cs:289,20512875.2515707)
--(axis cs:288,19522164.9205609)
--(axis cs:287,30733795.5632591)
--(axis cs:286,35519001.4304357)
--(axis cs:285,32386674.4926884)
--(axis cs:284,24188949.0578282)
--(axis cs:283,23389221.3706558)
--(axis cs:282,23113925.5753293)
--(axis cs:281,32396291.5189061)
--(axis cs:280,34611207.0343197)
--(axis cs:279,36029428.6493173)
--(axis cs:278,33853278.2553107)
--(axis cs:277,36272909.1599315)
--(axis cs:276,43440419.8901326)
--(axis cs:275,31454478.0865651)
--(axis cs:274,30072254.9479808)
--(axis cs:273,28396507.9796101)
--(axis cs:272,25163678.4042803)
--(axis cs:271,29238395.472369)
--(axis cs:270,14852773.3667126)
--(axis cs:269,15667345.8509048)
--(axis cs:268,14045222.0540293)
--(axis cs:267,12823850.1470753)
--(axis cs:266,12919640.9072227)
--(axis cs:265,12493416.2836185)
--(axis cs:264,13186129.5431684)
--(axis cs:263,13646874.4278113)
--(axis cs:262,12387824.1816021)
--(axis cs:261,15727854.0439084)
--(axis cs:260,19022695.6336935)
--(axis cs:259,28604995.4638123)
--(axis cs:258,20212779.3614108)
--(axis cs:257,28651958.9765477)
--(axis cs:256,18172268.4999406)
--(axis cs:255,23884321.3842848)
--(axis cs:254,18280383.5183954)
--(axis cs:253,12078451.6280455)
--(axis cs:252,14946155.2233105)
--(axis cs:251,14890723.8250225)
--(axis cs:250,15691104.3864117)
--(axis cs:249,18605757.043718)
--(axis cs:248,24775039.3687194)
--(axis cs:247,41889018.0453273)
--(axis cs:246,26730768.7959515)
--(axis cs:245,17777919.1618511)
--(axis cs:244,13842737.0994778)
--(axis cs:243,19621264.1503681)
--(axis cs:242,21470582.6014788)
--(axis cs:241,17878293.9285904)
--(axis cs:240,23387069.1131071)
--(axis cs:239,23496521.0407496)
--(axis cs:238,20635819.598145)
--(axis cs:237,15818889.9156357)
--(axis cs:236,15337722.8000725)
--(axis cs:235,14923637.8493062)
--(axis cs:234,15727771.9422658)
--(axis cs:233,18018458.1061706)
--(axis cs:232,21240404.9990911)
--(axis cs:231,20761167.0267365)
--(axis cs:230,23608374.3240058)
--(axis cs:229,29432545.105878)
--(axis cs:228,33059994.4740888)
--(axis cs:227,28653114.6019784)
--(axis cs:226,29428603.6943392)
--(axis cs:225,39112217.4867033)
--(axis cs:224,25432866.57446)
--(axis cs:223,26036547.2236748)
--(axis cs:222,29667572.3741466)
--(axis cs:221,43599277.0111432)
--(axis cs:220,32008930.4561818)
--(axis cs:219,25413344.2976898)
--(axis cs:218,35183818.3558282)
--(axis cs:217,29882667.9822318)
--(axis cs:216,32332001.7897428)
--(axis cs:215,28167752.34853)
--(axis cs:214,37176655.7966127)
--(axis cs:213,37853399.5042444)
--(axis cs:212,36398289.3139443)
--(axis cs:211,32115695.7026782)
--(axis cs:210,38263488.321848)
--(axis cs:209,50297718.9635114)
--(axis cs:208,40442431.8230993)
--(axis cs:207,34269336.3659192)
--(axis cs:206,25983577.021337)
--(axis cs:205,23750821.2456423)
--(axis cs:204,28935451.4638624)
--(axis cs:203,46635505.6784375)
--(axis cs:202,57970058.6873322)
--(axis cs:201,39566246.0729342)
--(axis cs:200,36698030.3028788)
--(axis cs:199,35612082.7028125)
--(axis cs:198,38977813.1521046)
--(axis cs:197,34060754.8175484)
--(axis cs:196,35517374.2286694)
--(axis cs:195,34330359.1593366)
--(axis cs:194,26760470.7243873)
--(axis cs:193,23913861.1097761)
--(axis cs:192,31512052.1389119)
--(axis cs:191,28705968.6041874)
--(axis cs:190,20578621.6075639)
--(axis cs:189,25261813.1491628)
--(axis cs:188,32452444.6518195)
--(axis cs:187,35158266.9208004)
--(axis cs:186,54486148.6845519)
--(axis cs:185,21959599.6078442)
--(axis cs:184,22820033.6283486)
--(axis cs:183,22955838.5631931)
--(axis cs:182,21445192.9581827)
--(axis cs:181,19324266.5742908)
--(axis cs:180,21162599.0390022)
--(axis cs:179,38051924.5109594)
--(axis cs:178,33848188.1493291)
--(axis cs:177,53443291.4282192)
--(axis cs:176,31670571.9577052)
--(axis cs:175,28738445.8496741)
--(axis cs:174,33840850.832019)
--(axis cs:173,26900557.1930931)
--(axis cs:172,31304362.8544272)
--(axis cs:171,26324433.0413459)
--(axis cs:170,21295834.0899418)
--(axis cs:169,23592798.1850167)
--(axis cs:168,31495189.833244)
--(axis cs:167,23452121.5026072)
--(axis cs:166,25744688.4411984)
--(axis cs:165,16757986.7768077)
--(axis cs:164,14582020.9506169)
--(axis cs:163,19680434.3521342)
--(axis cs:162,20305432.3103824)
--(axis cs:161,16118736.8927752)
--(axis cs:160,16085980.6709387)
--(axis cs:159,13713823.9686884)
--(axis cs:158,15553656.4443872)
--(axis cs:157,29510612.1266871)
--(axis cs:156,29777383.3950285)
--(axis cs:155,24118035.354245)
--(axis cs:154,23378823.4424762)
--(axis cs:153,16478229.8138015)
--(axis cs:152,14811172.1522502)
--(axis cs:151,28415788.3737234)
--(axis cs:150,22560087.0772682)
--(axis cs:149,23597794.2461632)
--(axis cs:148,41506843.5396058)
--(axis cs:147,22776231.8137639)
--(axis cs:146,29958327.6091365)
--(axis cs:145,28983437.9274137)
--(axis cs:144,24278733.2808669)
--(axis cs:143,27646809.2438933)
--(axis cs:142,28163026.9360034)
--(axis cs:141,16163446.4932492)
--(axis cs:140,17611996.9142068)
--(axis cs:139,14409577.9177382)
--(axis cs:138,20602810.7022285)
--(axis cs:137,18407672.990054)
--(axis cs:136,16196373.6846662)
--(axis cs:135,33986696.1842878)
--(axis cs:134,26324828.3836578)
--(axis cs:133,16625713.3914016)
--(axis cs:132,22356587.4012648)
--(axis cs:131,19838326.8484765)
--(axis cs:130,23847769.6442785)
--(axis cs:129,30951995.1245773)
--(axis cs:128,27956596.893078)
--(axis cs:127,27655763.2543288)
--(axis cs:126,31616544.3322112)
--(axis cs:125,22554996.2121439)
--(axis cs:124,23337886.8442278)
--(axis cs:123,14869521.9900331)
--(axis cs:122,15552339.8066642)
--(axis cs:121,18523846.4514782)
--(axis cs:120,15557770.6910397)
--(axis cs:119,11020087.1503537)
--(axis cs:118,15582070.3701088)
--(axis cs:117,12880429.109007)
--(axis cs:116,23449637.2434869)
--(axis cs:115,16531640.5065405)
--(axis cs:114,9317782.58704665)
--(axis cs:113,10700213.4175046)
--(axis cs:112,9893924.33157613)
--(axis cs:111,10687912.6453606)
--(axis cs:110,13166662.9250931)
--(axis cs:109,10632994.8815753)
--(axis cs:108,13394796.5162521)
--(axis cs:107,12736477.3140668)
--(axis cs:106,15730773.5704251)
--(axis cs:105,13987280.2100597)
--(axis cs:104,8714400.84640658)
--(axis cs:103,10104535.9282876)
--(axis cs:102,10422956.2065938)
--(axis cs:101,9462994.77297512)
--(axis cs:100,8386647.45610763)
--(axis cs:99,9067634.34115022)
--(axis cs:98,9113211.55539139)
--(axis cs:97,6945444.20341399)
--(axis cs:96,7285685.57657714)
--(axis cs:95,7836425.34477807)
--(axis cs:94,8029491.94798574)
--(axis cs:93,7446729.42798017)
--(axis cs:92,8394417.95188516)
--(axis cs:91,8831961.64456418)
--(axis cs:90,11150301.0433709)
--(axis cs:89,8296208.67737757)
--(axis cs:88,7995685.89619124)
--(axis cs:87,7657679.58861858)
--(axis cs:86,8604718.010999)
--(axis cs:85,9795642.38153118)
--(axis cs:84,8717864.0792706)
--(axis cs:83,8905078.04335921)
--(axis cs:82,7698707.45960012)
--(axis cs:81,8198711.668144)
--(axis cs:80,7429337.881026)
--(axis cs:79,7653063.25555805)
--(axis cs:78,7570232.4111104)
--(axis cs:77,7562146.62030881)
--(axis cs:76,8226124.24551952)
--(axis cs:75,9156867.98666176)
--(axis cs:74,10989053.501532)
--(axis cs:73,9765172.1297991)
--(axis cs:72,9339640.49174983)
--(axis cs:71,8371111.5746233)
--(axis cs:70,7521971.73048445)
--(axis cs:69,7722542.16266022)
--(axis cs:68,7833397.49937755)
--(axis cs:67,7899211.6970766)
--(axis cs:66,8286773.5177205)
--(axis cs:65,8258365.46019145)
--(axis cs:64,8361774.58147933)
--(axis cs:63,9406176.12462598)
--(axis cs:62,10467359.9772292)
--(axis cs:61,14491245.127714)
--(axis cs:60,23318457.8082782)
--(axis cs:59,18337743.8137552)
--(axis cs:58,13456583.0376267)
--(axis cs:57,11845039.2242703)
--(axis cs:56,8206271.70037219)
--(axis cs:55,8311304.32878818)
--(axis cs:54,7616731.02654012)
--(axis cs:53,8324269.41025345)
--(axis cs:52,10911570.3480424)
--(axis cs:51,12659428.8789377)
--(axis cs:50,9583124.9880465)
--(axis cs:49,11045349.6336961)
--(axis cs:48,8959333.86100508)
--(axis cs:47,8915235.17058496)
--(axis cs:46,10453907.6668681)
--(axis cs:45,11703402.6606984)
--(axis cs:44,10122094.6006791)
--(axis cs:43,10823555.4866531)
--(axis cs:42,10423879.4352118)
--(axis cs:41,9066873.51857266)
--(axis cs:40,8640223.41590778)
--(axis cs:39,9676082.42717595)
--(axis cs:38,10330133.5459703)
--(axis cs:37,9121286.52097248)
--(axis cs:36,10469272.8493879)
--(axis cs:35,9762762.40201712)
--(axis cs:34,9368509.11031362)
--(axis cs:33,11776387.8500759)
--(axis cs:32,9394815.22304077)
--(axis cs:31,10807706.3898616)
--(axis cs:30,9906102.47793522)
--(axis cs:29,11324483.9808545)
--(axis cs:28,10594494.7574081)
--(axis cs:27,12440622.5540883)
--(axis cs:26,12855755.7582377)
--(axis cs:25,13262581.4179207)
--(axis cs:24,14864168.3124409)
--(axis cs:23,13072803.5600099)
--(axis cs:22,14876116.3691556)
--(axis cs:21,24426453.3424818)
--(axis cs:20,15690469.8604012)
--(axis cs:19,18942708.7518357)
--(axis cs:18,18527983.2362802)
--(axis cs:17,26077662.6109674)
--(axis cs:16,29566191.1479305)
--(axis cs:15,16682601.4873887)
--(axis cs:14,13781958.4519084)
--(axis cs:13,16470559.6408434)
--(axis cs:12,17305627.1044374)
--(axis cs:11,19969792.8244549)
--(axis cs:10,22495229.7440649)
--(axis cs:9,20759229.5695093)
--(axis cs:8,18914055.1508619)
--(axis cs:7,26903979.9958883)
--(axis cs:6,37202559.529172)
--(axis cs:5,28901719.7488726)
--(axis cs:4,31048883.636374)
--(axis cs:3,28784146.110642)
--(axis cs:2,33592633.711925)
--(axis cs:1,36100935.0127416)
--(axis cs:0,35859165.7589368)
--cycle;

\addplot [only marks, red, mark=asterisk, mark size=1.2]
table {%
0 29755500
1 20228000
2 8513700
3 9502900
4 8909200
5 7004800
6 6305500
7 6348400
8 7321100
9 6421300
10 6164800
11 5360500
12 5074700
13 5074700
14 6804600
15 9612000
16 9154100
17 5291400
18 5456200
19 5819200
20 12820200
21 5305900
22 5067000
23 11001200
24 7051300
25 7450000
26 9014400
27 5198200
28 4916300
29 5534800
30 4719000
31 4596600
32 6568000
33 6431200
34 5034500
35 6933000
36 4533400
37 4446000
38 5301600
39 4517400
40 4370200
41 6208600
42 7137600
43 4905200
44 5415600
45 4461400
46 4170200
47 4549000
48 5623500
49 4354800
50 6282500
51 6528100
52 4165200
53 3940800
54 3894500
55 4728400
56 12964300
57 7101500
58 9158000
59 10692100
60 4636300
61 5787600
62 5199600
63 4093600
64 3925700
65 3818200
66 3756600
67 3702000
68 3632400
69 3676500
70 3599700
71 4028000
72 6895400
73 4511900
74 5519200
75 3725000
76 3495900
77 3418000
78 3308300
79 3208300
80 3241700
81 7029700
82 5106300
83 4259800
84 11266900
85 7029700
86 3754100
87 3833100
88 5472000
89 6634600
90 3756400
91 3932500
92 3587100
93 3677400
94 3490000
95 3271000
96 3169400
97 3754100
98 3754100
99 3289100
100 3907600
101 3268800
102 3747400
103 4939300
104 5061200
105 34188900
106 27994500
107 6776800
108 5061200
109 4065600
110 3948300
111 5492400
112 6815200
113 4162700
114 10287200
115 19116000
116 5162800
117 5390800
118 9111000
119 8332200
120 6176400
121 6007000
122 5449500
123 9217100
124 26947100
125 23633100
126 7943900
127 9619000
128 7515000
129 7323100
130 6271200
131 10975700
132 5934800
133 13255700
134 8424800
135 5537500
136 20382400
137 15957800
138 9571500
139 11587400
140 5961900
141 7573700
142 7627900
143 34868400
144 19389200
145 7959700
146 5842300
147 22768500
148 7232800
149 7546600
150 16027800
151 5833200
152 5336600
153 7476600
154 7512800
155 6573700
156 4891900
157 5083800
158 4998000
159 5007000
160 5165000
161 24100400
162 19050500
163 6097300
164 8512800
165 14923900
166 9695700
167 9673100
168 13014100
169 9045600
170 15720800
171 25333000
172 21901700
173 29845600
174 12422700
175 11126900
176 29179700
177 12285000
178 31793800
179 7835600
180 7106400
181 9212600
182 14449900
183 22425400
184 24443600
185 36209300
186 23874700
187 11655200
188 11354900
189 8605400
190 31960800
191 32662900
192 23240300
193 113027800
194 30163900
195 17425200
196 9767900
197 13793000
198 9176500
199 48431100
200 56194500
201 19000900
202 34642700
203 14820100
204 11009500
205 16849500
206 26134400
207 33843500
208 32130100
209 16087400
210 17438700
211 14080400
212 15384500
213 40475900
214 18610300
215 11987000
216 10382000
217 12513000
218 23102600
219 67402600
220 16682500
221 15589900
222 22150000
223 64014200
224 18127200
225 11294600
226 10635400
227 18295300
228 23097100
229 8678100
230 8156300
231 14513900
232 7707800
233 7069100
234 6833500
235 6511900
236 6344300
237 7089500
238 23791600
239 10674900
240 33965900
241 8903700
242 8915100
243 6704400
244 6631900
245 30353300
246 41741700
247 41295500
248 7716900
249 7114400
250 6434900
251 6253700
252 6063400
253 6285400
254 14942200
255 8441700
256 15918400
257 6738400
258 6738400
259 9590000
260 6439400
261 6058900
262 5855000
263 5691900
264 5549300
265 5404300
266 5284200
267 5207200
268 5669300
269 5574200
270 7710100
271 5907100
272 10505100
273 12550400
274 26328400
275 21633000
276 8083800
277 18942200
278 7873100
279 8684000
280 8011300
281 6641000
282 6031700
283 34190200
284 19821000
285 31746200
286 18552600
287 7839200
288 12471100
289 8468800
290 22822100
291 17691900
292 8228700
293 7429200
294 9578700
295 11988600
296 19771200
297 13257000
298 12799500
299 26049800
300 8720300
301 18244600
302 16595700
303 8638700
304 7465400
305 14095100
306 23268300
307 8117800
308 7275200
309 6851600
310 8199300
311 6978500
312 21503900
313 50912700
314 44235500
315 13080400
316 10482400
317 7288800
318 7094000
319 7544700
320 9150600
321 8079300
322 7254800
323 23109800
324 19705500
325 9900300
326 8115500
327 12027200
328 11605900
329 8681700
330 8289900
331 9282000
332 9334100
333 6591200
334 9687400
335 12131300
336 7669300
337 22799500
338 11479000
339 6181200
340 10036200
341 8566200
342 6348800
343 8690800
344 6029400
345 5472200
346 5182300
347 25941000
348 22471100
349 6924100
350 10926400
351 11340900
352 11861800
353 10104200
354 10158500
355 11315900
356 29934200
357 8493800
358 7152900
359 22946700
360 11884500
361 9943400
362 10604700
363 7551500
364 6679500
365 10824400
366 8539100
367 10095100
368 6169900
369 6054300
370 5782500
371 5451900
372 7485800
373 8675000
374 5766700
375 7368000
376 17732700
377 15408800
378 14600200
379 16423500
380 10514100
381 6847100
382 11556000
383 14274000
384 6534500
385 6545900
386 7393000
387 12083800
388 6767800
389 6280800
390 5789300
391 5710100
392 5662500
393 5395200
394 5094000
395 4915100
396 4906000
397 5605900
398 6364700
399 8364600
400 5513000
401 4910500
402 4724800
403 4636500
404 4645500
405 4625100
406 4575300
407 4518700
408 6396400
409 5037400
410 4656800
411 5141600
412 4847100
413 4466600
414 4971700
415 7485800
416 4645500
417 4308000
418 4185700
419 4208400
420 9963700
421 4541300
422 4280900
423 4249100
424 4262700
425 4792700
426 4940000
427 12004500
428 4865200
429 4317100
430 4405400
431 4178900
432 4077000
433 3970500
434 3990900
435 4215200
436 3959200
437 3877700
438 4516400
439 4643300
440 3959200
};
\addplot [blue]
table {%
0 15826469.8980623
1 13890178.8063679
2 15096361.405046
3 11811806.5700963
4 11918636.322228
5 12847422.3375388
6 15293403.6896312
7 11985752.8819613
8 9306359.97393407
9 9834607.91985424
10 11478744.7809624
11 9866148.80414203
12 8826276.10555772
13 8221938.36322751
14 6813320.55291467
15 8554323.67791479
16 14178812.4920901
17 12105638.6467034
18 9294322.19501664
19 9853628.87015821
20 8114001.98201694
21 12031826.885251
22 7783445.66918313
23 6914661.56838779
24 8006089.83608386
25 7079144.0474895
26 6774889.83536834
27 6614601.27910372
28 5755771.85409803
29 6025096.30381142
30 5439962.15886638
31 5505816.38709394
32 5242868.75522891
33 6395526.06244739
34 5068530.16814867
35 5263772.84807027
36 5627919.66380311
37 4851308.69700919
38 5446127.47190625
39 5296263.52663632
40 4658962.55961741
41 4805359.65649304
42 5655324.94236554
43 6037819.03440007
44 5546149.12475477
45 6444670.3946404
46 5780846.79567176
47 4860834.34643841
48 4847375.19321821
49 5658089.34309071
50 5007080.89150478
51 6538755.16458395
52 5607268.09327085
53 4388690.50686767
54 4060954.03567586
55 4410907.69744936
56 4288248.3733508
57 6261314.52543359
58 6408579.80356477
59 8723166.05124225
60 11336498.3182704
61 7435033.93922101
62 5600844.20346317
63 4900451.86706362
64 4351086.99756416
65 4455077.88278037
66 4493738.67899089
67 4152819.68804533
68 4128461.72031663
69 4042470.6242475
70 3971263.01153616
71 4350737.21430063
72 4599132.37410203
73 5046608.19142879
74 5245063.4838306
75 4615836.54723171
76 4226722.92915423
77 3936255.64524591
78 3981641.62288351
79 4060279.07164062
80 3933352.70057333
81 4185629.23397556
82 3792401.02782868
83 4605024.30002077
84 4703459.05989905
85 5072747.31066552
86 4424954.15124735
87 4088744.68764722
88 4109903.48460152
89 4444378.43354665
90 5963930.59688363
91 4557131.31028104
92 4329938.70128529
93 3981328.8783783
94 4145366.31192525
95 4106979.43905837
96 3837903.46210452
97 3613900.57420529
98 4949170.99659906
99 4764518.74915513
100 4277207.98916027
101 5056884.00762487
102 5223957.69853493
103 5279201.60084752
104 4484802.67996724
105 5774635.79412237
106 7730585.44388415
107 6316532.54229506
108 6582604.8195787
109 5484118.05056316
110 6707501.50791992
111 5357508.3029779
112 5166467.48506818
113 5706329.20620603
114 4848880.49232092
115 7951564.3744803
116 11227124.4823293
117 6633463.00006571
118 8085840.78907921
119 5672952.04793326
120 7788767.24877994
121 9084975.26703618
122 7789878.66578144
123 7021869.58313957
124 10935725.5875297
125 10518264.560361
126 14768585.4337426
127 14177545.9128157
128 12664576.1269321
129 15072572.7653078
130 12029341.3066232
131 10092894.6223984
132 10170772.6777657
133 8529026.47353447
134 12193407.0608647
135 14485154.3774527
136 7784303.69031454
137 8946082.09424879
138 9009732.47863212
139 6523875.52457012
140 8627565.64064753
141 7641147.46032086
142 12407829.3728936
143 12875906.493314
144 11735848.1255161
145 12351004.509655
146 12033054.3260766
147 10269627.4027834
148 16773378.3420551
149 11365785.0989328
150 10875327.2203308
151 11955466.3654089
152 7040993.26966251
153 7791812.74616047
154 9632186.79436342
155 11839002.2900899
156 11481970.3305063
157 12743493.1372107
158 7316596.68514431
159 6707474.31543956
160 7922064.66268666
161 8172459.26977817
162 9112272.60263549
163 10047819.1816943
164 7486935.08462391
165 8484300.67539048
166 12009122.1813745
167 10714919.359217
168 12730271.9100143
169 9503425.03002219
170 10035191.7589617
171 10638837.3633265
172 11824963.1288695
173 12154176.4697993
174 13602512.9665721
175 12112610.4382686
176 14547516.5218319
177 22397992.861924
178 15042324.4997206
179 14181836.4095872
180 9465188.02139962
181 8839389.75730367
182 9378787.60985305
183 10741454.5205908
184 9588042.31083911
185 9721540.64002977
186 18293324.3328807
187 12416862.5322764
188 14519791.3262812
189 12575132.6000205
190 10014074.7322162
191 13645582.3409162
192 14068837.915055
193 9545924.87075597
194 11690641.9405516
195 14571326.411576
196 13546198.1608272
197 14445592.753846
198 15893022.5229949
199 14577093.0740409
200 15907298.8289543
201 15065655.4505801
202 22475960.2240876
203 21504382.2826693
204 13305583.7214702
205 9790382.45891148
206 9978097.87462142
207 15129531.3485737
208 15971414.5123478
209 21697054.7852204
210 17051177.3406293
211 14417441.549699
212 13639824.4220283
213 14399380.050757
214 15185672.1509824
215 11912208.1155968
216 13811638.3223633
217 12715041.5425849
218 15579564.668382
219 10468240.8812555
220 12809663.7065974
221 17813034.8594964
222 12651664.2180262
223 11934932.32281
224 11465841.7244408
225 18217439.2681471
226 12799310.7045846
227 13541668.9295626
228 16066631.457315
229 14278160.4237686
230 11087874.0553596
231 9939695.29617311
232 9903909.41085332
233 9045363.24313234
234 7822880.57095546
235 7469580.68885376
236 7607422.94695202
237 7885280.61181847
238 7467591.05545598
239 9393938.18908682
240 8577725.36809147
241 6915645.69279896
242 8921254.87699148
243 8602019.52075932
244 6106248.63895921
245 7965523.77923371
246 11956879.2669323
247 17367721.3068529
248 10615467.8603096
249 8428909.1316292
250 7354490.09691006
251 6739118.58475163
252 6345854.57648866
253 5717628.35488461
254 8173055.31122561
255 9583167.42361628
256 8253660.06466805
257 12239518.3082909
258 9793904.86332897
259 12747606.6606436
260 6722908.92979105
261 7138160.1022346
262 5810426.44327788
263 6361574.16936861
264 5820740.73911549
265 5860657.35738008
266 6097301.38519483
267 6116438.09455744
268 6725675.01144979
269 7452478.12738557
270 7160503.44122573
271 10842569.6363122
272 10512450.55313
273 12019654.5827576
274 13209347.7815975
275 14002037.923105
276 19538319.6346323
277 16203324.7808076
278 12202137.2569851
279 14521605.4097628
280 14237864.379369
281 14167056.8777405
282 10441209.6716886
283 10065974.3540209
284 10674220.737656
285 14742411.784123
286 17505965.8779037
287 14520507.473388
288 9547128.99122263
289 8943998.04084427
290 9396412.38734365
291 14604076.9541395
292 15963459.3169718
293 15085477.2516407
294 12482762.0384249
295 10458214.3112143
296 11763591.8555754
297 11659909.8323902
298 12950644.2452665
299 13893039.5797548
300 14857285.4185632
301 12043664.0024143
302 13313656.8947127
303 17545836.5430793
304 14070473.1337651
305 10345926.8482487
306 12890000.3994891
307 18970382.085742
308 10665935.6515442
309 8796985.86527781
310 8638066.81455338
311 11562441.1008922
312 9368740.85846375
313 11803648.3845229
314 13146546.7960808
315 31477603.5404582
316 22377070.3974504
317 13063178.2474936
318 9651596.06130048
319 8559200.20349279
320 11296447.219784
321 14029070.0511709
322 16695940.3650826
323 12317020.4013914
324 13227951.4108037
325 14566554.4739373
326 11180360.9216377
327 18476779.9061207
328 8933865.11364914
329 13877516.4465433
330 11533808.7314156
331 11645210.8427025
332 12212871.7543633
333 11527474.4806306
334 8577054.12685642
335 11118896.5561879
336 8864846.03585907
337 11403435.3538484
338 12389614.669839
339 15063043.9693946
340 10305173.0236395
341 10781021.6041138
342 12151372.6418614
343 10678297.2523689
344 9793851.12803236
345 8612602.21852603
346 8201122.95078175
347 7860568.73297866
348 12711693.2014414
349 15378238.1557868
350 10423456.4243826
351 13308160.6081344
352 14054483.0654439
353 16307941.8677496
354 13545307.7469278
355 12835216.9354112
356 14471939.2553513
357 12927258.6399046
358 9387451.95282564
359 9559456.02917221
360 11080491.1988201
361 11016106.4811039
362 11252237.8200517
363 10248444.5199337
364 9453016.95189783
365 8195078.79950581
366 7892494.96997171
367 8326158.09666039
368 7906306.41380585
369 6637886.38952071
370 5995083.9675803
371 5594583.09754998
372 6314811.94976981
373 6939988.10629678
374 7400222.64533585
375 8587423.41147281
376 11434874.4546117
377 12227349.268219
378 21044989.1091776
379 13775921.7960002
380 11752297.8668866
381 8052580.25965203
382 7595447.60199101
383 7516126.15376678
384 8992856.27688834
385 6962295.12004236
386 7194774.41752758
387 7566625.68333011
388 6856411.40461641
389 6745822.75331521
390 6961998.48256792
391 5588002.55087924
392 5192861.33302107
393 4893260.15664458
394 4937569.10785365
395 4640570.06950405
396 5064876.24528889
397 4631966.77944795
398 4641388.72780091
399 4710830.17017306
400 4753923.71096068
401 5012691.80251906
402 4636999.27280131
403 4611444.47126019
404 4974737.95120794
405 4887312.93567655
406 4877062.02949554
407 4742478.01229074
408 4522020.94390151
409 5030538.94114607
410 5600061.27212043
411 5579521.748153
412 5150920.82991456
413 5317774.63345861
414 4405877.45830903
415 5342326.55829231
416 5476248.50326284
417 4899000.00706246
418 4357883.77047536
419 4639374.97737854
420 4507457.22943051
421 5559868.51587471
422 5172696.27074218
423 4986475.09044429
424 4694440.40360263
425 4358481.62907953
426 4696742.67353944
427 5332396.70088596
428 6464221.67920149
429 5153601.74120508
430 4492671.77759085
431 4711553.01008392
432 4405945.26612801
433 4289487.02039636
434 4488057.08536832
435 4399033.54564161
436 4337781.17151046
437 4168745.36751241
438 3956445.01262389
439 4255737.35042453
440 4128054.0892542
};
\addplot [black]
table {%
0 14513799.8380988
1 11997734.6570562
2 13926318.3141062
3 10334133.6428525
4 9958470.84662075
5 11402202.6224426
6 13550129.6374571
7 10843943.7325501
8 8655344.83142299
9 9127663.5975066
10 10897106.0189668
11 9389563.27405326
12 8294187.68982007
13 7650370.22456377
14 6462017.19295669
15 8144430.74347337
16 13145989.6669625
17 11189951.938536
18 8782620.07217859
19 9297105.97094859
20 7598072.42089846
21 11091527.1699948
22 7431312.75206885
23 6563715.6531024
24 7608633.39365388
25 6805896.50639101
26 6479950.73969281
27 6279885.72403746
28 5437120.50259435
29 5759253.68451983
30 5192281.77112159
31 5275733.66992699
32 4986380.16463402
33 6075434.47436919
34 4815167.04264531
35 5034785.29523917
36 5356030.13234045
37 4630684.66105801
38 5183705.52402385
39 5034474.70766014
40 4413533.32445559
41 4616104.90864873
42 5373919.51758261
43 5761944.54821891
44 5247626.74635172
45 6087196.29756592
46 5573167.61066205
47 4675787.82570844
48 4591589.05727662
49 5347307.69003432
50 4711935.98373315
51 6150545.77134198
52 5360900.65851669
53 4160205.34154938
54 3884967.1995714
55 4188253.04217977
56 4043893.16184862
57 5978536.39769865
58 6056264.13870443
59 8230562.60383947
60 10353687.0260249
61 7009826.71527854
62 5317997.35430633
63 4702420.46228523
64 4115674.07959938
65 4234819.13438126
66 4286232.50649051
67 3950020.09014486
68 3925663.6417468
69 3811804.78067241
70 3784746.0108492
71 4117785.22448902
72 4279200.45365794
73 4782752.31065586
74 4914955.35302271
75 4366516.93048869
76 3993858.70318035
77 3782039.89938986
78 3775922.13457493
79 3868316.62784394
80 3708221.79966934
81 3970525.64388312
82 3579207.07622521
83 4358551.66609247
84 4485204.75013877
85 4782631.49685908
86 4171167.09597852
87 3843021.27784045
88 3890628.50693151
89 4227778.92888952
90 5646689.07541012
91 4298645.42441965
92 4079860.14353544
93 3752889.7701809
94 3886022.71502549
95 3913491.67461534
96 3645530.70464709
97 3386510.84749076
98 4735255.32195135
99 4583353.77658583
100 4102840.6579079
101 4814070.70405841
102 4876078.61151717
103 4968696.77214159
104 4213286.90376583
105 5196499.04408636
106 7188344.3899628
107 5876354.09036024
108 6180297.165261
109 5095064.49292069
110 6382532.07973891
111 4959813.39800151
112 4900344.68593794
113 5438113.26463021
114 4572335.77090564
115 7303805.11654929
116 10399262.8670794
117 6307474.22838745
118 7670693.67712516
119 5334333.12690356
120 7251510.10793301
121 8456297.95554524
122 7230606.9152052
123 6357652.60388754
124 9953385.48217472
125 9545417.22514471
126 13872639.9177393
127 13382280.6264847
128 11842634.7989948
129 14226957.4409402
130 11398622.20328
131 9438016.15855245
132 9584604.70729933
133 7958994.15095528
134 11081990.8064685
135 13212144.4234015
136 7111775.1244712
137 8162300.22635183
138 8188531.07591223
139 6160464.68921721
140 8037738.6487428
141 7000656.80060468
142 11123490.4912629
143 11603092.8499659
144 10853246.8758639
145 11261173.0009865
146 10744262.4808593
147 9438736.71934883
148 14596894.1414891
149 10476960.6389706
150 10090759.6748402
151 11015913.8714511
152 6522072.95922899
153 7219761.63855951
154 8627867.2668744
155 11084610.7382039
156 9771164.25763344
157 11437186.2449123
158 6793813.45916451
159 6244286.80811059
160 7474200.70217799
161 7670305.99257746
162 8473768.746802
163 9479051.28300592
164 7018506.8941782
165 7949678.46457887
166 11267173.8896722
167 9750300.99005774
168 11507729.5304484
169 8384744.34933078
170 9274855.93376781
171 9171293.40725797
172 10089643.0449076
173 11273528.123843
174 12064431.4011415
175 10781327.9918076
176 13339902.6000145
177 20288509.8384442
178 13625782.1965228
179 11822586.1739796
180 8571120.49644061
181 7944176.09150501
182 8478234.27240016
183 9842845.46048396
184 8740019.10051665
185 8681942.8007533
186 15231276.3823056
187 10049002.6585377
188 13228091.4339353
189 11696605.4981725
190 9222808.0804182
191 12656452.0852827
192 12935355.9326909
193 8307867.84720593
194 10771716.1539171
195 13067974.7972681
196 11807724.5010454
197 12890676.8184093
198 14172236.6675937
199 12807874.3598829
200 14501786.9501401
201 12740240.6410292
202 19445175.7936876
203 19465641.5069675
204 12110543.2498374
205 8800669.44686595
206 8979194.06287254
207 13438043.5043497
208 13933267.8657384
209 19477764.7606327
210 15386280.5090599
211 12361291.2254156
212 12474179.3054787
213 12542060.1925001
214 13690369.3363523
215 10888646.2415646
216 12445747.6244065
217 11381340.5235341
218 13778426.4963531
219 9461104.16105605
220 10945413.4918566
221 15423580.6766309
222 11241356.6811614
223 11014931.301941
224 10237005.7224216
225 17031113.1259065
226 11552653.1444507
227 12551813.1849663
228 15189881.6238631
229 13215492.7444095
230 10259394.8953917
231 9029498.89735188
232 9346356.03201238
233 8485420.27204505
234 7322313.20831135
235 6983384.54903958
236 7152200.70174719
237 7574915.70991693
238 6456457.07992298
239 8286650.68811375
240 7404177.48651427
241 6060988.43190044
242 7497988.07644054
243 7633395.34810341
244 5587597.96290984
245 7239058.75426522
246 10924856.8173579
247 15641372.0094449
248 9545009.17979475
249 7869835.16428566
250 6916521.20324262
251 6218368.32390983
252 5875290.69182031
253 5276470.82687099
254 7504570.69712401
255 8125519.12826309
256 7592537.1011065
257 11195016.3127483
258 9109673.99214869
259 11524467.1302912
260 5690966.35240564
261 6551513.04592847
262 5397828.51225842
263 5864703.77725527
264 5310358.97611683
265 5362200.46965948
266 5661815.6093768
267 5644482.8561099
268 6188756.63975061
269 6962184.77710919
270 6612924.11286827
271 9201632.51566035
272 9405045.10925315
273 10985615.1424204
274 11925372.8905028
275 12548310.0026682
276 17925044.1267087
277 14293926.2842288
278 10481049.6650241
279 12975047.9142131
280 12646393.7511815
281 13106414.6504835
282 9665331.38380369
283 8978615.22050538
284 9357707.19656778
285 13587037.8681208
286 15999383.5998713
287 13189765.5735582
288 8972478.47997849
289 8065780.67179901
290 8614802.8515237
291 13528638.3705869
292 14459616.0813566
293 13453079.4229227
294 11485870.9453109
295 9643878.444476
296 10891589.6608762
297 10616201.3218075
298 11770441.981158
299 12678673.5216197
300 13824827.7202509
301 11242884.1542363
302 12339064.3745712
303 15391992.7934375
304 12610621.1660688
305 9540489.88608075
306 12189120.6718218
307 17385290.0335666
308 9851512.78989419
309 8174414.64020947
310 8065974.81120562
311 10887027.9884221
312 8674768.0583395
313 10484691.9789834
314 11915743.0813747
315 22199408.9375348
316 20307859.4959976
317 11889887.8404776
318 8854993.4738829
319 8191328.32414027
320 10504391.9305779
321 12746855.6793813
322 15125957.6854217
323 11179445.1528451
324 11962547.4975346
325 13585438.4640668
326 10436813.5301824
327 16661942.141247
328 7557859.94462749
329 12336300.0002208
330 10960627.7052738
331 10735870.6867559
332 11416229.7254178
333 10509541.3378788
334 8010515.74514146
335 10334429.363901
336 8142100.01844826
337 10576241.4048552
338 11437921.0269863
339 13974790.1015293
340 9574485.0206312
341 10048128.1271421
342 11413636.7039246
343 10004381.3290861
344 9294770.91955796
345 8063086.85667269
346 7715705.40139749
347 7340290.82192243
348 11177765.5541392
349 14370409.2481551
350 9613464.75319336
351 12127353.8152284
352 12527952.4333095
353 14842503.2977548
354 12628708.2680463
355 11758151.0713091
356 13560403.0633993
357 11810023.1563182
358 8728944.51588807
359 9008186.92532509
360 10583065.3702117
361 10338176.5314971
362 10532516.9726063
363 9523027.87302457
364 8931965.18650445
365 7676415.47323375
366 7506766.94361029
367 7956373.09206272
368 7471370.42178897
369 6318149.77952461
370 5689070.40599714
371 5382252.50309412
372 6009199.51852185
373 6566053.18016488
374 7052265.15786029
375 8164278.95201559
376 10649281.47789
377 11371994.9995037
378 18807640.0666399
379 12889345.3353523
380 10766435.4906063
381 7667113.10046201
382 7182555.75087727
383 7189002.12188892
384 8554640.43031846
385 6618691.84575953
386 6890635.67159285
387 7137232.2258313
388 6525560.77390444
389 6475327.89113438
390 6553932.76160802
391 5290907.3524688
392 4988624.10606071
393 4703607.98313683
394 4696939.19535323
395 4454962.23755428
396 4820573.87700234
397 4399311.8807416
398 4446954.18264995
399 4466982.15089548
400 4571064.67679657
401 4752229.34253947
402 4396939.57390559
403 4387770.98304549
404 4718646.76962992
405 4655328.68143151
406 4589147.91931558
407 4503231.455074
408 4333563.24001303
409 4854005.61543412
410 5403811.56770946
411 5383413.16098116
412 4981946.65339604
413 5087248.11769827
414 4161318.86045596
415 5087280.43551512
416 5208614.32310046
417 4698619.09636439
418 4150050.15814347
419 4394107.84180859
420 4302438.17904335
421 5312189.33417867
422 4914215.17775103
423 4772551.29645989
424 4515923.9133098
425 4150220.82929283
426 4490864.45407945
427 5061904.23290688
428 6151132.05472714
429 4878944.85422271
430 4274662.10704819
431 4482618.68877634
432 4215257.30467318
433 4095236.77711155
434 4228069.17035551
435 4164144.4112542
436 4078507.34382264
437 3970511.49225108
438 3781384.30384435
439 4033806.59003976
440 3951011.3368275
};


\end{axis}

\end{tikzpicture}
}
%		\subfloat[$O.$]{% This file was created with tikzplotlib v0.10.1.
\begin{tikzpicture}


\begin{axis}[
	axis lines=left,
	font=\small,
	width=15cm,
	height=7cm,
%	axis background/.style={fill=background_color},
%	axis line style={white},
%	tick align=inside,
%	tick pos=left,
%	x grid style={white},
%	y grid style={white},
%	xmajorgrids,
%	ymajorgrids,
	title={Reservoir Contributions (kWh)},
	xlabel={Time (days)},
	xmin=50, xmax=250,
%	xtick style={color=black},
	ymin=-10000000, % Adjusted to -0.1*10^8
	ymax=80000000,  % Adjusted to 0.8*10^8
%	ytick style={color=black},
%	ytick={-10000000,0,20000000,40000000,60000000,80000000}, % Updated ticks for new range
%	yticklabels={
%		\ensuremath{-}0.1,
%		0.0,
%		0.2,
%		0.4,
%		0.6,
%		0.8
%	},
	legend columns=1, 
	legend style={
		nodes={scale=0.8, transform shape},
		fill opacity=0.8,
		draw opacity=1,
		text opacity=1,
		at={(0.03,0.97)},
		anchor=north west,
		legend pos=north west,
%		draw=lightgray204,
%		fill=background_color
	}
	]

\path [draw=blue, fill=blue, opacity=0.5]
(axis cs:0,30907406.4514597)
--(axis cs:0,5885922.78993782)
--(axis cs:1,5674543.05807476)
--(axis cs:2,7301785.31760189)
--(axis cs:3,6956792.82687586)
--(axis cs:4,8183185.13398946)
--(axis cs:5,8195896.04875933)
--(axis cs:6,8977533.45208891)
--(axis cs:7,9019784.38992647)
--(axis cs:8,6563541.59216855)
--(axis cs:9,5024031.82018058)
--(axis cs:10,3951316.81423389)
--(axis cs:11,3403244.46128388)
--(axis cs:12,3782391.59382283)
--(axis cs:13,3504576.48729532)
--(axis cs:14,2707612.09794435)
--(axis cs:15,3163042.44660711)
--(axis cs:16,3974093.56616021)
--(axis cs:17,5852836.73089923)
--(axis cs:18,4263134.95795006)
--(axis cs:19,3959661.0661793)
--(axis cs:20,3964063.41803515)
--(axis cs:21,3525253.3595841)
--(axis cs:22,1978248.24136687)
--(axis cs:23,1930215.12969607)
--(axis cs:24,1795470.67961426)
--(axis cs:25,1521195.59313902)
--(axis cs:26,1741044.87681918)
--(axis cs:27,1501987.83486739)
--(axis cs:28,1175185.73026311)
--(axis cs:29,1396483.47880615)
--(axis cs:30,949149.462315896)
--(axis cs:31,1123228.24390233)
--(axis cs:32,818526.291392528)
--(axis cs:33,937686.908303233)
--(axis cs:34,737592.724089867)
--(axis cs:35,813494.274872942)
--(axis cs:36,960671.555528153)
--(axis cs:37,860559.099717059)
--(axis cs:38,1066637.65883783)
--(axis cs:39,880174.293292029)
--(axis cs:40,1203749.6207275)
--(axis cs:41,799299.435247251)
--(axis cs:42,1041410.8596224)
--(axis cs:43,1122440.68145136)
--(axis cs:44,1185025.34164734)
--(axis cs:45,1068031.32659201)
--(axis cs:46,976186.989842073)
--(axis cs:47,625172.089805166)
--(axis cs:48,670027.717485929)
--(axis cs:49,1184285.79550688)
--(axis cs:50,883777.483314366)
--(axis cs:51,1066163.18486263)
--(axis cs:52,1140114.9025892)
--(axis cs:53,575074.852245501)
--(axis cs:54,626863.146294142)
--(axis cs:55,632801.920292059)
--(axis cs:56,578851.716113477)
--(axis cs:57,1196437.80523812)
--(axis cs:58,989773.072905666)
--(axis cs:59,1797140.29591263)
--(axis cs:60,1798154.77964696)
--(axis cs:61,1304855.40957626)
--(axis cs:62,779553.20094473)
--(axis cs:63,569662.222496089)
--(axis cs:64,578825.174967114)
--(axis cs:65,500652.438646471)
--(axis cs:66,448378.423212281)
--(axis cs:67,420966.139404685)
--(axis cs:68,416336.675522104)
--(axis cs:69,388884.452952956)
--(axis cs:70,351795.021472162)
--(axis cs:71,416945.249087078)
--(axis cs:72,621184.216491492)
--(axis cs:73,437569.654576349)
--(axis cs:74,291952.967227485)
--(axis cs:75,406831.279136508)
--(axis cs:76,286448.731488487)
--(axis cs:77,307267.713257018)
--(axis cs:78,310281.944410744)
--(axis cs:79,291076.089038305)
--(axis cs:80,284069.499330887)
--(axis cs:81,298451.884961761)
--(axis cs:82,377695.916852841)
--(axis cs:83,354547.143998929)
--(axis cs:84,321760.601013329)
--(axis cs:85,468192.91339697)
--(axis cs:86,393200.776244804)
--(axis cs:87,312057.91216927)
--(axis cs:88,295412.833710447)
--(axis cs:89,554706.917688585)
--(axis cs:90,516360.598094657)
--(axis cs:91,494111.100014269)
--(axis cs:92,370889.854193524)
--(axis cs:93,358099.934615687)
--(axis cs:94,358539.275531733)
--(axis cs:95,468705.953740391)
--(axis cs:96,295965.310266312)
--(axis cs:97,197864.488907552)
--(axis cs:98,356521.415093029)
--(axis cs:99,276908.054028984)
--(axis cs:100,253172.4422219)
--(axis cs:101,350615.41508964)
--(axis cs:102,266061.615309853)
--(axis cs:103,449415.0290544)
--(axis cs:104,322254.859010478)
--(axis cs:105,524458.209057414)
--(axis cs:106,1334293.54515099)
--(axis cs:107,896890.548558071)
--(axis cs:108,948184.657988197)
--(axis cs:109,484743.186544185)
--(axis cs:110,1018441.13138522)
--(axis cs:111,1031671.3502079)
--(axis cs:112,597977.644558914)
--(axis cs:113,569193.262254381)
--(axis cs:114,579148.38560621)
--(axis cs:115,1062122.20096183)
--(axis cs:116,2465459.77493679)
--(axis cs:117,1234431.38923625)
--(axis cs:118,1383804.85495062)
--(axis cs:119,712382.348651562)
--(axis cs:120,985623.702975463)
--(axis cs:121,1065269.3816668)
--(axis cs:122,954909.247549378)
--(axis cs:123,1391683.66143444)
--(axis cs:124,4187814.01334828)
--(axis cs:125,2217462.29546424)
--(axis cs:126,2921683.35272755)
--(axis cs:127,2092471.97076557)
--(axis cs:128,2013094.92666548)
--(axis cs:129,2347429.64240725)
--(axis cs:130,1678117.56757269)
--(axis cs:131,1319292.24769294)
--(axis cs:132,1293322.03081131)
--(axis cs:133,1116452.77034514)
--(axis cs:134,2206196.11739621)
--(axis cs:135,4494548.99962802)
--(axis cs:136,1770540.6675187)
--(axis cs:137,2105862.53864624)
--(axis cs:138,1837669.25303643)
--(axis cs:139,2179670.96744299)
--(axis cs:140,3074966.30594941)
--(axis cs:141,2234087.37280266)
--(axis cs:142,2161899.1783829)
--(axis cs:143,2490254.16821019)
--(axis cs:144,3409825.17436242)
--(axis cs:145,3119550.55482692)
--(axis cs:146,3153374.39827717)
--(axis cs:147,3712193.23046244)
--(axis cs:148,5194753.34239912)
--(axis cs:149,4470694.92589561)
--(axis cs:150,2897807.47054268)
--(axis cs:151,7859520.64788529)
--(axis cs:152,5807236.26266111)
--(axis cs:153,6537234.69029279)
--(axis cs:154,5060620.91715702)
--(axis cs:155,4654802.77415325)
--(axis cs:156,6807040.69104126)
--(axis cs:157,5846368.87004621)
--(axis cs:158,3049321.67459329)
--(axis cs:159,2407199.20059873)
--(axis cs:160,2423266.93220722)
--(axis cs:161,2304686.08890137)
--(axis cs:162,2056799.04374939)
--(axis cs:163,2710417.98343284)
--(axis cs:164,1950640.59611637)
--(axis cs:165,2108626.80029966)
--(axis cs:166,3355919.31777448)
--(axis cs:167,3554253.21428481)
--(axis cs:168,4825345.21652113)
--(axis cs:169,8583443.04109458)
--(axis cs:170,7352386.69048913)
--(axis cs:171,8438919.30352168)
--(axis cs:172,7959027.84575208)
--(axis cs:173,9323736.16401268)
--(axis cs:174,12226641.1956416)
--(axis cs:175,9168070.65343566)
--(axis cs:176,9646519.74676877)
--(axis cs:177,7853798.91381352)
--(axis cs:178,9462950.47762435)
--(axis cs:179,13748834.5005414)
--(axis cs:180,11334287.0340279)
--(axis cs:181,9365122.73363676)
--(axis cs:182,8500298.65753706)
--(axis cs:183,6974809.83608806)
--(axis cs:184,4093380.09949642)
--(axis cs:185,4137784.01428194)
--(axis cs:186,4428680.53816348)
--(axis cs:187,6991351.65173873)
--(axis cs:188,5315324.82770793)
--(axis cs:189,4472481.41795101)
--(axis cs:190,3868805.84633733)
--(axis cs:191,3487904.20338225)
--(axis cs:192,3269921.01495045)
--(axis cs:193,5610934.16841571)
--(axis cs:194,6409566.46787521)
--(axis cs:195,5683457.85231325)
--(axis cs:196,6197642.06216503)
--(axis cs:197,5168836.20709938)
--(axis cs:198,7081783.41341364)
--(axis cs:199,8222176.50188293)
--(axis cs:200,6672683.07126275)
--(axis cs:201,7789608.95212143)
--(axis cs:202,8400243.36157936)
--(axis cs:203,9101659.25258377)
--(axis cs:204,8895883.87374998)
--(axis cs:205,6233890.2294476)
--(axis cs:206,7398549.90511274)
--(axis cs:207,6430356.26271581)
--(axis cs:208,7940391.41339114)
--(axis cs:209,6901018.81867239)
--(axis cs:210,11895836.1328784)
--(axis cs:211,6356548.59057485)
--(axis cs:212,10338172.6082507)
--(axis cs:213,10861931.1816734)
--(axis cs:214,11872467.3086382)
--(axis cs:215,9594892.87895865)
--(axis cs:216,9642674.81646281)
--(axis cs:217,13395779.6000364)
--(axis cs:218,8513769.7207618)
--(axis cs:219,8369877.68060045)
--(axis cs:220,12742191.1041494)
--(axis cs:221,17844527.5247142)
--(axis cs:222,12309350.2711541)
--(axis cs:223,8961720.65478013)
--(axis cs:224,8095881.73521033)
--(axis cs:225,7246445.98931923)
--(axis cs:226,5600278.62800738)
--(axis cs:227,5078266.29217994)
--(axis cs:228,6061725.77955029)
--(axis cs:229,8427322.37777764)
--(axis cs:230,9633530.8166702)
--(axis cs:231,9647537.97854844)
--(axis cs:232,8791207.46929713)
--(axis cs:233,5594834.67091531)
--(axis cs:234,6054270.84717083)
--(axis cs:235,6330371.29844263)
--(axis cs:236,6401054.58601598)
--(axis cs:237,7212764.8196859)
--(axis cs:238,9809815.68165749)
--(axis cs:239,15428721.5136098)
--(axis cs:240,10279396.2328681)
--(axis cs:241,7452596.29537626)
--(axis cs:242,9960130.84129409)
--(axis cs:243,11638915.7063169)
--(axis cs:244,10789041.8465765)
--(axis cs:245,9759680.11014523)
--(axis cs:246,9381321.64545867)
--(axis cs:247,8131430.18191267)
--(axis cs:248,9502108.30393242)
--(axis cs:249,11816373.1473228)
--(axis cs:250,10810543.4266018)
--(axis cs:251,18182887.3666676)
--(axis cs:252,25668170.5000103)
--(axis cs:253,23414250.6982734)
--(axis cs:254,12249167.6070946)
--(axis cs:255,13146536.9276894)
--(axis cs:256,13211775.666662)
--(axis cs:257,10956399.719752)
--(axis cs:258,13905908.41429)
--(axis cs:259,10229376.6760242)
--(axis cs:260,13749834.7425519)
--(axis cs:261,13502351.9181865)
--(axis cs:262,10952763.5736677)
--(axis cs:263,16911381.3101086)
--(axis cs:264,18048853.6073122)
--(axis cs:265,13540809.8481116)
--(axis cs:266,11088934.1988711)
--(axis cs:267,11440076.0192202)
--(axis cs:268,8479460.79407775)
--(axis cs:269,6496459.88255137)
--(axis cs:270,7511496.97898794)
--(axis cs:271,13871495.3357715)
--(axis cs:272,13218869.7271439)
--(axis cs:273,11306477.0735967)
--(axis cs:274,9073863.72767941)
--(axis cs:275,7243336.22443831)
--(axis cs:276,9016953.15442407)
--(axis cs:277,8415278.38614876)
--(axis cs:278,10947933.3322801)
--(axis cs:279,6998521.75621858)
--(axis cs:280,6909559.10969476)
--(axis cs:281,7379816.54111927)
--(axis cs:282,8080349.87329687)
--(axis cs:283,9636050.27412812)
--(axis cs:284,6674818.28170529)
--(axis cs:285,5643060.0877621)
--(axis cs:286,5190403.02260596)
--(axis cs:287,5007120.39754427)
--(axis cs:288,4631625.70947617)
--(axis cs:289,6368279.23901656)
--(axis cs:290,5961320.00721586)
--(axis cs:291,6676578.92018345)
--(axis cs:292,5439848.93986802)
--(axis cs:293,4599629.60502644)
--(axis cs:294,4618721.54446433)
--(axis cs:295,3571849.92165438)
--(axis cs:296,4048967.16713747)
--(axis cs:297,3441618.84813054)
--(axis cs:298,3219848.44612806)
--(axis cs:299,3696846.96266786)
--(axis cs:300,3247579.25179106)
--(axis cs:301,3283952.03138324)
--(axis cs:302,4642174.53896253)
--(axis cs:303,10102445.1414483)
--(axis cs:304,10170619.0929285)
--(axis cs:305,6666880.56787654)
--(axis cs:306,5067312.83377437)
--(axis cs:307,4400712.87731744)
--(axis cs:308,5042738.62250831)
--(axis cs:309,5013968.83767498)
--(axis cs:310,4029528.35661837)
--(axis cs:311,4385369.73331181)
--(axis cs:312,4596221.72229076)
--(axis cs:313,3508063.15146899)
--(axis cs:314,5431127.44258737)
--(axis cs:315,5351588.39306578)
--(axis cs:316,7362632.67197411)
--(axis cs:317,9420654.0854381)
--(axis cs:318,5682107.72658379)
--(axis cs:319,4256823.8750039)
--(axis cs:320,4697740.15697164)
--(axis cs:321,5707905.18248479)
--(axis cs:322,3883883.97735439)
--(axis cs:323,3918063.66695527)
--(axis cs:324,3335132.94007076)
--(axis cs:325,3143126.46797185)
--(axis cs:326,3245422.54779617)
--(axis cs:327,4002944.84741737)
--(axis cs:328,5611325.11617583)
--(axis cs:329,6662831.56306687)
--(axis cs:330,4380736.90980662)
--(axis cs:331,3640388.39934131)
--(axis cs:332,3298331.89462565)
--(axis cs:333,3450353.22629363)
--(axis cs:334,3756787.12673499)
--(axis cs:335,3632144.96065723)
--(axis cs:336,3173995.28178716)
--(axis cs:337,2936465.30958241)
--(axis cs:338,5688568.66478927)
--(axis cs:339,6354918.27897763)
--(axis cs:340,3612129.24388249)
--(axis cs:341,3655025.41338249)
--(axis cs:342,3946968.45189771)
--(axis cs:343,5373970.37347272)
--(axis cs:344,3384818.16340067)
--(axis cs:345,2699665.40736224)
--(axis cs:346,2821564.70720534)
--(axis cs:347,2727266.92077257)
--(axis cs:348,5758722.19064767)
--(axis cs:349,7596014.62233575)
--(axis cs:350,7443577.35666882)
--(axis cs:351,8356835.84293042)
--(axis cs:352,9627219.46011807)
--(axis cs:353,6845695.51505982)
--(axis cs:354,6065686.37170465)
--(axis cs:355,5708725.78605777)
--(axis cs:356,9203669.97008046)
--(axis cs:357,10651545.9824798)
--(axis cs:358,7059257.49950107)
--(axis cs:359,5729878.43461214)
--(axis cs:360,5346773.13236028)
--(axis cs:361,4986651.28279049)
--(axis cs:362,4858615.43991823)
--(axis cs:363,3635066.5917467)
--(axis cs:364,3261658.97302984)
--(axis cs:365,2729343.78025953)
--(axis cs:366,2393662.33304993)
--(axis cs:367,2397017.93032576)
--(axis cs:368,2183968.85073964)
--(axis cs:369,1561968.27568165)
--(axis cs:370,1504705.38734753)
--(axis cs:371,1115345.94915297)
--(axis cs:372,1548269.14966113)
--(axis cs:373,1725084.35071295)
--(axis cs:374,1637833.16178425)
--(axis cs:375,2253411.33341079)
--(axis cs:376,4063904.38212003)
--(axis cs:377,3893908.94085638)
--(axis cs:378,5503964.55824355)
--(axis cs:379,3655975.40685349)
--(axis cs:380,2837472.92147622)
--(axis cs:381,1975589.69351533)
--(axis cs:382,2074520.1697708)
--(axis cs:383,1569034.66048513)
--(axis cs:384,1593125.1907466)
--(axis cs:385,1188698.67094589)
--(axis cs:386,1258446.03761076)
--(axis cs:387,1129419.84001063)
--(axis cs:388,997809.681282995)
--(axis cs:389,1047942.25675551)
--(axis cs:390,1034396.71414504)
--(axis cs:391,834479.349800151)
--(axis cs:392,923140.32875804)
--(axis cs:393,782335.914563685)
--(axis cs:394,1019495.0995653)
--(axis cs:395,859245.071988825)
--(axis cs:396,929741.013921958)
--(axis cs:397,582710.68067314)
--(axis cs:398,564277.115197866)
--(axis cs:399,765149.908633839)
--(axis cs:400,662383.531390484)
--(axis cs:401,790968.714943065)
--(axis cs:402,656868.530271685)
--(axis cs:403,712265.826661303)
--(axis cs:404,956314.743010091)
--(axis cs:405,982663.792073087)
--(axis cs:406,776901.887679556)
--(axis cs:407,640899.799539382)
--(axis cs:408,623205.05472591)
--(axis cs:409,968760.52056564)
--(axis cs:410,972558.038259843)
--(axis cs:411,928068.87001041)
--(axis cs:412,1023035.3434562)
--(axis cs:413,873814.437637977)
--(axis cs:414,653636.052045234)
--(axis cs:415,784966.294567908)
--(axis cs:416,785282.572579655)
--(axis cs:417,666827.845717447)
--(axis cs:418,518345.578987251)
--(axis cs:419,505802.962987898)
--(axis cs:420,537978.068505467)
--(axis cs:421,639022.101410031)
--(axis cs:422,552775.651046558)
--(axis cs:423,507323.168394768)
--(axis cs:424,484162.927752758)
--(axis cs:425,404238.958893168)
--(axis cs:426,488064.294943239)
--(axis cs:427,449202.257629827)
--(axis cs:428,658603.676186485)
--(axis cs:429,469330.385086818)
--(axis cs:430,371638.602575499)
--(axis cs:431,512666.39473428)
--(axis cs:432,454228.701608211)
--(axis cs:433,374808.909598031)
--(axis cs:434,345175.884406033)
--(axis cs:435,397675.743865078)
--(axis cs:436,562026.310520272)
--(axis cs:437,514053.464560066)
--(axis cs:438,331948.934847177)
--(axis cs:439,771993.608710197)
--(axis cs:440,425095.600541194)
--(axis cs:440,4199289.41172533)
--(axis cs:440,4199289.41172533)
--(axis cs:439,5263544.99962118)
--(axis cs:438,3818942.60412088)
--(axis cs:437,4618188.69990215)
--(axis cs:436,4728135.30626294)
--(axis cs:435,4052103.13101315)
--(axis cs:434,4133688.57304111)
--(axis cs:433,3986905.48424963)
--(axis cs:432,4266202.43738488)
--(axis cs:431,4639610.26834708)
--(axis cs:430,4086673.67146971)
--(axis cs:429,4630651.88498612)
--(axis cs:428,5909738.08595537)
--(axis cs:427,4291683.26714343)
--(axis cs:426,4430276.03212385)
--(axis cs:425,4146270.05640581)
--(axis cs:424,4379158.57181653)
--(axis cs:423,4468609.3331454)
--(axis cs:422,4658124.13311879)
--(axis cs:421,4824686.54011011)
--(axis cs:420,4510879.42990366)
--(axis cs:419,4553589.50305825)
--(axis cs:418,4422728.93501595)
--(axis cs:417,4851426.62973183)
--(axis cs:416,5614774.7346897)
--(axis cs:415,5526964.00691462)
--(axis cs:414,5470393.57455906)
--(axis cs:413,5926840.43958532)
--(axis cs:412,6208355.04309339)
--(axis cs:411,5968121.02158074)
--(axis cs:410,5884214.47079459)
--(axis cs:409,6025731.40357646)
--(axis cs:408,4987703.05070686)
--(axis cs:407,5308124.66512419)
--(axis cs:406,5767149.89908403)
--(axis cs:405,6474017.77728772)
--(axis cs:404,6406287.84291345)
--(axis cs:403,5877550.19988039)
--(axis cs:402,5655336.47440097)
--(axis cs:401,6099774.67390184)
--(axis cs:400,5291037.08656023)
--(axis cs:399,6130470.43969601)
--(axis cs:398,5258439.13700402)
--(axis cs:397,5250051.64528962)
--(axis cs:396,8007819.0213936)
--(axis cs:395,6382625.3894454)
--(axis cs:394,6810077.26067491)
--(axis cs:393,5738505.43750242)
--(axis cs:392,6509519.83214562)
--(axis cs:391,6256760.80901626)
--(axis cs:390,7268595.62931472)
--(axis cs:389,7176817.46126866)
--(axis cs:388,7420154.99863558)
--(axis cs:387,8203979.65226962)
--(axis cs:386,8208120.54934509)
--(axis cs:385,8527250.74451589)
--(axis cs:384,10114216.0523928)
--(axis cs:383,10077226.1671876)
--(axis cs:382,12515729.3942162)
--(axis cs:381,12029744.16748)
--(axis cs:380,18382389.4497451)
--(axis cs:379,21931659.4898066)
--(axis cs:378,32119867.070031)
--(axis cs:377,20820209.8705683)
--(axis cs:376,24948150.4232784)
--(axis cs:375,13294629.1234139)
--(axis cs:374,10965545.7905627)
--(axis cs:373,11314382.9071257)
--(axis cs:372,10593742.0637399)
--(axis cs:371,8576076.23809106)
--(axis cs:370,9069698.93993837)
--(axis cs:369,9485139.24966531)
--(axis cs:368,12852205.4554322)
--(axis cs:367,13762894.4832561)
--(axis cs:366,13687612.4645395)
--(axis cs:365,15566675.848878)
--(axis cs:364,17061337.2891881)
--(axis cs:363,19664924.6194977)
--(axis cs:362,25144748.6747629)
--(axis cs:361,24232926.5112259)
--(axis cs:360,25016658.7708773)
--(axis cs:359,26422032.2018139)
--(axis cs:358,32996183.5984118)
--(axis cs:357,47398028.8214451)
--(axis cs:356,40219331.6917484)
--(axis cs:355,28905967.3930454)
--(axis cs:354,32007322.0913327)
--(axis cs:353,31985076.1367082)
--(axis cs:352,43111394.2949487)
--(axis cs:351,37219593.2611589)
--(axis cs:350,36268949.2467646)
--(axis cs:349,36098827.4871247)
--(axis cs:348,32821909.2721048)
--(axis cs:347,15787497.5914623)
--(axis cs:346,16445861.5089191)
--(axis cs:345,14812096.2438514)
--(axis cs:344,18756905.9335852)
--(axis cs:343,30971165.7016194)
--(axis cs:342,22302483.5983266)
--(axis cs:341,20993698.1816097)
--(axis cs:340,20348169.9786559)
--(axis cs:339,33284668.8577015)
--(axis cs:338,33237255.123715)
--(axis cs:337,17936012.1471995)
--(axis cs:336,18016255.2876366)
--(axis cs:335,18959121.8845204)
--(axis cs:334,21275356.8717298)
--(axis cs:333,22358248.7255727)
--(axis cs:332,19026992.0625653)
--(axis cs:331,20314532.2896306)
--(axis cs:330,24161823.2796418)
--(axis cs:329,37377306.3708081)
--(axis cs:328,34282242.611517)
--(axis cs:327,24366583.5055708)
--(axis cs:326,18149451.0588443)
--(axis cs:325,18191804.0407082)
--(axis cs:324,19769761.1426609)
--(axis cs:323,22918496.4287368)
--(axis cs:322,22929127.0902387)
--(axis cs:321,29626099.0200389)
--(axis cs:320,23163977.3193191)
--(axis cs:319,21877404.9032236)
--(axis cs:318,28570111.2474229)
--(axis cs:317,42804253.2748846)
--(axis cs:316,36196246.8076043)
--(axis cs:315,36674359.4645165)
--(axis cs:314,31421425.1906672)
--(axis cs:313,21217535.1550024)
--(axis cs:312,22553204.4858835)
--(axis cs:311,22101177.0678996)
--(axis cs:310,20993247.0917678)
--(axis cs:309,25858639.8454585)
--(axis cs:308,25264769.7295539)
--(axis cs:307,23365468.291973)
--(axis cs:306,26392697.1962685)
--(axis cs:305,35255657.2822883)
--(axis cs:304,45904336.2582476)
--(axis cs:303,49064582.6554454)
--(axis cs:302,27899613.7706029)
--(axis cs:301,19496350.7354926)
--(axis cs:300,18904851.617763)
--(axis cs:299,20863260.8189016)
--(axis cs:298,19989921.1833747)
--(axis cs:297,20731501.2251396)
--(axis cs:296,21875505.3174502)
--(axis cs:295,20617078.4280775)
--(axis cs:294,25773286.5491508)
--(axis cs:293,29997732.4868623)
--(axis cs:292,30877926.3490141)
--(axis cs:291,32257968.4820293)
--(axis cs:290,32282313.1228062)
--(axis cs:289,37502244.6044886)
--(axis cs:288,27348258.8395536)
--(axis cs:287,27813690.4435484)
--(axis cs:286,27002598.9074423)
--(axis cs:285,29326763.3269021)
--(axis cs:284,33061960.5431371)
--(axis cs:283,44810607.6673006)
--(axis cs:282,38839697.0114324)
--(axis cs:281,39548602.7998612)
--(axis cs:280,36969658.6218567)
--(axis cs:279,38966399.2340311)
--(axis cs:278,59325482.663408)
--(axis cs:277,39858016.07084)
--(axis cs:276,42014756.5951939)
--(axis cs:275,38124287.3146453)
--(axis cs:274,46059938.7169577)
--(axis cs:273,49881722.1286587)
--(axis cs:272,56914531.8272636)
--(axis cs:271,73079825.3809388)
--(axis cs:270,32375885.3533778)
--(axis cs:269,28433886.9913605)
--(axis cs:268,34480946.230896)
--(axis cs:267,47154270.6506619)
--(axis cs:266,44160137.7425085)
--(axis cs:265,58089025.7558108)
--(axis cs:264,78333884.5980151)
--(axis cs:263,68816427.0143886)
--(axis cs:262,47470360.2480301)
--(axis cs:261,56747625.9222415)
--(axis cs:260,63250019.5038976)
--(axis cs:259,48845449.5107154)
--(axis cs:258,58811523.6211435)
--(axis cs:257,48893925.3812921)
--(axis cs:256,57910547.9343189)
--(axis cs:255,56681689.1139916)
--(axis cs:254,51247492.3237838)
--(axis cs:253,119579645.037083)
--(axis cs:252,305785769.363122)
--(axis cs:251,93327319.9002433)
--(axis cs:250,50756456.9957249)
--(axis cs:249,54073495.033407)
--(axis cs:248,47756510.6623956)
--(axis cs:247,41158152.994058)
--(axis cs:246,45924689.6402783)
--(axis cs:245,45185173.0286492)
--(axis cs:244,53157935.394833)
--(axis cs:243,57040752.9040771)
--(axis cs:242,49593942.0901391)
--(axis cs:241,40671142.563746)
--(axis cs:240,54778407.4952251)
--(axis cs:239,86206805.1205929)
--(axis cs:238,51149787.9960311)
--(axis cs:237,31982658.7044678)
--(axis cs:236,29061661.9785873)
--(axis cs:235,29983347.3115835)
--(axis cs:234,29605645.9352351)
--(axis cs:233,27147369.9770555)
--(axis cs:232,41730796.2947655)
--(axis cs:231,45727997.4309129)
--(axis cs:230,45993485.0198137)
--(axis cs:229,41338497.6753431)
--(axis cs:228,32287820.9056374)
--(axis cs:227,26211792.5515864)
--(axis cs:226,29856739.813329)
--(axis cs:225,36558914.4728581)
--(axis cs:224,39561108.8255466)
--(axis cs:223,41430871.6979982)
--(axis cs:222,56331651.5976199)
--(axis cs:221,94489742.6225434)
--(axis cs:220,238728692.658533)
--(axis cs:219,43860154.7920334)
--(axis cs:218,40256613.7793458)
--(axis cs:217,61529186.099065)
--(axis cs:216,49180430.2715794)
--(axis cs:215,51696071.3398773)
--(axis cs:214,60579010.4360006)
--(axis cs:213,54738534.6118452)
--(axis cs:212,52057275.3123109)
--(axis cs:211,37252920.7073407)
--(axis cs:210,55323960.7667376)
--(axis cs:209,38957899.1281953)
--(axis cs:208,44643183.5954432)
--(axis cs:207,35501894.8048815)
--(axis cs:206,37247947.5079995)
--(axis cs:205,33347775.6830233)
--(axis cs:204,41750122.8307517)
--(axis cs:203,42883287.5820727)
--(axis cs:202,38415374.5349056)
--(axis cs:201,41729699.734216)
--(axis cs:200,37240659.001287)
--(axis cs:199,46862055.7254133)
--(axis cs:198,36458114.0304558)
--(axis cs:197,33249517.6092754)
--(axis cs:196,39595135.8804486)
--(axis cs:195,32508123.0477923)
--(axis cs:194,36862176.2543281)
--(axis cs:193,32365151.0246501)
--(axis cs:192,25105088.2294362)
--(axis cs:191,22903797.8731198)
--(axis cs:190,24826961.7395492)
--(axis cs:189,26177037.6405506)
--(axis cs:188,32606151.7361513)
--(axis cs:187,42149932.4141385)
--(axis cs:186,35605431.2187158)
--(axis cs:185,30071075.7194126)
--(axis cs:184,26808768.2281718)
--(axis cs:183,37379003.3201648)
--(axis cs:182,43715898.5868172)
--(axis cs:181,47041899.3997055)
--(axis cs:180,56451508.3282476)
--(axis cs:179,74059373.3070983)
--(axis cs:178,45741177.7066333)
--(axis cs:177,42628409.3226226)
--(axis cs:176,45925586.2285622)
--(axis cs:175,45635323.8654739)
--(axis cs:174,58526888.4029096)
--(axis cs:173,43252798.7037189)
--(axis cs:172,46219241.4597439)
--(axis cs:171,44248732.8907601)
--(axis cs:170,38167804.8096629)
--(axis cs:169,48360630.5680163)
--(axis cs:168,30380202.9341655)
--(axis cs:167,22075636.8426656)
--(axis cs:166,21417404.9796468)
--(axis cs:165,15449271.2575766)
--(axis cs:164,14193505.2428978)
--(axis cs:163,16410203.2887796)
--(axis cs:162,15293227.4690749)
--(axis cs:161,15988802.8175019)
--(axis cs:160,16410006.6267682)
--(axis cs:159,17006613.5068585)
--(axis cs:158,22148215.8249944)
--(axis cs:157,30412818.7667384)
--(axis cs:156,37451413.0768894)
--(axis cs:155,25907320.4800542)
--(axis cs:154,29796321.8137097)
--(axis cs:153,33675433.1359814)
--(axis cs:152,33651699.566187)
--(axis cs:151,48399966.3322372)
--(axis cs:150,21043173.7399156)
--(axis cs:149,26437965.7828873)
--(axis cs:148,30659116.8284679)
--(axis cs:147,26370942.4893997)
--(axis cs:146,23154414.7396531)
--(axis cs:145,23407901.3196011)
--(axis cs:144,24662706.2701574)
--(axis cs:143,21932521.3781466)
--(axis cs:142,19513936.2353838)
--(axis cs:141,20461046.5476423)
--(axis cs:140,25103300.2639423)
--(axis cs:139,21912393.5078485)
--(axis cs:138,19231517.9150977)
--(axis cs:137,20288123.8723443)
--(axis cs:136,17659691.892833)
--(axis cs:135,28992565.606704)
--(axis cs:134,18909537.0122337)
--(axis cs:133,9846382.82989278)
--(axis cs:132,11443441.0379957)
--(axis cs:131,10800821.2126233)
--(axis cs:130,13530505.3159609)
--(axis cs:129,19360701.8583425)
--(axis cs:128,17021778.2061125)
--(axis cs:127,16766165.6684989)
--(axis cs:126,25152410.3523359)
--(axis cs:125,16988109.2911192)
--(axis cs:124,27920254.6201206)
--(axis cs:123,17385837.2854033)
--(axis cs:122,11276396.7988257)
--(axis cs:121,12630869.7752584)
--(axis cs:120,8901250.0818607)
--(axis cs:119,7668918.75883771)
--(axis cs:118,12616812.0524451)
--(axis cs:117,11817036.3463745)
--(axis cs:116,24852128.2608868)
--(axis cs:115,15463589.1890112)
--(axis cs:114,6297129.69147975)
--(axis cs:113,6293631.04049408)
--(axis cs:112,6439658.64207188)
--(axis cs:111,11755927.0095615)
--(axis cs:110,12269578.218331)
--(axis cs:109,7293472.90140121)
--(axis cs:108,10327909.2357345)
--(axis cs:107,10222168.2334478)
--(axis cs:106,12969254.8018652)
--(axis cs:105,13244789.1176147)
--(axis cs:104,5044976.20097081)
--(axis cs:103,5710747.26996729)
--(axis cs:102,6728089.9427976)
--(axis cs:101,4368750.58573883)
--(axis cs:100,4415523.74963993)
--(axis cs:99,4000092.37141886)
--(axis cs:98,4284010.08869339)
--(axis cs:97,3490376.60940341)
--(axis cs:96,3865187.75572023)
--(axis cs:95,4519671.4742157)
--(axis cs:94,4613062.24608463)
--(axis cs:93,4177861.27667571)
--(axis cs:92,5033785.94701024)
--(axis cs:91,5707540.58931229)
--(axis cs:90,5599455.31266141)
--(axis cs:89,4747961.61216315)
--(axis cs:88,3850863.80318259)
--(axis cs:87,4103481.71630969)
--(axis cs:86,4675993.77923644)
--(axis cs:85,4940873.40835879)
--(axis cs:84,3791606.47415043)
--(axis cs:83,4792824.65342116)
--(axis cs:82,4428362.75821313)
--(axis cs:81,4242658.82889773)
--(axis cs:80,3657509.10868021)
--(axis cs:79,4054534.82920661)
--(axis cs:78,4037710.36183407)
--(axis cs:77,3898448.61515546)
--(axis cs:76,4516804.16665704)
--(axis cs:75,5729245.3165944)
--(axis cs:74,8263451.89912157)
--(axis cs:73,5516158.77842074)
--(axis cs:72,7603119.96047562)
--(axis cs:71,4462300.92007251)
--(axis cs:70,4086342.9563068)
--(axis cs:69,4251436.98869144)
--(axis cs:68,4239491.38409234)
--(axis cs:67,4503542.3388013)
--(axis cs:66,4338614.14249858)
--(axis cs:65,4714606.74619845)
--(axis cs:64,5358833.73768554)
--(axis cs:63,5627121.03027346)
--(axis cs:62,6838349.51114454)
--(axis cs:61,13254043.2033068)
--(axis cs:60,19111471.0362781)
--(axis cs:59,14279685.6789195)
--(axis cs:58,10268089.8490413)
--(axis cs:57,8097279.19854255)
--(axis cs:56,4998346.9356354)
--(axis cs:55,5275477.89839847)
--(axis cs:54,5393959.72502817)
--(axis cs:53,5457645.64223258)
--(axis cs:52,9206202.37065737)
--(axis cs:51,9731647.30718806)
--(axis cs:50,7086408.94073138)
--(axis cs:49,9981135.63327471)
--(axis cs:48,5411792.86035991)
--(axis cs:47,5147326.35877483)
--(axis cs:46,6261454.67355194)
--(axis cs:45,6985651.31913097)
--(axis cs:44,7160892.75926486)
--(axis cs:43,6363552.28692704)
--(axis cs:42,6541472.34047879)
--(axis cs:41,5918323.0243611)
--(axis cs:40,6747156.21716429)
--(axis cs:39,6177516.19461915)
--(axis cs:38,7267928.26480801)
--(axis cs:37,5963307.52297306)
--(axis cs:36,6539234.16054271)
--(axis cs:35,5762756.94620261)
--(axis cs:34,5481795.56676344)
--(axis cs:33,6685318.96357387)
--(axis cs:32,5617334.43244526)
--(axis cs:31,8421154.50384803)
--(axis cs:30,6396627.25635895)
--(axis cs:29,9362920.18032182)
--(axis cs:28,7393880.28051603)
--(axis cs:27,9801266.61087961)
--(axis cs:26,10542239.2729267)
--(axis cs:25,9100827.61774091)
--(axis cs:24,10613262.965633)
--(axis cs:23,11461534.9194487)
--(axis cs:22,11383551.213304)
--(axis cs:21,18294344.5244007)
--(axis cs:20,20287612.7119578)
--(axis cs:19,19027300.8296004)
--(axis cs:18,20189562.5128942)
--(axis cs:17,28261813.569889)
--(axis cs:16,22987212.2921109)
--(axis cs:15,17012807.2491663)
--(axis cs:14,15427795.4761646)
--(axis cs:13,18200869.7162744)
--(axis cs:12,19240157.355112)
--(axis cs:11,17737071.6739917)
--(axis cs:10,20061172.1473353)
--(axis cs:9,26305564.9902142)
--(axis cs:8,29539023.8489169)
--(axis cs:7,43610232.6175159)
--(axis cs:6,44239662.2620842)
--(axis cs:5,40578806.3473961)
--(axis cs:4,44812877.4860693)
--(axis cs:3,35931535.608027)
--(axis cs:2,37271950.6111963)
--(axis cs:1,33355288.1255068)
--(axis cs:0,30907406.4514597)
--cycle;

\addplot [only marks, red, mark=asterisk, mark size=1.2]
table {%
0 17920100
1 20248400
2 18087300
3 18316900
4 18044700
5 22639100
6 19789300
7 14686600
8 11502400
9 9352700
10 8582100
11 11649900
12 9782300
13 7409700
14 7937700
15 11951600
16 13791300
17 10920300
18 10713700
19 12009000
20 11282600
21 7367100
22 6603000
23 6539000
24 5942200
25 5765100
26 5302700
27 4925600
28 4325500
29 3851600
30 3472800
31 3169500
32 3046500
33 3474500
34 3267900
35 3308900
36 3322000
37 3338400
38 2776000
39 2633300
40 2776000
41 2375900
42 2215200
43 2999000
44 2917000
45 2402100
46 2274200
47 2433300
48 4332000
49 2994100
50 2869400
51 5627400
52 4043500
53 3120300
54 2528400
55 2233200
56 2666100
57 3199000
58 4791100
59 4153300
60 4274600
61 3363000
62 2495600
63 1993900
64 1765900
65 1702000
66 1592100
67 1464200
68 1372400
69 1347800
70 1311700
71 1508500
72 1659400
73 1702000
74 2341500
75 1420000
76 1308500
77 1223200
78 1192000
79 1136300
80 1149400
81 1190400
82 1129700
83 1101900
84 1167500
85 1208400
86 1551100
87 1272400
88 1116600
89 1388800
90 1728200
91 1726600
92 1623300
93 1160900
94 1170700
95 1015000
96 1167500
97 1231400
98 1115000
99 1129700
100 1228100
101 1549500
102 1251100
103 1665900
104 1821700
105 2193900
106 2202100
107 2903900
108 2428400
109 4323800
110 4814100
111 2985900
112 2297200
113 2305400
114 6327500
115 4494400
116 2805500
117 2077500
118 2008600
119 1970900
120 2710400
121 2228300
122 6494800
123 14245500
124 5363400
125 9349500
126 6568600
127 6753800
128 5697900
129 4345200
130 3863100
131 3307200
132 3369500
133 5522400
134 15231000
135 5953700
136 6871900
137 8970700
138 8341100
139 8483700
140 6193100
141 4950200
142 4327100
143 6491500
144 8265600
145 8141000
146 8800200
147 19151500
148 16346000
149 10290600
150 40800200
151 17308500
152 24067200
153 15927900
154 11067800
155 25603600
156 15375300
157 9887300
158 8095100
159 7209700
160 6063500
161 5289600
162 4866600
163 4815700
164 5242100
165 8669000
166 9542900
167 14517700
168 52177900
169 24288600
170 29417500
171 28182800
172 25541300
173 30627600
174 25752800
175 26618600
176 21983200
177 25367500
178 47919600
179 28571400
180 22909600
181 20286100
182 16623100
183 13602800
184 15458900
185 24282000
186 27756500
187 15190000
188 11868000
189 10492300
190 9096900
191 9602000
192 23829500
193 26753000
194 21184700
195 21588000
196 19899200
197 22309500
198 24147600
199 23190000
200 49352700
201 27041600
202 30930900
203 27174400
204 18992400
205 29878200
206 22434100
207 31185100
208 24457500
209 37904500
210 25323200
211 37802800
212 39396600
213 44325500
214 33485500
215 24906700
216 31806500
217 25085500
218 25892200
219 51576100
220 38050400
221 29583100
222 23290000
223 21069900
224 18662800
225 17557700
226 13837300
227 13066600
228 22163600
229 29327300
230 25428200
231 19318700
232 14771900
233 15132600
234 16177100
235 19100600
236 19381000
237 27615500
238 35612200
239 26056200
240 27774500
241 26820200
242 31006300
243 21214200
244 18223400
245 20350100
246 15621200
247 20760000
248 22301300
249 18144700
250 39506400
251 46898100
252 64052500
253 37488000
254 39360500
255 33231400
256 27777800
257 25434700
258 24719800
259 41657700
260 29101000
261 22374000
262 43966500
263 44027000
264 36616400
265 30216200
266 28524100
267 20749000
268 16987300
269 23598500
270 36276400
271 27510500
272 35321700
273 29521400
274 22944500
275 22181100
276 26176500
277 35362500
278 25040400
279 25037100
280 22156600
281 19184500
282 27206500
283 17706600
284 15772600
285 11930800
286 11368500
287 10760300
288 32362700
289 14795000
290 18355600
291 17165500
292 13585300
293 10915600
294 10258400
295 9548900
296 8646500
297 8092300
298 7856900
299 7505400
300 7456400
301 9874200
302 37072500
303 43216100
304 14688700
305 10892700
306 10770100
307 12025600
308 11775500
309 9648600
310 8914600
311 9890600
312 8917900
313 12215300
314 35035600
315 21501000
316 23971200
317 14410800
318 10387600
319 9969100
320 14520400
321 12476900
322 10735800
323 8744600
324 7547900
325 7343600
326 8913000
327 17425400
328 13748700
329 10690000
330 9058500
331 7732600
332 10789700
333 9107500
334 7734300
335 7060700
336 7121200
337 12414700
338 16140500
339 8811600
340 7513600
341 7546300
342 11554800
343 9539100
344 8471600
345 8579500
346 7785000
347 18571400
348 21973500
349 23778300
350 23215900
351 34620300
352 21178900
353 13477400
354 15167700
355 29351300
356 44963800
357 20335400
358 14605400
359 12252900
360 10199600
361 15777500
362 9835000
363 9583200
364 8497700
365 7502100
366 7086900
367 7266700
368 6712500
369 6231900
370 5965400
371 5161100
372 9223600
373 6310400
374 5527300
375 8223100
376 10583700
377 12437600
378 10085100
379 7753900
380 6228600
381 5417800
382 5015600
383 3457600
384 3439600
385 3336600
386 3197700
387 3408600
388 3338300
389 3375900
390 3305600
391 3357900
392 3335000
393 2476700
394 3181300
395 3163400
396 2633700
397 2883800
398 2865800
399 2744800
400 2661500
401 2594400
402 2538900
403 2507800
404 2679500
405 2543800
406 2460400
407 2529000
408 2468600
409 2350900
410 2243000
411 1978100
412 1816300
413 1790100
414 1427200
415 1401000
416 1334000
417 1271900
418 1185200
419 1185200
420 1101900
421 1062600
422 1029900
423 1052800
424 1054500
425 1070800
426 1041400
427 1798300
428 1412500
429 1257200
430 1255500
431 1193400
432 1093700
433 1039700
434 951500
435 904000
436 874600
437 954700
438 994000
439 966200
440 948200
};
\addplot [blue]
table {%
0 15817166.9029387
1 17145190.5225306
2 19703475.5105783
3 18366736.5824923
4 22129341.9521366
5 21306174.908615
6 23187246.6687183
7 23371020.6348475
8 15896690.0053414
9 13741863.5486684
10 10424577.0164218
11 9080490.87441059
12 10013378.7443896
13 9182104.06261942
14 7846247.88242018
15 8595205.26428845
16 11511290.9300549
17 14899460.0106026
18 10670458.049483
19 9979722.79245217
20 10794865.9503605
21 9709820.95568015
22 5640538.97860684
23 5616153.9244689
24 5197156.93713589
25 4548266.10054598
26 5314543.00332894
27 4721815.68558829
28 3573737.71570562
29 4519775.75500272
30 3100719.75848485
31 3870869.16124664
32 2694471.66639465
33 3200701.2371607
34 2616177.07934864
35 2755737.97811234
36 3201327.49819082
37 2809819.71292162
38 3490717.52570412
39 2904829.74159846
40 3400483.45573234
41 2730737.02965754
42 3170535.39629684
43 3220589.57931373
44 3582313.62486965
45 3311635.31864139
46 3037045.60066886
47 2352686.80081
48 2499687.28697661
49 4538686.53775737
50 3224267.9289801
51 4285912.76483627
52 4238386.82476412
53 2391420.73346951
54 2417153.56286799
55 2391550.3385881
56 2223965.5403143
57 3943611.29506132
58 4484577.14456321
59 6602806.22063108
60 7862122.88719539
61 5732097.37389192
62 3052571.74047431
63 2451709.22492641
64 2364887.0271427
65 2007080.50751847
66 1908264.74471907
67 1941827.7473877
68 1813393.26031246
69 1806817.21795047
70 1689273.10304811
71 1898369.86212898
72 3109582.06755396
73 2258302.59010804
74 2874504.26571522
75 2317164.25553454
76 1761716.56760604
77 1565062.8292228
78 1614692.05972512
79 1612752.99904145
80 1475797.31956482
81 1694741.26618903
82 1847645.141592
83 1942621.24493914
84 1575096.52578418
85 2114560.76283538
86 1951151.33717167
87 1618964.81480327
88 1542587.96726445
89 2158072.71553427
90 2355808.13646589
91 2375383.76644832
92 2064138.4550911
93 1705341.59540626
94 1875553.09285395
95 1996577.57431946
96 1616251.69613577
97 1329501.32287569
98 1764291.54413542
99 1498404.07424741
100 1706750.81492763
101 1828069.79457661
102 2335417.30557071
103 2312260.3853618
104 1910045.63439667
105 4689766.24085449
106 5681508.23436574
107 4221029.90303453
108 4375039.47411462
109 2835057.11608126
110 5026304.10091156
111 4944164.04608513
112 2688040.45641937
113 2640485.54634326
114 2685431.78192499
115 6268336.11662248
116 10810267.7751976
117 5131443.81071823
118 5591279.25522077
119 3229836.56317685
120 3948910.39269931
121 5281301.48351834
122 4521541.23626272
123 7150497.90137576
124 13304132.8565417
125 7831691.15026142
126 11305386.2794738
127 7760746.66352563
128 7786368.01499926
129 8859583.31097712
130 6158088.54079236
131 4914631.07092061
132 5077495.31563268
133 4392648.03245461
134 8443942.0368202
135 14395306.8687438
136 7644108.8624477
137 8769869.23282104
138 7970041.60712091
139 9460650.0986257
140 11364259.721617
141 9141525.56233264
142 8729581.71874131
143 9502865.7819968
144 11889987.4793034
145 10914131.8706718
146 11087038.9307394
147 12520983.3460805
148 14958314.0298709
149 12990465.8431474
150 9781807.80972213
151 23117339.2727531
152 16691438.6513235
153 17690181.3809365
154 14947696.774802
155 13052704.4048142
156 18481656.5983629
157 15670104.684711
158 10114210.8777156
159 8118929.95797702
160 7869362.10966852
161 7575244.81212652
162 7178454.91680303
163 8019116.8496968
164 6772678.8687646
165 7111980.83840174
166 10513614.3091575
167 10751949.1640092
168 14902944.9211813
169 24369524.2102699
170 19842279.4558989
171 22469725.3368968
172 22684036.6355168
173 23185777.8241614
174 31098812.2636249
175 24130611.45815
176 24119749.950085
177 21744844.1856101
178 23781403.5908344
179 35846457.8776203
180 29300834.7852873
181 24429275.8890735
182 22947298.1274158
183 18822381.8122583
184 12850334.2635998
185 14190204.1007543
186 15717137.5875855
187 20493636.7845707
188 15969208.4300027
189 12965556.2345471
190 12059048.9434909
191 11115882.943589
192 11670584.380552
193 15945009.0818837
194 18883807.6422594
195 16031881.0225333
196 19278213.1580589
197 16035549.9606281
198 18984525.9336109
199 23109339.4407823
200 19203536.1529147
201 21460367.0913395
202 20813820.024191
203 22579939.7274051
204 22720952.2997906
205 17112860.0100969
206 19764576.7848033
207 17805341.6972037
208 22263459.3083684
209 19749110.7406204
210 29986885.1709166
211 18589813.6964119
212 26858300.2111651
213 28712323.1470952
214 31076024.7571954
215 26104543.7860655
216 25260201.6447637
217 32972642.9163986
218 21549993.9082119
219 22845599.0259692
220 61310619.5845119
221 46429670.0344703
222 30353582.8232426
223 22253140.9758421
224 21008577.2396638
225 19162323.4044949
226 15410261.0997889
227 13493714.5859948
228 16640599.2950091
229 21812942.3092414
230 24344877.2890107
231 24395789.1426938
232 22230274.5852992
233 14505771.4005165
234 15578556.29157
235 15699371.558797
236 15608658.4668774
237 16972435.9899214
238 26397340.9133122
239 40463018.6028649
240 27693970.2287068
241 20374161.1848208
242 25642702.3162933
243 30207771.8296936
244 27912000.6090829
245 24282004.4819338
246 24218980.0735596
247 21813125.0750976
248 24251126.6147131
249 28757133.215754
250 26763331.5055557
251 47589853.6306894
252 85876621.7765533
253 59640960.0422263
254 28236662.3118587
255 30941156.7329272
256 31712491.2340658
257 26116171.5182185
258 32278329.3513308
259 25948446.4489517
260 34185380.1474541
261 31476736.7663939
262 25867181.1734402
263 38561932.7340806
264 42004420.431799
265 32089234.4417422
266 25070680.1878136
267 26090450.0579419
268 19340712.3791421
269 15382610.9820694
270 17908802.8710566
271 35853662.7616761
272 31031921.0115907
273 27239953.6316347
274 24039254.1785751
275 19728196.3418067
276 22516160.4085836
277 21549256.3199576
278 28517092.6667087
279 19985147.0645098
280 18780448.392693
281 20361337.2777619
282 20593379.2603232
283 24294555.8596071
284 17305416.1559645
285 15022872.467961
286 13953284.3598747
287 14018742.5673007
288 13742789.7099644
289 18142349.1759614
290 16558625.4635079
291 17126433.2227616
292 15670226.8263745
293 14489078.3597249
294 13195014.9273231
295 10221924.3231966
296 11005332.6689434
297 10025015.5445261
298 10076816.4759409
299 10525537.0395549
300 9511758.69481277
301 9636988.69669458
302 13843218.6769109
303 25492439.4068606
304 24575228.9280661
305 18335494.3954573
306 13382913.3809106
307 12115748.53758
308 13327074.7777563
309 13180924.9403318
310 10702158.1141293
311 11515697.5966926
312 11924140.0358933
313 10553377.0174844
314 15338368.1638863
315 17237039.4147474
316 19186021.9218797
317 23136697.873801
318 15082110.3103782
319 11316361.910599
320 12206432.9626541
321 15055667.1144521
322 11739311.5697793
323 11516879.8466384
324 9864045.41846232
325 9002723.62918769
326 9132162.68355163
327 12131969.7256949
328 16812339.1203243
329 19035766.7389835
330 12285851.4132363
331 10366553.3670918
332 9591379.30356058
333 11032456.4520627
334 10879575.1739264
335 9786896.16415578
336 9165904.66111255
337 8781190.46220768
338 16280379.0704675
339 17344299.4952445
340 10375784.1723856
341 10465232.8667995
342 11346649.7962284
343 15613148.4189859
344 9611537.47116247
345 7618522.80730687
346 8193692.00952102
347 7989624.65768473
348 16520311.6308929
349 19097798.7951571
350 19298130.6792488
351 20167112.8390764
352 22680905.8693014
353 17388777.8578242
354 16336914.7986623
355 15003465.6896234
356 22230434.5939455
357 25575774.0824056
358 17784424.7100276
359 14138349.4487893
360 13572671.572507
361 12767175.4381347
362 12930872.824674
363 9843073.09357965
364 8738997.80090356
365 7901298.03283077
366 6923214.0744638
367 6856832.46367288
368 6282491.50254044
369 4697144.4203149
370 4436390.54578279
371 3920976.32494381
372 5086547.26526382
373 5592935.36769166
374 5143068.56560185
375 6508300.35611906
376 12388450.0824522
377 10676372.5851138
378 16128647.3037612
379 11080481.6862442
380 9169247.30820924
381 5856733.92241032
382 6243287.77184244
383 4806598.0195022
384 4998467.84292711
385 4044365.85973294
386 3927530.78502294
387 3766024.795146
388 3474533.68850079
389 3403666.64329599
390 3397706.48159642
391 2885670.90664532
392 2999857.754135
393 2630684.39242121
394 3335288.79249748
395 2965618.12093371
396 3515697.99065885
397 2374499.1779725
398 2369302.89321954
399 2802952.79158214
400 2434885.63361194
401 2857970.54188138
402 2518604.35254323
403 2680708.61104026
404 3011243.95615938
405 3065257.53455731
406 2722669.28019223
407 2454667.83664112
408 2240404.92492015
409 2970434.37890756
410 2905033.21132058
411 2892472.64514179
412 3088245.72557693
413 2816044.52759822
414 2517857.43940275
415 2633785.67434188
416 2616730.79755236
417 2212522.49731027
418 1948999.30887365
419 2023939.67282425
420 1961149.00377695
421 2226555.29859337
422 2093701.50693544
423 1970964.37156584
424 1936975.79495566
425 1829219.61860749
426 1957949.99227134
427 1847840.65904331
428 2592558.61059873
429 2043245.09087693
430 1702394.71530586
431 2044106.11330845
432 1824513.8661629
433 1676446.45022735
434 1721964.45290067
435 1746150.46709673
436 2113311.35151209
437 2025922.64921875
438 1614084.80221801
439 2573838.69135064
440 1812585.62323777
};
\addplot [black]
table {%
0 14892701.5549125
1 16105592.4363421
2 18266448.1806344
3 17504638.8336834
4 21400453.7368507
5 20197528.8711394
6 21933579.5864444
7 22306413.5272016
8 15161858.0266189
9 12732650.5528128
10 9852728.79590429
11 8711352.15680785
12 9487736.38601163
13 8715691.20762302
14 7470644.70052106
15 8137756.20158925
16 10749444.8108311
17 14249938.6522305
18 10245973.9018169
19 9495904.18325791
20 10245028.155478
21 9107149.49607602
22 5298490.9264577
23 5271424.11227119
24 4933130.71642757
25 4223993.71173469
26 4979192.33891741
27 4391027.2671512
28 3349503.90639006
29 4241459.01136462
30 2836111.09340364
31 3536230.93088559
32 2533624.60962837
33 2946222.76300503
34 2431064.41473476
35 2560682.52508425
36 2950159.44312263
37 2618580.52901552
38 3250866.46071164
39 2684336.16544333
40 3225205.66368502
41 2537848.66499306
42 2961451.1565054
43 3042510.69768472
44 3307986.45856947
45 3153004.587146
46 2795617.85515411
47 2136250.81385285
48 2287282.98990377
49 4167849.03199317
50 2971301.03140765
51 3921446.38514843
52 3782382.33049124
53 2170190.19030649
54 2206193.65302136
55 2158720.99863844
56 2040275.0560086
57 3644375.00720007
58 4011493.42088385
59 6096953.50194743
60 7064601.82303547
61 5144218.68767642
62 2767207.76146467
63 2234457.356348
64 2170290.75241437
65 1820249.21669804
66 1742083.73024891
67 1731752.61328364
68 1630910.2490352
69 1618756.81924232
70 1472845.02410038
71 1704947.383727
72 2803437.56455603
73 1973958.28363113
74 2330894.741921
75 2009199.68417525
76 1549890.57707243
77 1384106.19810937
78 1436964.51029405
79 1391649.56410332
80 1284995.43578754
81 1485210.42808339
82 1643147.74731137
83 1752580.84604355
84 1374996.04872073
85 1884917.12516795
86 1746080.49596359
87 1432146.91938598
88 1369912.81550229
89 1988106.6690747
90 2136416.3484932
91 2148372.40429351
92 1814696.15511319
93 1527141.27705308
94 1666052.64323002
95 1805798.5592818
96 1416536.73995068
97 1157550.12248053
98 1590462.72223883
99 1307117.51546335
100 1466083.72646004
101 1633900.31644726
102 1943594.41359681
103 2005827.28669708
104 1651817.19071959
105 3958118.46912934
106 5136255.01569268
107 3815979.02812778
108 3842673.83270983
109 2491667.70794794
110 4460530.77571751
111 4434575.6246253
112 2371831.16739132
113 2411385.52189616
114 2376848.10909736
115 5418366.67464636
116 9649965.56107295
117 4635634.65068756
118 5136031.38952198
119 2880608.00140485
120 3494165.93817339
121 4714874.9787833
122 4079398.24475713
123 6401915.05277008
124 12339053.1417716
125 7305034.99377738
126 10319842.8394841
127 7126237.5171012
128 6996577.57826913
129 8208296.65977568
130 5584902.73210995
131 4445154.63284783
132 4546088.84091157
133 4033928.2713101
134 7765392.61659981
135 13290473.5468912
136 6842693.72237758
137 7889128.20066047
138 7186240.2434083
139 8409819.68348522
140 10355423.2343793
141 8252386.77082822
142 7853715.07013867
143 8528314.14932212
144 10926916.6812138
145 10058718.0275423
146 10214879.948713
147 11743707.531887
148 14196491.8794839
149 12197634.625411
150 9256023.64847716
151 21610346.9993928
152 15712395.6831525
153 16869908.8899271
154 14019281.8485531
155 12272950.4462683
156 17459629.0984731
157 14976765.554702
158 9447583.31738906
159 7595020.11168568
160 7244850.10933867
161 7025626.01532164
162 6618734.04034122
163 7507924.0417813
164 6302388.79630156
165 6584564.95651218
166 9760720.46563324
167 10094599.7857546
168 14098168.2901059
169 23104678.9887904
170 18859079.6122961
171 21350501.1040146
172 21678961.4123353
173 21935071.1835384
174 29406406.6882252
175 22634113.2619574
176 22704898.9741285
177 20578518.4684799
178 22932293.3980214
179 33879443.335289
180 27502650.5605419
181 23012559.3408345
182 21724465.3927037
183 17969943.2769561
184 12049259.1800632
185 13022981.5158371
186 13927726.0554185
187 19180715.1948025
188 14954180.9622705
189 12371123.7221246
190 11395289.7094494
191 10371062.5531947
192 10837926.4992647
193 15266289.8351919
194 17588826.508124
195 15329037.2274309
196 17852272.0517497
197 15079100.3533849
198 18008072.1243269
199 21807011.3605691
200 17607605.0564492
201 19863833.4118281
202 20002965.0324689
203 21612643.2320905
204 21561211.4557563
205 16311532.5058479
206 18342450.8485463
207 16865824.2053881
208 20772594.1868633
209 18675499.5556948
210 28316255.8861119
211 17508132.1843205
212 25458873.3414471
213 26926893.8285132
214 28911927.3121327
215 24511000.8431349
216 23986525.0752304
217 31627378.7846685
218 20748996.0987056
219 21718881.1586193
220 35604175.4100394
221 42655110.1562589
222 29082075.7567452
223 21431750.3932708
224 19934197.0331458
225 18127844.9138457
226 14443264.3959176
227 12839419.4532014
228 15733550.9008249
229 20704138.7381817
230 23198780.8625109
231 23212532.3954736
232 20873170.2830999
233 13749350.3957616
234 14695946.0307301
235 14915535.7620647
236 14965353.8325148
237 16364069.5737344
238 25000041.9612734
239 37700772.578519
240 26254951.354144
241 19224465.6801539
242 24234438.1538962
243 28295755.2027793
244 26481032.0550915
245 22904420.80896
246 22948573.3227791
247 20414469.7216052
248 23136966.1552481
249 27190934.9314836
250 25532399.3302991
251 43947906.7470062
252 66767222.1307156
253 54635435.4502686
254 27210122.648439
255 29884669.8700173
256 30209828.1952439
257 24835300.7352778
258 31010501.7941556
259 24605246.2279667
260 32333191.0109222
261 30120780.2907996
262 24644278.6532865
263 37094318.5037047
264 40203136.8319307
265 30759755.9478505
266 23880010.4896208
267 25372818.7254138
268 18407516.1977322
269 14817284.3115973
270 17071487.726248
271 32939627.8004205
272 29913551.6845149
273 26045078.9028003
274 22232660.9620587
275 18580850.4179655
276 21268204.3278441
277 20412785.5964816
278 27248634.8304861
279 18658430.9368129
280 17909454.8465971
281 19160321.3607153
282 19373276.1740477
283 22978056.7436543
284 16545932.105681
285 14411187.8646251
286 13158783.807441
287 13165822.5165997
288 12895682.9502299
289 17424622.3575661
290 15453265.7706959
291 16395604.8810162
292 14597068.1111017
293 13397197.7744849
294 12348463.86287
295 9542576.57762315
296 10449441.45127
297 9375039.72707347
298 9370128.58162165
299 9950854.84494464
300 8829046.41081527
301 9044288.06905944
302 12976185.0680466
303 24320324.0034909
304 23490274.5603481
305 17517517.5080175
306 12454951.5772011
307 11239046.9734587
308 12490104.8978153
309 12581273.8406782
310 10168647.6732363
311 10900569.8435776
312 11229646.9542005
313 9901158.88159031
314 14499439.7469843
315 16080900.5855952
316 18133804.6799823
317 21978182.7211381
318 14256483.1367079
319 10726133.8940739
320 11462781.3887298
321 14261101.2702967
322 11013643.9805025
323 10911327.6595703
324 9148996.2914564
325 8440220.70886438
326 8608913.16840721
327 11366133.1590038
328 15792385.6039002
329 17958581.6254609
330 11669686.9490569
331 9657166.77658437
332 9030583.85481668
333 10482960.2814032
334 10200640.5399405
335 9304149.31024398
336 8608966.68211998
337 8342295.42721414
338 15461819.6287262
339 16183805.946172
340 9959221.17877958
341 9862344.63798509
342 10672169.3682607
343 14493235.8577101
344 8989597.00709746
345 7130656.87893723
346 7701670.69103186
347 7606712.53644355
348 15492279.0737609
349 17906967.1340456
350 18283248.0579907
351 19104772.1744042
352 21932117.4916859
353 16485208.0063793
354 15401832.2990553
355 14296767.7418352
356 21279424.0463911
357 24410793.7851165
358 16787302.1117872
359 13401312.8271559
360 12848201.5166925
361 11986417.7005987
362 12361185.5523782
363 9270781.12218172
364 8186290.73779104
365 7456075.23648666
366 6608857.72557701
367 6353937.30758582
368 5819086.55135636
369 4409864.5992719
370 4177808.24906268
371 3671825.48218606
372 4748482.89888408
373 5107663.65536728
374 4748172.72948324
375 6040003.09607141
376 11646338.1951229
377 10087409.7725359
378 15125433.2033337
379 10350397.2867323
380 8533435.51486708
381 5510883.08612065
382 5782323.52905837
383 4536404.51064012
384 4706043.77911151
385 3745112.25545179
386 3710607.97515886
387 3518527.37086626
388 3249731.06599588
389 3202323.26701014
390 3158972.04397173
391 2650725.46466819
392 2801758.85354954
393 2399380.81359136
394 3072032.23641454
395 2739373.60959914
396 3212076.94336684
397 2147750.66743433
398 2153353.59065654
399 2595524.19832103
400 2260078.75834455
401 2649259.65865974
402 2315614.86396452
403 2412562.38681832
404 2814204.30974039
405 2874105.08628668
406 2490250.31890561
407 2258460.4126195
408 2067826.8410591
409 2780769.0874866
410 2678979.87938628
411 2657700.88819395
412 2859977.59763358
413 2599068.83675615
414 2348091.23576721
415 2444429.14962575
416 2446622.21113604
417 2074785.25112385
418 1762580.55606247
419 1862162.87622667
420 1784952.86244942
421 2058947.71190814
422 1892510.37292223
423 1828710.66211916
424 1785631.23256978
425 1631599.59165952
426 1775645.56377612
427 1641542.84688509
428 2399552.2487258
429 1832891.61257615
430 1517379.41010488
431 1858803.33354891
432 1672913.7647959
433 1518716.21056749
434 1534931.1238448
435 1540388.50403076
436 1919667.44575147
437 1856968.49570591
438 1454597.48933966
439 2396299.68286054
440 1606874.44519124
};
\addlegendentry{Real Observations}
\addlegendentry{Prediction mean}
\addlegendentry{Prediction median}
\addlegendimage{line width=5pt,draw=blue,opacity=0.5}
\addlegendentry{$95\%$ CI}
\end{axis}

\end{tikzpicture}
}
%	\end{figure}
%	\vspace{-1.5em}
%	\begin{block}{}
%		Test data for four reservoirs in one day ahead of ChdGP Gamma model prediction. This setting does not allocate predictive distribution for negative values. Mean and Median become different in peak presence, suggesting asymmetric distribution.
%	\end{block}
%\end{frame}

%\begin{frame}{To Conclude}
%		
%		\begin{block}{}
%			\justifying
%			The ChdGP model generalizes all previously developed GP-based models, enhancing expressiveness by modeling likelihood parameters and enabling the handling of natural output restrictions.	
%		\end{block}
%		
%		\begin{block}{}
%			\justifying
%			The ChdGP Normal model outperformed the LMCGP model across all forecasting horizons, primarily due to its ability to adaptively vary data noise over the input space, providing a more refined capture of the underlying data structure.
%		\end{block}
%		
%		\begin{block}{}
%			\justifying
%			The ChdGP with Gamma likelihood ensured non-negative predictions. The tuning process revealed a significant improvement in model stability as the number of independent GPs ($Q$) increased, suggesting superior data modeling capabilities.
%		\end{block}
%		
%		\begin{block}{}
%			\justifying
%			The Gamma likelihood configuration outperformed the Gaussian likelihood across all evaluated horizons by avoiding the allocation of predictive distribution mass to negative values and utilizing an asymmetric distribution to more effectively handle peak outliers.
%		\end{block}
		
%\end{frame}

