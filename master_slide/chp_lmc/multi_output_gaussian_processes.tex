\section{Multi-Output Gaussian Processes: Modeling Inter-Output Dependencies}

\subsection{Mathematical Framework}
\begin{frame}{Multi Output Gaussian Processes}
	We start to extending the notation developed so far to learning multiple outputs with a single model, we consider
	\begin{equation*}
		\boldsymbol{f}(\mathbf{x}) = \left[f_1(\mathbf{x}), f_2(\mathbf{x}), \cdots, f_d(\mathbf{x}), \cdots, f_D(\mathbf{x})\right]^\top
	\end{equation*}
	as a multi-output Gaussian process, comprising $D$ single-output Gaussian processes. Now our target is a vector-valued function.
\end{frame}

\begin{frame}{Independent Gaussian Process (IGP)}
	Combining all independent GP-based models evaluated in the $N_*$ test point set $X_*$, assuming that all outputs are fully observed for each input, the vector function space inference corresponds to the following generative model:
	
	
	\begin{equation*}\label{eq:igp}
		\left[ \begin{array}{c}
			\mathbf{f}_{1*}\\
			\mathbf{f}_{2*}\\
			\vdots\\
			\mathbf{f}_{D*}\\
		\end{array}
		\right]
		\sim
		\mathcal{N} \left(
		\begin{array}{c}
			\mathbf{0}\\
		\end{array},
		\left[ \begin{array}{cccccc}
			K_{1**} & 0 & \cdots & 0 \\
			0 & K_{2**} & \cdots & 0 \\
			\vdots & \vdots & \ddots & \vdots\\
			0 & 0 & \cdots & K_{D**}\\
		\end{array}
		\right] \right)
	\end{equation*}
	
	Here, $K_{d**} \in \mathbb{R}^{N_* \times N_*}$ correspond to the covariance matrix of the $d$-th output. Due to the independence assumption, the grand covariance matrix of the multi-output model $\mathbf{K}_{**} \in \mathbb{R}^{DN_* \times DN_*}$ is diagonal by block.\\~\\
	
	We call this model Independent Gaussian process (IGP).
\end{frame}

\begin{frame}{Modeling Output Interactions}
	We introduce a framework based on a set of independent Gaussian processes $\{u_q(\mathbf{x})\}_{q=1}^Q$, with their own covariance function $k_q(\mathbf{x}, \mathbf{x}')$ and each latent process $f_d(\mathbf{x})$ is represented as an instantaneous, time-invariant linear combination of the independent processes:
	
	\begin{equation*}
		\begin{aligned}
			f_{d}(\mathbf{x}) &= \sum_{q=1}^Q a_{d,q} u_{q}(\mathbf{x}) \quad &
			u_{q}(\mathbf{x}) &\sim \mathcal{GP}(0, k_{q}(\mathbf{x}, \mathbf{x}')) 
		\end{aligned}\label{eq:LMC_cov}
	\end{equation*}
	

\end{frame}


\begin{frame}{Graphical Representation}
	\begin{figure}
	\setlength\figurewidth{0.5\textwidth} 
	\setlength\figureheight{0.08\textwidth}
	% NEURAL NETWORK activation
% https://www.youtube.com/watch?v=aircAruvnKk&list=PLZHQObOWTQDNU6R1_67000Dx_ZCJB-3pi&index=1
\begin{tikzpicture}[x=\figurewidth,y=\figureheight]
  \message{^^JNeural network activation}
  \def\NI{4} % number of nodes in input layers
  \def\NO{4} % number of nodes in output layers
  \def\yshift{0.4} % shift last node for dots
  
  % INPUT LAYER
  \foreach \i [evaluate={\c=int(\i==\NI); \y=\NI/2-\i-\c*\yshift; \index=(\i<\NI?int(\i):"Q");}]
              in {1,...,\NI}{ % loop over nodes
    \node[node in,outer sep=0.6] (NI-\i) at (0,\y) {$u_{\index}$};
  }
  
  % OUTPUT LAYER
  \foreach \i [evaluate={\c=int(\i==\NO); \y=\NO/2-\i-\c*\yshift; \index=(\i<\NO?int(\i):"D");}]
    in {\NO,...,1}{ % loop over nodes
    \ifnum\i=1 % high-lighted node
      \node[node hidden]
        (NO-\i) at (1,\y) {$f_{\index}$};
      \foreach \j [evaluate={\index=(\j<\NI?int(\j):"Q");}] in {1,...,\NI}{ % loop over nodes in previous layer
        \draw[connect,white,line width=1.2] (NI-\j) -- (NO-\i);
        \draw[connect] (NI-\j) -- (NO-\i)
          node[pos=0.50] {\contour{white}{$a_{1,\index}$}};
      }
    \else % other light-colored nodes
      \node[node,blue!20!black!80,draw=myblue!20,fill=myblue!5]
        (NO-\i) at (1,\y) {$f_{\index}$};
      \foreach \j in {1,...,\NI}{ % loop over nodes in previous layer
        %\draw[connect,white,line width=1.2] (NI-\j) -- (NO-\i);
        \draw[connect,myblue!20] (NI-\j) -- (NO-\i);
      }
    \fi
  }
  
  % DOTS
  \path (NI-\NI) --++ (0,1+\yshift) node[midway,scale=1.2] {$\vdots$};
  \path (NO-\NO) --++ (0,1+\yshift) node[midway,scale=1.2] {$\vdots$};
  

  
\end{tikzpicture}

	\caption{Graphical representation of the Linear Model of Coregionalization GP} 
	\end{figure}

\end{frame}

\begin{frame}{Cross-covariance function}
	The cross-covariance function of the latent Gaussian process is expressed as:
	
	\begin{equation}
		\begin{split}
			k_{d,d'}(\mathbf{x}, \mathbf{x}') &= \text{cov}\{f_{d}(\mathbf{x}), f_{d'}(\mathbf{x}')\}\\
			&=\sum_{q=1}^Q \sum_{q'=1}^Q a_{d,q}a_{d',q'}\text{cov}\{u_{q}(\mathbf{x}), u_{q'}(\mathbf{x}')\}\\
			&=\sum_{q=1}^Q a_{d,q}a_{d',q} k_{q}(\mathbf{x}, \mathbf{x}')\\
			&=\sum_{q=1}^Q b^q_{d, d'} k_{q}(\mathbf{x}, \mathbf{x}')
		\end{split}
	\end{equation}
	
	Here, $b^q_{d, d'} = a_{d,q}a_{d',q}$ captures the interactions among the outputs induced by the $q$-th independent process, while $k_{q}(\mathbf{x}, \mathbf{x}')$ characterizes the interaction among input spaces viewed from the perspective of the $q$-th independent process. 
\end{frame}

\begin{frame}{The Matrix Kernel Function}
	Instead of a scalar kernel function, we now have a matrix kernel function $\mathbf{k}: \mathcal{X} \times \mathcal{X} \to \mathbb{R}^{D \times D}$ with elements $k_{d,d'}(\mathbf{x}, \mathbf{x}')$:
	
	\begin{equation}\label{eq:lmc_covariance_function}
		\mathbf{k}(\mathbf{x}, \mathbf{x}') = \sum_{q=1}^Q \mathbf{B}_q k_{q}(\mathbf{x}, \mathbf{x}')
	\end{equation}
	
	Here, each $\mathbf{B}_q \in \mathbb{R}^{D \times D}$ is referred to as the coregionalization matrix, with elements $(\mathbf{B}_q)_{d,d'} = b^q_{d, d'}$. Evaluating this at all test points $X_*$ allows us to recover the covariance matrix as follows:
	
	\begin{equation}
		\mathbf{K}_{**} = \sum_{q=1}^Q \mathbf{B}_q \otimes K_{q**}
	\end{equation}
	where $\otimes$ denotes the Kronecker product between matrices.
\end{frame}

\begin{frame}{Linear Model of Coregionalization GP (LMCGP)}
	This model can effectively capture the cross-covariances of the output given by $k_{d,d'}(\mathbf{x}, \mathbf{x}')$, allowing to fill zeros in covariance matrix of IGP as follows:
	
	\begin{equation*}
		\left[ \begin{array}{c}
			\mathbf{f}_{1*}\\
			\mathbf{f}_{2*}\\
			\vdots\\
			\mathbf{f}_{D*}\\
		\end{array}
		\right]
		\sim
		\mathcal{N} \left(
		\begin{array}{c}
			\mathbf{0}\\
		\end{array},
		\sum_{q=1}^Q
		\left[ \begin{array}{cccccc}
			b^q_{1, 1} & b^q_{1, 2} & \cdots & b^q_{1, D} \\
			b^q_{2, 1} & b^q_{2, 2} & \cdots & b^q_{2, D} \\
			\vdots & \vdots & \ddots & \vdots\\
			b^q_{D, 1} & b^q_{D, 2} & \cdots & b^q_{D, D}\\
		\end{array}
		\right] \otimes K_{q**}\right)
	\end{equation*}
	We call this model Linear Model of Coregionalization Gaussian Process (LMCGP).
\end{frame}

\begin{frame}{Variational Inference and ELBO}
	We can extent our variational inference to include the independent set. Furthermore, instead of utilizing the latent processes $f_d$, we utilize the inducing variables derived from the independent processes $u_q$ providing the following ELBO:
	
	\begin{equation}\label{eq:lmc_elbo}
		\mathcal{L} = \sum_{d=1}^D\sum_{n=1}^{N} \mathbb{E}_{q(f_{dn})}\{\log p(y_{dn} \mid f_{dn})\} - \sum_{q=1}^Q \text{KL}\{q(\mathbf{u}_q)\parallel p(\mathbf{u}_q)\}
	\end{equation}
	
\end{frame}

\begin{frame}{Latent posterior and Predictive distribution}
	The posterior over test points $X_*$, denoted as $p(\mathbf{f}_* \mid \mathbf{y})$, is approximated as follows:
	\begin{equation}
		p(\mathbf{f}_* \mid \mathbf{y}) \approx q(\mathbf{f}_*) = \int p(\mathbf{f}_* \mid \mathbf{u}) q(\mathbf{u}) d \mathbf{u}
	\end{equation}
	We add Gaussian noise $\sigma_{Nd}^2$ to the corresponding task into the above random vector to obtain the predictive distribution.
\end{frame}

\begin{frame}{Adam + Natural Gradient Optimization} \textbf{Optimization Parameters for LMCGP:} \begin{itemize} \item \textbf{Kernel:} lengthscales, output scales \item \textbf{Likelihood:} data noise $\{ \sigma^2_{Nd} \}_{d=1}^D$ \item \textbf{Inducing Points:} $Z$ \item \textbf{Variational Parameters:} $\{\mathbf{\mu_q}, S_q\}_{q=1}^Q$ \end{itemize}
	
	\textbf{Challenges:} Strong dependency between variational parameters and others makes the model sensitive. ELBO loss function is non-convex, leading to poor local minima with traditional optimizers.
	
	\textbf{Solution:} Combine Natural Gradient (NG) with Adam. NG optimizes variational parameters, while Adam optimizes other parameters.
	
	\textbf{Benefits:} This hybrid method, Adam + NG, improves optimization performance, allowing better convergence.
\end{frame}

\begin{frame}{Model Setup}
	The proposed methodology constructs the LMCGP covariance function using the widely applied squared exponential kernel:
	\begin{equation}\label{eq:nonscaled_squared_exponential_kernel}
		k_{q}\left(\mathbf{x}, \mathbf{x'} \mid \Theta_q \right) = \exp\left(-\frac{1}{2}(\mathbf{x} - \mathbf{x'})^\top \Theta_q^{-2} (\mathbf{x} - \mathbf{x'})\right)
	\end{equation}
	Upon examining, one might question the absence of an output scale parameter. However, a detailed look at shows that the elements in the matrix $\mathbf{B}_q$ perform the function of rescaling the exponential term in the $k_q$ kernel.\\~\\ 
	The trainable covariance parameters are $\{ \mathbf{B}_q, \Theta_q\}_{q=1}^Q$.
\end{frame}

\begin{frame}{Negative Log Predictive Density}
	To evaluate the models' performance, we use the implemented metrics MSE, MSLL, and CRPS. Additionally, we propose a new metric called Negative Log Predictive Density (NLPD), which applies to any type of predictive distribution and is defined as:
	
	\begin{equation}
		\begin{split}
			\text{NLPD} &= -\frac{1}{N_*}\log p(\mathbf{y}_* \mid \mathbf{y})\\
			&= -\frac{1}{N_*}\sum_{n=1}^{N_*} \log p(\mathbf{y}_{n*} \mid \mathbf{y}) 
		\end{split}
	\end{equation}
\end{frame}

\subsection{Results and Discussions}

\subsubsection{Hyperparameter Tuning}

\begin{frame}{Tuning Q}
	\begin{figure}[htbp]
	\centering
	\setlength\figurewidth{0.5\columnwidth} 
	\setlength\figureheight{0.28\columnwidth}
	
	\subfloat[CRPS]{% This file was created with tikzplotlib v0.10.1.
\begin{tikzpicture}

\definecolor{darkgray176}{RGB}{176,176,176}
\definecolor{steelblue31119180}{RGB}{31,119,180}

\begin{axis}[
width=\figurewidth,
height=\figureheight,
axis background/.style={fill=background_color},
axis line style={white},
tick align=inside,
tick pos=left,
x grid style={white},
y grid style={white},
xmajorgrids,
ymajorgrids,
xmin=-1.25, xmax=48.25,
xtick style={color=black},
ymin=0.239929987456654, ymax=0.438839652236985,
ytick style={color=black},
xlabel style={font=\tiny},
tick label style={font=\tiny}
]
\addplot [semithick, steelblue31119180, mark=*, mark size=1, mark options={solid}]
table {%
1 0.429798303837879
2 0.374510552905612
3 0.345960005084818
4 0.323892519212847
5 0.311383175599372
6 0.299542632982873
7 0.297875424155548
8 0.28784981072712
9 0.286844690564962
10 0.279610258905049
11 0.279515690474725
12 0.273041146856969
13 0.271082282952741
14 0.271889987179417
15 0.270210632285698
16 0.26759372049199
17 0.266678895529016
18 0.264912747643612
19 0.26441644238815
20 0.309002018166923
21 0.257220132139326
22 0.291456973758776
23 0.2613559059979
24 0.287804261488934
25 0.285577216964857
26 0.287627602712463
27 0.300194088705333
28 0.24897133585576
29 0.293655299555358
30 0.287299736020118
31 0.296865072134467
32 0.291250790206968
33 0.307497232174908
34 0.289441525631671
35 0.294363581321622
36 0.291544275421493
37 0.305791393660394
38 0.296662343784081
39 0.308247437640682
40 0.298021172730888
41 0.29625972317634
42 0.307272161655389
43 0.303820351549033
44 0.297729094232099
45 0.299530085504188
46 0.293446316295075
};
\end{axis}

\end{tikzpicture}
}\hspace{0.1em}
	\subfloat[MSE]{% This file was created with tikzplotlib v0.10.1.
\begin{tikzpicture}

\definecolor{darkgray176}{RGB}{176,176,176}
\definecolor{steelblue31119180}{RGB}{31,119,180}

\begin{axis}[
width=\figurewidth,
height=\figureheight,
axis background/.style={fill=background_color},
axis line style={white},
tick align=inside,
tick pos=left,
x grid style={white},
y grid style={white},
xmajorgrids,
ymajorgrids,
xmin=-1.25, xmax=48.25,
xtick style={color=black},
ymin=0.259536486825269, ymax=0.731194606213278,
ytick style={color=black},
xlabel style={font=\tiny},
tick label style={font=\tiny}
]
\addplot [semithick, steelblue31119180, mark=*, mark size=1, mark options={solid}]
table {%
1 0.70975560078655
2 0.570005452889205
3 0.524956091611488
4 0.468590469300411
5 0.435353470792991
6 0.411259722878218
7 0.409096034731835
8 0.385485371917808
9 0.381016960281645
10 0.360132840745332
11 0.362921694624643
12 0.347941662750589
13 0.33894395440443
14 0.342577112243772
15 0.343123787309784
16 0.334095395016974
17 0.330290643463412
18 0.322687081772815
19 0.326757528243904
20 0.437607617331642
21 0.304623270578551
22 0.389408892826712
23 0.316685484093948
24 0.381513763041057
25 0.375840283026447
26 0.377810182111291
27 0.414747999915877
28 0.280975492251997
29 0.393662339819777
30 0.373947243132724
31 0.402964916925179
32 0.385485092508175
33 0.431905374190024
34 0.38280924563172
35 0.397469992032685
36 0.390029348756403
37 0.427310239658667
38 0.408634597214847
39 0.437765674807815
40 0.40614578730405
41 0.40052469480261
42 0.43521136441382
43 0.423471188409076
44 0.405748603781294
45 0.409759054806833
46 0.390775042857023
};
\end{axis}

\end{tikzpicture}
}\\
	\vspace{0.1em} % Adjust spacing between rows
	
	\subfloat[MSLL]{% This file was created with tikzplotlib v0.10.1.
\begin{tikzpicture}

\definecolor{darkgray176}{RGB}{176,176,176}
\definecolor{steelblue31119180}{RGB}{31,119,180}

\begin{axis}[
width=\figurewidth,
height=\figureheight,
axis background/.style={fill=background_color},
axis line style={white},
tick align=inside,
tick pos=left,
x grid style={white},
y grid style={white},
xmajorgrids,
ymajorgrids,
xmin=-1.25, xmax=48.25,
xtick style={color=black},
ymin=-1.06651615899946, ymax=-0.510154630298139,
ytick style={color=black},
xlabel={Number of Independent GPs (Q)},
xlabel style={font=\tiny},
tick label style={font=\tiny}
]
\addplot [semithick, steelblue31119180, mark=*, mark size=1, mark options={solid}]
table {%
1 -0.535443790693654
2 -0.656143802651295
3 -0.738178467100046
4 -0.800136632373368
5 -0.840255253429766
6 -0.875183228780147
7 -0.883212638166813
8 -0.916060696072918
9 -0.92732556592439
10 -0.941608614832543
11 -0.941373520408176
12 -0.963328948986684
13 -0.959595962050077
14 -0.96451770455758
15 -0.95974756275392
16 -0.971143985423901
17 -0.981752140746748
18 -0.952261104902915
19 -0.979157271636092
20 -0.642391297788161
21 -1.00087763478619
22 -0.720088184779209
23 -0.954280692847926
24 -0.728105102288219
25 -0.735847580651938
26 -0.735567459714915
27 -0.672249112260655
28 -1.04122699860395
29 -0.705361482919378
30 -0.729647651122807
31 -0.678594473020219
32 -0.711949327335191
33 -0.64218758924277
34 -0.719052195633076
35 -0.694440862143111
36 -0.703501481048209
37 -0.654029715100638
38 -0.668137210996388
39 -0.621833430969577
40 -0.671693822590583
41 -0.67987459279109
42 -0.638762954793032
43 -0.660630779693286
44 -0.675898190836243
45 -0.674508388989054
46 -0.695663541852463
};
\end{axis}

\end{tikzpicture}
}\hspace{0.1em}
	\subfloat[NLPD]{% This file was created with tikzplotlib v0.10.1.
\begin{tikzpicture}

\definecolor{darkgray176}{RGB}{176,176,176}
\definecolor{steelblue31119180}{RGB}{31,119,180}

\begin{axis}[
width=\figurewidth,
height=\figureheight,
axis background/.style={fill=background_color},
axis line style={white},
tick align=inside,
tick pos=left,
x grid style={white},
y grid style={white},
xmajorgrids,
ymajorgrids,
xmin=-1.25, xmax=48.25,
xtick style={color=black},
ymin=8.86884898711136, ymax=21.6651641472417,
ytick style={color=black},
xlabel={Number of Independent GPs (Q)},
]
\addplot [semithick, steelblue31119180, mark=*, mark size=1, mark options={solid}]
table {%
1 21.0835134581449
2 18.3074131831191
3 16.4206159007979
4 14.9955780995115
5 14.0728498152143
6 13.2695063821556
7 13.0848299662623
8 12.3293246344218
9 12.070232627838
10 11.7417225029505
11 11.7471296747109
12 11.2421548174052
13 11.3280135169472
14 11.2148134392746
15 11.3245267007588
16 11.0624089793492
17 10.8184214069237
18 11.4967152313319
19 10.8781033964688
20 18.6237207949712
21 10.3785350440165
22 16.8366923941771
23 11.4502647085967
24 16.6523032914699
25 16.4742262891044
26 16.4806690706559
27 17.9369910621039
28 9.4504996762082
29 17.1754065369533
30 16.6168246682744
31 17.7910477646339
32 17.0238861153895
33 18.6284060915152
34 16.8605201445382
35 17.4265808148074
36 17.2181865799902
37 18.3560371967843
38 18.031564791182
39 19.0965517317987
40 17.9497627245155
41 17.7616050099039
42 18.7071726838592
43 18.2042127111534
44 17.8530622548654
45 17.8850276973507
46 17.3984591814923
};
\end{axis}

\end{tikzpicture}
}
	\caption{Performance metrics for LMCGP models as a function of the number of independent GPs. We finally select $Q=17$ as the proper parameter}
	\end{figure}
\end{frame}

\begin{frame}{Lengthscale and $a_{d,q}$ Values (H=1)}
	\begin{figure}[htbp]
		\centering
		\setlength\figurewidth{0.46\columnwidth} 
		\setlength\figureheight{0.46\columnwidth}
		\subfloat[\label{fig:lmc_lengthscales}]{% This file was created with tikzplotlib v0.10.1.
\begin{tikzpicture}

\definecolor{darkgray176}{RGB}{176,176,176}

\begin{axis}[
width=\figurewidth,%
height=\figureheight,%
title={$H=1$},
%colorbar,
%colorbar style={ylabel={}},
colormap/viridis,
point meta max=40,
point meta min=20,
%tick align=outside,
tick pos=left,
x grid style={darkgray176},
xmin=0, xmax=23,
xtick style={color=black},
xtick={0.5,2.5,4.5,6.5,8.5,10.5,12.5,14.5,16.5,18.5,20.5,22.5},
xticklabel style={rotate=0.0},
xticklabels={A,C,E,G,I,K,M,O,Q,S,U,W},
y dir=reverse,
y grid style={darkgray176},
ymin=0, ymax=17,
ytick style={color=black},
ytick={0.5,2.5,4.5,6.5,8.5,10.5,12.5,14.5,16.5,18.5,20.5,22.5},
yticklabel style={rotate=0.0},
yticklabels={1,3,5,7,9,11,13,15,17},
xlabel style={font=\tiny},
tick label style={font=\tiny}
]
\addplot graphics [includegraphics cmd=\pgfimage,xmin=0, xmax=23, ymin=17, ymax=0] {chp_lmc/figures/lengthscale_matrix_1-000.png};
\end{axis}

\end{tikzpicture}
}\hspace{-0.8em}
		\subfloat[\label{fig:lmc_W_matrix}]{% This file was created with tikzplotlib v0.10.1.
\begin{tikzpicture}

\definecolor{darkgray176}{RGB}{176,176,176}

\begin{axis}[
width=\figurewidth,%
height=\figureheight,%
colorbar,
colorbar style={ylabel={}},
colormap/viridis,
point meta max=1.5,
point meta min=0,
tick align=outside,
tick pos=left,
x grid style={darkgray176},
xmin=0, xmax=17,
xtick style={color=black},
xtick={0.5,2.5,4.5,6.5,8.5,10.5,12.5,14.5,16.5,18.5,20.5,22.5},
xticklabel style={rotate=0.0},
xticklabels={1,3,5,7,9,11,13,15,17},
y dir=reverse,
y grid style={darkgray176},
ymin=0, ymax=23,
ytick style={color=black},
ytick={0.5,2.5,4.5,6.5,8.5,10.5,12.5,14.5,16.5,18.5,20.5,22.5},
yticklabel style={rotate=0.0},
yticklabels={A,C,E,G,I,K,M,O,Q,S,U,W},
]
\addplot graphics [includegraphics cmd=\pgfimage,xmin=0, xmax=17, ymin=23, ymax=0] {chp_lmc/figures/coregionalization_1-000.png};
\end{axis}

\end{tikzpicture}
}
		\caption{Lengthscale values of Input Features for Each Independent GP Model (left) and coefficients $a_{d,q}$ (right) for horizon $H=1$.}
		\label{fig:lmc_parameters_h1} 
	\end{figure}
	
\end{frame}

\begin{frame}{Lengthscale and $a_{d,q}$ Values (H=30)}
	\begin{figure}[htbp]
		%\centering
		\setlength\figurewidth{0.46\columnwidth} 
		\setlength\figureheight{0.46\columnwidth}
		\subfloat[\label{fig:lmc_lengthscales_h30}]{% This file was created with tikzplotlib v0.10.1.
\begin{tikzpicture}

\definecolor{darkgray176}{RGB}{176,176,176}

\begin{axis}[
width=\figurewidth,%
height=\figureheight,%
colorbar,
colorbar style={ylabel={}},
colormap/viridis,
point meta max=40,
point meta min=20,
tick align=outside,
tick pos=left,
x grid style={darkgray176},
xmin=0, xmax=23,
xtick style={color=black},
xtick={0.5,2.5,4.5,6.5,8.5,10.5,12.5,14.5,16.5,18.5,20.5,22.5},
xticklabel style={rotate=0.0},
xticklabels={A,C,E,G,I,K,M,O,Q,S,U,W},
y dir=reverse,
y grid style={darkgray176},
ymin=0, ymax=17,
ytick style={color=black},
ytick={0.5,2.5,4.5,6.5,8.5,10.5,12.5,14.5,16.5,18.5,20.5,22.5},
yticklabel style={rotate=0.0},
yticklabels={1,3,5,7,9,11,13,15,17},
xlabel style={font=\tiny},
tick label style={font=\tiny}
]
\addplot graphics [includegraphics cmd=\pgfimage,xmin=0, xmax=23, ymin=17, ymax=0] {chp_lmc/figures/lengthscale_matrix_30-000.png};
\end{axis}

\end{tikzpicture}
}\hspace{-0.8em}
		\subfloat[\label{fig:lmc_W_matrix_h30}]{% This file was created with tikzplotlib v0.10.1.
\begin{tikzpicture}

\definecolor{darkgray176}{RGB}{176,176,176}

\begin{axis}[
width=\figurewidth,%
height=\figureheight,%
colorbar,
colorbar style={ylabel={}},
colormap/viridis,
point meta max=1.5,
point meta min=0,
tick align=outside,
tick pos=left,
x grid style={darkgray176},
xmin=0, xmax=17,
xtick style={color=black},
xtick={0.5,2.5,4.5,6.5,8.5,10.5,12.5,14.5,16.5,18.5,20.5,22.5},
xticklabel style={rotate=0.0},
xticklabels={1,3,5,7,9,11,13,15,17},
y dir=reverse,
y grid style={darkgray176},
ymin=0, ymax=23,
ytick style={color=black},
ytick={0.5,2.5,4.5,6.5,8.5,10.5,12.5,14.5,16.5,18.5,20.5,22.5},
yticklabel style={rotate=0.0},
yticklabels={A,C,E,G,I,K,M,O,Q,S,U,W},
xlabel style={font=\tiny},
tick label style={font=\tiny}
]
\addplot graphics [includegraphics cmd=\pgfimage,xmin=0, xmax=17, ymin=23, ymax=0] {chp_lmc/figures/coregionalization_30-000.png};
\end{axis}

\end{tikzpicture}
}
		\caption{Lengthscale values of Input Features for Each Independent GP Model (left) and coefficients $a_{d,q}$ (right) for horizon $H=30$.}
		\label{fig:lmc_parameters_h30} 
	\end{figure}
\end{frame}

\subsubsection{Performance Analysis}

\begin{frame}{LMCGP vs IGP+ vs IGP}
The performance analysis compares LMCGP against two multi-output GP models: the IGP implemented, trained using only Adam optimizer, and another IGP trained into Adam + NG framework, called here IGP+.
	\begin{figure}[htbp]
		\centering
		\setlength\figurewidth{0.52\columnwidth}
		\setlength\figureheight{0.42\columnwidth}
		
		\subfloat[NLPD]{% This file was created with tikzplotlib v0.10.1.
\begin{tikzpicture}

\definecolor{darkgray176}{RGB}{176,176,176}
\definecolor{gray}{RGB}{128,128,128}
\definecolor{green01270}{RGB}{0,127,0}

\begin{axis}[
width=\figurewidth,%
height=\figureheight,%
axis background/.style={fill=background_color},
tick align=outside,
tick pos=left,
axis line style={white},
tick align=inside,
tick pos=left,
x grid style={white},
y grid style={white},
xmajorgrids,
ymajorgrids,
xlabel={Horizon},
xtick style={color=black},
xtick={0.25,1.25,2.25,3.25,4.25,5.25,6.25,7.25,8.25,9.25},
xticklabels={1,2,3,4,5,6,7,14,21,30},
ytick style={color=black},
xmin=-0.6125, xmax=10.1125,
ymin=0, ymax=29.8449605648958,
xlabel style={font=\tiny},
tick label style={font=\tiny}
]
\draw[draw=gray,fill=blue] (axis cs:-0.125,0) rectangle (axis cs:0.125,21.1569115523988);
\draw[draw=gray,fill=blue] (axis cs:0.875,0) rectangle (axis cs:1.125,23.5500376665929);
\draw[draw=gray,fill=blue] (axis cs:1.875,0) rectangle (axis cs:2.125,24.4596859590331);
\draw[draw=gray,fill=blue] (axis cs:2.875,0) rectangle (axis cs:3.125,25.3025311179923);
\draw[draw=gray,fill=blue] (axis cs:3.875,0) rectangle (axis cs:4.125,25.7392946447281);
\draw[draw=gray,fill=blue] (axis cs:4.875,0) rectangle (axis cs:5.125,25.8577961476045);
\draw[draw=gray,fill=blue] (axis cs:5.875,0) rectangle (axis cs:6.125,26.1182370685187);
\draw[draw=gray,fill=blue] (axis cs:6.875,0) rectangle (axis cs:7.125,27.0243046679054);
\draw[draw=gray,fill=blue] (axis cs:7.875,0) rectangle (axis cs:8.125,27.7866695852108);
\draw[draw=gray,fill=blue] (axis cs:8.875,0) rectangle (axis cs:9.125,28.4237719665674);
\draw[draw=gray,fill=green01270] (axis cs:0.125,0) rectangle (axis cs:0.375,11.7354752059831);

\draw[draw=gray,fill=green01270] (axis cs:1.125,0) rectangle (axis cs:1.375,14.4672232695331);
\draw[draw=gray,fill=green01270] (axis cs:2.125,0) rectangle (axis cs:2.375,15.1059474613183);
\draw[draw=gray,fill=green01270] (axis cs:3.125,0) rectangle (axis cs:3.375,15.8355933150668);
\draw[draw=gray,fill=green01270] (axis cs:4.125,0) rectangle (axis cs:4.375,15.7177185284159);
\draw[draw=gray,fill=green01270] (axis cs:5.125,0) rectangle (axis cs:5.375,16.2355496277881);
\draw[draw=gray,fill=green01270] (axis cs:6.125,0) rectangle (axis cs:6.375,16.1205779992887);
\draw[draw=gray,fill=green01270] (axis cs:7.125,0) rectangle (axis cs:7.375,16.4226456565551);
\draw[draw=gray,fill=green01270] (axis cs:8.125,0) rectangle (axis cs:8.375,17.167740838524);
\draw[draw=gray,fill=green01270] (axis cs:9.125,0) rectangle (axis cs:9.375,17.0291942568183);
\draw[draw=gray,fill=red] (axis cs:0.375,0) rectangle (axis cs:0.625,10.8184214069237);

\draw[draw=gray,fill=red] (axis cs:1.375,0) rectangle (axis cs:1.625,12.4999191375716);
\draw[draw=gray,fill=red] (axis cs:2.375,0) rectangle (axis cs:2.625,13.5104526402534);
\draw[draw=gray,fill=red] (axis cs:3.375,0) rectangle (axis cs:3.625,13.8888512386877);
\draw[draw=gray,fill=red] (axis cs:4.375,0) rectangle (axis cs:4.625,13.6898117257208);
\draw[draw=gray,fill=red] (axis cs:5.375,0) rectangle (axis cs:5.625,14.5583942742105);
\draw[draw=gray,fill=red] (axis cs:6.375,0) rectangle (axis cs:6.625,13.6712080903002);
\draw[draw=gray,fill=red] (axis cs:7.375,0) rectangle (axis cs:7.625,14.2560387120429);
\draw[draw=gray,fill=red] (axis cs:8.375,0) rectangle (axis cs:8.625,14.4738108583833);
\draw[draw=gray,fill=red] (axis cs:9.375,0) rectangle (axis cs:9.625,14.8850707781946);
\end{axis}

\end{tikzpicture}
}
		\hfill
		\subfloat[MSLL]{% This file was created with tikzplotlib v0.10.1.
\begin{tikzpicture}

\definecolor{darkgray176}{RGB}{176,176,176}
\definecolor{gray}{RGB}{128,128,128}
\definecolor{green01270}{RGB}{0,127,0}
\definecolor{lightgray204}{RGB}{204,204,204}

\begin{axis}[
width=\figurewidth,%
height=\figureheight,%
axis background/.style={fill=background_color},
tick align=outside,
tick pos=left,
axis line style={white},
tick align=inside,
tick pos=left,
x grid style={white},
y grid style={white},
xmajorgrids,
ymajorgrids,
xlabel={Horizon},
xmin=-0.6125, xmax=10.1125,
xtick style={color=black},
xtick={0.25,1.25,2.25,3.25,4.25,5.25,6.25,7.25,8.25,9.25},
xticklabels={1,2,3,4,5,6,7,14,21,30},
ymin=-1.03083974778409, ymax=0,
ytick style={color=black},
legend columns=3, 
legend style={
  nodes={scale=0.8, transform shape},
  fill opacity=1,
  draw opacity=1,
  text opacity=1,
  at={(0.03,0.97)},
  anchor=north west,
  legend pos=north west,
  draw=lightgray204,
  fill=background_color,
  font=\tiny
},
xlabel style={font=\tiny},
tick label style={font=\tiny}
]
\draw[draw=gray,fill=blue] (axis cs:-0.125,0) rectangle (axis cs:0.125,-0.532252569204355);
\addlegendimage{ybar,ybar legend,draw=gray,fill=blue}
\addlegendentry{IGP}

\draw[draw=gray,fill=blue] (axis cs:0.875,0) rectangle (axis cs:1.125,-0.428154570305303);
\draw[draw=gray,fill=blue] (axis cs:1.875,0) rectangle (axis cs:2.125,-0.388555233274582);
\draw[draw=gray,fill=blue] (axis cs:2.875,0) rectangle (axis cs:3.125,-0.351859932002441);
\draw[draw=gray,fill=blue] (axis cs:3.875,0) rectangle (axis cs:4.125,-0.332820583869734);
\draw[draw=gray,fill=blue] (axis cs:4.875,0) rectangle (axis cs:5.125,-0.327618686894435);
\draw[draw=gray,fill=blue] (axis cs:5.875,0) rectangle (axis cs:6.125,-0.316243282995642);
\draw[draw=gray,fill=blue] (axis cs:6.875,0) rectangle (axis cs:7.125,-0.272667322281103);
\draw[draw=gray,fill=blue] (axis cs:7.875,0) rectangle (axis cs:8.125,-0.238004115713702);
\draw[draw=gray,fill=blue] (axis cs:8.875,0) rectangle (axis cs:9.125,-0.209382344201211);
\draw[draw=gray,fill=green01270] (axis cs:0.125,0) rectangle (axis cs:0.375,-0.941880236439821);
\addlegendimage{ybar,ybar legend,draw=gray,fill=green01270}
\addlegendentry{IGP+}

\draw[draw=gray,fill=green01270] (axis cs:1.125,0) rectangle (axis cs:1.375,-0.823059544090509);
\draw[draw=gray,fill=green01270] (axis cs:2.125,0) rectangle (axis cs:2.375,-0.795239515783919);
\draw[draw=gray,fill=green01270] (axis cs:3.125,0) rectangle (axis cs:3.375,-0.763465923433985);
\draw[draw=gray,fill=green01270] (axis cs:4.125,0) rectangle (axis cs:4.375,-0.768541284578963);
\draw[draw=gray,fill=green01270] (axis cs:5.125,0) rectangle (axis cs:5.375,-0.745977231234278);
\draw[draw=gray,fill=green01270] (axis cs:6.125,0) rectangle (axis cs:6.375,-0.750924112092601);
\draw[draw=gray,fill=green01270] (axis cs:7.125,0) rectangle (axis cs:7.375,-0.733609018426772);
\draw[draw=gray,fill=green01270] (axis cs:8.125,0) rectangle (axis cs:8.375,-0.699696669917477);
\draw[draw=gray,fill=green01270] (axis cs:9.125,0) rectangle (axis cs:9.375,-0.704798766364212);
\draw[draw=gray,fill=red] (axis cs:0.375,0) rectangle (axis cs:0.625,-0.981752140746748);
\addlegendimage{ybar,ybar legend,draw=gray,fill=red}
\addlegendentry{LMCGP}

\draw[draw=gray,fill=red] (axis cs:1.375,0) rectangle (axis cs:1.625,-0.908594506349704);
\draw[draw=gray,fill=red] (axis cs:2.375,0) rectangle (axis cs:2.625,-0.86460885583022);
\draw[draw=gray,fill=red] (axis cs:3.375,0) rectangle (axis cs:3.625,-0.848106883276553);
\draw[draw=gray,fill=red] (axis cs:4.375,0) rectangle (axis cs:4.625,-0.856711145565708);
\draw[draw=gray,fill=red] (axis cs:5.375,0) rectangle (axis cs:5.625,-0.81889702921591);
\draw[draw=gray,fill=red] (axis cs:6.375,0) rectangle (axis cs:6.625,-0.857418455961666);
\draw[draw=gray,fill=red] (axis cs:7.375,0) rectangle (axis cs:7.625,-0.827809320362084);
\draw[draw=gray,fill=red] (axis cs:8.375,0) rectangle (axis cs:8.625,-0.816824060358377);
\draw[draw=gray,fill=red] (axis cs:9.375,0) rectangle (axis cs:9.625,-0.798021526304375);
\end{axis}

\end{tikzpicture}
}
		
		\caption{Bar plots comparing the performance of LMCGP, IGP+, and IGP models for different \(H\) values.}
	\end{figure}
\end{frame}

\subsubsection{Performance Analysis}
\subsubsection{Concluding Remark}