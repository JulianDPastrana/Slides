\section{Objective 2: Multi-Output Gaussian Processes}

\subsection{Mathematical Framework}

%\begin{frame}{Independent Gaussian Process (IGP)}
%	
%	\begin{columns}[T] % Top alignment
%		\begin{column}{0.5\textwidth}
%			
%			\textbf{Independent Process} 
%			\begin{equation*}
%				u_{d}(\mathbf{x}) \sim \mathcal{GP}(0, k_{d}(\mathbf{x}, \mathbf{x}'))
%			\end{equation*}
%			
%			\textbf{Latent Process}
%			\begin{equation*}
%				f_{d}(\mathbf{x}) = u_{d}(\mathbf{x})
%			\end{equation*}
%			
%			
%			\textbf{Multi-Output Model}
%			\begin{equation*}
%				\mathbf{f}_{*}
%				\sim
%				\mathcal{N} \left(
%				\begin{array}{c}
%					\mathbf{0}\\
%				\end{array},
%				\sum_{d=1}^D \mathbf{B}_d \otimes K_{d**}\right)
%			\end{equation*}
%			
%			\begin{itemize}
%				\item  $\mathbf{B}_d = (\delta_{d,d'}) \in \mathbb{R}^{D \times D}$.
%				
%				\item $K_{d**} \in \mathbb{R}^{N_* \times N_*}$ is the $d$-th covariance matrix at $X_*$ test inputs.
%				
%				\item $\otimes$ denotes the Kronecker product.
%			\end{itemize}	
%			
%		\end{column}
%		
%		\begin{column}{0.5\textwidth}
%			\setlength\figurewidth{0.8\textwidth} 
%			\setlength\figureheight{0.32\textwidth}
%			% NEURAL NETWORK activation
% https://www.youtube.com/watch?v=aircAruvnKk&list=PLZHQObOWTQDNU6R1_67000Dx_ZCJB-3pi&index=1
\begin{tikzpicture}[x=\figurewidth,y=\figureheight]
  \message{^^JNeural network activation}
  \def\NI{4} % number of nodes in input layers
  \def\NO{4} % number of nodes in output layers
  \def\yshift{0.4} % shift last node for dots
  
  % INPUT LAYER
  \foreach \i [evaluate={\c=int(\i==\NI); \y=\NI/2-\i-\c*\yshift; \index=(\i<\NI?int(\i):"D");}]
              in {1,...,\NI}{ % loop over nodes
    \node[node in,outer sep=0.6] (NI-\i) at (0,\y) {$u_{\index}$};
  }
  
  % OUTPUT LAYER
  \foreach \i [evaluate={\c=int(\i==\NO); \y=\NO/2-\i-\c*\yshift; \index=(\i<\NO?int(\i):"D");}]
    in {\NO,...,1}{ % loop over nodes
    \ifnum\i=1 % high-lighted node
      \node[node hidden]
        (NO-\i) at (1,\y) {$f_{\index}$};
      % \foreach \j [evaluate={\index=(\j<\NI?int(\j):"D");}] in {1,...,\NI}{ % loop over nodes in previous layer
        \draw[connect,white,line width=1.2] (NI-\i) -- (NO-\i);
        \draw[connect] (NI-\i) -- (NO-\i);
      %     % node[pos=0.50] {\contour{white}{$a_{1,\index}$}};
      % }
    \else % other light-colored nodes
      \node[node,blue!20!black!80,draw=myblue!20,fill=myblue!5]
        (NO-\i) at (1,\y) {$f_{\index}$};
      % \foreach \j in {1,...,\NI}{ % loop over nodes in previous layer
        %\draw[connect,white,line width=1.2] (NI-\j) -- (NO-\i);
        \draw[connect,myblue!20] (NI-\i) -- (NO-\i);
      % }
    \fi
  }
  
  % DOTS
  \path (NI-\NI) --++ (0,1+\yshift) node[midway,scale=1.2] {$\vdots$};
  \path (NO-\NO) --++ (0,1+\yshift) node[midway,scale=1.2] {$\vdots$};
  

  
\end{tikzpicture}
	
%		\end{column}
%		
%	\end{columns}
%\end{frame}


\begin{frame}{Linear Model of Coregionalization GP (LMCGP)}
	\centering
	\setlength\figurewidth{0.55\textwidth} 
	\setlength\figureheight{0.11\textwidth}
	% NEURAL NETWORK activation
% https://www.youtube.com/watch?v=aircAruvnKk&list=PLZHQObOWTQDNU6R1_67000Dx_ZCJB-3pi&index=1
\begin{tikzpicture}[x=\figurewidth,y=\figureheight]
  \message{^^JNeural network activation}
  \def\NI{4} % number of nodes in input layers
  \def\NO{4} % number of nodes in output layers
  \def\yshift{0.4} % shift last node for dots
  
  % INPUT LAYER
  \foreach \i [evaluate={\c=int(\i==\NI); \y=\NI/2-\i-\c*\yshift; \index=(\i<\NI?int(\i):"Q");}]
              in {1,...,\NI}{ % loop over nodes
    \node[node in,outer sep=0.6] (NI-\i) at (0,\y) {$u_{\index}$};
  }
  
  % OUTPUT LAYER
  \foreach \i [evaluate={\c=int(\i==\NO); \y=\NO/2-\i-\c*\yshift; \index=(\i<\NO?int(\i):"D");}]
    in {\NO,...,1}{ % loop over nodes
    \ifnum\i=1 % high-lighted node
      \node[node hidden]
        (NO-\i) at (1,\y) {$f_{\index}$};
      \foreach \j [evaluate={\index=(\j<\NI?int(\j):"Q");}] in {1,...,\NI}{ % loop over nodes in previous layer
        \draw[connect,white,line width=1.2] (NI-\j) -- (NO-\i);
        \draw[connect] (NI-\j) -- (NO-\i)
          node[pos=0.50] {\contour{white}{$a_{1,\index}$}};
      }
    \else % other light-colored nodes
      \node[node,blue!20!black!80,draw=myblue!20,fill=myblue!5]
        (NO-\i) at (1,\y) {$f_{\index}$};
      \foreach \j in {1,...,\NI}{ % loop over nodes in previous layer
        %\draw[connect,white,line width=1.2] (NI-\j) -- (NO-\i);
        \draw[connect,myblue!20] (NI-\j) -- (NO-\i);
      }
    \fi
  }
  
  % DOTS
  \path (NI-\NI) --++ (0,1+\yshift) node[midway,scale=1.2] {$\vdots$};
  \path (NO-\NO) --++ (0,1+\yshift) node[midway,scale=1.2] {$\vdots$};
  

  
\end{tikzpicture}

	
	\begin{columns}[T] % Top alignment

		\begin{column}{0.5\textwidth}
			\centering
	\textcolor{mygreen}{
		\textbf{Independent Process (IGP)}
		\begin{equation*}
			u_{q}(\mathbf{x}) \sim \mathcal{GP}(0, k_{q}(\mathbf{x}, \mathbf{x}'))
		\end{equation*}
	}
	\end{column}
	
	
	\begin{column}{0.5\textwidth}
		\centering
		\textcolor{myblue}{
			\textbf{Latent Process (LMCGP)}
			\begin{equation*}
				f_{d}(\mathbf{x}) = \sum_{q=1}^Q a_{d,q} u_{q}(\mathbf{x})
			\end{equation*}	
		}
	\end{column}
	
	\end{columns}
\end{frame}


\begin{frame}{Variational Inference, ELBO, and Predictive Distribution}
	We extend variational inference to include the independent set, utilizing the inducing variables $u_q$ derived from independent processes. The ELBO is given by:
	
	\begin{equation*}\label{eq:lmc_elbo}
		\mathcal{L} = \sum_{d=1}^D\sum_{n=1}^{N} \mathbb{E}_{q(f_{dn})}\{\log p(y_{dn} \mid f_{dn})\} - \sum_{q=1}^Q \text{KL}\{q(\mathbf{u}_q)\parallel p(\mathbf{u}_q)\}
	\end{equation*}
	
	The posterior over test points $X_*$, $p(\mathbf{f}_* \mid \mathbf{y})$, is given by:
	
	\begin{equation*}
		p(\mathbf{f}_* \mid \mathbf{y}) \approx q(\mathbf{f}_*) = \int p(\mathbf{f}_* \mid \mathbf{u}) q(\mathbf{u}) d \mathbf{u}
	\end{equation*}
	
	Gaussian noise $\sigma_{Nd}^2$ is added to obtain the predictive distribution.
\end{frame}

\begin{frame}{Model Setup}

	\begin{block}{Covariance Function (LMCGP)}
		The LMCGP model uses a squared exponential kernel:
		\begin{equation*}
			k_{q}\left(\mathbf{x}, \mathbf{x'} \mid \Theta_q \right) = \exp\left(-\frac{1}{2}(\mathbf{x} - \mathbf{x'})^\top \Theta_q^{-2} (\mathbf{x} - \mathbf{x'})\right)
		\end{equation*}
		Here, \( \Theta_q \) is the lengthscale matrix, and \( a_{d, q} \) work as outputscales.
	\end{block}
	
	\begin{block}{Optimization and Model Variants}
		Strong dependencies between parameters may cause poor local minima \cite{giraldo2021fully}. We address this by combining Natural Gradient (NG) to optimize variational parameters, and Adam for the rest \cite{pmlr-v84-salimbeni18a}.

		\textbf{Variants:}
		\begin{itemize}
			\item IGP: Independent GP (Adam).
			\item IGP+: Independent GP (Adam+NG).
			\item LMCGP: Correlated GP (Adam+NG)
		\end{itemize}
	\end{block}
\end{frame}



\subsection{Results and Discussions}

\subsubsection{Hyperparameter Tuning}

\begin{frame}{Performance metrics vs Q}
	\begin{figure}[htbp]
	\tiny
	\centering
	\setlength\figurewidth{0.5\columnwidth} 
	\setlength\figureheight{0.28\columnwidth}
	
	\subfloat[MSE]{% This file was created with tikzplotlib v0.10.1.
\begin{tikzpicture}

\definecolor{darkgray176}{RGB}{176,176,176}
\definecolor{steelblue31119180}{RGB}{31,119,180}

\begin{axis}[
width=\figurewidth,
height=\figureheight,
axis background/.style={fill=background_color},
axis line style={white},
tick align=inside,
tick pos=left,
x grid style={white},
y grid style={white},
xmajorgrids,
ymajorgrids,
xmin=-1.25, xmax=48.25,
xtick style={color=black},
ymin=0.259536486825269, ymax=0.731194606213278,
ytick style={color=black},
xlabel style={font=\tiny},
tick label style={font=\tiny}
]
\addplot [semithick, steelblue31119180, mark=*, mark size=1, mark options={solid}]
table {%
1 0.70975560078655
2 0.570005452889205
3 0.524956091611488
4 0.468590469300411
5 0.435353470792991
6 0.411259722878218
7 0.409096034731835
8 0.385485371917808
9 0.381016960281645
10 0.360132840745332
11 0.362921694624643
12 0.347941662750589
13 0.33894395440443
14 0.342577112243772
15 0.343123787309784
16 0.334095395016974
17 0.330290643463412
18 0.322687081772815
19 0.326757528243904
20 0.437607617331642
21 0.304623270578551
22 0.389408892826712
23 0.316685484093948
24 0.381513763041057
25 0.375840283026447
26 0.377810182111291
27 0.414747999915877
28 0.280975492251997
29 0.393662339819777
30 0.373947243132724
31 0.402964916925179
32 0.385485092508175
33 0.431905374190024
34 0.38280924563172
35 0.397469992032685
36 0.390029348756403
37 0.427310239658667
38 0.408634597214847
39 0.437765674807815
40 0.40614578730405
41 0.40052469480261
42 0.43521136441382
43 0.423471188409076
44 0.405748603781294
45 0.409759054806833
46 0.390775042857023
};
\end{axis}

\end{tikzpicture}
}\hspace{-0.1em}
	\subfloat[MSLL]{% This file was created with tikzplotlib v0.10.1.
\begin{tikzpicture}

\definecolor{darkgray176}{RGB}{176,176,176}
\definecolor{steelblue31119180}{RGB}{31,119,180}

\begin{axis}[
width=\figurewidth,
height=\figureheight,
axis background/.style={fill=background_color},
axis line style={white},
tick align=inside,
tick pos=left,
x grid style={white},
y grid style={white},
xmajorgrids,
ymajorgrids,
xmin=-1.25, xmax=48.25,
xtick style={color=black},
ymin=-1.06651615899946, ymax=-0.510154630298139,
ytick style={color=black},
xlabel={Number of Independent GPs (Q)},
xlabel style={font=\tiny},
tick label style={font=\tiny}
]
\addplot [semithick, steelblue31119180, mark=*, mark size=1, mark options={solid}]
table {%
1 -0.535443790693654
2 -0.656143802651295
3 -0.738178467100046
4 -0.800136632373368
5 -0.840255253429766
6 -0.875183228780147
7 -0.883212638166813
8 -0.916060696072918
9 -0.92732556592439
10 -0.941608614832543
11 -0.941373520408176
12 -0.963328948986684
13 -0.959595962050077
14 -0.96451770455758
15 -0.95974756275392
16 -0.971143985423901
17 -0.981752140746748
18 -0.952261104902915
19 -0.979157271636092
20 -0.642391297788161
21 -1.00087763478619
22 -0.720088184779209
23 -0.954280692847926
24 -0.728105102288219
25 -0.735847580651938
26 -0.735567459714915
27 -0.672249112260655
28 -1.04122699860395
29 -0.705361482919378
30 -0.729647651122807
31 -0.678594473020219
32 -0.711949327335191
33 -0.64218758924277
34 -0.719052195633076
35 -0.694440862143111
36 -0.703501481048209
37 -0.654029715100638
38 -0.668137210996388
39 -0.621833430969577
40 -0.671693822590583
41 -0.67987459279109
42 -0.638762954793032
43 -0.660630779693286
44 -0.675898190836243
45 -0.674508388989054
46 -0.695663541852463
};
\end{axis}

\end{tikzpicture}
}\\
	\vspace{-0.1em} % Adjust spacing between rows
	
	\subfloat[CRPS]{% This file was created with tikzplotlib v0.10.1.
\begin{tikzpicture}

\definecolor{darkgray176}{RGB}{176,176,176}
\definecolor{steelblue31119180}{RGB}{31,119,180}

\begin{axis}[
width=\figurewidth,
height=\figureheight,
axis background/.style={fill=background_color},
axis line style={white},
tick align=inside,
tick pos=left,
x grid style={white},
y grid style={white},
xmajorgrids,
ymajorgrids,
xmin=-1.25, xmax=48.25,
xtick style={color=black},
ymin=0.239929987456654, ymax=0.438839652236985,
ytick style={color=black},
xlabel style={font=\tiny},
tick label style={font=\tiny}
]
\addplot [semithick, steelblue31119180, mark=*, mark size=1, mark options={solid}]
table {%
1 0.429798303837879
2 0.374510552905612
3 0.345960005084818
4 0.323892519212847
5 0.311383175599372
6 0.299542632982873
7 0.297875424155548
8 0.28784981072712
9 0.286844690564962
10 0.279610258905049
11 0.279515690474725
12 0.273041146856969
13 0.271082282952741
14 0.271889987179417
15 0.270210632285698
16 0.26759372049199
17 0.266678895529016
18 0.264912747643612
19 0.26441644238815
20 0.309002018166923
21 0.257220132139326
22 0.291456973758776
23 0.2613559059979
24 0.287804261488934
25 0.285577216964857
26 0.287627602712463
27 0.300194088705333
28 0.24897133585576
29 0.293655299555358
30 0.287299736020118
31 0.296865072134467
32 0.291250790206968
33 0.307497232174908
34 0.289441525631671
35 0.294363581321622
36 0.291544275421493
37 0.305791393660394
38 0.296662343784081
39 0.308247437640682
40 0.298021172730888
41 0.29625972317634
42 0.307272161655389
43 0.303820351549033
44 0.297729094232099
45 0.299530085504188
46 0.293446316295075
};
\end{axis}

\end{tikzpicture}
}\hspace{-0.1em}
	\subfloat[NLPD]{% This file was created with tikzplotlib v0.10.1.
\begin{tikzpicture}

\definecolor{darkgray176}{RGB}{176,176,176}
\definecolor{steelblue31119180}{RGB}{31,119,180}

\begin{axis}[
width=\figurewidth,
height=\figureheight,
axis background/.style={fill=background_color},
axis line style={white},
tick align=inside,
tick pos=left,
x grid style={white},
y grid style={white},
xmajorgrids,
ymajorgrids,
xmin=-1.25, xmax=48.25,
xtick style={color=black},
ymin=8.86884898711136, ymax=21.6651641472417,
ytick style={color=black},
xlabel={Number of Independent GPs (Q)},
]
\addplot [semithick, steelblue31119180, mark=*, mark size=1, mark options={solid}]
table {%
1 21.0835134581449
2 18.3074131831191
3 16.4206159007979
4 14.9955780995115
5 14.0728498152143
6 13.2695063821556
7 13.0848299662623
8 12.3293246344218
9 12.070232627838
10 11.7417225029505
11 11.7471296747109
12 11.2421548174052
13 11.3280135169472
14 11.2148134392746
15 11.3245267007588
16 11.0624089793492
17 10.8184214069237
18 11.4967152313319
19 10.8781033964688
20 18.6237207949712
21 10.3785350440165
22 16.8366923941771
23 11.4502647085967
24 16.6523032914699
25 16.4742262891044
26 16.4806690706559
27 17.9369910621039
28 9.4504996762082
29 17.1754065369533
30 16.6168246682744
31 17.7910477646339
32 17.0238861153895
33 18.6284060915152
34 16.8605201445382
35 17.4265808148074
36 17.2181865799902
37 18.3560371967843
38 18.031564791182
39 19.0965517317987
40 17.9497627245155
41 17.7616050099039
42 18.7071726838592
43 18.2042127111534
44 17.8530622548654
45 17.8850276973507
46 17.3984591814923
};
\end{axis}

\end{tikzpicture}
}
	\end{figure}
	\vspace{-1em}
	\begin{block}{}
	 Beyond $Q=19$, adding more independent GPs becomes counterproductive due to handling a more complex model. We select $Q=17$ as the proper parameter.
	\end{block}
\end{frame}

\begin{frame}{Reservoir-wise Lengthscale and $a_{d, q}$}
\begin{figure}[htbp]
	\tiny
	\centering
	\setlength\figurewidth{0.4\columnwidth} 
	\setlength\figureheight{0.35\columnwidth}
	\vspace{-1.5em}
	% This file was created with tikzplotlib v0.10.1.
\begin{tikzpicture}

\definecolor{darkgray176}{RGB}{176,176,176}

\begin{axis}[
width=\figurewidth,%
height=\figureheight,%
title={$H=1$},
%colorbar,
%colorbar style={ylabel={}},
colormap/viridis,
point meta max=40,
point meta min=20,
%tick align=outside,
tick pos=left,
x grid style={darkgray176},
xmin=0, xmax=23,
xtick style={color=black},
xtick={0.5,2.5,4.5,6.5,8.5,10.5,12.5,14.5,16.5,18.5,20.5,22.5},
xticklabel style={rotate=0.0},
xticklabels={A,C,E,G,I,K,M,O,Q,S,U,W},
y dir=reverse,
y grid style={darkgray176},
ymin=0, ymax=17,
ytick style={color=black},
ytick={0.5,2.5,4.5,6.5,8.5,10.5,12.5,14.5,16.5,18.5,20.5,22.5},
yticklabel style={rotate=0.0},
yticklabels={1,3,5,7,9,11,13,15,17},
xlabel style={font=\tiny},
tick label style={font=\tiny}
]
\addplot graphics [includegraphics cmd=\pgfimage,xmin=0, xmax=23, ymin=17, ymax=0] {chp_lmc/figures/lengthscale_matrix_1-000.png};
\end{axis}

\end{tikzpicture}
 % Direct inclusion, no subcaption
	% This file was created with tikzplotlib v0.10.1.
\begin{tikzpicture}

\definecolor{darkgray176}{RGB}{176,176,176}

\begin{axis}[
width=\figurewidth,%
height=\figureheight,%
colorbar,
colorbar style={ylabel={}},
colormap/viridis,
point meta max=40,
point meta min=20,
tick align=outside,
tick pos=left,
x grid style={darkgray176},
xmin=0, xmax=23,
xtick style={color=black},
xtick={0.5,2.5,4.5,6.5,8.5,10.5,12.5,14.5,16.5,18.5,20.5,22.5},
xticklabel style={rotate=0.0},
xticklabels={A,C,E,G,I,K,M,O,Q,S,U,W},
y dir=reverse,
y grid style={darkgray176},
ymin=0, ymax=17,
ytick style={color=black},
ytick={0.5,2.5,4.5,6.5,8.5,10.5,12.5,14.5,16.5,18.5,20.5,22.5},
yticklabel style={rotate=0.0},
yticklabels={1,3,5,7,9,11,13,15,17},
xlabel style={font=\tiny},
tick label style={font=\tiny}
]
\addplot graphics [includegraphics cmd=\pgfimage,xmin=0, xmax=23, ymin=17, ymax=0] {chp_lmc/figures/lengthscale_matrix_30-000.png};
\end{axis}

\end{tikzpicture}
\\ % Direct inclusion, no subcaption
%	\vspace{-50.0em}
	% This file was created with tikzplotlib v0.10.1.
\begin{tikzpicture}

\definecolor{darkgray176}{RGB}{176,176,176}

\begin{axis}[
width=\figurewidth,%
height=\figureheight,%
colorbar,
colorbar style={ylabel={}},
colormap/viridis,
point meta max=1.5,
point meta min=0,
tick align=outside,
tick pos=left,
x grid style={darkgray176},
xmin=0, xmax=17,
xtick style={color=black},
xtick={0.5,2.5,4.5,6.5,8.5,10.5,12.5,14.5,16.5,18.5,20.5,22.5},
xticklabel style={rotate=0.0},
xticklabels={1,3,5,7,9,11,13,15,17},
y dir=reverse,
y grid style={darkgray176},
ymin=0, ymax=23,
ytick style={color=black},
ytick={0.5,2.5,4.5,6.5,8.5,10.5,12.5,14.5,16.5,18.5,20.5,22.5},
yticklabel style={rotate=0.0},
yticklabels={A,C,E,G,I,K,M,O,Q,S,U,W},
]
\addplot graphics [includegraphics cmd=\pgfimage,xmin=0, xmax=17, ymin=23, ymax=0] {chp_lmc/figures/coregionalization_1-000.png};
\end{axis}

\end{tikzpicture}
 % Direct inclusion, no subcaption
	% This file was created with tikzplotlib v0.10.1.
\begin{tikzpicture}

\definecolor{darkgray176}{RGB}{176,176,176}

\begin{axis}[
width=\figurewidth,%
height=\figureheight,%
colorbar,
colorbar style={ylabel={}},
colormap/viridis,
point meta max=1.5,
point meta min=0,
tick align=outside,
tick pos=left,
x grid style={darkgray176},
xmin=0, xmax=17,
xtick style={color=black},
xtick={0.5,2.5,4.5,6.5,8.5,10.5,12.5,14.5,16.5,18.5,20.5,22.5},
xticklabel style={rotate=0.0},
xticklabels={1,3,5,7,9,11,13,15,17},
y dir=reverse,
y grid style={darkgray176},
ymin=0, ymax=23,
ytick style={color=black},
ytick={0.5,2.5,4.5,6.5,8.5,10.5,12.5,14.5,16.5,18.5,20.5,22.5},
yticklabel style={rotate=0.0},
yticklabels={A,C,E,G,I,K,M,O,Q,S,U,W},
xlabel style={font=\tiny},
tick label style={font=\tiny}
]
\addplot graphics [includegraphics cmd=\pgfimage,xmin=0, xmax=17, ymin=23, ymax=0] {chp_lmc/figures/coregionalization_30-000.png};
\end{axis}

\end{tikzpicture}
 % Direct inclusion, no subcaption
\end{figure}


	\vspace{-1.0em}
	\begin{block}{}
		Increase horizon, lengthscale values show less selective. All independent GPs incorporate more features due to the extended time gap. The $a_{d,q}$ coefficients are smaller, indicating weaker individual feature contributions to each output.
	\end{block}
	
	
\end{frame}


\subsubsection{Performance Analysis}

\begin{frame}{Bar plots comparing IGP, IGP+, and LMCGP}
	\begin{figure}[htbp]
		\centering
		\setlength\figurewidth{0.56\columnwidth}
		\setlength\figureheight{0.48\columnwidth}
		
		\subfloat[MSLL]{% This file was created with tikzplotlib v0.10.1.
\begin{tikzpicture}

\definecolor{darkgray176}{RGB}{176,176,176}
\definecolor{gray}{RGB}{128,128,128}
\definecolor{green01270}{RGB}{0,127,0}
\definecolor{lightgray204}{RGB}{204,204,204}

\begin{axis}[
width=\figurewidth,%
height=\figureheight,%
axis background/.style={fill=background_color},
tick align=outside,
tick pos=left,
axis line style={white},
tick align=inside,
tick pos=left,
x grid style={white},
y grid style={white},
xmajorgrids,
ymajorgrids,
xlabel={Horizon},
xmin=-0.6125, xmax=10.1125,
xtick style={color=black},
xtick={0.25,1.25,2.25,3.25,4.25,5.25,6.25,7.25,8.25,9.25},
xticklabels={1,2,3,4,5,6,7,14,21,30},
ymin=-1.03083974778409, ymax=0,
ytick style={color=black},
legend columns=3, 
legend style={
  nodes={scale=0.8, transform shape},
  fill opacity=1,
  draw opacity=1,
  text opacity=1,
  at={(0.03,0.97)},
  anchor=north west,
  legend pos=north west,
  draw=lightgray204,
  fill=background_color,
  font=\tiny
},
xlabel style={font=\tiny},
tick label style={font=\tiny}
]
\draw[draw=gray,fill=blue] (axis cs:-0.125,0) rectangle (axis cs:0.125,-0.532252569204355);
\addlegendimage{ybar,ybar legend,draw=gray,fill=blue}
\addlegendentry{IGP}

\draw[draw=gray,fill=blue] (axis cs:0.875,0) rectangle (axis cs:1.125,-0.428154570305303);
\draw[draw=gray,fill=blue] (axis cs:1.875,0) rectangle (axis cs:2.125,-0.388555233274582);
\draw[draw=gray,fill=blue] (axis cs:2.875,0) rectangle (axis cs:3.125,-0.351859932002441);
\draw[draw=gray,fill=blue] (axis cs:3.875,0) rectangle (axis cs:4.125,-0.332820583869734);
\draw[draw=gray,fill=blue] (axis cs:4.875,0) rectangle (axis cs:5.125,-0.327618686894435);
\draw[draw=gray,fill=blue] (axis cs:5.875,0) rectangle (axis cs:6.125,-0.316243282995642);
\draw[draw=gray,fill=blue] (axis cs:6.875,0) rectangle (axis cs:7.125,-0.272667322281103);
\draw[draw=gray,fill=blue] (axis cs:7.875,0) rectangle (axis cs:8.125,-0.238004115713702);
\draw[draw=gray,fill=blue] (axis cs:8.875,0) rectangle (axis cs:9.125,-0.209382344201211);
\draw[draw=gray,fill=green01270] (axis cs:0.125,0) rectangle (axis cs:0.375,-0.941880236439821);
\addlegendimage{ybar,ybar legend,draw=gray,fill=green01270}
\addlegendentry{IGP+}

\draw[draw=gray,fill=green01270] (axis cs:1.125,0) rectangle (axis cs:1.375,-0.823059544090509);
\draw[draw=gray,fill=green01270] (axis cs:2.125,0) rectangle (axis cs:2.375,-0.795239515783919);
\draw[draw=gray,fill=green01270] (axis cs:3.125,0) rectangle (axis cs:3.375,-0.763465923433985);
\draw[draw=gray,fill=green01270] (axis cs:4.125,0) rectangle (axis cs:4.375,-0.768541284578963);
\draw[draw=gray,fill=green01270] (axis cs:5.125,0) rectangle (axis cs:5.375,-0.745977231234278);
\draw[draw=gray,fill=green01270] (axis cs:6.125,0) rectangle (axis cs:6.375,-0.750924112092601);
\draw[draw=gray,fill=green01270] (axis cs:7.125,0) rectangle (axis cs:7.375,-0.733609018426772);
\draw[draw=gray,fill=green01270] (axis cs:8.125,0) rectangle (axis cs:8.375,-0.699696669917477);
\draw[draw=gray,fill=green01270] (axis cs:9.125,0) rectangle (axis cs:9.375,-0.704798766364212);
\draw[draw=gray,fill=red] (axis cs:0.375,0) rectangle (axis cs:0.625,-0.981752140746748);
\addlegendimage{ybar,ybar legend,draw=gray,fill=red}
\addlegendentry{LMCGP}

\draw[draw=gray,fill=red] (axis cs:1.375,0) rectangle (axis cs:1.625,-0.908594506349704);
\draw[draw=gray,fill=red] (axis cs:2.375,0) rectangle (axis cs:2.625,-0.86460885583022);
\draw[draw=gray,fill=red] (axis cs:3.375,0) rectangle (axis cs:3.625,-0.848106883276553);
\draw[draw=gray,fill=red] (axis cs:4.375,0) rectangle (axis cs:4.625,-0.856711145565708);
\draw[draw=gray,fill=red] (axis cs:5.375,0) rectangle (axis cs:5.625,-0.81889702921591);
\draw[draw=gray,fill=red] (axis cs:6.375,0) rectangle (axis cs:6.625,-0.857418455961666);
\draw[draw=gray,fill=red] (axis cs:7.375,0) rectangle (axis cs:7.625,-0.827809320362084);
\draw[draw=gray,fill=red] (axis cs:8.375,0) rectangle (axis cs:8.625,-0.816824060358377);
\draw[draw=gray,fill=red] (axis cs:9.375,0) rectangle (axis cs:9.625,-0.798021526304375);
\end{axis}

\end{tikzpicture}
}
		\hfill
		\subfloat[NLPD]{% This file was created with tikzplotlib v0.10.1.
\begin{tikzpicture}

\definecolor{darkgray176}{RGB}{176,176,176}
\definecolor{gray}{RGB}{128,128,128}
\definecolor{green01270}{RGB}{0,127,0}

\begin{axis}[
width=\figurewidth,%
height=\figureheight,%
axis background/.style={fill=background_color},
tick align=outside,
tick pos=left,
axis line style={white},
tick align=inside,
tick pos=left,
x grid style={white},
y grid style={white},
xmajorgrids,
ymajorgrids,
xlabel={Horizon},
xtick style={color=black},
xtick={0.25,1.25,2.25,3.25,4.25,5.25,6.25,7.25,8.25,9.25},
xticklabels={1,2,3,4,5,6,7,14,21,30},
ytick style={color=black},
xmin=-0.6125, xmax=10.1125,
ymin=0, ymax=29.8449605648958,
xlabel style={font=\tiny},
tick label style={font=\tiny}
]
\draw[draw=gray,fill=blue] (axis cs:-0.125,0) rectangle (axis cs:0.125,21.1569115523988);
\draw[draw=gray,fill=blue] (axis cs:0.875,0) rectangle (axis cs:1.125,23.5500376665929);
\draw[draw=gray,fill=blue] (axis cs:1.875,0) rectangle (axis cs:2.125,24.4596859590331);
\draw[draw=gray,fill=blue] (axis cs:2.875,0) rectangle (axis cs:3.125,25.3025311179923);
\draw[draw=gray,fill=blue] (axis cs:3.875,0) rectangle (axis cs:4.125,25.7392946447281);
\draw[draw=gray,fill=blue] (axis cs:4.875,0) rectangle (axis cs:5.125,25.8577961476045);
\draw[draw=gray,fill=blue] (axis cs:5.875,0) rectangle (axis cs:6.125,26.1182370685187);
\draw[draw=gray,fill=blue] (axis cs:6.875,0) rectangle (axis cs:7.125,27.0243046679054);
\draw[draw=gray,fill=blue] (axis cs:7.875,0) rectangle (axis cs:8.125,27.7866695852108);
\draw[draw=gray,fill=blue] (axis cs:8.875,0) rectangle (axis cs:9.125,28.4237719665674);
\draw[draw=gray,fill=green01270] (axis cs:0.125,0) rectangle (axis cs:0.375,11.7354752059831);

\draw[draw=gray,fill=green01270] (axis cs:1.125,0) rectangle (axis cs:1.375,14.4672232695331);
\draw[draw=gray,fill=green01270] (axis cs:2.125,0) rectangle (axis cs:2.375,15.1059474613183);
\draw[draw=gray,fill=green01270] (axis cs:3.125,0) rectangle (axis cs:3.375,15.8355933150668);
\draw[draw=gray,fill=green01270] (axis cs:4.125,0) rectangle (axis cs:4.375,15.7177185284159);
\draw[draw=gray,fill=green01270] (axis cs:5.125,0) rectangle (axis cs:5.375,16.2355496277881);
\draw[draw=gray,fill=green01270] (axis cs:6.125,0) rectangle (axis cs:6.375,16.1205779992887);
\draw[draw=gray,fill=green01270] (axis cs:7.125,0) rectangle (axis cs:7.375,16.4226456565551);
\draw[draw=gray,fill=green01270] (axis cs:8.125,0) rectangle (axis cs:8.375,17.167740838524);
\draw[draw=gray,fill=green01270] (axis cs:9.125,0) rectangle (axis cs:9.375,17.0291942568183);
\draw[draw=gray,fill=red] (axis cs:0.375,0) rectangle (axis cs:0.625,10.8184214069237);

\draw[draw=gray,fill=red] (axis cs:1.375,0) rectangle (axis cs:1.625,12.4999191375716);
\draw[draw=gray,fill=red] (axis cs:2.375,0) rectangle (axis cs:2.625,13.5104526402534);
\draw[draw=gray,fill=red] (axis cs:3.375,0) rectangle (axis cs:3.625,13.8888512386877);
\draw[draw=gray,fill=red] (axis cs:4.375,0) rectangle (axis cs:4.625,13.6898117257208);
\draw[draw=gray,fill=red] (axis cs:5.375,0) rectangle (axis cs:5.625,14.5583942742105);
\draw[draw=gray,fill=red] (axis cs:6.375,0) rectangle (axis cs:6.625,13.6712080903002);
\draw[draw=gray,fill=red] (axis cs:7.375,0) rectangle (axis cs:7.625,14.2560387120429);
\draw[draw=gray,fill=red] (axis cs:8.375,0) rectangle (axis cs:8.625,14.4738108583833);
\draw[draw=gray,fill=red] (axis cs:9.375,0) rectangle (axis cs:9.625,14.8850707781946);
\end{axis}

\end{tikzpicture}
}
		
	\end{figure}
	\vspace{-2.0em}
	\begin{block}{}
		The Adam+NG optimizer significantly boosts performance, and with LMCGP showing the most improvement, especially for larger horizons.
	\end{block}
	
	
\end{frame}

%\begin{frame}{Model Forecasting}
%	\centering
%	\begin{figure}[htbp]
%	\tiny
%	\setlength\figurewidth{0.55\columnwidth} 
%	\setlength\figureheight{0.5\columnwidth}
%
%	\subfloat[$A$.]{% This file was created with tikzplotlib v0.10.1.
\begin{tikzpicture}

\definecolor{darkgray176}{RGB}{176,176,176}

\begin{axis}[
width=\figurewidth,
height=\figureheight,
axis background/.style={fill=background_color},
axis line style={white},
tick align=inside,
tick pos=left,
ylabel={Contributions (kWh)},
x grid style={white},
xmajorgrids,
xmin=50, xmax=200,
xtick style={color=black},
y grid style={white},
ymajorgrids,
ymin=-3414130.1187411, ymax=12497779.277438,
ytick style={color=black},
ytick={-4000000,-2000000,0,2000000,4000000,6000000,8000000,10000000,12000000,14000000},
]
\path [draw=blue, fill=blue, opacity=0.5]
(axis cs:0,7679984.24705729)
--(axis cs:0,2917489.43104134)
--(axis cs:1,2982475.2004108)
--(axis cs:2,3412409.28949479)
--(axis cs:3,3203869.38989402)
--(axis cs:4,2912864.85847748)
--(axis cs:5,3294361.27001976)
--(axis cs:6,2504446.52562919)
--(axis cs:7,2455496.5359375)
--(axis cs:8,2063730.61385061)
--(axis cs:9,2471527.08738114)
--(axis cs:10,2087061.93914864)
--(axis cs:11,2068758.04016285)
--(axis cs:12,1653180.33918535)
--(axis cs:13,3859736.02751677)
--(axis cs:14,4537897.20831926)
--(axis cs:15,3277620.00039835)
--(axis cs:16,1695284.85875567)
--(axis cs:17,1966959.62473449)
--(axis cs:18,3134134.78771934)
--(axis cs:19,2828709.56750116)
--(axis cs:20,3876449.59478391)
--(axis cs:21,2848093.35253936)
--(axis cs:22,3600507.52062665)
--(axis cs:23,3022347.70935306)
--(axis cs:24,2905676.13474874)
--(axis cs:25,2481211.30106356)
--(axis cs:26,2564268.35362987)
--(axis cs:27,2030429.25208577)
--(axis cs:28,2294638.86030943)
--(axis cs:29,3322096.97925028)
--(axis cs:30,2397133.9884203)
--(axis cs:31,2053047.60767589)
--(axis cs:32,2311818.05999175)
--(axis cs:33,2697391.55598018)
--(axis cs:34,2157240.38803831)
--(axis cs:35,2401106.17088506)
--(axis cs:36,2299723.00322058)
--(axis cs:37,3800216.20377714)
--(axis cs:38,3703461.88262605)
--(axis cs:39,3019779.81502086)
--(axis cs:40,3413201.44657928)
--(axis cs:41,2881025.07959644)
--(axis cs:42,4291389.08062548)
--(axis cs:43,4060142.8237833)
--(axis cs:44,3481997.07579305)
--(axis cs:45,3891492.14582372)
--(axis cs:46,4110304.48482329)
--(axis cs:47,3532827.15842848)
--(axis cs:48,3634901.65677719)
--(axis cs:49,2869282.93787973)
--(axis cs:50,4494561.93350498)
--(axis cs:51,3216161.45753051)
--(axis cs:52,4034747.58598494)
--(axis cs:53,3317984.37464376)
--(axis cs:54,3594001.88822107)
--(axis cs:55,3117094.45043948)
--(axis cs:56,3065348.27955064)
--(axis cs:57,4650171.09004554)
--(axis cs:58,3310559.73327812)
--(axis cs:59,2809454.99209545)
--(axis cs:60,2996747.62458147)
--(axis cs:61,2426189.79359569)
--(axis cs:62,2813148.52216307)
--(axis cs:63,2811657.34525501)
--(axis cs:64,3335729.43511588)
--(axis cs:65,2537340.37919239)
--(axis cs:66,2164190.19690098)
--(axis cs:67,1654852.02503031)
--(axis cs:68,1399690.08274928)
--(axis cs:69,1237163.36781349)
--(axis cs:70,1209226.19800315)
--(axis cs:71,1436966.65871377)
--(axis cs:72,3513173.94671417)
--(axis cs:73,2820010.91609189)
--(axis cs:74,2353687.92180664)
--(axis cs:75,1902465.34870207)
--(axis cs:76,1624368.23086607)
--(axis cs:77,2095649.46641945)
--(axis cs:78,1574748.80152594)
--(axis cs:79,1683137.36360548)
--(axis cs:80,1229993.81825738)
--(axis cs:81,1066761.80772511)
--(axis cs:82,3111364.08941993)
--(axis cs:83,1701936.78221748)
--(axis cs:84,1831707.83885457)
--(axis cs:85,2562040.28739348)
--(axis cs:86,3113548.58271729)
--(axis cs:87,1877180.53775377)
--(axis cs:88,2281224.20268079)
--(axis cs:89,2586727.81812645)
--(axis cs:90,1952498.79589295)
--(axis cs:91,1802213.10017979)
--(axis cs:92,1288777.44523775)
--(axis cs:93,1105400.53027211)
--(axis cs:94,1120194.97881654)
--(axis cs:95,1275151.52814121)
--(axis cs:96,935119.511475899)
--(axis cs:97,2125290.88469931)
--(axis cs:98,2730128.1621721)
--(axis cs:99,2341227.18780366)
--(axis cs:100,1493226.36825061)
--(axis cs:101,1161461.47985358)
--(axis cs:102,1129224.43332816)
--(axis cs:103,1059155.30887248)
--(axis cs:104,1341415.34656189)
--(axis cs:105,983744.755914505)
--(axis cs:106,1664866.5549998)
--(axis cs:107,2420746.70900163)
--(axis cs:108,1655608.96741228)
--(axis cs:109,1695656.71105454)
--(axis cs:110,1453713.18470758)
--(axis cs:111,605962.766181772)
--(axis cs:112,1163845.7545825)
--(axis cs:113,577191.472753175)
--(axis cs:114,1516529.69718251)
--(axis cs:115,3358582.923284)
--(axis cs:116,2702482.98495262)
--(axis cs:117,2015092.41675752)
--(axis cs:118,1997318.90056699)
--(axis cs:119,1965145.60743524)
--(axis cs:120,2010626.58147187)
--(axis cs:121,1299869.15564852)
--(axis cs:122,1532434.83647141)
--(axis cs:123,2135592.4581391)
--(axis cs:124,2314684.70553052)
--(axis cs:125,2846091.68720357)
--(axis cs:126,2305832.55878519)
--(axis cs:127,2467803.82965228)
--(axis cs:128,1921492.81952138)
--(axis cs:129,2458008.96321451)
--(axis cs:130,2325944.66787883)
--(axis cs:131,2429545.75979456)
--(axis cs:132,1815136.82704659)
--(axis cs:133,2264282.25902703)
--(axis cs:134,2786897.04742897)
--(axis cs:135,1242794.96059963)
--(axis cs:136,1365941.21273608)
--(axis cs:137,1553723.35083154)
--(axis cs:138,1289906.37631477)
--(axis cs:139,788932.163283331)
--(axis cs:140,940767.004119291)
--(axis cs:141,1105773.15870677)
--(axis cs:142,1851531.81487777)
--(axis cs:143,1980425.71421969)
--(axis cs:144,2270204.48383211)
--(axis cs:145,2405585.91819742)
--(axis cs:146,1675951.01406862)
--(axis cs:147,1429079.19995716)
--(axis cs:148,3332519.62155422)
--(axis cs:149,1970398.27253148)
--(axis cs:150,2212308.53497331)
--(axis cs:151,731321.690293832)
--(axis cs:152,742401.862329931)
--(axis cs:153,696481.653166119)
--(axis cs:154,844265.149900103)
--(axis cs:155,965383.799802638)
--(axis cs:156,1448044.83825602)
--(axis cs:157,1038315.64862774)
--(axis cs:158,2121144.6338735)
--(axis cs:159,1381243.99103028)
--(axis cs:160,1617778.68211952)
--(axis cs:161,1290676.22403618)
--(axis cs:162,1594225.52892163)
--(axis cs:163,1180084.83564248)
--(axis cs:164,1263719.17512619)
--(axis cs:165,1232759.83590769)
--(axis cs:166,2241237.79852954)
--(axis cs:167,2547958.48127936)
--(axis cs:168,3979832.30590487)
--(axis cs:169,2684384.55364342)
--(axis cs:170,3287709.67916509)
--(axis cs:171,3432549.29471511)
--(axis cs:172,3448630.89848808)
--(axis cs:173,3017009.45836024)
--(axis cs:174,2746960.6610996)
--(axis cs:175,2735298.36892186)
--(axis cs:176,3064884.2230575)
--(axis cs:177,3645100.07441874)
--(axis cs:178,2753781.31003333)
--(axis cs:179,2463743.76010546)
--(axis cs:180,1904047.78126948)
--(axis cs:181,2267442.74088608)
--(axis cs:182,2062314.14372937)
--(axis cs:183,2132300.28432497)
--(axis cs:184,3747428.53643517)
--(axis cs:185,4033076.96487381)
--(axis cs:186,3605624.41010253)
--(axis cs:187,2356068.7823238)
--(axis cs:188,2976651.89941397)
--(axis cs:189,2440791.2162352)
--(axis cs:190,2306345.2306331)
--(axis cs:191,3351800.0534161)
--(axis cs:192,2261309.62684204)
--(axis cs:193,1819279.53589965)
--(axis cs:194,1464569.85686153)
--(axis cs:195,1698819.78271284)
--(axis cs:196,2188578.81351247)
--(axis cs:197,2694323.45912929)
--(axis cs:198,3474396.36982183)
--(axis cs:199,2938411.45915798)
--(axis cs:200,3869974.85491382)
--(axis cs:201,3120349.86858124)
--(axis cs:202,2683422.49830802)
--(axis cs:203,3724868.45667511)
--(axis cs:204,3245032.53837236)
--(axis cs:205,2132785.10054987)
--(axis cs:206,2974878.85806041)
--(axis cs:207,3196914.03575752)
--(axis cs:208,4199883.83119749)
--(axis cs:209,3451290.728535)
--(axis cs:210,3105455.07919413)
--(axis cs:211,3128299.23365977)
--(axis cs:212,2433218.17485911)
--(axis cs:213,2769390.67800514)
--(axis cs:214,3050205.51425521)
--(axis cs:215,2879942.76903341)
--(axis cs:216,2537096.49545219)
--(axis cs:217,1821998.67525135)
--(axis cs:218,1294846.54573787)
--(axis cs:219,2022770.21523863)
--(axis cs:220,1550706.08933325)
--(axis cs:221,689285.743333825)
--(axis cs:222,1123458.94219666)
--(axis cs:223,2091519.73641097)
--(axis cs:224,2221600.6607308)
--(axis cs:225,2588898.8711105)
--(axis cs:226,2063643.09785299)
--(axis cs:227,1802367.97327628)
--(axis cs:228,1249671.50296711)
--(axis cs:229,736612.423290445)
--(axis cs:230,102494.637866587)
--(axis cs:231,75155.2371517671)
--(axis cs:232,-145397.268552344)
--(axis cs:233,494222.041646762)
--(axis cs:234,616546.236871024)
--(axis cs:235,342706.927493482)
--(axis cs:236,159769.545503355)
--(axis cs:237,112298.063791818)
--(axis cs:238,-391148.456104159)
--(axis cs:239,-504510.593138072)
--(axis cs:240,372555.188068123)
--(axis cs:241,922619.762367975)
--(axis cs:242,-498242.310442616)
--(axis cs:243,-834211.405757753)
--(axis cs:244,-544254.336859646)
--(axis cs:245,-261888.349011427)
--(axis cs:246,257954.199219426)
--(axis cs:247,-196752.683902787)
--(axis cs:248,-1064551.72945586)
--(axis cs:249,-1060880.93599958)
--(axis cs:250,-788384.100683442)
--(axis cs:251,-1418055.27242892)
--(axis cs:252,-1190275.53927862)
--(axis cs:253,-978849.892340882)
--(axis cs:254,-817894.323294024)
--(axis cs:255,-58360.6458918303)
--(axis cs:256,-418422.068856314)
--(axis cs:257,-286019.019800927)
--(axis cs:258,-675867.655906836)
--(axis cs:259,-566296.756191047)
--(axis cs:260,-1617590.49495109)
--(axis cs:261,-1100575.73697092)
--(axis cs:262,-768917.460955109)
--(axis cs:263,-1671845.75501035)
--(axis cs:264,-1352822.13736866)
--(axis cs:265,-1315059.16599414)
--(axis cs:266,-1200662.26779044)
--(axis cs:267,-1192114.44657334)
--(axis cs:268,-1009209.38367926)
--(axis cs:269,-251308.632440381)
--(axis cs:270,269169.755357018)
--(axis cs:271,-961498.785289288)
--(axis cs:272,-1024219.01019484)
--(axis cs:273,1115751.80977258)
--(axis cs:274,1077096.35865129)
--(axis cs:275,2807879.67676026)
--(axis cs:276,1773298.47465474)
--(axis cs:277,3209451.75713373)
--(axis cs:278,1967719.89400045)
--(axis cs:279,2417301.04230861)
--(axis cs:280,2822802.70517876)
--(axis cs:281,3081151.00053684)
--(axis cs:282,2550851.12953923)
--(axis cs:283,1682341.6431942)
--(axis cs:284,2367219.44165324)
--(axis cs:285,2592886.59713286)
--(axis cs:286,2225712.97144436)
--(axis cs:287,2481842.45757226)
--(axis cs:288,2055939.50134904)
--(axis cs:289,2560241.14735253)
--(axis cs:290,2011428.63100231)
--(axis cs:291,1592480.28189806)
--(axis cs:292,2724382.7163194)
--(axis cs:293,2054264.09948943)
--(axis cs:294,1891202.54059085)
--(axis cs:295,1501417.80656133)
--(axis cs:296,1293635.38956902)
--(axis cs:297,998888.85364853)
--(axis cs:298,1687795.56441785)
--(axis cs:299,1156354.96763287)
--(axis cs:300,2045969.09083358)
--(axis cs:301,1309088.90913885)
--(axis cs:302,1346530.20706875)
--(axis cs:303,583675.187761881)
--(axis cs:304,423265.493671787)
--(axis cs:305,444675.38511499)
--(axis cs:306,1027462.28289674)
--(axis cs:307,1276332.78455816)
--(axis cs:308,667446.147486211)
--(axis cs:309,441178.537505197)
--(axis cs:310,361036.530339733)
--(axis cs:311,359848.942644017)
--(axis cs:312,388001.013866929)
--(axis cs:313,2120319.55307975)
--(axis cs:314,3532925.51091316)
--(axis cs:315,1668211.78606755)
--(axis cs:316,1189355.03839513)
--(axis cs:317,774533.518755079)
--(axis cs:318,484053.943311654)
--(axis cs:319,431951.171165565)
--(axis cs:320,452226.455621021)
--(axis cs:321,646648.558764158)
--(axis cs:322,656320.749266473)
--(axis cs:323,824994.071183939)
--(axis cs:324,1664474.82067434)
--(axis cs:325,1420105.02429988)
--(axis cs:326,1443991.67860116)
--(axis cs:327,1943207.27550379)
--(axis cs:328,672602.616723479)
--(axis cs:329,59565.213614882)
--(axis cs:330,972909.308632754)
--(axis cs:331,1928137.10569285)
--(axis cs:332,1367180.63002391)
--(axis cs:333,2496436.94992171)
--(axis cs:334,1646695.27865691)
--(axis cs:335,1746751.84391372)
--(axis cs:336,1716044.32219471)
--(axis cs:337,1660888.20296363)
--(axis cs:338,1730947.72984214)
--(axis cs:339,1493665.66778162)
--(axis cs:340,1290860.94633602)
--(axis cs:341,1674580.04091971)
--(axis cs:342,1002324.79565706)
--(axis cs:343,801582.916510339)
--(axis cs:344,1853343.79561231)
--(axis cs:345,1590230.56587636)
--(axis cs:346,1381269.06669185)
--(axis cs:347,1507060.94565381)
--(axis cs:348,2325146.24438242)
--(axis cs:349,2345382.89679721)
--(axis cs:350,2174982.0761801)
--(axis cs:351,2190262.48084903)
--(axis cs:352,2014551.65643376)
--(axis cs:353,2645309.35506452)
--(axis cs:354,2125941.13618584)
--(axis cs:355,2220225.50154373)
--(axis cs:356,2704948.47310419)
--(axis cs:357,3159007.19823957)
--(axis cs:358,2433518.07616621)
--(axis cs:359,2059444.41484432)
--(axis cs:360,2064578.35051255)
--(axis cs:361,2151582.14322951)
--(axis cs:362,2693986.49320545)
--(axis cs:363,3315718.48649008)
--(axis cs:364,2856572.253689)
--(axis cs:365,3439388.09630195)
--(axis cs:366,1940223.42255239)
--(axis cs:367,2272070.3962789)
--(axis cs:368,3117417.83929134)
--(axis cs:369,2332994.13105902)
--(axis cs:370,3146596.38026869)
--(axis cs:371,3468065.85506536)
--(axis cs:372,4086706.28704517)
--(axis cs:373,3759930.65356188)
--(axis cs:374,3391430.10968748)
--(axis cs:375,4034443.79796314)
--(axis cs:376,3959819.23371528)
--(axis cs:377,4235317.34346317)
--(axis cs:378,4137485.1030638)
--(axis cs:379,3496015.34363113)
--(axis cs:380,2181152.63113452)
--(axis cs:381,3092585.75026244)
--(axis cs:382,3868144.3505766)
--(axis cs:383,4011925.16037722)
--(axis cs:384,3172547.53836776)
--(axis cs:385,3890018.75265465)
--(axis cs:386,3772780.42133329)
--(axis cs:387,3254358.07719239)
--(axis cs:388,3362255.26897569)
--(axis cs:389,3492707.89090529)
--(axis cs:390,3166640.42813365)
--(axis cs:391,2892856.89934353)
--(axis cs:392,4092752.53567151)
--(axis cs:393,3771200.82051161)
--(axis cs:394,4646147.65032412)
--(axis cs:395,4364989.99027751)
--(axis cs:396,3506508.07958188)
--(axis cs:397,3323581.45236484)
--(axis cs:398,2812454.85617854)
--(axis cs:399,3998005.03209272)
--(axis cs:400,3582073.38247949)
--(axis cs:401,3763943.21289781)
--(axis cs:402,2834831.10336599)
--(axis cs:403,3506516.10781756)
--(axis cs:404,4311980.64557083)
--(axis cs:405,4283312.42445459)
--(axis cs:406,4042480.6427551)
--(axis cs:407,3340675.95059722)
--(axis cs:408,3279738.95930303)
--(axis cs:409,4198697.81156371)
--(axis cs:410,3524230.56256518)
--(axis cs:411,3050900.49603025)
--(axis cs:412,3960144.5076375)
--(axis cs:413,3613529.19174047)
--(axis cs:414,4048265.56227727)
--(axis cs:415,4317130.13664659)
--(axis cs:416,3474861.10805085)
--(axis cs:417,3519516.34854784)
--(axis cs:418,3138067.57331856)
--(axis cs:419,2803690.17493645)
--(axis cs:420,2348548.99812738)
--(axis cs:421,2263872.12187547)
--(axis cs:422,2237429.69808405)
--(axis cs:423,2851102.42285992)
--(axis cs:424,2431060.61794783)
--(axis cs:425,2043255.3426351)
--(axis cs:426,1875798.78806705)
--(axis cs:427,1980876.27960109)
--(axis cs:428,1074587.97042321)
--(axis cs:429,1194882.20566753)
--(axis cs:430,1448647.15581405)
--(axis cs:431,3885615.18217673)
--(axis cs:432,2395431.24552416)
--(axis cs:433,1726188.80308921)
--(axis cs:434,1518835.88124687)
--(axis cs:435,1369885.93183798)
--(axis cs:436,3963059.87060979)
--(axis cs:437,3867796.84753531)
--(axis cs:438,3174165.87244418)
--(axis cs:439,4290185.78046883)
--(axis cs:440,3256120.16507369)
--(axis cs:440,8022356.31003915)
--(axis cs:440,8022356.31003915)
--(axis cs:439,9691203.45842095)
--(axis cs:438,7866419.53012419)
--(axis cs:437,8710386.69306469)
--(axis cs:436,8729814.6984532)
--(axis cs:435,6064181.06058305)
--(axis cs:434,6210636.1558785)
--(axis cs:433,6410236.85286229)
--(axis cs:432,7076933.46019765)
--(axis cs:431,8597866.44141556)
--(axis cs:430,6147228.07995845)
--(axis cs:429,5897067.85148066)
--(axis cs:428,5855273.49397249)
--(axis cs:427,6691312.28447242)
--(axis cs:426,6595059.88172283)
--(axis cs:425,6820851.89778449)
--(axis cs:424,7161416.94647428)
--(axis cs:423,7551280.76531755)
--(axis cs:422,6948631.19252782)
--(axis cs:421,7048335.28621586)
--(axis cs:420,7133489.86256111)
--(axis cs:419,7541349.78049977)
--(axis cs:418,7850287.46441707)
--(axis cs:417,8258372.3385759)
--(axis cs:416,8176402.06116528)
--(axis cs:415,9038374.97633848)
--(axis cs:414,8740661.28399606)
--(axis cs:413,8308951.35332157)
--(axis cs:412,8741203.63788666)
--(axis cs:411,7745851.07282461)
--(axis cs:410,8210095.86799942)
--(axis cs:409,8970816.99739047)
--(axis cs:408,7955101.18326571)
--(axis cs:407,8024635.58924367)
--(axis cs:406,8746584.26977279)
--(axis cs:405,9263590.65680052)
--(axis cs:404,9163335.29814785)
--(axis cs:403,8241139.79166751)
--(axis cs:402,7536376.66274259)
--(axis cs:401,8474613.8239344)
--(axis cs:400,8267840.11074639)
--(axis cs:399,8690879.77504601)
--(axis cs:398,7500819.66104002)
--(axis cs:397,8028173.07387978)
--(axis cs:396,8344917.75035064)
--(axis cs:395,9512468.64800802)
--(axis cs:394,9672833.67191725)
--(axis cs:393,8527351.07688676)
--(axis cs:392,8829111.93168207)
--(axis cs:391,7563055.79117382)
--(axis cs:390,7882167.05603462)
--(axis cs:389,8171488.55926914)
--(axis cs:388,8044000.56481757)
--(axis cs:387,7980366.180205)
--(axis cs:386,8470653.80442838)
--(axis cs:385,8595415.37318593)
--(axis cs:384,7901820.00849699)
--(axis cs:383,8741656.22342907)
--(axis cs:382,8592232.92436555)
--(axis cs:381,7800415.71810708)
--(axis cs:380,7110305.73232726)
--(axis cs:379,8460103.66663251)
--(axis cs:378,9200822.92396957)
--(axis cs:377,9124663.08237261)
--(axis cs:376,8864320.06526396)
--(axis cs:375,8750093.95695064)
--(axis cs:374,8082852.41242962)
--(axis cs:373,8453698.03346741)
--(axis cs:372,8839859.85695364)
--(axis cs:371,8185143.61466612)
--(axis cs:370,7821365.91307012)
--(axis cs:369,7017415.705728)
--(axis cs:368,7807572.4543371)
--(axis cs:367,6998019.46415688)
--(axis cs:366,6845152.37655341)
--(axis cs:365,8136843.28837586)
--(axis cs:364,7601084.20756068)
--(axis cs:363,8027974.65681743)
--(axis cs:362,7529218.4817143)
--(axis cs:361,6877212.73448065)
--(axis cs:360,6851887.02627603)
--(axis cs:359,6800988.46511969)
--(axis cs:358,7276617.9784904)
--(axis cs:357,8127833.33000954)
--(axis cs:356,7431637.60907517)
--(axis cs:355,6922444.82617644)
--(axis cs:354,6874955.51636513)
--(axis cs:353,7468384.46957303)
--(axis cs:352,6773385.30364753)
--(axis cs:351,6937483.15904326)
--(axis cs:350,6997257.5453839)
--(axis cs:349,7161727.86128985)
--(axis cs:348,7105470.07659208)
--(axis cs:347,6223681.96220715)
--(axis cs:346,6084913.93935312)
--(axis cs:345,6272301.29539062)
--(axis cs:344,6585660.21973364)
--(axis cs:343,5532374.23038601)
--(axis cs:342,5872041.8784377)
--(axis cs:341,6485634.62363162)
--(axis cs:340,5988356.17167386)
--(axis cs:339,6196712.2547643)
--(axis cs:338,6480575.78146034)
--(axis cs:337,6446053.10374163)
--(axis cs:336,6574919.70291316)
--(axis cs:335,6441446.09879662)
--(axis cs:334,6352856.7679818)
--(axis cs:333,7294094.45899704)
--(axis cs:332,6147059.7659915)
--(axis cs:331,6681151.99913654)
--(axis cs:330,5749354.00157312)
--(axis cs:329,4902034.73928589)
--(axis cs:328,5575140.32927956)
--(axis cs:327,6724172.21912285)
--(axis cs:326,6185848.04308381)
--(axis cs:325,6136087.94975728)
--(axis cs:324,6474423.21248024)
--(axis cs:323,5628376.34755757)
--(axis cs:322,5586059.78764098)
--(axis cs:321,5414616.19288484)
--(axis cs:320,5213347.95859968)
--(axis cs:319,5140517.35701155)
--(axis cs:318,5225350.38348705)
--(axis cs:317,5489158.71378632)
--(axis cs:316,5935537.49263528)
--(axis cs:315,7029171.1370364)
--(axis cs:314,8539768.54562119)
--(axis cs:313,6946513.07394425)
--(axis cs:312,5099785.40781596)
--(axis cs:311,5100793.0596111)
--(axis cs:310,5074612.0960969)
--(axis cs:309,5128503.53265326)
--(axis cs:308,5362234.55971839)
--(axis cs:307,6030681.9891503)
--(axis cs:306,5720210.09157174)
--(axis cs:305,5137178.88732451)
--(axis cs:304,5248585.24454535)
--(axis cs:303,5395877.84086812)
--(axis cs:302,6063216.58703186)
--(axis cs:301,6021727.77580961)
--(axis cs:300,6809391.43228823)
--(axis cs:299,5937227.24141796)
--(axis cs:298,6538314.61786301)
--(axis cs:297,5884606.19087736)
--(axis cs:296,6004482.08810512)
--(axis cs:295,6180307.90881648)
--(axis cs:294,6615756.21228378)
--(axis cs:293,6862426.82672978)
--(axis cs:292,7604555.75380777)
--(axis cs:291,6342216.36659028)
--(axis cs:290,6710358.89109285)
--(axis cs:289,7490808.33116744)
--(axis cs:288,6752426.02442935)
--(axis cs:287,7214057.49616407)
--(axis cs:286,6987570.7206934)
--(axis cs:285,7351296.7302903)
--(axis cs:284,7270179.03615997)
--(axis cs:283,6473017.82257516)
--(axis cs:282,7313202.84727305)
--(axis cs:281,7868196.99427038)
--(axis cs:280,7631796.17498143)
--(axis cs:279,7358026.72616718)
--(axis cs:278,6895203.86710297)
--(axis cs:277,8039495.4315072)
--(axis cs:276,6517470.84407483)
--(axis cs:275,7600518.37380187)
--(axis cs:274,5862226.22870356)
--(axis cs:273,5918627.17450617)
--(axis cs:272,3799402.5583802)
--(axis cs:271,4144691.39833375)
--(axis cs:270,4997222.53172352)
--(axis cs:269,4474377.23469855)
--(axis cs:268,3763895.7140109)
--(axis cs:267,3569744.86627753)
--(axis cs:266,3570175.51008694)
--(axis cs:265,3471600.53079333)
--(axis cs:264,3568839.35642576)
--(axis cs:263,3103638.5953311)
--(axis cs:262,3979428.0755234)
--(axis cs:261,3660072.03999091)
--(axis cs:260,3582017.31357931)
--(axis cs:259,4215693.98740307)
--(axis cs:258,4058865.95164573)
--(axis cs:257,4555947.83435833)
--(axis cs:256,4363657.76015522)
--(axis cs:255,4710584.64527587)
--(axis cs:254,4089524.97193181)
--(axis cs:253,4177425.72182261)
--(axis cs:252,3879893.12886764)
--(axis cs:251,3564783.85836179)
--(axis cs:250,4057534.5185709)
--(axis cs:249,3810589.1321966)
--(axis cs:248,4095326.64372283)
--(axis cs:247,4945073.40193915)
--(axis cs:246,5073915.87791109)
--(axis cs:245,4469787.05552874)
--(axis cs:244,4225886.88042692)
--(axis cs:243,3994193.8450365)
--(axis cs:242,4336683.66130794)
--(axis cs:241,5979827.12068389)
--(axis cs:240,5125492.40553625)
--(axis cs:239,4645917.94904521)
--(axis cs:238,4540225.07378281)
--(axis cs:237,4809571.21403723)
--(axis cs:236,4855362.01881906)
--(axis cs:235,5030360.20909669)
--(axis cs:234,5300132.03578323)
--(axis cs:233,5175127.29155159)
--(axis cs:232,4727029.98233835)
--(axis cs:231,4797343.38275115)
--(axis cs:230,4816296.87730774)
--(axis cs:229,5473531.17287031)
--(axis cs:228,5953622.24253577)
--(axis cs:227,6541505.17458394)
--(axis cs:226,6845008.74534634)
--(axis cs:225,7317777.98370203)
--(axis cs:224,7477608.63668965)
--(axis cs:223,6829629.63183662)
--(axis cs:222,5982290.0699364)
--(axis cs:221,5861995.45939569)
--(axis cs:220,6972680.34843589)
--(axis cs:219,6942842.35688233)
--(axis cs:218,6084359.90535575)
--(axis cs:217,6607539.83881067)
--(axis cs:216,7312393.95659604)
--(axis cs:215,7900605.9830517)
--(axis cs:214,8028995.86076962)
--(axis cs:213,7839179.97006602)
--(axis cs:212,7604756.13279482)
--(axis cs:211,8608018.08291848)
--(axis cs:210,7916329.93450038)
--(axis cs:209,8467271.64633882)
--(axis cs:208,9465953.67527066)
--(axis cs:207,7973369.03780106)
--(axis cs:206,7867600.79482359)
--(axis cs:205,7013974.55411695)
--(axis cs:204,8058118.41554364)
--(axis cs:203,8566881.28533567)
--(axis cs:202,7762703.9048781)
--(axis cs:201,8506678.63455296)
--(axis cs:200,8999991.03141956)
--(axis cs:199,7834257.12383957)
--(axis cs:198,8293559.26307454)
--(axis cs:197,7671251.96914194)
--(axis cs:196,7018499.83210553)
--(axis cs:195,6708084.48817406)
--(axis cs:194,7092479.59374169)
--(axis cs:193,6623697.25765675)
--(axis cs:192,7128797.73804549)
--(axis cs:191,8122079.22580202)
--(axis cs:190,7005831.45343372)
--(axis cs:189,7142485.30690615)
--(axis cs:188,7734119.90854094)
--(axis cs:187,7667217.27026429)
--(axis cs:186,8760889.25525625)
--(axis cs:185,8969829.48438775)
--(axis cs:184,8696524.60397946)
--(axis cs:183,6830897.01066938)
--(axis cs:182,6848667.03451863)
--(axis cs:181,7030307.72378642)
--(axis cs:180,6666893.86124969)
--(axis cs:179,7413270.73944194)
--(axis cs:178,7608298.51197949)
--(axis cs:177,8454965.59160259)
--(axis cs:176,7851560.0645369)
--(axis cs:175,7539087.83161556)
--(axis cs:174,7597022.79454206)
--(axis cs:173,8111629.08014248)
--(axis cs:172,8460508.1652476)
--(axis cs:171,8269135.93007002)
--(axis cs:170,8030037.5577928)
--(axis cs:169,7843572.20304232)
--(axis cs:168,8855513.83080661)
--(axis cs:167,7277434.23525222)
--(axis cs:166,6981310.7156479)
--(axis cs:165,5949064.48217294)
--(axis cs:164,5941589.08559362)
--(axis cs:163,5913077.36251603)
--(axis cs:162,6456745.70609371)
--(axis cs:161,5974388.27070282)
--(axis cs:160,6348246.07492212)
--(axis cs:159,6055094.9635329)
--(axis cs:158,6852948.12814345)
--(axis cs:157,5802820.07066115)
--(axis cs:156,6360101.54756293)
--(axis cs:155,5688410.40971653)
--(axis cs:154,5617267.20853434)
--(axis cs:153,5517640.80548702)
--(axis cs:152,5434638.04990845)
--(axis cs:151,5821721.89087053)
--(axis cs:150,6954165.21558927)
--(axis cs:149,6673093.01398498)
--(axis cs:148,8266053.16671285)
--(axis cs:147,6133498.37307363)
--(axis cs:146,6394821.79600721)
--(axis cs:145,7337268.85564099)
--(axis cs:144,7104010.23216164)
--(axis cs:143,6690035.52343236)
--(axis cs:142,6580301.95706702)
--(axis cs:141,5857267.6137795)
--(axis cs:140,5708734.07977597)
--(axis cs:139,5552060.95484431)
--(axis cs:138,6100646.02124984)
--(axis cs:137,6284286.97887067)
--(axis cs:136,6076872.34192094)
--(axis cs:135,6057978.31506259)
--(axis cs:134,7633551.11817586)
--(axis cs:133,7005642.82215158)
--(axis cs:132,6871682.35362213)
--(axis cs:131,7195455.61239436)
--(axis cs:130,7144874.59020811)
--(axis cs:129,7264017.00082095)
--(axis cs:128,6808651.9720635)
--(axis cs:127,7275645.98244101)
--(axis cs:126,7273809.96141489)
--(axis cs:125,7656584.12289611)
--(axis cs:124,7086409.23081471)
--(axis cs:123,7316064.61168085)
--(axis cs:122,6282566.86410872)
--(axis cs:121,6108860.49712517)
--(axis cs:120,6960947.94839998)
--(axis cs:119,6704489.41414463)
--(axis cs:118,6719501.71596854)
--(axis cs:117,6715890.0598456)
--(axis cs:116,7702009.8260955)
--(axis cs:115,8094022.27309296)
--(axis cs:114,6224478.86133071)
--(axis cs:113,5403289.30444477)
--(axis cs:112,5921970.01045392)
--(axis cs:111,5773587.5852807)
--(axis cs:110,6197216.15960382)
--(axis cs:109,6455005.24173789)
--(axis cs:108,6363503.93425422)
--(axis cs:107,7253276.86750313)
--(axis cs:106,6537754.5198939)
--(axis cs:105,5801660.3046735)
--(axis cs:104,6037670.00341317)
--(axis cs:103,5773836.71712868)
--(axis cs:102,5867265.40550506)
--(axis cs:101,5872362.67622323)
--(axis cs:100,6228839.39756818)
--(axis cs:99,7054917.96675261)
--(axis cs:98,7431220.63180688)
--(axis cs:97,6809495.36445855)
--(axis cs:96,5668234.35987965)
--(axis cs:95,6071114.35838793)
--(axis cs:94,5829748.90535881)
--(axis cs:93,5804871.37892263)
--(axis cs:92,6000743.31709687)
--(axis cs:91,6494569.94744923)
--(axis cs:90,6698150.86011434)
--(axis cs:89,7308882.20505177)
--(axis cs:88,6987550.6117297)
--(axis cs:87,6581850.45899539)
--(axis cs:86,7855726.59571309)
--(axis cs:85,7394539.45379704)
--(axis cs:84,6554035.45594732)
--(axis cs:83,6394975.67781797)
--(axis cs:82,7850606.08262153)
--(axis cs:81,5779838.93524948)
--(axis cs:80,5949856.55831945)
--(axis cs:79,6383454.03809912)
--(axis cs:78,6271410.28224737)
--(axis cs:77,6791990.24153959)
--(axis cs:76,6308177.97159257)
--(axis cs:75,6601279.44819712)
--(axis cs:74,7085841.84714626)
--(axis cs:73,7532699.33918604)
--(axis cs:72,8629621.44408564)
--(axis cs:71,6124766.7654831)
--(axis cs:70,5922364.37606678)
--(axis cs:69,5944655.5857685)
--(axis cs:68,6092849.17724834)
--(axis cs:67,6350420.35214876)
--(axis cs:66,6853546.93339051)
--(axis cs:65,7223705.95219183)
--(axis cs:64,8042296.51818)
--(axis cs:63,7511761.62983328)
--(axis cs:62,7494078.48788662)
--(axis cs:61,7159752.87235976)
--(axis cs:60,7979838.39618029)
--(axis cs:59,7575224.59594134)
--(axis cs:58,8015850.69312602)
--(axis cs:57,9405281.85405485)
--(axis cs:56,7820355.94316565)
--(axis cs:55,7832005.57964444)
--(axis cs:54,8350857.76497807)
--(axis cs:53,8065462.53297554)
--(axis cs:52,9052961.64103819)
--(axis cs:51,7950741.33034869)
--(axis cs:50,9233994.24260177)
--(axis cs:49,7605572.95366764)
--(axis cs:48,8337490.74567383)
--(axis cs:47,8221615.13451748)
--(axis cs:46,8830784.18158457)
--(axis cs:45,8608553.29241206)
--(axis cs:44,8720181.54875344)
--(axis cs:43,8975446.78313228)
--(axis cs:42,9106663.87325399)
--(axis cs:41,7552354.96632876)
--(axis cs:40,8254323.21389199)
--(axis cs:39,7697260.87393141)
--(axis cs:38,8387255.9309216)
--(axis cs:37,8513166.48357755)
--(axis cs:36,6996659.79328977)
--(axis cs:35,7071389.41326878)
--(axis cs:34,6827846.4819399)
--(axis cs:33,7441741.47343554)
--(axis cs:32,6982897.01352642)
--(axis cs:31,6723379.5961083)
--(axis cs:30,7082756.28995205)
--(axis cs:29,8024871.52077783)
--(axis cs:28,6995268.95702948)
--(axis cs:27,6743084.58716952)
--(axis cs:26,7259051.64086478)
--(axis cs:25,7184740.06182591)
--(axis cs:24,7602442.59011252)
--(axis cs:23,7712469.15797597)
--(axis cs:22,8302425.66899976)
--(axis cs:21,7715006.54235887)
--(axis cs:20,8728141.64079677)
--(axis cs:19,7556424.53620037)
--(axis cs:18,7926601.75760115)
--(axis cs:17,6800967.931237)
--(axis cs:16,7766133.5586506)
--(axis cs:15,7986714.32678255)
--(axis cs:14,9321241.78467989)
--(axis cs:13,8641134.16753677)
--(axis cs:12,6389653.07323644)
--(axis cs:11,6749407.69242221)
--(axis cs:10,6827428.59381815)
--(axis cs:9,7232825.07862385)
--(axis cs:8,6953409.42891104)
--(axis cs:7,7326635.23231807)
--(axis cs:6,7414173.69228216)
--(axis cs:5,8129876.40119676)
--(axis cs:4,8144611.9796668)
--(axis cs:3,8093423.58946947)
--(axis cs:2,8324253.76512555)
--(axis cs:1,8271244.11171538)
--(axis cs:0,7679984.24705729)
--cycle;

\addplot [only marks, red, mark=asterisk, mark size=1.2]
table {%
0 7149100
1 5658700
2 5278900
3 6046300
4 5661300
5 5928400
6 5221600
7 4428300
8 4796900
9 4144900
10 3767800
11 3453700
12 7709600
13 7709600
14 5534200
15 5959400
16 4790300
17 4820800
18 4843600
19 7379500
20 7029500
21 6528800
22 5612900
23 5680700
24 4828500
25 4393800
26 4046200
27 4235000
28 5846400
29 4380600
30 3798100
31 4207000
32 4958500
33 3948900
34 4204500
35 4362400
36 6925600
37 6458900
38 5364400
39 5995100
40 4883700
41 8473900
42 7491200
43 7058500
44 6415100
45 6576600
46 5782900
47 5479600
48 4524700
49 7920800
50 5019300
51 8852200
52 5798500
53 5956900
54 5249100
55 6044900
56 8281700
57 5647300
58 4850400
59 7208100
60 4797400
61 5228600
62 5174700
63 6327900
64 4725400
65 3995200
66 3364900
67 3007100
68 2773200
69 2709100
70 2954100
71 10084800
72 5496600
73 4964200
74 4087900
75 3579900
76 4194200
77 3393200
78 3388600
79 2876100
80 2652000
81 6194500
82 3557200
83 3872300
84 5395800
85 6194500
86 3897200
87 4466600
88 5014200
89 4057300
90 3780500
91 3121200
92 3035300
93 2766900
94 2847700
95 2480800
96 4124200
97 4759100
98 4759100
99 3726000
100 3217600
101 3130400
102 3074100
103 2881200
104 2418100
105 2747900
106 4268200
107 3733900
108 3835400
109 3862200
110 3172400
111 3248600
112 2834700
113 3209700
114 6202700
115 7724000
116 4160700
117 3946100
118 4168600
119 4587900
120 3794300
121 3508300
122 4960300
123 5194600
124 4757200
125 5037300
126 4577300
127 4343500
128 4838900
129 5547100
130 4331800
131 4591600
132 3884200
133 4230800
134 3919900
135 3447000
136 3375500
137 3653600
138 2934400
139 2704200
140 2984100
141 4224400
142 4231700
143 3532700
144 5728500
145 3739000
146 3302000
147 6888800
148 4376700
149 4427800
150 3371200
151 3079700
152 3038200
153 3045500
154 2783900
155 4552900
156 3226900
157 5103700
158 3608000
159 3633800
160 3034400
161 3512000
162 2962500
163 3088900
164 2882200
165 4026900
166 4617400
167 5124600
168 7773700
169 5641200
170 6349000
171 6525700
172 5305400
173 4846800
174 5394400
175 5407800
176 5351500
177 5015200
178 5513000
179 4517400
180 4271000
181 4220700
182 4023200
183 5952700
184 6192100
185 6109600
186 6124700
187 6094300
188 4790000
189 4472500
190 5605800
191 5186300
192 4291800
193 4173200
194 4022700
195 4806200
196 4205000
197 6726900
198 5977600
199 5216300
200 7415200
201 6250300
202 6979200
203 7303600
204 5190000
205 5815200
206 5235700
207 9781600
208 6692300
209 6353100
210 5431300
211 5204300
212 4729600
213 5068200
214 6122000
215 5059400
216 3967400
217 3688700
218 3742700
219 3118400
220 2941300
221 3205200
222 3930500
223 4250700
224 5111600
225 4421400
226 3775000
227 3451100
228 2984700
229 2712000
230 2500800
231 2312500
232 2314000
233 2447400
234 2275900
235 2090500
236 2015900
237 1912100
238 1875000
239 2692700
240 4866700
241 2094200
242 1805300
243 1607100
244 1474500
245 1870800
246 1663200
247 1451800
248 1333600
249 1245100
250 1207500
251 1180600
252 1154200
253 1123100
254 2155000
255 1463400
256 1873100
257 1340100
258 1340100
259 1230700
260 1160200
261 1104100
262 1055500
263 1022100
264 989200
265 959500
266 932600
267 905300
268 1649700
269 2359000
270 1346500
271 1139800
272 3925300
273 3750000
274 6550700
275 4871300
276 7055400
277 5808600
278 5128100
279 6022300
280 6574800
281 4805000
282 3766700
283 4804600
284 5674600
285 4063700
286 5243600
287 4373900
288 6620200
289 4486100
290 4234800
291 7840200
292 4896400
293 4088400
294 3589600
295 3241900
296 3224300
297 3727800
298 3196100
299 4726700
300 3278200
301 3079800
302 3213200
303 3031100
304 2591200
305 3070900
306 3489500
307 2517000
308 2212500
309 1990900
310 1849000
311 1947800
312 4983000
313 3274900
314 7221400
315 3673500
316 2890600
317 2486900
318 2228700
319 2038600
320 2631000
321 3451500
322 2727900
323 4411500
324 3607800
325 3217900
326 4213100
327 3042600
328 2697800
329 2699600
330 3654500
331 3325900
332 5257800
333 3309200
334 3438000
335 3834900
336 3367600
337 3426900
338 3866300
339 3354100
340 3635000
341 3162800
342 3024600
343 3957700
344 3484400
345 3246600
346 3211900
347 4457900
348 5736300
349 4331200
350 4098200
351 3952600
352 3992500
353 3673100
354 3894600
355 4672000
356 7140300
357 4707200
358 3695700
359 3964600
360 3656800
361 4355000
362 5632900
363 4689600
364 5576800
365 4433700
366 4036500
367 5008900
368 3930400
369 5319100
370 5797900
371 6756400
372 6488500
373 5668600
374 6442200
375 5849400
376 7113400
377 8568000
378 6312900
379 5103100
380 5300900
381 5946300
382 6725000
383 5804400
384 7214000
385 6788500
386 5741800
387 5898600
388 6116900
389 5599500
390 5021900
391 8005800
392 7747600
393 10203800
394 10338700
395 6655400
396 5403000
397 4469400
398 6673500
399 6047800
400 7027700
401 5099400
402 6563300
403 9234600
404 10016500
405 7341900
406 5782100
407 5582800
408 8059100
409 5688600
410 4980300
411 7363700
412 6216500
413 6770000
414 6974300
415 5061300
416 5165700
417 4762300
418 4461900
419 4335500
420 3995200
421 3494100
422 4353000
423 3605400
424 3149300
425 2922100
426 3169300
427 2875400
428 2509600
429 2772800
430 7051400
431 4182000
432 3320200
433 2970400
434 2779800
435 8009900
436 8343200
437 5654700
438 10685900
439 6363000
440 5326000
};
\addplot [blue]
table {%
0 5298736.83904932
1 5626859.65606309
2 5868331.52731017
3 5648646.48968174
4 5528738.41907214
5 5712118.83560826
6 4959310.10895567
7 4891065.88412778
8 4508570.02138083
9 4852176.08300249
10 4457245.2664834
11 4409082.86629253
12 4021416.7062109
13 6250435.09752677
14 6929569.49649957
15 5632167.16359045
16 4730709.20870313
17 4383963.77798575
18 5530368.27266025
19 5192567.05185076
20 6302295.61779034
21 5281549.94744911
22 5951466.59481321
23 5367408.43366451
24 5254059.36243063
25 4832975.68144474
26 4911659.99724732
27 4386756.91962765
28 4644953.90866946
29 5673484.25001405
30 4739945.13918617
31 4388213.6018921
32 4647357.53675909
33 5069566.51470786
34 4492543.4349891
35 4736247.79207692
36 4648191.39825518
37 6156691.34367734
38 6045358.90677382
39 5358520.34447614
40 5833762.33023563
41 5216690.0229626
42 6699026.47693974
43 6517794.80345779
44 6101089.31227324
45 6250022.71911789
46 6470544.33320393
47 5877221.14647298
48 5986196.20122551
49 5237427.94577368
50 6864278.08805338
51 5583451.3939396
52 6543854.61351157
53 5691723.45380965
54 5972429.82659957
55 5474550.01504196
56 5442852.11135815
57 7027726.4720502
58 5663205.21320207
59 5192339.79401839
60 5488293.01038088
61 4792971.33297773
62 5153613.50502484
63 5161709.48754414
64 5689012.97664794
65 4880523.16569211
66 4508868.56514575
67 4002636.18858953
68 3746269.62999881
69 3590909.476791
70 3565795.28703497
71 3780866.71209844
72 6071397.6953999
73 5176355.12763896
74 4719764.88447645
75 4251872.3984496
76 3966273.10122932
77 4443819.85397952
78 3923079.54188665
79 4033295.7008523
80 3589925.18828841
81 3423300.37148729
82 5480985.08602073
83 4048456.23001773
84 4192871.64740095
85 4978289.87059526
86 5484637.58921519
87 4229515.49837458
88 4634387.40720525
89 4947805.01158911
90 4325324.82800364
91 4148391.52381451
92 3644760.38116731
93 3455135.95459737
94 3474971.94208767
95 3673132.94326457
96 3301676.93567778
97 4467393.12457893
98 5080674.39698949
99 4698072.57727813
100 3861032.88290939
101 3516912.07803841
102 3498244.91941661
103 3416496.01300058
104 3689542.67498753
105 3392702.530294
106 4101310.53744685
107 4837011.78825238
108 4009556.45083325
109 4075330.97639622
110 3825464.6721557
111 3189775.17573124
112 3542907.88251821
113 2990240.38859897
114 3870504.27925661
115 5726302.59818848
116 5202246.40552406
117 4365491.23830156
118 4358410.30826777
119 4334817.51078993
120 4485787.26493593
121 3704364.82638684
122 3907500.85029006
123 4725828.53490997
124 4700546.96817261
125 5251337.90504984
126 4789821.26010004
127 4871724.90604665
128 4365072.39579244
129 4861012.98201773
130 4735409.62904347
131 4812500.68609446
132 4343409.59033436
133 4634962.54058931
134 5210224.08280241
135 3650386.63783111
136 3721406.77732851
137 3919005.16485111
138 3695276.19878231
139 3170496.55906382
140 3324750.54194763
141 3481520.38624314
142 4215916.88597239
143 4335230.61882603
144 4687107.35799687
145 4871427.3869192
146 4035386.40503791
147 3781288.78651539
148 5799286.39413354
149 4321745.64325823
150 4583236.87528129
151 3276521.79058218
152 3088519.95611919
153 3107061.22932657
154 3230766.17921722
155 3326897.10475958
156 3904073.19290947
157 3420567.85964444
158 4487046.38100847
159 3718169.47728159
160 3983012.37852082
161 3632532.2473695
162 4025485.61750767
163 3546581.09907926
164 3602654.13035991
165 3590912.15904031
166 4611274.25708872
167 4912696.35826579
168 6417673.06835574
169 5263978.37834287
170 5658873.61847895
171 5850842.61239257
172 5954569.53186784
173 5564319.26925136
174 5171991.72782083
175 5137193.10026871
176 5458222.1437972
177 6050032.83301066
178 5181039.91100641
179 4938507.2497737
180 4285470.82125958
181 4648875.23233625
182 4455490.589124
183 4481598.64749718
184 6221976.57020731
185 6501453.22463078
186 6183256.83267939
187 5011643.02629405
188 5355385.90397746
189 4791638.26157068
190 4656088.34203341
191 5736939.63960906
192 4695053.68244377
193 4221488.3967782
194 4278524.72530161
195 4203452.13544345
196 4603539.322809
197 5182787.71413561
198 5883977.81644818
199 5386334.29149877
200 6434982.94316669
201 5813514.2515671
202 5223063.20159306
203 6145874.87100539
204 5651575.476958
205 4573379.82733341
206 5421239.826442
207 5585141.53677929
208 6832918.75323408
209 5959281.18743691
210 5510892.50684725
211 5868158.65828912
212 5018987.15382696
213 5304285.32403558
214 5539600.68751242
215 5390274.37604256
216 4924745.22602412
217 4214769.25703101
218 3689603.22554681
219 4482806.28606048
220 4261693.21888457
221 3275640.60136476
222 3552874.50606653
223 4460574.6841238
224 4849604.64871022
225 4953338.42740627
226 4454325.92159966
227 4171936.57393011
228 3601646.87275144
229 3105071.79808038
230 2459395.75758716
231 2436249.30995146
232 2290816.356893
233 2834674.66659918
234 2958339.13632713
235 2686533.56829509
236 2507565.78216121
237 2460934.63891452
238 2074538.30883932
239 2070703.67795357
240 2749023.79680219
241 3451223.44152593
242 1919220.67543266
243 1579991.21963937
244 1840816.27178364
245 2103949.35325866
246 2665935.03856526
247 2374160.35901818
248 1515387.45713348
249 1374854.09809851
250 1634575.20894373
251 1073364.29296644
252 1344808.79479451
253 1599287.91474086
254 1635815.3243189
255 2326111.99969202
256 1972617.84564945
257 2134964.4072787
258 1691499.14786945
259 1824698.61560601
260 982213.409314109
261 1279748.15150999
262 1605255.30728415
263 715896.420160378
264 1108008.60952855
265 1078270.6823996
266 1184756.62114825
267 1188815.20985209
268 1377343.16516582
269 2111534.30112908
270 2633196.14354027
271 1591596.30652223
272 1387591.77409268
273 3517189.49213937
274 3469661.29367743
275 5204199.02528106
276 4145384.65936478
277 5624473.59432047
278 4431461.88055171
279 4887663.88423789
280 5227299.44008009
281 5474673.99740361
282 4932026.98840614
283 4077679.73288468
284 4818699.2389066
285 4972091.66371158
286 4606641.84606888
287 4847949.97686817
288 4404182.76288919
289 5025524.73925998
290 4360893.76104758
291 3967348.32424417
292 5164469.23506358
293 4458345.4631096
294 4253479.37643732
295 3840862.8576889
296 3649058.73883707
297 3441747.52226294
298 4113055.09114043
299 3546791.10452542
300 4427680.26156091
301 3665408.34247423
302 3704873.3970503
303 2989776.514315
304 2835925.36910857
305 2790927.13621975
306 3373836.18723424
307 3653507.38685423
308 3014840.3536023
309 2784841.03507923
310 2717824.31321831
311 2730321.00112756
312 2743893.21084144
313 4533416.313512
314 6036347.02826717
315 4348691.46155198
316 3562446.26551521
317 3131846.1162707
318 2854702.16339935
319 2786234.26408856
320 2832787.20711035
321 3030632.3758245
322 3121190.26845373
323 3226685.20937076
324 4069449.01657729
325 3778096.48702858
326 3814919.86084248
327 4333689.74731332
328 3123871.47300152
329 2480799.97645038
330 3361131.65510294
331 4304644.55241469
332 3757120.19800771
333 4895265.70445937
334 3999776.02331935
335 4094098.97135517
336 4145482.01255394
337 4053470.65335263
338 4105761.75565124
339 3845188.96127296
340 3639608.55900494
341 4080107.33227566
342 3437183.33704738
343 3166978.57344817
344 4219502.00767298
345 3931265.93063349
346 3733091.50302249
347 3865371.45393048
348 4715308.16048725
349 4753555.37904353
350 4586119.810782
351 4563872.81994614
352 4393968.48004065
353 5056846.91231878
354 4500448.32627549
355 4571335.16386009
356 5068293.04108968
357 5643420.26412455
358 4855068.02732831
359 4430216.43998201
360 4458232.68839429
361 4514397.43885508
362 5111602.48745987
363 5671846.57165375
364 5228828.23062484
365 5788115.69233891
366 4392687.8995529
367 4635044.93021789
368 5462495.14681422
369 4675204.91839351
370 5483981.14666941
371 5826604.73486574
372 6463283.07199941
373 6106814.34351465
374 5737141.26105855
375 6392268.87745689
376 6412069.64948962
377 6679990.21291789
378 6669154.01351669
379 5978059.50513182
380 4645729.18173089
381 5446500.73418476
382 6230188.63747107
383 6376790.69190314
384 5537183.77343238
385 6242717.06292029
386 6121717.11288083
387 5617362.1286987
388 5703127.91689663
389 5832098.22508721
390 5524403.74208413
391 5227956.34525868
392 6460932.23367679
393 6149275.94869918
394 7159490.66112068
395 6938729.31914276
396 5925712.91496626
397 5675877.26312231
398 5156637.25860928
399 6344442.40356936
400 5924956.74661294
401 6119278.5184161
402 5185603.88305429
403 5873827.94974253
404 6737657.97185934
405 6773451.54062755
406 6394532.45626395
407 5682655.76992045
408 5617420.07128437
409 6584757.40447709
410 5867163.2152823
411 5398375.78442743
412 6350674.07276208
413 5961240.27253102
414 6394463.42313667
415 6677752.55649254
416 5825631.58460806
417 5888944.34356187
418 5494177.51886781
419 5172519.97771811
420 4741019.43034424
421 4656103.70404567
422 4593030.44530593
423 5201191.59408874
424 4796238.78221106
425 4432053.62020979
426 4235429.33489494
427 4336094.28203675
428 3464930.73219785
429 3545975.02857409
430 3797937.61788625
431 6241740.81179615
432 4736182.35286091
433 4068212.82797575
434 3864736.01856268
435 3717033.49621051
436 6346437.28453149
437 6289091.7703
438 5520292.70128419
439 6990694.61944489
440 5639238.23755642
};
\end{axis}

\end{tikzpicture}
}
%	\subfloat[$I$.]{% This file was created with tikzplotlib v0.10.1.
\begin{tikzpicture}
\definecolor{green01270}{RGB}{0,127,0}
\definecolor{steelblue31119180}{RGB}{31,119,180}
\definecolor{darkgray176}{RGB}{176,176,176}
\definecolor{lightgray204}{RGB}{204,204,204}

\begin{axis}[
width=\figurewidth,
height=\figureheight,
axis background/.style={fill=background_color},
axis line style={white},
tick align=inside,
tick pos=left,
ylabel={Contributions (kWh)},
x grid style={white},
xmajorgrids,
xmin=150, xmax=300,
xtick style={color=white},
y grid style={white},
ymajorgrids,
xlabel={Time (days)},
xtick style={color=black},
ymin=-13237205.6599335, ymax=119040419.31714,
ytick style={color=black},
ytick={-20000000,0,20000000,40000000,60000000,80000000,100000000,120000000},
yticklabels={\ensuremath{-}0.2,0.0,0.2,0.4,0.6,0.8,1.0,1.2},
legend columns=3, 
legend style={
  nodes={scale=0.8, transform shape},
  fill opacity=0.8,
  draw opacity=1,
  text opacity=1,
  at={(0.03,0.97)},
  anchor=north west,
  legend pos=north east,
  draw=lightgray204,
  fill=background_color
},
]
\path [draw=blue, fill=blue, opacity=0.5]
(axis cs:0,31925052.2091835)
--(axis cs:0,1558183.05204632)
--(axis cs:1,-2349178.42999233)
--(axis cs:2,-5126183.6097403)
--(axis cs:3,7024335.79619604)
--(axis cs:4,-5739164.84421941)
--(axis cs:5,-3654257.77045015)
--(axis cs:6,-5519537.2407308)
--(axis cs:7,-5765991.04656563)
--(axis cs:8,-7934752.50211442)
--(axis cs:9,-6512049.72912347)
--(axis cs:10,-3992347.33414757)
--(axis cs:11,-5316205.56090487)
--(axis cs:12,-7888235.74583745)
--(axis cs:13,-4324348.01138568)
--(axis cs:14,-5820781.84119326)
--(axis cs:15,-6001510.58673185)
--(axis cs:16,-11084451.5735503)
--(axis cs:17,-8505380.47099489)
--(axis cs:18,-6317763.97851044)
--(axis cs:19,-6043994.00266101)
--(axis cs:20,-7252956.23418731)
--(axis cs:21,-9618673.39036574)
--(axis cs:22,-9508850.00971471)
--(axis cs:23,-7867763.64765651)
--(axis cs:24,-7686728.24722127)
--(axis cs:25,-8944613.44307322)
--(axis cs:26,-9203281.84058987)
--(axis cs:27,-12453972.7203834)
--(axis cs:28,-10550814.3180554)
--(axis cs:29,-9691860.75423588)
--(axis cs:30,-10218377.0003186)
--(axis cs:31,-9508590.36013648)
--(axis cs:32,-9857440.87546588)
--(axis cs:33,-9074201.31928582)
--(axis cs:34,-10317932.3402869)
--(axis cs:35,-9950043.87293975)
--(axis cs:36,-11075159.002645)
--(axis cs:37,-10174903.9501448)
--(axis cs:38,-10208930.3893856)
--(axis cs:39,-9248027.75112463)
--(axis cs:40,-8619619.34733264)
--(axis cs:41,-9614853.35479374)
--(axis cs:42,-8186434.7662968)
--(axis cs:43,-9915719.53105286)
--(axis cs:44,-12058437.8203891)
--(axis cs:45,-7871106.80845477)
--(axis cs:46,-7684395.08465428)
--(axis cs:47,-8683689.24909451)
--(axis cs:48,-7640804.68908913)
--(axis cs:49,-8386188.30279893)
--(axis cs:50,-6805722.15997268)
--(axis cs:51,-5590380.43944634)
--(axis cs:52,-8412910.03861637)
--(axis cs:53,-9695413.20699674)
--(axis cs:54,-10997308.6131284)
--(axis cs:55,-11806959.4750052)
--(axis cs:56,-9590038.68815796)
--(axis cs:57,-3881932.28977642)
--(axis cs:58,-2873247.41678162)
--(axis cs:59,999975.446154563)
--(axis cs:60,-2879697.2291535)
--(axis cs:61,-6746630.71858448)
--(axis cs:62,-8133593.36079635)
--(axis cs:63,-9808614.55604794)
--(axis cs:64,-8912413.63587536)
--(axis cs:65,-10595047.389887)
--(axis cs:66,-10176361.9092951)
--(axis cs:67,-10822283.2051554)
--(axis cs:68,-11819859.7472139)
--(axis cs:69,-12190168.3045937)
--(axis cs:70,-11239421.6797118)
--(axis cs:71,-10391904.972889)
--(axis cs:72,-8193895.91549887)
--(axis cs:73,-7480118.93072725)
--(axis cs:74,-6562568.40929183)
--(axis cs:75,-10190673.2652991)
--(axis cs:76,-10196818.2091364)
--(axis cs:77,-11388375.9572337)
--(axis cs:78,-10458419.1367197)
--(axis cs:79,-11476382.4226019)
--(axis cs:80,-12161082.352936)
--(axis cs:81,-10413567.1377354)
--(axis cs:82,-10181157.6861187)
--(axis cs:83,-9463024.05623508)
--(axis cs:84,-11226987.1542857)
--(axis cs:85,-8206325.73694822)
--(axis cs:86,-9750624.47650647)
--(axis cs:87,-11608829.2764832)
--(axis cs:88,-9898938.5198346)
--(axis cs:89,-11260335.1551459)
--(axis cs:90,-11675299.3615661)
--(axis cs:91,-10101100.4853032)
--(axis cs:92,-12278903.7568894)
--(axis cs:93,-12868703.5835604)
--(axis cs:94,-12989606.6385469)
--(axis cs:95,-14206425.8247409)
--(axis cs:96,-14002650.0815878)
--(axis cs:97,-12246115.1844074)
--(axis cs:98,-8961808.1432364)
--(axis cs:99,-10761121.7090604)
--(axis cs:100,-11053442.2467161)
--(axis cs:101,-12246281.4725702)
--(axis cs:102,-9879437.24328485)
--(axis cs:103,-12150198.4976553)
--(axis cs:104,-10383555.0509901)
--(axis cs:105,-366851.773162145)
--(axis cs:106,1682683.8349628)
--(axis cs:107,-7779164.36966727)
--(axis cs:108,-7923937.64003375)
--(axis cs:109,-11175532.2257438)
--(axis cs:110,-8963421.81197097)
--(axis cs:111,-11720419.2035788)
--(axis cs:112,-10939223.0135852)
--(axis cs:113,-12238728.7816918)
--(axis cs:114,-7866876.97170314)
--(axis cs:115,-1052507.0173686)
--(axis cs:116,-4289961.94679056)
--(axis cs:117,-9100466.9765065)
--(axis cs:118,-4813478.10293579)
--(axis cs:119,-8961011.58963506)
--(axis cs:120,-5252400.85605777)
--(axis cs:121,-9800714.86309974)
--(axis cs:122,-7715383.05326862)
--(axis cs:123,-5916935.19178989)
--(axis cs:124,-578362.954528984)
--(axis cs:125,593699.592696244)
--(axis cs:126,2314584.49466387)
--(axis cs:127,-2637917.72364703)
--(axis cs:128,-3266867.202149)
--(axis cs:129,316786.819850437)
--(axis cs:130,-7032436.79560497)
--(axis cs:131,3374600.8491869)
--(axis cs:132,-9156326.43965145)
--(axis cs:133,-6945960.33932743)
--(axis cs:134,-2144874.69709749)
--(axis cs:135,-2772530.35299499)
--(axis cs:136,-7732827.46535706)
--(axis cs:137,-1292919.52696036)
--(axis cs:138,-785066.349835197)
--(axis cs:139,-3445723.03501045)
--(axis cs:140,-203108.971859908)
--(axis cs:141,-1509748.63741338)
--(axis cs:142,379150.266811427)
--(axis cs:143,20709.6645837016)
--(axis cs:144,4744948.25751818)
--(axis cs:145,-3301579.0380511)
--(axis cs:146,45122.0313422345)
--(axis cs:147,1562533.69548005)
--(axis cs:148,-1698715.79518692)
--(axis cs:149,-905689.069364598)
--(axis cs:150,-4108932.19826762)
--(axis cs:151,-13079226.4851881)
--(axis cs:152,-3192418.99894056)
--(axis cs:153,2222276.85105362)
--(axis cs:154,-4791347.64706116)
--(axis cs:155,-248073.938784458)
--(axis cs:156,8837189.47384572)
--(axis cs:157,-6454322.05522016)
--(axis cs:158,-3229375.26130272)
--(axis cs:159,-6384249.25738256)
--(axis cs:160,-8708637.90086552)
--(axis cs:161,-6298587.35511497)
--(axis cs:162,-4707749.7346411)
--(axis cs:163,-3829980.91429561)
--(axis cs:164,-7337736.53589446)
--(axis cs:165,-5168010.10449969)
--(axis cs:166,745499.95747696)
--(axis cs:167,-4559099.62492303)
--(axis cs:168,-84431.084169345)
--(axis cs:169,-4303216.5154261)
--(axis cs:170,3918002.50425644)
--(axis cs:171,6985184.21120862)
--(axis cs:172,12675692.2032619)
--(axis cs:173,10618426.0757076)
--(axis cs:174,-814248.452844203)
--(axis cs:175,-2909265.61995627)
--(axis cs:176,-499015.022293478)
--(axis cs:177,-1620754.03801222)
--(axis cs:178,7351054.12826618)
--(axis cs:179,-8258343.85496712)
--(axis cs:180,-3280326.29834601)
--(axis cs:181,2131225.77778679)
--(axis cs:182,-1462677.88899643)
--(axis cs:183,3690627.24892992)
--(axis cs:184,5842298.98514461)
--(axis cs:185,16432308.6725312)
--(axis cs:186,21281139.8305038)
--(axis cs:187,-4472900.08934802)
--(axis cs:188,-2202226.28799476)
--(axis cs:189,-3350400.07992559)
--(axis cs:190,-2328484.81517838)
--(axis cs:191,6336578.61174068)
--(axis cs:192,8764073.82973653)
--(axis cs:193,25901757.9152522)
--(axis cs:194,20394524.7527422)
--(axis cs:195,2812282.90936599)
--(axis cs:196,4259804.9059528)
--(axis cs:197,1279836.27846912)
--(axis cs:198,-1252450.20391109)
--(axis cs:199,8475398.71222135)
--(axis cs:200,36825318.7224212)
--(axis cs:201,9021208.21018066)
--(axis cs:202,195298.788273076)
--(axis cs:203,435029.921350647)
--(axis cs:204,5678258.22176882)
--(axis cs:205,934244.1210443)
--(axis cs:206,7323924.95539739)
--(axis cs:207,7204040.78363095)
--(axis cs:208,12479063.4198653)
--(axis cs:209,-919939.564798279)
--(axis cs:210,5186715.38002818)
--(axis cs:211,21386553.9029942)
--(axis cs:212,-1579158.20875236)
--(axis cs:213,7108443.91654442)
--(axis cs:214,-5941755.69235916)
--(axis cs:215,-3625740.48573226)
--(axis cs:216,-2668942.69055533)
--(axis cs:217,-3838966.2631075)
--(axis cs:218,14468840.6339481)
--(axis cs:219,15672440.7025596)
--(axis cs:220,7939348.36185779)
--(axis cs:221,-3717474.32934823)
--(axis cs:222,3000447.0702942)
--(axis cs:223,12585582.5447384)
--(axis cs:224,2881721.21555357)
--(axis cs:225,3265751.28548337)
--(axis cs:226,-4541305.94211222)
--(axis cs:227,689822.42365418)
--(axis cs:228,-2261143.88942303)
--(axis cs:229,-383863.913648708)
--(axis cs:230,-4708664.91143827)
--(axis cs:231,-3142078.52519059)
--(axis cs:232,-7774746.54997696)
--(axis cs:233,-5282461.14722271)
--(axis cs:234,-7845759.78078065)
--(axis cs:235,-7582433.93274541)
--(axis cs:236,-4626678.36031134)
--(axis cs:237,-4017774.64428618)
--(axis cs:238,8733603.25581942)
--(axis cs:239,-1852390.11798437)
--(axis cs:240,15700827.0186618)
--(axis cs:241,-3003346.74903443)
--(axis cs:242,-1958493.87660743)
--(axis cs:243,-7322090.70865847)
--(axis cs:244,-5145740.85409056)
--(axis cs:245,2436020.753153)
--(axis cs:246,10920395.2591011)
--(axis cs:247,5124668.85916309)
--(axis cs:248,-4128684.09758325)
--(axis cs:249,-8816196.22128519)
--(axis cs:250,-8052250.91874349)
--(axis cs:251,-5077077.16364152)
--(axis cs:252,-6331015.13114138)
--(axis cs:253,-7982763.43616026)
--(axis cs:254,8079677.3621595)
--(axis cs:255,6919896.66031865)
--(axis cs:256,2057689.50909814)
--(axis cs:257,-224988.159488656)
--(axis cs:258,-927370.683494868)
--(axis cs:259,7303811.56633783)
--(axis cs:260,-6709166.44938457)
--(axis cs:261,-4198965.1839363)
--(axis cs:262,-5675773.8987025)
--(axis cs:263,-10747278.5740637)
--(axis cs:264,-14888096.7685547)
--(axis cs:265,-6543280.48251369)
--(axis cs:266,-10537126.7210858)
--(axis cs:267,-12836986.9649774)
--(axis cs:268,-7817245.99357252)
--(axis cs:269,-5655818.35049332)
--(axis cs:270,-3942387.54602846)
--(axis cs:271,-5472566.71012365)
--(axis cs:272,-1519103.24536341)
--(axis cs:273,-3633118.85919705)
--(axis cs:274,-1998811.21812547)
--(axis cs:275,4916528.65427052)
--(axis cs:276,-1150301.32548789)
--(axis cs:277,6658898.0258016)
--(axis cs:278,-2672208.59684595)
--(axis cs:279,-2954985.33492053)
--(axis cs:280,-471033.765561892)
--(axis cs:281,-3766971.25024104)
--(axis cs:282,-796979.105053034)
--(axis cs:283,-3500082.46696732)
--(axis cs:284,-2449309.53580593)
--(axis cs:285,-5397370.14119626)
--(axis cs:286,775122.449253762)
--(axis cs:287,-117921.479802124)
--(axis cs:288,-5808964.123311)
--(axis cs:289,-2865349.47467298)
--(axis cs:290,-3100333.76831)
--(axis cs:291,2893625.77789757)
--(axis cs:292,-2105200.55447249)
--(axis cs:293,-5986145.96593835)
--(axis cs:294,-5651700.59759589)
--(axis cs:295,-250026.212397311)
--(axis cs:296,1288371.56347809)
--(axis cs:297,3125219.5235871)
--(axis cs:298,1140581.10862351)
--(axis cs:299,518916.163653873)
--(axis cs:300,676044.219172895)
--(axis cs:301,-1605513.45558587)
--(axis cs:302,5454284.30100858)
--(axis cs:303,-572799.026433457)
--(axis cs:304,-7549702.51513581)
--(axis cs:305,-6714725.28570318)
--(axis cs:306,-1993622.99019448)
--(axis cs:307,-4447103.84677581)
--(axis cs:308,-6961875.57578766)
--(axis cs:309,-6957437.87394874)
--(axis cs:310,-4127747.68718458)
--(axis cs:311,-2645422.11963348)
--(axis cs:312,-407109.279278984)
--(axis cs:313,2437900.30792989)
--(axis cs:314,28822742.725176)
--(axis cs:315,1184600.58098856)
--(axis cs:316,-2066237.6451567)
--(axis cs:317,5878216.84809564)
--(axis cs:318,-6838110.7408238)
--(axis cs:319,-6180608.24577861)
--(axis cs:320,-4015879.7811871)
--(axis cs:321,-5092235.02825681)
--(axis cs:322,-6379543.09223432)
--(axis cs:323,-942438.842127299)
--(axis cs:324,856689.024505397)
--(axis cs:325,-1353666.58415581)
--(axis cs:326,-3720176.90352517)
--(axis cs:327,1959543.25176384)
--(axis cs:328,3864019.14557752)
--(axis cs:329,-3676171.24807593)
--(axis cs:330,-3590186.00506706)
--(axis cs:331,-32253.3700084165)
--(axis cs:332,-1616623.49092088)
--(axis cs:333,-4084865.89205346)
--(axis cs:334,-2895955.2276528)
--(axis cs:335,-1348947.35607245)
--(axis cs:336,-3466253.48573287)
--(axis cs:337,-2525303.4693416)
--(axis cs:338,-971093.557603233)
--(axis cs:339,-7099395.82810367)
--(axis cs:340,-4315499.32319484)
--(axis cs:341,62331.2809176389)
--(axis cs:342,-1806981.5735455)
--(axis cs:343,-6983769.5642515)
--(axis cs:344,-5724185.45903565)
--(axis cs:345,-6333634.46210776)
--(axis cs:346,-7603192.57254876)
--(axis cs:347,-3978146.26133446)
--(axis cs:348,4422665.37988386)
--(axis cs:349,-6565592.91870088)
--(axis cs:350,643051.459906681)
--(axis cs:351,3558832.00327458)
--(axis cs:352,2541305.64126211)
--(axis cs:353,-5214143.26673375)
--(axis cs:354,-2684596.3567352)
--(axis cs:355,-3807671.42530994)
--(axis cs:356,-339871.723502874)
--(axis cs:357,-5842403.1332742)
--(axis cs:358,-6136824.94669376)
--(axis cs:359,-4013118.53470249)
--(axis cs:360,-5766348.3628489)
--(axis cs:361,-3616617.3141033)
--(axis cs:362,-3759699.71121432)
--(axis cs:363,-6910818.05283808)
--(axis cs:364,-8880245.43715982)
--(axis cs:365,-4681542.96467946)
--(axis cs:366,-9369520.50944607)
--(axis cs:367,-4634835.48882621)
--(axis cs:368,-4339646.66726598)
--(axis cs:369,-6917627.07580572)
--(axis cs:370,-6979306.96651278)
--(axis cs:371,-7252840.70150443)
--(axis cs:372,-7955050.53245939)
--(axis cs:373,-6347402.33488897)
--(axis cs:374,-5555166.42674215)
--(axis cs:375,-2428721.77877922)
--(axis cs:376,1592489.14203584)
--(axis cs:377,-2142017.05971332)
--(axis cs:378,-7367561.03988581)
--(axis cs:379,-9900628.49364917)
--(axis cs:380,-11564176.4938226)
--(axis cs:381,-8924371.42075885)
--(axis cs:382,-6827682.90673274)
--(axis cs:383,-7438517.9373647)
--(axis cs:384,-6287035.23299627)
--(axis cs:385,-5966732.6596797)
--(axis cs:386,-5675633.57390694)
--(axis cs:387,-6727392.4676682)
--(axis cs:388,-6477449.466322)
--(axis cs:389,-6628169.29796252)
--(axis cs:390,-5203233.91729769)
--(axis cs:391,-8917686.83899321)
--(axis cs:392,-7918904.98372306)
--(axis cs:393,-9040797.2855427)
--(axis cs:394,-7688559.45463303)
--(axis cs:395,-9209191.2304942)
--(axis cs:396,-8958116.55621583)
--(axis cs:397,-9944916.01480553)
--(axis cs:398,-9179177.91944674)
--(axis cs:399,-8016894.61824447)
--(axis cs:400,-9727060.09788186)
--(axis cs:401,-9733658.99045403)
--(axis cs:402,-10792326.8587067)
--(axis cs:403,-9879890.6213648)
--(axis cs:404,-7081582.96955026)
--(axis cs:405,-7985604.0927485)
--(axis cs:406,-5700945.79484526)
--(axis cs:407,-8744974.86359898)
--(axis cs:408,-9913388.17994212)
--(axis cs:409,-9745405.08527431)
--(axis cs:410,-7312426.12676026)
--(axis cs:411,-7656846.79881373)
--(axis cs:412,-11131917.7448056)
--(axis cs:413,-9045677.80445304)
--(axis cs:414,-9448598.83601603)
--(axis cs:415,-7127846.21555243)
--(axis cs:416,-7054250.62536332)
--(axis cs:417,-10186030.2502215)
--(axis cs:418,-10530531.4496492)
--(axis cs:419,-12068525.8750916)
--(axis cs:420,-11804300.242239)
--(axis cs:421,-10854162.5215055)
--(axis cs:422,-10045247.4160132)
--(axis cs:423,-8869189.03546282)
--(axis cs:424,-10393591.1961976)
--(axis cs:425,-11178616.6971534)
--(axis cs:426,-10217213.559802)
--(axis cs:427,-11004154.3647045)
--(axis cs:428,-11694457.6975439)
--(axis cs:429,-10188609.9338197)
--(axis cs:430,-11077471.8828855)
--(axis cs:431,-7823382.36761869)
--(axis cs:432,-10426773.2511418)
--(axis cs:433,-10949175.1493761)
--(axis cs:434,-10452084.9037486)
--(axis cs:435,-11304130.2051657)
--(axis cs:436,-9560220.10306126)
--(axis cs:437,-9372052.86961551)
--(axis cs:438,-10830191.1953708)
--(axis cs:439,-11380050.1642548)
--(axis cs:440,-10675003.7696071)
--(axis cs:440,19269425.9866216)
--(axis cs:440,19269425.9866216)
--(axis cs:439,19640647.2693746)
--(axis cs:438,19020723.6416415)
--(axis cs:437,20639957.2108487)
--(axis cs:436,20335850.1715696)
--(axis cs:435,18494749.5862448)
--(axis cs:434,19331435.1674132)
--(axis cs:433,18806153.5582905)
--(axis cs:432,19321588.2045393)
--(axis cs:431,21962758.3859266)
--(axis cs:430,18737792.00468)
--(axis cs:429,19590998.7648917)
--(axis cs:428,18796852.7348172)
--(axis cs:427,18924937.8609915)
--(axis cs:426,19701731.6103114)
--(axis cs:425,18976156.7492356)
--(axis cs:424,19625265.6589473)
--(axis cs:423,21033271.999057)
--(axis cs:422,19864183.5279189)
--(axis cs:421,19297754.0258822)
--(axis cs:420,18486836.8761797)
--(axis cs:419,18010386.9957854)
--(axis cs:418,19605017.3381678)
--(axis cs:417,19938124.154487)
--(axis cs:416,22818711.9023564)
--(axis cs:415,22865368.0636351)
--(axis cs:414,20314695.1352992)
--(axis cs:413,20786963.2716108)
--(axis cs:412,18844625.0233313)
--(axis cs:411,22285915.021438)
--(axis cs:410,22545371.4460086)
--(axis cs:409,20282684.9283811)
--(axis cs:408,19837605.1452264)
--(axis cs:407,21073040.1654105)
--(axis cs:406,24143090.8492918)
--(axis cs:405,22031674.317949)
--(axis cs:404,22933365.1859467)
--(axis cs:403,20629032.8299722)
--(axis cs:402,19068772.2620246)
--(axis cs:401,20090180.6126877)
--(axis cs:400,20076308.1193671)
--(axis cs:399,21776675.7817981)
--(axis cs:398,20659158.1984162)
--(axis cs:397,19994713.4402426)
--(axis cs:396,21627355.9930479)
--(axis cs:395,21267056.8130254)
--(axis cs:394,22375090.1200801)
--(axis cs:393,20860582.8612837)
--(axis cs:392,21943204.3718551)
--(axis cs:391,20864468.0923135)
--(axis cs:390,24767274.2028565)
--(axis cs:389,23138626.6503239)
--(axis cs:388,23337419.0750379)
--(axis cs:387,23242643.421837)
--(axis cs:386,24160192.4090843)
--(axis cs:385,23869879.8195718)
--(axis cs:384,23735308.0349425)
--(axis cs:383,22641086.1736571)
--(axis cs:382,23289026.7615771)
--(axis cs:381,21092565.3437877)
--(axis cs:380,19739620.6703748)
--(axis cs:379,22038847.5769604)
--(axis cs:378,24460616.4125181)
--(axis cs:377,29179198.4091185)
--(axis cs:376,32685171.734137)
--(axis cs:375,27523658.1803533)
--(axis cs:374,24405625.7128298)
--(axis cs:373,23596860.378555)
--(axis cs:372,22093912.7632429)
--(axis cs:371,22822856.4234762)
--(axis cs:370,22828571.676611)
--(axis cs:369,22994587.4215753)
--(axis cs:368,25617930.7091296)
--(axis cs:367,25536670.6597363)
--(axis cs:366,22110841.9480306)
--(axis cs:365,25231117.1398694)
--(axis cs:364,21394499.3145963)
--(axis cs:363,23147919.8890693)
--(axis cs:362,27094361.9211554)
--(axis cs:361,26440493.2345099)
--(axis cs:360,24696215.6579044)
--(axis cs:359,26054020.9083346)
--(axis cs:358,24423935.3636656)
--(axis cs:357,25426024.2752054)
--(axis cs:356,29800757.7200575)
--(axis cs:355,26196266.3364167)
--(axis cs:354,27425956.0321598)
--(axis cs:353,25941003.6730595)
--(axis cs:352,33102025.3031111)
--(axis cs:351,33702948.4861921)
--(axis cs:350,31153767.8958894)
--(axis cs:349,24258232.6093165)
--(axis cs:348,34959542.9492213)
--(axis cs:347,26093936.7295133)
--(axis cs:346,22273672.9971589)
--(axis cs:345,23462700.7714909)
--(axis cs:344,24356343.5198528)
--(axis cs:343,23217328.7678497)
--(axis cs:342,28814516.3733742)
--(axis cs:341,30558253.7436295)
--(axis cs:340,25564664.3856557)
--(axis cs:339,22872123.5148956)
--(axis cs:338,29328759.5228704)
--(axis cs:337,27777727.1559723)
--(axis cs:336,28810332.4553675)
--(axis cs:335,28597954.5398856)
--(axis cs:334,27132910.4612543)
--(axis cs:333,26372968.1079824)
--(axis cs:332,28623613.0526695)
--(axis cs:331,30149163.9909585)
--(axis cs:330,26484238.963219)
--(axis cs:329,28222462.1412297)
--(axis cs:328,37712512.6380888)
--(axis cs:327,32291999.2118638)
--(axis cs:326,26485316.5664193)
--(axis cs:325,28854738.4941449)
--(axis cs:324,31801580.931825)
--(axis cs:323,29666976.2759318)
--(axis cs:322,24551786.4359335)
--(axis cs:321,25489170.0476276)
--(axis cs:320,26198136.3222077)
--(axis cs:319,23768798.0380697)
--(axis cs:318,23340056.6480787)
--(axis cs:317,36087476.0269351)
--(axis cs:316,28415596.3126493)
--(axis cs:315,34139567.6807584)
--(axis cs:314,60863838.3730434)
--(axis cs:313,33366257.4687737)
--(axis cs:312,29513458.4266879)
--(axis cs:311,27355663.4758028)
--(axis cs:310,25820949.9027381)
--(axis cs:309,23027934.6790122)
--(axis cs:308,23034383.5956228)
--(axis cs:307,25804698.4832121)
--(axis cs:306,27946791.5099236)
--(axis cs:305,23380529.9797668)
--(axis cs:304,23539078.1827214)
--(axis cs:303,30645727.9146455)
--(axis cs:302,35497595.6395359)
--(axis cs:301,28680658.4510061)
--(axis cs:300,31065619.5016689)
--(axis cs:299,30972133.9664309)
--(axis cs:298,31884585.8371734)
--(axis cs:297,34371213.3265747)
--(axis cs:296,31214388.7081396)
--(axis cs:295,29621514.9098342)
--(axis cs:294,24586386.6182542)
--(axis cs:293,25245090.0768507)
--(axis cs:292,28683499.901802)
--(axis cs:291,33347791.0430751)
--(axis cs:290,26928491.4321136)
--(axis cs:289,28410915.2395854)
--(axis cs:288,24202614.951331)
--(axis cs:287,30239234.1086275)
--(axis cs:286,31060097.8930927)
--(axis cs:285,24964239.1570495)
--(axis cs:284,28818711.79781)
--(axis cs:283,27117228.7172924)
--(axis cs:282,29569322.3469263)
--(axis cs:281,26525478.7476235)
--(axis cs:280,30021073.5912945)
--(axis cs:279,28112171.8573654)
--(axis cs:278,29338456.1984213)
--(axis cs:277,37195866.5460679)
--(axis cs:276,29222125.9950931)
--(axis cs:275,35252237.8132786)
--(axis cs:274,28642486.7615062)
--(axis cs:273,27104403.0099693)
--(axis cs:272,29208046.46708)
--(axis cs:271,27740715.2787662)
--(axis cs:270,26461553.5305262)
--(axis cs:269,24588390.1586955)
--(axis cs:268,22698203.8038261)
--(axis cs:267,17396306.8400042)
--(axis cs:266,19799969.3186595)
--(axis cs:265,23954282.8992644)
--(axis cs:264,16066991.6593592)
--(axis cs:263,19755390.250983)
--(axis cs:262,24791122.0648047)
--(axis cs:261,25972497.138684)
--(axis cs:260,25804846.0633435)
--(axis cs:259,37800480.2073878)
--(axis cs:258,29290885.9196329)
--(axis cs:257,30366227.4692394)
--(axis cs:256,32574565.9064221)
--(axis cs:255,37555802.3679874)
--(axis cs:254,39234226.8088434)
--(axis cs:253,24623182.9131283)
--(axis cs:252,26034806.0361543)
--(axis cs:251,26618422.2944383)
--(axis cs:250,22910499.7189953)
--(axis cs:249,22124243.4499929)
--(axis cs:248,28050519.8081102)
--(axis cs:247,39070036.2228537)
--(axis cs:246,41440270.7029597)
--(axis cs:245,32850043.6753078)
--(axis cs:244,25708014.0710282)
--(axis cs:243,23413756.7006655)
--(axis cs:242,28975712.7762289)
--(axis cs:241,30683941.561438)
--(axis cs:240,46259080.2631186)
--(axis cs:239,31496419.2008197)
--(axis cs:238,40174475.8829105)
--(axis cs:237,26171851.850441)
--(axis cs:236,25649024.6792549)
--(axis cs:235,22429623.0555503)
--(axis cs:234,22151997.690262)
--(axis cs:233,24615247.0072366)
--(axis cs:232,23223621.7653129)
--(axis cs:231,27071048.7250858)
--(axis cs:230,25423241.1624265)
--(axis cs:229,29880359.325936)
--(axis cs:228,27798465.4342151)
--(axis cs:227,30822339.7019223)
--(axis cs:226,26869858.9957163)
--(axis cs:225,33563535.0229065)
--(axis cs:224,35807777.4929778)
--(axis cs:223,42841456.8329565)
--(axis cs:222,34389869.6706888)
--(axis cs:221,29077177.9436712)
--(axis cs:220,41615414.1209878)
--(axis cs:219,47024780.6863132)
--(axis cs:218,45240399.5765361)
--(axis cs:217,26956051.066773)
--(axis cs:216,27944202.9746584)
--(axis cs:215,28588851.4750366)
--(axis cs:214,26400568.4765216)
--(axis cs:213,39034024.8959944)
--(axis cs:212,30956936.6976745)
--(axis cs:211,59561941.7908069)
--(axis cs:210,35896700.2708636)
--(axis cs:209,31790822.2941905)
--(axis cs:208,43942722.113551)
--(axis cs:207,37866539.9953383)
--(axis cs:206,38637210.6060093)
--(axis cs:205,32147143.8962265)
--(axis cs:204,36049708.5101149)
--(axis cs:203,31238223.0739249)
--(axis cs:202,33690031.0910221)
--(axis cs:201,42254320.3978042)
--(axis cs:200,69681832.0603227)
--(axis cs:199,40459630.5989072)
--(axis cs:198,29194091.5553931)
--(axis cs:197,32554661.9658286)
--(axis cs:196,35048740.8634254)
--(axis cs:195,35778857.7099913)
--(axis cs:194,53623946.9098459)
--(axis cs:193,56965672.0446213)
--(axis cs:192,39459513.125689)
--(axis cs:191,36702779.964499)
--(axis cs:190,27684794.8987143)
--(axis cs:189,26655822.9834415)
--(axis cs:188,28257644.1390571)
--(axis cs:187,29837530.6907703)
--(axis cs:186,54354530.3444295)
--(axis cs:185,48137229.6378947)
--(axis cs:184,37183067.4513279)
--(axis cs:183,33715366.2172923)
--(axis cs:182,29438337.936182)
--(axis cs:181,32464968.5799909)
--(axis cs:180,27265203.1957237)
--(axis cs:179,23920719.0678456)
--(axis cs:178,38431665.4868771)
--(axis cs:177,29278138.0427443)
--(axis cs:176,29986401.279054)
--(axis cs:175,27743578.5046792)
--(axis cs:174,30450307.7608892)
--(axis cs:173,43194608.107606)
--(axis cs:172,44664902.7554429)
--(axis cs:171,38157155.3617688)
--(axis cs:170,34346822.3520057)
--(axis cs:169,28050076.916275)
--(axis cs:168,30505794.2879546)
--(axis cs:167,25787863.0257074)
--(axis cs:166,30858385.5505781)
--(axis cs:165,24930015.6004494)
--(axis cs:164,22683276.2829642)
--(axis cs:163,26336136.6844731)
--(axis cs:162,26414245.1418528)
--(axis cs:161,23718828.2832113)
--(axis cs:160,21357599.358753)
--(axis cs:159,23562431.4407949)
--(axis cs:158,26939491.9777115)
--(axis cs:157,24609634.8091765)
--(axis cs:156,40474962.032026)
--(axis cs:155,29799652.454181)
--(axis cs:154,26407867.7423083)
--(axis cs:153,33036009.9893627)
--(axis cs:152,26954942.3494751)
--(axis cs:151,19875702.687444)
--(axis cs:150,26337795.1928648)
--(axis cs:149,29129267.012863)
--(axis cs:148,29633608.0840516)
--(axis cs:147,31473199.0275476)
--(axis cs:146,30282359.8630144)
--(axis cs:145,28264764.0445535)
--(axis cs:144,35293404.2318773)
--(axis cs:143,30028845.5890718)
--(axis cs:142,30530170.1826362)
--(axis cs:141,28586905.3030635)
--(axis cs:140,30477218.2223347)
--(axis cs:139,26978726.7507574)
--(axis cs:138,29717794.4357272)
--(axis cs:137,28774903.1597303)
--(axis cs:136,22297225.4582833)
--(axis cs:135,27578205.1716035)
--(axis cs:134,28318263.8842657)
--(axis cs:133,23239059.0042328)
--(axis cs:132,24599392.0679622)
--(axis cs:131,33751259.137669)
--(axis cs:130,23606544.2925807)
--(axis cs:129,30760848.5866496)
--(axis cs:128,27836049.9438673)
--(axis cs:127,28000257.0609644)
--(axis cs:126,34138121.0141403)
--(axis cs:125,31741096.9488111)
--(axis cs:124,29723786.0086987)
--(axis cs:123,25290957.1597012)
--(axis cs:122,22328731.4137595)
--(axis cs:121,20710143.34285)
--(axis cs:120,25819673.9663331)
--(axis cs:119,21097692.6045773)
--(axis cs:118,25228872.4266819)
--(axis cs:117,20792934.8938373)
--(axis cs:116,26478280.2006977)
--(axis cs:115,29075195.7027124)
--(axis cs:114,22055813.6310329)
--(axis cs:113,18393717.0254932)
--(axis cs:112,19114063.1724384)
--(axis cs:111,19577103.2045981)
--(axis cs:110,21218181.1188125)
--(axis cs:109,18929353.4185457)
--(axis cs:108,22042108.0200946)
--(axis cs:107,22859724.5604021)
--(axis cs:106,33278352.6834227)
--(axis cs:105,30245154.4428577)
--(axis cs:104,19423553.8682815)
--(axis cs:103,17768325.5590965)
--(axis cs:102,20164448.0430835)
--(axis cs:101,17647914.6207141)
--(axis cs:100,18944133.8403092)
--(axis cs:99,19165750.9856944)
--(axis cs:98,20974068.7558441)
--(axis cs:97,17630353.2494886)
--(axis cs:96,16104966.4232986)
--(axis cs:95,15839846.2715459)
--(axis cs:94,16897139.3979525)
--(axis cs:93,17026731.6626238)
--(axis cs:92,17689065.8713372)
--(axis cs:91,19792325.7248864)
--(axis cs:90,18327814.858991)
--(axis cs:89,18656719.4848975)
--(axis cs:88,19998077.2931614)
--(axis cs:87,18238201.5553992)
--(axis cs:86,20191434.886759)
--(axis cs:85,22453309.6165387)
--(axis cs:84,18800506.8532522)
--(axis cs:83,20381069.3561732)
--(axis cs:82,19721136.589673)
--(axis cs:81,19465282.1341626)
--(axis cs:80,17826147.803433)
--(axis cs:79,18400177.1220038)
--(axis cs:78,19343669.466588)
--(axis cs:77,18451222.6737694)
--(axis cs:76,19567594.7133333)
--(axis cs:75,19763529.2343535)
--(axis cs:74,23391787.4287728)
--(axis cs:73,22448232.5986782)
--(axis cs:72,22013188.6809488)
--(axis cs:71,19362252.3726255)
--(axis cs:70,19401354.2643512)
--(axis cs:69,17699739.5403701)
--(axis cs:68,17959493.5853639)
--(axis cs:67,19103861.3243429)
--(axis cs:66,19767251.5204452)
--(axis cs:65,19292319.0685236)
--(axis cs:64,20902683.4677022)
--(axis cs:63,20077393.1908966)
--(axis cs:62,21661919.9850012)
--(axis cs:61,23383391.9241443)
--(axis cs:60,27770806.2505149)
--(axis cs:59,31731351.2805811)
--(axis cs:58,27055950.7418718)
--(axis cs:57,26174852.42186)
--(axis cs:56,20586350.3810005)
--(axis cs:55,18190146.9387878)
--(axis cs:54,19171699.5483401)
--(axis cs:53,20535731.6871493)
--(axis cs:52,22158113.0477397)
--(axis cs:51,24506487.0540148)
--(axis cs:50,23222419.0863687)
--(axis cs:49,21653575.7628479)
--(axis cs:48,22326651.7200357)
--(axis cs:47,21183672.0007156)
--(axis cs:46,22233192.5736002)
--(axis cs:45,22088158.5384926)
--(axis cs:44,19480935.0467992)
--(axis cs:43,20579175.6859623)
--(axis cs:42,21963012.2324751)
--(axis cs:41,20153795.6548025)
--(axis cs:40,21947713.1863646)
--(axis cs:39,20519409.6062547)
--(axis cs:38,19579310.7835012)
--(axis cs:37,19686438.3265004)
--(axis cs:36,18881799.996467)
--(axis cs:35,19877241.1948065)
--(axis cs:34,19531546.293232)
--(axis cs:33,21059611.7424309)
--(axis cs:32,19972850.7981965)
--(axis cs:31,20323129.4298751)
--(axis cs:30,19706721.2221154)
--(axis cs:29,20258316.1972449)
--(axis cs:28,19562252.2555616)
--(axis cs:27,17572093.7839409)
--(axis cs:26,20645004.1439194)
--(axis cs:25,21146171.9279127)
--(axis cs:24,22268206.7737548)
--(axis cs:23,22150019.2535872)
--(axis cs:22,20421757.3667733)
--(axis cs:21,21009930.3828662)
--(axis cs:20,23537302.4460774)
--(axis cs:19,24163399.8554157)
--(axis cs:18,24095814.1359468)
--(axis cs:17,22073839.4432389)
--(axis cs:16,22581102.3097662)
--(axis cs:15,24024950.9419489)
--(axis cs:14,24325335.6721734)
--(axis cs:13,25746068.2520929)
--(axis cs:12,22434351.5923761)
--(axis cs:11,24566511.446166)
--(axis cs:10,26068500.8330347)
--(axis cs:9,23899564.4726533)
--(axis cs:8,22806903.39768)
--(axis cs:7,25040625.0923553)
--(axis cs:6,25345124.7440987)
--(axis cs:5,27166416.1786298)
--(axis cs:4,27464872.7561989)
--(axis cs:3,38738461.5213158)
--(axis cs:2,27359656.8054173)
--(axis cs:1,31938252.4538893)
--(axis cs:0,31925052.2091835)
--cycle;

\addplot [only marks, red, mark=asterisk, mark size=1.2]
table {%
0 29755500
1 20228000
2 8513700
3 9502900
4 8909200
5 7004800
6 6305500
7 6348400
8 7321100
9 6421300
10 6164800
11 5360500
12 5074700
13 5074700
14 6804600
15 9612000
16 9154100
17 5291400
18 5456200
19 5819200
20 12820200
21 5305900
22 5067000
23 11001200
24 7051300
25 7450000
26 9014400
27 5198200
28 4916300
29 5534800
30 4719000
31 4596600
32 6568000
33 6431200
34 5034500
35 6933000
36 4533400
37 4446000
38 5301600
39 4517400
40 4370200
41 6208600
42 7137600
43 4905200
44 5415600
45 4461400
46 4170200
47 4549000
48 5623500
49 4354800
50 6282500
51 6528100
52 4165200
53 3940800
54 3894500
55 4728400
56 12964300
57 7101500
58 9158000
59 10692100
60 4636300
61 5787600
62 5199600
63 4093600
64 3925700
65 3818200
66 3756600
67 3702000
68 3632400
69 3676500
70 3599700
71 4028000
72 6895400
73 4511900
74 5519200
75 3725000
76 3495900
77 3418000
78 3308300
79 3208300
80 3241700
81 7029700
82 5106300
83 4259800
84 11266900
85 7029700
86 3754100
87 3833100
88 5472000
89 6634600
90 3756400
91 3932500
92 3587100
93 3677400
94 3490000
95 3271000
96 3169400
97 3754100
98 3754100
99 3289100
100 3907600
101 3268800
102 3747400
103 4939300
104 5061200
105 34188900
106 27994500
107 6776800
108 5061200
109 4065600
110 3948300
111 5492400
112 6815200
113 4162700
114 10287200
115 19116000
116 5162800
117 5390800
118 9111000
119 8332200
120 6176400
121 6007000
122 5449500
123 9217100
124 26947100
125 23633100
126 7943900
127 9619000
128 7515000
129 7323100
130 6271200
131 10975700
132 5934800
133 13255700
134 8424800
135 5537500
136 20382400
137 15957800
138 9571500
139 11587400
140 5961900
141 7573700
142 7627900
143 34868400
144 19389200
145 7959700
146 5842300
147 22768500
148 7232800
149 7546600
150 16027800
151 5833200
152 5336600
153 7476600
154 7512800
155 6573700
156 4891900
157 5083800
158 4998000
159 5007000
160 5165000
161 24100400
162 19050500
163 6097300
164 8512800
165 14923900
166 9695700
167 9673100
168 13014100
169 9045600
170 15720800
171 25333000
172 21901700
173 29845600
174 12422700
175 11126900
176 29179700
177 12285000
178 31793800
179 7835600
180 7106400
181 9212600
182 14449900
183 22425400
184 24443600
185 36209300
186 23874700
187 11655200
188 11354900
189 8605400
190 31960800
191 32662900
192 23240300
193 113027800
194 30163900
195 17425200
196 9767900
197 13793000
198 9176500
199 48431100
200 56194500
201 19000900
202 34642700
203 14820100
204 11009500
205 16849500
206 26134400
207 33843500
208 32130100
209 16087400
210 17438700
211 14080400
212 15384500
213 40475900
214 18610300
215 11987000
216 10382000
217 12513000
218 23102600
219 67402600
220 16682500
221 15589900
222 22150000
223 64014200
224 18127200
225 11294600
226 10635400
227 18295300
228 23097100
229 8678100
230 8156300
231 14513900
232 7707800
233 7069100
234 6833500
235 6511900
236 6344300
237 7089500
238 23791600
239 10674900
240 33965900
241 8903700
242 8915100
243 6704400
244 6631900
245 30353300
246 41741700
247 41295500
248 7716900
249 7114400
250 6434900
251 6253700
252 6063400
253 6285400
254 14942200
255 8441700
256 15918400
257 6738400
258 6738400
259 9590000
260 6439400
261 6058900
262 5855000
263 5691900
264 5549300
265 5404300
266 5284200
267 5207200
268 5669300
269 5574200
270 7710100
271 5907100
272 10505100
273 12550400
274 26328400
275 21633000
276 8083800
277 18942200
278 7873100
279 8684000
280 8011300
281 6641000
282 6031700
283 34190200
284 19821000
285 31746200
286 18552600
287 7839200
288 12471100
289 8468800
290 22822100
291 17691900
292 8228700
293 7429200
294 9578700
295 11988600
296 19771200
297 13257000
298 12799500
299 26049800
300 8720300
301 18244600
302 16595700
303 8638700
304 7465400
305 14095100
306 23268300
307 8117800
308 7275200
309 6851600
310 8199300
311 6978500
312 21503900
313 50912700
314 44235500
315 13080400
316 10482400
317 7288800
318 7094000
319 7544700
320 9150600
321 8079300
322 7254800
323 23109800
324 19705500
325 9900300
326 8115500
327 12027200
328 11605900
329 8681700
330 8289900
331 9282000
332 9334100
333 6591200
334 9687400
335 12131300
336 7669300
337 22799500
338 11479000
339 6181200
340 10036200
341 8566200
342 6348800
343 8690800
344 6029400
345 5472200
346 5182300
347 25941000
348 22471100
349 6924100
350 10926400
351 11340900
352 11861800
353 10104200
354 10158500
355 11315900
356 29934200
357 8493800
358 7152900
359 22946700
360 11884500
361 9943400
362 10604700
363 7551500
364 6679500
365 10824400
366 8539100
367 10095100
368 6169900
369 6054300
370 5782500
371 5451900
372 7485800
373 8675000
374 5766700
375 7368000
376 17732700
377 15408800
378 14600200
379 16423500
380 10514100
381 6847100
382 11556000
383 14274000
384 6534500
385 6545900
386 7393000
387 12083800
388 6767800
389 6280800
390 5789300
391 5710100
392 5662500
393 5395200
394 5094000
395 4915100
396 4906000
397 5605900
398 6364700
399 8364600
400 5513000
401 4910500
402 4724800
403 4636500
404 4645500
405 4625100
406 4575300
407 4518700
408 6396400
409 5037400
410 4656800
411 5141600
412 4847100
413 4466600
414 4971700
415 7485800
416 4645500
417 4308000
418 4185700
419 4208400
420 9963700
421 4541300
422 4280900
423 4249100
424 4262700
425 4792700
426 4940000
427 12004500
428 4865200
429 4317100
430 4405400
431 4178900
432 4077000
433 3970500
434 3990900
435 4215200
436 3959200
437 3877700
438 4516400
439 4643300
440 3959200
};
\addplot [blue]
table {%
0 16741617.6306149
1 14794537.0119485
2 11116736.5978385
3 22881398.6587559
4 10862853.9559898
5 11756079.2040898
6 9912793.75168397
7 9637317.02289481
8 7436075.44778281
9 8693757.37176492
10 11038076.7494436
11 9625152.94263056
12 7273057.9232693
13 10710860.1203536
14 9252276.91549008
15 9011720.17760852
16 5748325.36810795
17 6784229.486122
18 8889025.07871816
19 9059702.92637735
20 8142173.10594504
21 5695628.49625022
22 5456453.67852928
23 7141127.80296532
24 7290739.26326678
25 6100779.24241973
26 5720861.15166478
27 2559060.53177878
28 4505718.9687531
29 5283227.72150452
30 4744172.11089837
31 5407269.5348693
32 5057704.96136532
33 5992705.21157252
34 4606806.97647257
35 4963598.66093337
36 3903320.49691103
37 4755767.18817779
38 4685190.19705777
39 5635690.92756503
40 6664046.91951597
41 5269471.15000439
42 6888288.73308916
43 5331728.07745474
44 3711248.61320507
45 7108525.86501894
46 7274398.74447294
47 6249991.37581053
48 7342923.5154733
49 6633693.73002447
50 8208348.463198
51 9458053.30728421
52 6872601.50456166
53 5420159.24007628
54 4087195.46760585
55 3191593.73189132
56 5498155.84642128
57 11146460.0660418
58 12091351.6625451
59 16365663.3633678
60 12445554.5106807
61 8318380.60277989
62 6764163.31210244
63 5134389.31742434
64 5995134.91591342
65 4348635.8393183
66 4795444.80557506
67 4140789.05959375
68 3069816.91907504
69 2754785.6178882
70 4080966.29231971
71 4485173.69986827
72 6909646.38272495
73 7484056.8339755
74 8414609.50974051
75 4786427.98452724
76 4685388.25209842
77 3531423.35826787
78 4442625.16493415
79 3461897.34970092
80 2832532.72524851
81 4525857.49821363
82 4769989.45177718
83 5459022.64996904
84 3786759.84948323
85 7123491.93979526
86 5220405.20512629
87 3314686.13945798
88 5049569.38666342
89 3698192.1648758
90 3326257.74871249
91 4845612.61979155
92 2705081.0572239
93 2079014.0395317
94 1953766.37970277
95 816710.223402533
96 1051158.17085538
97 2692119.03254059
98 6006130.30630387
99 4202314.63831697
100 3945345.79679654
101 2700816.57407194
102 5142505.3998993
103 2809063.53072061
104 4519999.40864573
105 14939151.3348478
106 17480518.2591927
107 7540280.09536742
108 7059085.1900304
109 3876910.59640093
110 6127379.65342075
111 3928342.00050963
112 4087420.07942659
113 3077494.12190068
114 7094468.32966486
115 14011344.3426719
116 11094159.1269536
117 5846233.95866538
118 10207697.1618731
119 6068340.5074711
120 10283636.5551377
121 5454714.23987512
122 7306674.18024546
123 9687010.98395565
124 14572711.5270849
125 16167398.2707537
126 18226352.7544021
127 12681169.6686587
128 12284591.3708591
129 15538817.70325
130 8287053.74848787
131 18562929.993428
132 7721532.81415535
133 8146549.33245267
134 13086694.5935841
135 12402837.4093042
136 7282198.99646311
137 13740991.816385
138 14466364.042946
139 11766501.8578735
140 15137054.6252374
141 13538578.3328251
142 15454660.2247238
143 15024777.6268277
144 20019176.2446977
145 12481592.5032512
146 15163740.9471783
147 16517866.3615138
148 13967446.1444323
149 14111788.9717492
150 11114431.4972986
151 3398238.10112796
152 11881261.6752673
153 17629143.4202082
154 10808260.0476236
155 14775789.2576983
156 24656075.7529358
157 9077656.37697818
158 11855058.3582044
159 8589091.09170616
160 6324480.72894372
161 8710120.46404816
162 10853247.7036058
163 11253077.8850887
164 7672769.87353489
165 9881002.74797484
166 15801942.7540276
167 10614381.7003922
168 15210681.6018926
169 11873430.2004245
170 19132412.4281311
171 22571169.7864887
172 28670297.4793524
173 26906517.0916568
174 14818029.6540225
175 12417156.4423615
176 14743693.1283802
177 13828692.002366
178 22891359.8075716
179 7831187.60643924
180 11992438.4486888
181 17298097.1788889
182 13987830.0235928
183 18702996.7331111
184 21512683.2182362
185 32284769.1552129
186 37817835.0874667
187 12682315.3007111
188 13027708.9255312
189 11652711.451758
190 12678155.041768
191 21519679.2881199
192 24111793.4777128
193 41433714.9799368
194 37009235.831294
195 19295570.3096787
196 19654272.8846891
197 16917249.1221489
198 13970820.675741
199 24467514.6555643
200 53253575.391372
201 25637764.3039924
202 16942664.9396476
203 15836626.4976378
204 20863983.3659418
205 16540694.0086354
206 22980567.7807034
207 22535290.3894846
208 28210892.7667081
209 15435441.3646961
210 20541707.8254459
211 40474247.8469006
212 14688889.2444611
213 23071234.4062694
214 10229406.3920812
215 12481555.4946522
216 12637630.1420516
217 11558542.4018327
218 29854620.1052421
219 31348610.6944364
220 24777381.2414228
221 12679851.8071615
222 18695158.3704915
223 27713519.6888474
224 19344749.3542657
225 18414643.1541949
226 11164276.526802
227 15756081.0627882
228 12768660.772396
229 14748247.7061436
230 10357288.1254941
231 11964485.0999476
232 7724437.60766799
233 9666392.93000696
234 7153118.95474069
235 7423594.56140242
236 10511173.1594718
237 11077038.6030774
238 24454039.569365
239 14822014.5414177
240 30979953.6408902
241 13840297.4062018
242 13508609.4498107
243 8045832.99600352
244 10281136.6084688
245 17643032.2142304
246 26180332.9810304
247 22097352.5410084
248 11960917.8552635
249 6654023.61435388
250 7429124.40012589
251 10770672.5653984
252 9851895.45250647
253 8320209.738484
254 23656952.0855015
255 22237849.514153
256 17316127.7077601
257 15070619.6548754
258 14181757.618069
259 22552145.8868628
260 9547839.80697949
261 10886765.9773739
262 9557674.08305113
263 4504055.83845966
264 589447.445402256
265 8705501.20837534
266 4631421.29878683
267 2279659.93751341
268 7440478.9051268
269 9466285.90410108
270 11259582.9922489
271 11134074.2843213
272 13844471.6108583
273 11735642.0753861
274 13321837.7716904
275 20084383.2337746
276 14035912.3348026
277 21927382.2859348
278 13333123.8007877
279 12578593.2612224
280 14775019.9128663
281 11379253.7486912
282 14386171.6209366
283 11808573.1251625
284 13184701.131002
285 9783434.50792663
286 15917610.1711732
287 15060656.3144127
288 9196825.41401001
289 12772782.8824562
290 11914078.8319018
291 18120708.4104864
292 13289149.6736648
293 9629472.0554562
294 9467343.01032915
295 14685744.3487184
296 16251380.1358089
297 18748216.4250809
298 16512583.4728985
299 15745525.0650424
300 15870831.8604209
301 13537572.4977101
302 20475939.9702723
303 15036464.444106
304 7994687.83379279
305 8332902.34703181
306 12976584.2598646
307 10678797.3182182
308 8036254.00991759
309 8035248.40253174
310 10846601.1077768
311 12355120.6780847
312 14553174.5737045
313 17902078.8883518
314 44843290.5491097
315 17662084.1308735
316 13174679.3337463
317 20982846.4375154
318 8250972.95362743
319 8794094.89614557
320 11091128.2705103
321 10198467.5096854
322 9086121.67184957
323 14362268.7169022
324 16329134.9781652
325 13750535.9549945
326 11382569.8314471
327 17125771.2318138
328 20788265.8918331
329 12273145.4465769
330 11447026.479076
331 15058455.3104751
332 13503494.7808743
333 11144051.1079645
334 12118477.6168008
335 13624503.5919066
336 12672039.4848173
337 12626211.8433153
338 14178832.9826336
339 7886363.84339595
340 10624582.5312304
341 15310292.5122736
342 13503767.3999144
343 8116779.60179909
344 9316079.03040858
345 8564533.15469158
346 7335240.21230509
347 11057895.2340894
348 19691104.1645526
349 8846319.84530781
350 15898409.677898
351 18630890.2447333
352 17821665.4721866
353 10363430.2031629
354 12370679.8377123
355 11194297.4555534
356 14730442.9982773
357 9791810.57096562
358 9143555.20848592
359 11020451.1868161
360 9464933.64752774
361 11411937.9602033
362 11667331.1049705
363 8118550.91811559
364 6257126.93871824
365 10274787.087595
366 6370660.71929226
367 10450917.585455
368 10639142.0209318
369 8038480.17288477
370 7924632.3550491
371 7785007.86098587
372 7069431.11539175
373 8624729.02183302
374 9425229.64304384
375 12547468.2007871
376 17138830.4380864
377 13518590.6747026
378 8546527.68631615
379 6069109.54165559
380 4087722.08827614
381 6084096.96151443
382 8230671.92742219
383 7601284.1181462
384 8724136.40097314
385 8951573.57994605
386 9242279.41758871
387 8257625.47708441
388 8429984.80435794
389 8255228.67618069
390 9782020.14277939
391 5973390.62666014
392 7012149.69406604
393 5909892.78787048
394 7343265.33272353
395 6028932.79126561
396 6334619.71841604
397 5024898.71271853
398 5739990.13948472
399 6879890.58177681
400 5174624.01074264
401 5178260.81111684
402 4138222.70165893
403 5374571.1043037
404 7925891.10819822
405 7023035.11260027
406 9221072.52722329
407 6164032.65090576
408 4962108.48264212
409 5268639.9215534
410 7616472.65962418
411 7314534.11131214
412 3856353.63926288
413 5870642.7335789
414 5433048.1496416
415 7868760.92404132
416 7882230.63849655
417 4876046.95213273
418 4537242.94425931
419 2970930.56034695
420 3341268.31697037
421 4221795.75218834
422 4909468.05595287
423 6082041.48179708
424 4615837.23137485
425 3898770.02604107
426 4742259.02525468
427 3960391.74814353
428 3551197.51863669
429 4701194.41553602
430 3830160.06089725
431 7069688.00915394
432 4447407.47669873
433 3928489.20445722
434 4439675.13183228
435 3595309.69053955
436 5387815.03425416
437 5633952.17061659
438 4095266.22313538
439 4130298.5525599
440 4297211.10850723
};


\addlegendentry{Test data}
\addlegendentry{LMCGP mean}
\addlegendimage{line width=5pt,draw=blue,opacity=0.5}
\addlegendentry{$95\%$ LMCGP CI}

\end{axis}

\end{tikzpicture}
}
%	\end{figure}
%	\vspace{-1.5em}
%	\begin{block}{}
%		Test data for two reservoirs in one day ahead LMCGP model prediction. The model more accurately follows the peaks due its complex behavior.
%	\end{block}
%\end{frame}

\begin{frame}{One-day-ahead Models Forecasting}
	\centering
	\begin{figure}[htbp]
		\tiny
		\setlength\figurewidth{0.55\columnwidth} 
		\setlength\figureheight{0.5\columnwidth}
		% This file was created with tikzplotlib v0.10.1.
\begin{tikzpicture}

\definecolor{darkgray176}{RGB}{176,176,176}
\definecolor{goldenrod1911910}{RGB}{191,191,0}
\definecolor{green01270}{RGB}{0,127,0}


\begin{axis}[
	width=\figurewidth,
	height=\figureheight,
	axis background/.style={fill=background_color},
	axis line style={white},
	tick align=inside,
	tick pos=left,
	%	ylabel={Contributions (kWh)},
	ylabel style={font=\tiny},
	xlabel={Time (days)},
	xlabel style={font=\tiny},
	x grid style={white},
	xmajorgrids,
	xmin=150, xmax=300,
	xtick style={color=black},
	xticklabel style={font=\tiny},
	y grid style={white},
	ymajorgrids,
	ymin=-0.1e8, ymax=0.6e8, % Set the y-axis limits here
	ytick style={color=black},
	ytick={-1e7,0,1e7,2e7,3e7,4e7,5e7,6e7}, % Define tick marks for y-axis
	yticklabels={\tiny-0.1,\tiny0,\tiny0.1,\tiny0.2,\tiny0.3,\tiny0.4,\tiny0.5,\tiny0.6}, % Corresponding labels
	yticklabel style={font=\tiny},
	legend columns=2, 
	legend style={
		nodes={scale=0.8, transform shape},
		font=\tiny,
		fill opacity=0.8,
		draw opacity=1,
		text opacity=1,
		at={(0.03,0.97)},
		anchor=north west,
		legend pos=north east,
		draw=lightgray204,
		fill=background_color
	},
	]

\path [draw=blue, fill=blue, opacity=0.5]
(axis cs:0,20567813.5861097)
--(axis cs:0,-457851.163993286)
--(axis cs:1,4379547.55076812)
--(axis cs:2,5798994.60948945)
--(axis cs:3,1842958.24257836)
--(axis cs:4,648180.343430167)
--(axis cs:5,770243.110356908)
--(axis cs:6,449711.059269443)
--(axis cs:7,-1005650.70936927)
--(axis cs:8,-2461976.57896788)
--(axis cs:9,-2257891.2229201)
--(axis cs:10,-2108172.04659882)
--(axis cs:11,-1251596.87516947)
--(axis cs:12,-2602830.21175721)
--(axis cs:13,-3021153.03376156)
--(axis cs:14,-3505559.15703869)
--(axis cs:15,-3422748.09515213)
--(axis cs:16,-2698161.28483729)
--(axis cs:17,846823.001511512)
--(axis cs:18,-2540128.73622349)
--(axis cs:19,-2335553.57571424)
--(axis cs:20,-3171943.8347535)
--(axis cs:21,881363.932491079)
--(axis cs:22,-2076545.16415234)
--(axis cs:23,-3106389.41287656)
--(axis cs:24,190888.196209187)
--(axis cs:25,-2168342.48670729)
--(axis cs:26,-1587875.06201098)
--(axis cs:27,-680754.416065868)
--(axis cs:28,-3510154.27658349)
--(axis cs:29,-3835116.49530571)
--(axis cs:30,-3766905.59324834)
--(axis cs:31,-4592993.96524563)
--(axis cs:32,-4645861.21110778)
--(axis cs:33,-3136855.89473533)
--(axis cs:34,-3631946.25328285)
--(axis cs:35,-4323892.97510871)
--(axis cs:36,-3281956.58132615)
--(axis cs:37,-5165564.70775089)
--(axis cs:38,-4767641.72407158)
--(axis cs:39,-4656967.94574451)
--(axis cs:40,-4846653.60794884)
--(axis cs:41,-5446257.27289035)
--(axis cs:42,-4548045.89127652)
--(axis cs:43,-4840610.77120704)
--(axis cs:44,-5140232.84052945)
--(axis cs:45,-4159138.1469877)
--(axis cs:46,-5012942.06012804)
--(axis cs:47,-5674361.23928053)
--(axis cs:48,-5643428.31577908)
--(axis cs:49,-4856103.2554688)
--(axis cs:50,-5704705.56061951)
--(axis cs:51,-4520834.87187164)
--(axis cs:52,-4648669.18999553)
--(axis cs:53,-5674198.08292057)
--(axis cs:54,-6717557.10923175)
--(axis cs:55,-6381476.21140914)
--(axis cs:56,-5886877.24971247)
--(axis cs:57,-1685745.12594382)
--(axis cs:58,-3802408.04049428)
--(axis cs:59,-1258043.74578805)
--(axis cs:60,-1774735.72337276)
--(axis cs:61,-4120161.10289316)
--(axis cs:62,-4413628.68745106)
--(axis cs:63,-4706356.25198635)
--(axis cs:64,-5864780.73777867)
--(axis cs:65,-5751926.10136926)
--(axis cs:66,-5764132.42070409)
--(axis cs:67,-6043691.93486427)
--(axis cs:68,-6180519.98538256)
--(axis cs:69,-6252439.36438838)
--(axis cs:70,-6127105.62522247)
--(axis cs:71,-6092622.78926599)
--(axis cs:72,-5997998.50350158)
--(axis cs:73,-4430705.99348932)
--(axis cs:74,-5300908.19847347)
--(axis cs:75,-5308693.16332413)
--(axis cs:76,-6141981.00245738)
--(axis cs:77,-6521372.16493178)
--(axis cs:78,-6415053.14966976)
--(axis cs:79,-6498708.00245543)
--(axis cs:80,-6624851.75917559)
--(axis cs:81,-6419864.16650434)
--(axis cs:82,-4795660.15298983)
--(axis cs:83,-5154533.44076017)
--(axis cs:84,-5404213.59732283)
--(axis cs:85,-2073408.10373814)
--(axis cs:86,-4510935.54500436)
--(axis cs:87,-6243985.98327141)
--(axis cs:88,-6022629.21973124)
--(axis cs:89,-5308299.86310456)
--(axis cs:90,-4266185.04207257)
--(axis cs:91,-6165041.89093815)
--(axis cs:92,-6285318.3951759)
--(axis cs:93,-6662957.14722593)
--(axis cs:94,-6583907.35326646)
--(axis cs:95,-6927856.28687304)
--(axis cs:96,-6829937.10369222)
--(axis cs:97,-7065693.74867333)
--(axis cs:98,-5775906.280058)
--(axis cs:99,-5988300.89334308)
--(axis cs:100,-6725261.00620497)
--(axis cs:101,-5746326.14401152)
--(axis cs:102,-6453777.94737232)
--(axis cs:103,-5907391.51245219)
--(axis cs:104,-5544910.15669456)
--(axis cs:105,-4132204.5336014)
--(axis cs:106,756631.587565307)
--(axis cs:107,599182.88311285)
--(axis cs:108,-4216217.65254998)
--(axis cs:109,-5974783.86014221)
--(axis cs:110,-5161473.72490173)
--(axis cs:111,-5206028.00246126)
--(axis cs:112,-5148279.99423728)
--(axis cs:113,-4130283.71293343)
--(axis cs:114,-5912969.55530582)
--(axis cs:115,-1206567.90702783)
--(axis cs:116,-206630.45992321)
--(axis cs:117,-4544840.41136602)
--(axis cs:118,-4171808.20702549)
--(axis cs:119,-3255911.88245623)
--(axis cs:120,-2110692.65696449)
--(axis cs:121,-2770517.21632921)
--(axis cs:122,-3337506.54310994)
--(axis cs:123,-4276915.48188306)
--(axis cs:124,1448186.79868233)
--(axis cs:125,4652806.77698381)
--(axis cs:126,3861352.73509552)
--(axis cs:127,1976726.41673387)
--(axis cs:128,3191543.22563768)
--(axis cs:129,1445393.4411639)
--(axis cs:130,-902965.534040298)
--(axis cs:131,559963.253029941)
--(axis cs:132,890631.999782026)
--(axis cs:133,-1628673.95424007)
--(axis cs:134,239214.56711679)
--(axis cs:135,1203329.8279615)
--(axis cs:136,-3045424.76033118)
--(axis cs:137,3705234.5153756)
--(axis cs:138,1279639.10077686)
--(axis cs:139,-1088381.08050959)
--(axis cs:140,1093045.49587454)
--(axis cs:141,-2775138.74518299)
--(axis cs:142,-667391.761495819)
--(axis cs:143,-734881.062573425)
--(axis cs:144,2482408.08741663)
--(axis cs:145,2495177.11647915)
--(axis cs:146,1076994.07216818)
--(axis cs:147,-922032.918605417)
--(axis cs:148,7063760.74516851)
--(axis cs:149,448739.728201555)
--(axis cs:150,660008.080641745)
--(axis cs:151,1901175.5209676)
--(axis cs:152,-2414956.66947538)
--(axis cs:153,-2469949.13650539)
--(axis cs:154,-708432.95392473)
--(axis cs:155,-198358.076952763)
--(axis cs:156,809399.913601086)
--(axis cs:157,-261600.377892243)
--(axis cs:158,-3744083.87975491)
--(axis cs:159,-3666231.66945427)
--(axis cs:160,-3626244.9078111)
--(axis cs:161,-2584108.31700201)
--(axis cs:162,404887.206128825)
--(axis cs:163,3167444.27019527)
--(axis cs:164,-2602466.5175267)
--(axis cs:165,-1076266.51137068)
--(axis cs:166,1440541.44673666)
--(axis cs:167,880406.415013889)
--(axis cs:168,1594378.53861047)
--(axis cs:169,2228717.11473723)
--(axis cs:170,767148.616965733)
--(axis cs:171,5181158.6011894)
--(axis cs:172,7306794.81572697)
--(axis cs:173,2708418.70272985)
--(axis cs:174,6852793.19129355)
--(axis cs:175,3562220.330872)
--(axis cs:176,5000021.37367984)
--(axis cs:177,11368032.5645057)
--(axis cs:178,5415890.9526477)
--(axis cs:179,6095172.69386468)
--(axis cs:180,346790.970387992)
--(axis cs:181,-734265.861954018)
--(axis cs:182,-795266.067483995)
--(axis cs:183,2928370.04084115)
--(axis cs:184,310290.200829802)
--(axis cs:185,794145.72504483)
--(axis cs:186,5582632.39810256)
--(axis cs:187,1660194.49013201)
--(axis cs:188,2448160.61968907)
--(axis cs:189,2302203.69188063)
--(axis cs:190,-312525.432705095)
--(axis cs:191,3829869.67568848)
--(axis cs:192,2185845.0198232)
--(axis cs:193,3422506.65714097)
--(axis cs:194,-2875557.69581851)
--(axis cs:195,3330805.41862599)
--(axis cs:196,5017012.42054473)
--(axis cs:197,2145412.93721954)
--(axis cs:198,4962710.2700887)
--(axis cs:199,2912337.91210424)
--(axis cs:200,4593416.13075422)
--(axis cs:201,2007340.74531968)
--(axis cs:202,5791574.97070479)
--(axis cs:203,9958776.66973711)
--(axis cs:204,3832850.54200688)
--(axis cs:205,2782635.46979772)
--(axis cs:206,5279764.74609653)
--(axis cs:207,8570263.89891973)
--(axis cs:208,7688903.32257591)
--(axis cs:209,8981755.65423668)
--(axis cs:210,7918697.24573993)
--(axis cs:211,3636204.10706673)
--(axis cs:212,1544383.08848532)
--(axis cs:213,1842195.75270413)
--(axis cs:214,7753554.31275103)
--(axis cs:215,2352688.0337412)
--(axis cs:216,4636298.07425619)
--(axis cs:217,3372059.04822173)
--(axis cs:218,3157775.35160619)
--(axis cs:219,864587.874175165)
--(axis cs:220,-2485178.87993928)
--(axis cs:221,-238615.873827768)
--(axis cs:222,1363444.29950674)
--(axis cs:223,5376063.55334506)
--(axis cs:224,2004659.51536958)
--(axis cs:225,6763791.99062668)
--(axis cs:226,1694784.57232471)
--(axis cs:227,1067029.79819494)
--(axis cs:228,5414793.1191644)
--(axis cs:229,6564501.40076189)
--(axis cs:230,411701.055876773)
--(axis cs:231,-563873.269205032)
--(axis cs:232,-1653521.21716839)
--(axis cs:233,-1040614.98864784)
--(axis cs:234,-1785701.93215009)
--(axis cs:235,-2292809.05117698)
--(axis cs:236,-2410823.09581087)
--(axis cs:237,-2565785.10298668)
--(axis cs:238,-2170845.04218552)
--(axis cs:239,657505.213229347)
--(axis cs:240,-957268.530488774)
--(axis cs:241,913809.076981004)
--(axis cs:242,-1952191.68669192)
--(axis cs:243,-2897824.65334037)
--(axis cs:244,-3920401.47434827)
--(axis cs:245,-1880150.33620658)
--(axis cs:246,2726689.5116163)
--(axis cs:247,2936125.86847129)
--(axis cs:248,-1261844.04651939)
--(axis cs:249,-3721757.68209203)
--(axis cs:250,-3992856.64830083)
--(axis cs:251,-4727199.40703468)
--(axis cs:252,-4649697.42533202)
--(axis cs:253,-5342443.57122845)
--(axis cs:254,-2222361.65035468)
--(axis cs:255,1929084.52002756)
--(axis cs:256,-776997.7138004)
--(axis cs:257,2149588.57283299)
--(axis cs:258,-2243050.04728639)
--(axis cs:259,239965.923569988)
--(axis cs:260,-4295956.81890338)
--(axis cs:261,-3675893.97157736)
--(axis cs:262,-3803502.15789793)
--(axis cs:263,-4277013.89884523)
--(axis cs:264,-4681695.97351347)
--(axis cs:265,-4260818.7437979)
--(axis cs:266,-4050241.60226333)
--(axis cs:267,-3834070.14535276)
--(axis cs:268,-3726501.80190208)
--(axis cs:269,-3001535.37466138)
--(axis cs:270,-2565622.56696047)
--(axis cs:271,-725042.271519611)
--(axis cs:272,-2879599.05193483)
--(axis cs:273,1492025.27866454)
--(axis cs:274,2723967.6494226)
--(axis cs:275,3663002.69195927)
--(axis cs:276,6288762.96788061)
--(axis cs:277,3043593.41175102)
--(axis cs:278,3753274.7857041)
--(axis cs:279,170560.934292248)
--(axis cs:280,2056331.97399582)
--(axis cs:281,2644508.46500712)
--(axis cs:282,-44519.6313869376)
--(axis cs:283,-284856.886745505)
--(axis cs:284,1324652.82049185)
--(axis cs:285,3560451.86321047)
--(axis cs:286,5537488.9286017)
--(axis cs:287,5082262.78434371)
--(axis cs:288,-56957.872794833)
--(axis cs:289,928442.496084996)
--(axis cs:290,-231240.585890986)
--(axis cs:291,3677406.4819505)
--(axis cs:292,4868535.40588061)
--(axis cs:293,2503642.23652991)
--(axis cs:294,1221799.49041558)
--(axis cs:295,992522.598322365)
--(axis cs:296,2121840.71310183)
--(axis cs:297,945419.650760872)
--(axis cs:298,2658173.0676718)
--(axis cs:299,1360127.43227344)
--(axis cs:300,5055420.65560201)
--(axis cs:301,689807.561724951)
--(axis cs:302,4316912.03218145)
--(axis cs:303,5608853.66116197)
--(axis cs:304,2079512.71656168)
--(axis cs:305,-697597.207936317)
--(axis cs:306,2460496.31065578)
--(axis cs:307,6162597.77512236)
--(axis cs:308,542833.02257734)
--(axis cs:309,-1042365.56786553)
--(axis cs:310,-1313770.46599596)
--(axis cs:311,288105.605476536)
--(axis cs:312,-837669.712932564)
--(axis cs:313,384095.097007589)
--(axis cs:314,750956.19479719)
--(axis cs:315,4277037.25837336)
--(axis cs:316,3468360.78603218)
--(axis cs:317,2228662.36568047)
--(axis cs:318,-461727.814759752)
--(axis cs:319,-1039856.49340293)
--(axis cs:320,113583.158395682)
--(axis cs:321,317749.638408436)
--(axis cs:322,-702032.845606195)
--(axis cs:323,-1113549.22613245)
--(axis cs:324,1730789.75219064)
--(axis cs:325,4016594.02153596)
--(axis cs:326,389794.249960586)
--(axis cs:327,501545.915838685)
--(axis cs:328,321525.072389957)
--(axis cs:329,-1540930.04813397)
--(axis cs:330,-67762.567014128)
--(axis cs:331,-457492.899342567)
--(axis cs:332,-1322633.99091335)
--(axis cs:333,470983.56416201)
--(axis cs:334,-1682303.19921342)
--(axis cs:335,740188.368959615)
--(axis cs:336,1495379.75374007)
--(axis cs:337,-305351.172719695)
--(axis cs:338,3676605.79614777)
--(axis cs:339,2062521.60760278)
--(axis cs:340,-765405.811756965)
--(axis cs:341,-1305447.38559785)
--(axis cs:342,-1471197.69526924)
--(axis cs:343,-2523737.42156704)
--(axis cs:344,-1941033.84839337)
--(axis cs:345,-2289832.94877388)
--(axis cs:346,-2841859.64160595)
--(axis cs:347,-3462811.90541187)
--(axis cs:348,5012448.29392079)
--(axis cs:349,5227275.4752994)
--(axis cs:350,-1387994.82165509)
--(axis cs:351,2463880.20909783)
--(axis cs:352,4444244.20959021)
--(axis cs:353,3348931.54886066)
--(axis cs:354,1590015.65603757)
--(axis cs:355,2239902.84350292)
--(axis cs:356,3093117.40969291)
--(axis cs:357,6801563.59939758)
--(axis cs:358,-1047579.06461357)
--(axis cs:359,-833132.976984704)
--(axis cs:360,2712990.39465247)
--(axis cs:361,1294413.50693234)
--(axis cs:362,-471382.971580721)
--(axis cs:363,813393.797896964)
--(axis cs:364,-1184449.43068807)
--(axis cs:365,-2336477.92649151)
--(axis cs:366,-2278650.37880598)
--(axis cs:367,-1750572.71509925)
--(axis cs:368,-630539.338279866)
--(axis cs:369,-3092645.20703212)
--(axis cs:370,-3564482.47034445)
--(axis cs:371,-4214916.38796306)
--(axis cs:372,-4603885.48665718)
--(axis cs:373,-2868648.73759504)
--(axis cs:374,-2119663.56377276)
--(axis cs:375,-2209524.57616181)
--(axis cs:376,-2004813.78642532)
--(axis cs:377,3326460.3830375)
--(axis cs:378,6966439.82418853)
--(axis cs:379,3542662.0670127)
--(axis cs:380,3625728.32108727)
--(axis cs:381,457739.515311543)
--(axis cs:382,-2386079.94543634)
--(axis cs:383,-458840.431189843)
--(axis cs:384,24199.1491447911)
--(axis cs:385,-2906996.74127531)
--(axis cs:386,-2709830.05964686)
--(axis cs:387,-1918076.58828717)
--(axis cs:388,-45997.0821049325)
--(axis cs:389,-2646599.54926826)
--(axis cs:390,-2497719.02391191)
--(axis cs:391,-3823530.11131555)
--(axis cs:392,-4365608.7801052)
--(axis cs:393,-4513437.57799164)
--(axis cs:394,-5041255.09310351)
--(axis cs:395,-5791665.01094564)
--(axis cs:396,-5108816.89079137)
--(axis cs:397,-5304093.23032886)
--(axis cs:398,-4881334.66822381)
--(axis cs:399,-4480292.45569259)
--(axis cs:400,-3445643.47612115)
--(axis cs:401,-5586316.95515614)
--(axis cs:402,-5787995.530784)
--(axis cs:403,-6420723.10793686)
--(axis cs:404,-5634716.72495943)
--(axis cs:405,-5459455.29349363)
--(axis cs:406,-5396236.82165925)
--(axis cs:407,-5371975.40260226)
--(axis cs:408,-5561593.82205096)
--(axis cs:409,-3963148.21994973)
--(axis cs:410,-3927154.26375179)
--(axis cs:411,-4379127.51614937)
--(axis cs:412,-5563228.57431301)
--(axis cs:413,-5246023.28836136)
--(axis cs:414,-5925460.77555143)
--(axis cs:415,-5070236.60032986)
--(axis cs:416,-3992331.54497015)
--(axis cs:417,-5996487.36883271)
--(axis cs:418,-6232501.67796589)
--(axis cs:419,-6059200.85659184)
--(axis cs:420,-6358032.19379449)
--(axis cs:421,-3269061.1050918)
--(axis cs:422,-5232582.12986724)
--(axis cs:423,-5632818.48748321)
--(axis cs:424,-5842462.9006638)
--(axis cs:425,-5947663.82048385)
--(axis cs:426,-5197779.66593442)
--(axis cs:427,-4409394.44318021)
--(axis cs:428,-1355750.55734923)
--(axis cs:429,-4902936.01718779)
--(axis cs:430,-5549280.4755149)
--(axis cs:431,-5598182.24254013)
--(axis cs:432,-5661397.04616395)
--(axis cs:433,-5816722.65985516)
--(axis cs:434,-5855440.2521593)
--(axis cs:435,-5842334.02543801)
--(axis cs:436,-6061352.97697448)
--(axis cs:437,-6030359.28589319)
--(axis cs:438,-6444752.95989352)
--(axis cs:439,-6210768.04907734)
--(axis cs:440,-5788492.90924436)
--(axis cs:440,14745443.3003051)
--(axis cs:440,14745443.3003051)
--(axis cs:439,15329829.4461088)
--(axis cs:438,14001604.6847446)
--(axis cs:437,14557360.5158273)
--(axis cs:436,14474378.5555333)
--(axis cs:435,14609003.5607608)
--(axis cs:434,14593483.0772268)
--(axis cs:433,14621458.033704)
--(axis cs:432,14805370.5996932)
--(axis cs:431,14897507.1391359)
--(axis cs:430,14958490.4526154)
--(axis cs:429,15592117.6974863)
--(axis cs:428,19598793.4533385)
--(axis cs:427,16175208.8158732)
--(axis cs:426,15389845.9072242)
--(axis cs:425,15008941.5279417)
--(axis cs:424,14836154.9709689)
--(axis cs:423,14989832.6196037)
--(axis cs:422,15427770.2082004)
--(axis cs:421,17645253.8058495)
--(axis cs:420,14795652.0014587)
--(axis cs:419,14790932.5346887)
--(axis cs:418,14423736.9993941)
--(axis cs:417,14917589.6025204)
--(axis cs:416,16649570.0227177)
--(axis cs:415,15633968.3572645)
--(axis cs:414,14664495.8555486)
--(axis cs:413,15358243.2494484)
--(axis cs:412,15256297.0333104)
--(axis cs:411,16146631.6167766)
--(axis cs:410,16559529.4115829)
--(axis cs:409,16757216.0415833)
--(axis cs:408,14878480.0125308)
--(axis cs:407,15082645.7093772)
--(axis cs:406,15144349.4865658)
--(axis cs:405,15341310.6895639)
--(axis cs:404,15152715.939938)
--(axis cs:403,14381462.4456936)
--(axis cs:402,14899959.5675186)
--(axis cs:401,15120370.3460822)
--(axis cs:400,17104664.2719325)
--(axis cs:399,16054262.6813341)
--(axis cs:398,15712074.9624487)
--(axis cs:397,15419298.8948457)
--(axis cs:396,16097442.6497114)
--(axis cs:395,15768972.146662)
--(axis cs:394,15894850.6038076)
--(axis cs:393,16069188.0754446)
--(axis cs:392,16159633.8313386)
--(axis cs:391,16611278.797356)
--(axis cs:390,18077502.5781254)
--(axis cs:389,17850511.5474936)
--(axis cs:388,20548562.3886686)
--(axis cs:387,18928091.3930283)
--(axis cs:386,17912856.3244158)
--(axis cs:385,17752796.7996056)
--(axis cs:384,21012836.212902)
--(axis cs:383,20501276.9128934)
--(axis cs:382,18418150.4907527)
--(axis cs:381,21229088.3417466)
--(axis cs:380,25417753.8488478)
--(axis cs:379,25646099.5163965)
--(axis cs:378,28632632.9299319)
--(axis cs:377,24953023.9173427)
--(axis cs:376,19706679.8462528)
--(axis cs:375,18486654.5744325)
--(axis cs:374,18595549.3256686)
--(axis cs:373,17810177.2221866)
--(axis cs:372,16372851.1371476)
--(axis cs:371,16388482.2391857)
--(axis cs:370,16914757.6581229)
--(axis cs:369,17480629.4129563)
--(axis cs:368,19990090.0734573)
--(axis cs:367,19030222.9264378)
--(axis cs:366,19667984.7274958)
--(axis cs:365,18237974.9831669)
--(axis cs:364,19739660.0551635)
--(axis cs:363,21678100.3914496)
--(axis cs:362,21032302.9256613)
--(axis cs:361,22003532.0608905)
--(axis cs:360,24265431.4994737)
--(axis cs:359,19957486.682513)
--(axis cs:358,20301765.3154506)
--(axis cs:357,29333278.5066915)
--(axis cs:356,24102011.6187152)
--(axis cs:355,23055542.0529885)
--(axis cs:354,22575197.1861343)
--(axis cs:353,24967996.5706843)
--(axis cs:352,25442740.6585455)
--(axis cs:351,23343795.593486)
--(axis cs:350,20020468.2555524)
--(axis cs:349,26982620.0769321)
--(axis cs:348,26697892.6858636)
--(axis cs:347,17138124.9078266)
--(axis cs:346,17770456.2258951)
--(axis cs:345,18235899.2855318)
--(axis cs:344,18853034.1589005)
--(axis cs:343,18254013.4677841)
--(axis cs:342,19752633.9968855)
--(axis cs:341,19609634.5114517)
--(axis cs:340,19793436.2509245)
--(axis cs:339,22809515.4639679)
--(axis cs:338,25003155.8939402)
--(axis cs:337,20592635.0729267)
--(axis cs:336,22826510.6028891)
--(axis cs:335,21283225.3566988)
--(axis cs:334,18816824.7567203)
--(axis cs:333,21252258.9536903)
--(axis cs:332,19535043.0966921)
--(axis cs:331,20312609.518302)
--(axis cs:330,20569989.8213966)
--(axis cs:329,20061741.2321975)
--(axis cs:328,22155098.34201)
--(axis cs:327,21392412.855048)
--(axis cs:326,21130909.0188506)
--(axis cs:325,24851012.6389568)
--(axis cs:324,23220317.5485361)
--(axis cs:323,20005679.9464835)
--(axis cs:322,20952811.1621611)
--(axis cs:321,21074171.0511866)
--(axis cs:320,20692872.6006225)
--(axis cs:319,19461886.3399422)
--(axis cs:318,20117615.2861894)
--(axis cs:317,22856929.3402411)
--(axis cs:316,24540475.2287955)
--(axis cs:315,27525189.9746347)
--(axis cs:314,24100775.6356097)
--(axis cs:313,21892742.3485866)
--(axis cs:312,19636244.9094854)
--(axis cs:311,20879035.6173244)
--(axis cs:310,19152631.10482)
--(axis cs:309,19369855.9371832)
--(axis cs:308,20998312.673151)
--(axis cs:307,27335322.5101307)
--(axis cs:306,23051750.3490251)
--(axis cs:305,19796617.4299001)
--(axis cs:304,22961244.857102)
--(axis cs:303,26864724.2026166)
--(axis cs:302,25097078.9707909)
--(axis cs:301,21392507.2882368)
--(axis cs:300,26493811.6879362)
--(axis cs:299,22335020.932939)
--(axis cs:298,23804733.6760705)
--(axis cs:297,22036868.9969103)
--(axis cs:296,22589951.5186431)
--(axis cs:295,21441792.2230152)
--(axis cs:294,21898462.4632792)
--(axis cs:293,23608486.9272564)
--(axis cs:292,26043658.4739686)
--(axis cs:291,24667233.416042)
--(axis cs:290,20268191.5191987)
--(axis cs:289,22234032.1459713)
--(axis cs:288,20476713.1754046)
--(axis cs:287,25900978.8291508)
--(axis cs:286,27622454.7044241)
--(axis cs:285,24846329.7565306)
--(axis cs:284,24029467.3687784)
--(axis cs:283,20703491.2936753)
--(axis cs:282,20888424.5241203)
--(axis cs:281,23663505.2578577)
--(axis cs:280,23368263.9198422)
--(axis cs:279,22149241.371403)
--(axis cs:278,25567071.1004372)
--(axis cs:277,23952355.3366565)
--(axis cs:276,27536268.2836184)
--(axis cs:275,25326585.7378776)
--(axis cs:274,23758187.8463243)
--(axis cs:273,22645572.8996452)
--(axis cs:272,18341635.4029362)
--(axis cs:271,21298440.705493)
--(axis cs:270,18008324.2823228)
--(axis cs:269,17501053.0908919)
--(axis cs:268,16741867.6790792)
--(axis cs:267,16671499.1748016)
--(axis cs:266,16500633.1321812)
--(axis cs:265,16346900.4383558)
--(axis cs:264,16475593.4626487)
--(axis cs:263,16355306.7611344)
--(axis cs:262,16745337.9732892)
--(axis cs:261,17111797.4233793)
--(axis cs:260,18265945.1516499)
--(axis cs:259,21016371.0595821)
--(axis cs:258,18372001.6595159)
--(axis cs:257,23543726.7731218)
--(axis cs:256,20005646.3472692)
--(axis cs:255,23084087.3988612)
--(axis cs:254,19251641.4792702)
--(axis cs:253,16422316.2561897)
--(axis cs:252,17161357.3803281)
--(axis cs:251,16929167.0094113)
--(axis cs:250,17019165.1064012)
--(axis cs:249,17618896.4002477)
--(axis cs:248,22012110.081327)
--(axis cs:247,25893991.982007)
--(axis cs:246,24664648.8811072)
--(axis cs:245,18754985.5880779)
--(axis cs:244,16906383.1280238)
--(axis cs:243,18483306.2150475)
--(axis cs:242,19610875.4566176)
--(axis cs:241,23966922.0567467)
--(axis cs:240,20507495.9313444)
--(axis cs:239,22905237.2548259)
--(axis cs:238,19532986.3973005)
--(axis cs:237,17928794.7056189)
--(axis cs:236,18056155.3401788)
--(axis cs:235,18172057.1911632)
--(axis cs:234,18661560.9069527)
--(axis cs:233,19403720.3961766)
--(axis cs:232,20054727.9063606)
--(axis cs:231,20068929.4121016)
--(axis cs:230,21067099.627894)
--(axis cs:229,27835871.3861002)
--(axis cs:228,26292084.0514839)
--(axis cs:227,21861150.8962083)
--(axis cs:226,22965821.9886409)
--(axis cs:225,27532498.218986)
--(axis cs:224,25382677.2627534)
--(axis cs:223,26678416.7184528)
--(axis cs:222,23226558.0612882)
--(axis cs:221,22896357.0281449)
--(axis cs:220,21018089.1253756)
--(axis cs:219,23386123.0351814)
--(axis cs:218,24669938.9688988)
--(axis cs:217,24530870.4611541)
--(axis cs:216,25976937.2117181)
--(axis cs:215,25006448.1387917)
--(axis cs:214,30664338.6693458)
--(axis cs:213,24661570.9584487)
--(axis cs:212,24543362.6147041)
--(axis cs:211,26451255.4187172)
--(axis cs:210,29432225.5271943)
--(axis cs:209,31762461.6442146)
--(axis cs:208,30461088.0041135)
--(axis cs:207,30141142.5809737)
--(axis cs:206,26533648.0776952)
--(axis cs:205,24169760.2716729)
--(axis cs:204,24933370.0380436)
--(axis cs:203,32149532.8538786)
--(axis cs:202,28289713.571508)
--(axis cs:201,25393905.8615019)
--(axis cs:200,27875007.8172843)
--(axis cs:199,24706336.5519218)
--(axis cs:198,26258927.7082822)
--(axis cs:197,23886315.1637996)
--(axis cs:196,26408573.9896166)
--(axis cs:195,26088254.1003656)
--(axis cs:194,20628548.5469914)
--(axis cs:193,25040186.2273029)
--(axis cs:192,24450980.6674102)
--(axis cs:191,26004551.6361259)
--(axis cs:190,20217180.1794068)
--(axis cs:189,23023651.0102046)
--(axis cs:188,23730321.0228286)
--(axis cs:187,24952275.6855448)
--(axis cs:186,28711434.8413678)
--(axis cs:185,23376074.136161)
--(axis cs:184,22550706.9452235)
--(axis cs:183,23635881.9686943)
--(axis cs:182,20249364.6792889)
--(axis cs:181,19993400.3500406)
--(axis cs:180,21463709.2368113)
--(axis cs:179,28855223.6528093)
--(axis cs:178,26847277.4348443)
--(axis cs:177,33243096.8285464)
--(axis cs:176,26407485.8066734)
--(axis cs:175,25215972.408438)
--(axis cs:174,29103367.8109799)
--(axis cs:173,25412589.3002929)
--(axis cs:172,29803991.807395)
--(axis cs:171,26915663.7803368)
--(axis cs:170,21729642.3514278)
--(axis cs:169,24741173.6795462)
--(axis cs:168,23129428.7558878)
--(axis cs:167,21684118.3498707)
--(axis cs:166,22303471.9298606)
--(axis cs:165,19415989.8079185)
--(axis cs:164,17784370.3853035)
--(axis cs:163,23849788.810057)
--(axis cs:162,22168609.4273174)
--(axis cs:161,17825403.6935759)
--(axis cs:160,16880230.7494419)
--(axis cs:159,16754613.4813434)
--(axis cs:158,16872176.230218)
--(axis cs:157,20667288.0666119)
--(axis cs:156,22328392.7907636)
--(axis cs:155,20426583.9904946)
--(axis cs:154,20250307.0391024)
--(axis cs:153,18348008.9250843)
--(axis cs:152,18099981.3265525)
--(axis cs:151,24072849.330555)
--(axis cs:150,21284535.1348382)
--(axis cs:149,21021400.9410076)
--(axis cs:148,28897211.1686719)
--(axis cs:147,19633868.9615404)
--(axis cs:146,21847897.5712905)
--(axis cs:145,24523483.7690951)
--(axis cs:144,25019544.749624)
--(axis cs:143,19942774.8977432)
--(axis cs:142,20167736.9810622)
--(axis cs:141,17958211.9746924)
--(axis cs:140,21874686.4324527)
--(axis cs:139,19825829.1978133)
--(axis cs:138,22755627.9038919)
--(axis cs:137,24818997.951275)
--(axis cs:136,17630786.1874296)
--(axis cs:135,22437617.9190079)
--(axis cs:134,21586300.4972536)
--(axis cs:133,19454330.1811505)
--(axis cs:132,23280735.5434417)
--(axis cs:131,21847512.0218748)
--(axis cs:130,20809523.2572669)
--(axis cs:129,23108816.7963468)
--(axis cs:128,25290704.7013453)
--(axis cs:127,23784251.2354348)
--(axis cs:126,26528110.0654766)
--(axis cs:125,26521228.6552225)
--(axis cs:124,22735765.2990474)
--(axis cs:123,18142734.7145187)
--(axis cs:122,17528838.0508343)
--(axis cs:121,18881659.6699593)
--(axis cs:120,19608143.0609074)
--(axis cs:119,17505803.3930152)
--(axis cs:118,16541354.152016)
--(axis cs:117,16041223.4016841)
--(axis cs:116,21929939.9632995)
--(axis cs:115,19500869.1776342)
--(axis cs:114,14749583.9820734)
--(axis cs:113,17259821.8695882)
--(axis cs:112,15835138.9748872)
--(axis cs:111,16883184.6437067)
--(axis cs:110,16019829.6323104)
--(axis cs:109,14930333.4771453)
--(axis cs:108,16385248.9330607)
--(axis cs:107,22616182.4113571)
--(axis cs:106,23252283.5623378)
--(axis cs:105,16708842.0636667)
--(axis cs:104,14886771.8151099)
--(axis cs:103,14636661.9028565)
--(axis cs:102,14392840.2625883)
--(axis cs:101,14828507.2263242)
--(axis cs:100,13961834.2701822)
--(axis cs:99,14553268.7583165)
--(axis cs:98,14750461.2077919)
--(axis cs:97,13351495.2287373)
--(axis cs:96,13605980.3588847)
--(axis cs:95,13753259.4306712)
--(axis cs:94,13864368.6528539)
--(axis cs:93,13783470.9260065)
--(axis cs:92,14194141.3000629)
--(axis cs:91,14287627.868541)
--(axis cs:90,16367191.4526384)
--(axis cs:89,15211344.6997998)
--(axis cs:88,14379078.0201046)
--(axis cs:87,14134871.3244115)
--(axis cs:86,15946489.8106093)
--(axis cs:85,18854198.5816276)
--(axis cs:84,15049165.3535934)
--(axis cs:83,15225803.9744175)
--(axis cs:82,15677171.6667603)
--(axis cs:81,13981495.9760559)
--(axis cs:80,13758586.8407444)
--(axis cs:79,13894914.122287)
--(axis cs:78,14004294.2774831)
--(axis cs:77,13889755.7656876)
--(axis cs:76,14255040.3090596)
--(axis cs:75,15148360.1115497)
--(axis cs:74,15219325.6780488)
--(axis cs:73,16023531.0735337)
--(axis cs:72,14692537.3007586)
--(axis cs:71,14283810.9808516)
--(axis cs:70,14345602.8812539)
--(axis cs:69,14165596.0666614)
--(axis cs:68,14216510.5120339)
--(axis cs:67,14389988.0446991)
--(axis cs:66,14677000.1155772)
--(axis cs:65,14688862.0298404)
--(axis cs:64,14610646.8801393)
--(axis cs:63,15800901.6147541)
--(axis cs:62,16055473.9551502)
--(axis cs:61,16629407.09792)
--(axis cs:60,20049939.746391)
--(axis cs:59,19633150.417315)
--(axis cs:58,16743397.9763188)
--(axis cs:57,19193024.464239)
--(axis cs:56,15151689.8866474)
--(axis cs:55,14466658.8976274)
--(axis cs:54,14335293.4670583)
--(axis cs:53,15114640.2575339)
--(axis cs:52,16548670.2257674)
--(axis cs:51,16264154.2531195)
--(axis cs:50,14992690.0656409)
--(axis cs:49,15762930.6626073)
--(axis cs:48,15055572.519831)
--(axis cs:47,14921101.0546879)
--(axis cs:46,15761876.8959895)
--(axis cs:45,16541607.5678721)
--(axis cs:44,17301667.5206446)
--(axis cs:43,16472888.7176119)
--(axis cs:42,16265148.6037983)
--(axis cs:41,14995377.8298552)
--(axis cs:40,16285445.2365921)
--(axis cs:39,15822112.2007067)
--(axis cs:38,15744619.845116)
--(axis cs:37,15354116.8797189)
--(axis cs:36,17261387.0047658)
--(axis cs:35,16079238.2065388)
--(axis cs:34,16777793.2200205)
--(axis cs:33,17566741.7445091)
--(axis cs:32,15752923.5072787)
--(axis cs:31,15812907.0583281)
--(axis cs:30,16690950.3450855)
--(axis cs:29,16700992.9556087)
--(axis cs:28,16985452.0217668)
--(axis cs:27,19892941.8060927)
--(axis cs:26,18938683.968433)
--(axis cs:25,18361252.4294165)
--(axis cs:24,20759681.132355)
--(axis cs:23,17382383.1781399)
--(axis cs:22,18528348.7867409)
--(axis cs:21,22351616.5878632)
--(axis cs:20,18032425.2804925)
--(axis cs:19,18665953.4367719)
--(axis cs:18,18627011.6623077)
--(axis cs:17,22241551.4776493)
--(axis cs:16,20801293.0745797)
--(axis cs:15,17246000.9860322)
--(axis cs:14,17282984.4167113)
--(axis cs:13,17720494.1000185)
--(axis cs:12,18031941.1585183)
--(axis cs:11,19225432.9328421)
--(axis cs:10,18792719.3949508)
--(axis cs:9,18875710.6078602)
--(axis cs:8,19400837.0451375)
--(axis cs:7,20939622.3476151)
--(axis cs:6,22644485.4462647)
--(axis cs:5,22468883.9554915)
--(axis cs:4,23457906.7567972)
--(axis cs:3,23517055.4078285)
--(axis cs:2,27873280.9951064)
--(axis cs:1,27239165.744712)
--(axis cs:0,20567813.5861097)
--cycle;

\path [draw=goldenrod1911910, fill=goldenrod1911910, opacity=0.5]
(axis cs:0,20095184.5598595)
--(axis cs:0,-327982.067231344)
--(axis cs:1,8525981.06611937)
--(axis cs:2,6154474.93528846)
--(axis cs:3,2519791.9085786)
--(axis cs:4,2553153.65184665)
--(axis cs:5,993989.291548958)
--(axis cs:6,580024.971421062)
--(axis cs:7,78936.4630851764)
--(axis cs:8,-1313376.37149423)
--(axis cs:9,-1455949.76749097)
--(axis cs:10,-847362.439740345)
--(axis cs:11,-933064.144892262)
--(axis cs:12,-1492896.78051071)
--(axis cs:13,-1924765.9257097)
--(axis cs:14,-2719872.70812686)
--(axis cs:15,-2687779.86123491)
--(axis cs:16,-941276.113071991)
--(axis cs:17,957338.015581573)
--(axis cs:18,-213688.132339342)
--(axis cs:19,-788473.478375221)
--(axis cs:20,-1818534.7622713)
--(axis cs:21,2222479.24807901)
--(axis cs:22,-789366.524811707)
--(axis cs:23,-2271010.5216846)
--(axis cs:24,-542819.380169623)
--(axis cs:25,-2223186.66092153)
--(axis cs:26,-1644063.61464356)
--(axis cs:27,-1022788.68287746)
--(axis cs:28,-2796671.85904457)
--(axis cs:29,-3393871.75645888)
--(axis cs:30,-3152888.75109469)
--(axis cs:31,-4134669.52744208)
--(axis cs:32,-4019531.21947485)
--(axis cs:33,-3095044.80057744)
--(axis cs:34,-3373666.10713374)
--(axis cs:35,-3770846.28579746)
--(axis cs:36,-3417502.03193961)
--(axis cs:37,-4692495.64631009)
--(axis cs:38,-3441200.47682409)
--(axis cs:39,-4235123.75781293)
--(axis cs:40,-3433486.98245591)
--(axis cs:41,-4650610.04439952)
--(axis cs:42,-3856803.75597674)
--(axis cs:43,-1197202.26444195)
--(axis cs:44,-1565140.02897999)
--(axis cs:45,-2476572.03365216)
--(axis cs:46,-3163992.25769176)
--(axis cs:47,-4537121.85546144)
--(axis cs:48,-4413308.14753784)
--(axis cs:49,-4669885.79190248)
--(axis cs:50,-4857872.59283867)
--(axis cs:51,-4854267.72847487)
--(axis cs:52,-5127441.22899806)
--(axis cs:53,-5146729.08976134)
--(axis cs:54,-4685458.21281492)
--(axis cs:55,-5119016.28152834)
--(axis cs:56,-6374549.94505057)
--(axis cs:57,-2784883.86223149)
--(axis cs:58,-4058325.55771458)
--(axis cs:59,-2453299.36930479)
--(axis cs:60,-2352522.6349478)
--(axis cs:61,-4264205.61440473)
--(axis cs:62,-4370006.58329235)
--(axis cs:63,-4470061.17618822)
--(axis cs:64,-5014806.48431441)
--(axis cs:65,-4512402.52885564)
--(axis cs:66,-4592004.53695105)
--(axis cs:67,-4977276.12328488)
--(axis cs:68,-5128911.89994237)
--(axis cs:69,-5184105.32051442)
--(axis cs:70,-5203299.35628517)
--(axis cs:71,-4910190.47475148)
--(axis cs:72,-5546885.91761856)
--(axis cs:73,-4512855.30955732)
--(axis cs:74,-4529200.58763956)
--(axis cs:75,-5099216.94704768)
--(axis cs:76,-5102698.69244709)
--(axis cs:77,-5457010.11831894)
--(axis cs:78,-5159946.12737862)
--(axis cs:79,-5474446.75790032)
--(axis cs:80,-5488005.5414418)
--(axis cs:81,-5080681.98040372)
--(axis cs:82,-4523036.34201543)
--(axis cs:83,-4265953.59515314)
--(axis cs:84,-4528261.40140674)
--(axis cs:85,-2176245.46702133)
--(axis cs:86,-4429371.176625)
--(axis cs:87,-5117654.16563546)
--(axis cs:88,-4980392.37230774)
--(axis cs:89,-4264651.24782914)
--(axis cs:90,-2816905.9784693)
--(axis cs:91,-5010183.22634405)
--(axis cs:92,-5494857.89181997)
--(axis cs:93,-5851770.17317551)
--(axis cs:94,-5473110.37019144)
--(axis cs:95,-4781801.16376131)
--(axis cs:96,-5897075.45371105)
--(axis cs:97,-5816985.21376749)
--(axis cs:98,-5178944.01112027)
--(axis cs:99,-5368783.02053176)
--(axis cs:100,-7224586.34279377)
--(axis cs:101,-5118846.83966859)
--(axis cs:102,-6071367.77489944)
--(axis cs:103,-5401805.75544043)
--(axis cs:104,-4838171.77766267)
--(axis cs:105,-4186596.53768772)
--(axis cs:106,3360594.30262651)
--(axis cs:107,1457983.87513238)
--(axis cs:108,-4858377.06072852)
--(axis cs:109,-4753374.37447578)
--(axis cs:110,-5268903.71088376)
--(axis cs:111,-6537081.99893873)
--(axis cs:112,-6118406.40488253)
--(axis cs:113,-5188997.60445258)
--(axis cs:114,-5746448.27304974)
--(axis cs:115,-2472594.54236302)
--(axis cs:116,-564776.020626375)
--(axis cs:117,-4296133.7529762)
--(axis cs:118,-4407629.42501508)
--(axis cs:119,-3932644.37637787)
--(axis cs:120,-2020088.05327648)
--(axis cs:121,-2362246.58057457)
--(axis cs:122,-3250263.99679521)
--(axis cs:123,-6187629.89456107)
--(axis cs:124,1224531.01793912)
--(axis cs:125,5015436.5249786)
--(axis cs:126,7278349.36013981)
--(axis cs:127,4416351.6664131)
--(axis cs:128,6402507.01677439)
--(axis cs:129,3646350.59765421)
--(axis cs:130,938991.373206792)
--(axis cs:131,1934087.55029659)
--(axis cs:132,3266259.53233226)
--(axis cs:133,-65317.9161400776)
--(axis cs:134,1898189.91549616)
--(axis cs:135,588998.30376399)
--(axis cs:136,-2445495.04632213)
--(axis cs:137,1752191.04065494)
--(axis cs:138,691727.142898425)
--(axis cs:139,-2119614.88912989)
--(axis cs:140,-284804.046198091)
--(axis cs:141,-2963599.49298154)
--(axis cs:142,-704421.89374535)
--(axis cs:143,-1193004.73606347)
--(axis cs:144,6286959.77695577)
--(axis cs:145,3208481.69782204)
--(axis cs:146,491439.607521647)
--(axis cs:147,-1088612.3339346)
--(axis cs:148,6155263.56225797)
--(axis cs:149,-500041.476540953)
--(axis cs:150,188238.921633292)
--(axis cs:151,2623143.09062183)
--(axis cs:152,-2816571.75872411)
--(axis cs:153,-2712721.67136062)
--(axis cs:154,-1142045.9913144)
--(axis cs:155,-822942.538391845)
--(axis cs:156,647430.96189945)
--(axis cs:157,143932.278291093)
--(axis cs:158,-3311814.44329534)
--(axis cs:159,-3196902.09019082)
--(axis cs:160,-2089000.37266744)
--(axis cs:161,-2150004.95864659)
--(axis cs:162,3554967.04933067)
--(axis cs:163,1075126.59002237)
--(axis cs:164,-2472598.27244449)
--(axis cs:165,-1607171.78287048)
--(axis cs:166,824530.414844149)
--(axis cs:167,-230670.408547187)
--(axis cs:168,-92865.5886136293)
--(axis cs:169,671706.498136519)
--(axis cs:170,-407230.897963146)
--(axis cs:171,2129091.23544932)
--(axis cs:172,5391417.65607099)
--(axis cs:173,4513106.96572887)
--(axis cs:174,7337526.58704013)
--(axis cs:175,1722827.40494839)
--(axis cs:176,3520563.94521682)
--(axis cs:177,10589034.4999929)
--(axis cs:178,3327656.53372983)
--(axis cs:179,9177360.57939566)
--(axis cs:180,-344402.131816098)
--(axis cs:181,-1764201.87202547)
--(axis cs:182,-1072957.05170617)
--(axis cs:183,190736.636576589)
--(axis cs:184,2799188.01625326)
--(axis cs:185,1644408.26198969)
--(axis cs:186,11713686.2271488)
--(axis cs:187,5663618.18479691)
--(axis cs:188,2694819.61167085)
--(axis cs:189,1479975.40568081)
--(axis cs:190,-674561.973448306)
--(axis cs:191,7149027.00276058)
--(axis cs:192,5993173.79886327)
--(axis cs:193,4053634.00922102)
--(axis cs:194,27024722.6153965)
--(axis cs:195,7214028.39109296)
--(axis cs:196,4082547.67548249)
--(axis cs:197,3489649.01097382)
--(axis cs:198,2842378.94997196)
--(axis cs:199,1860608.15360076)
--(axis cs:200,12358066.8885473)
--(axis cs:201,13774927.2752539)
--(axis cs:202,10606649.6028822)
--(axis cs:203,10541434.9135545)
--(axis cs:204,1282502.12256871)
--(axis cs:205,2099951.026733)
--(axis cs:206,2634530.0222056)
--(axis cs:207,6263602.88669217)
--(axis cs:208,8360015.05598752)
--(axis cs:209,10927859.3067516)
--(axis cs:210,4055564.32269401)
--(axis cs:211,5889696.24095503)
--(axis cs:212,1795049.63046847)
--(axis cs:213,2237000.66346705)
--(axis cs:214,10040593.4898287)
--(axis cs:215,2363306.92207092)
--(axis cs:216,2490209.91988005)
--(axis cs:217,1467357.23635095)
--(axis cs:218,3489073.11119061)
--(axis cs:219,2836542.72003203)
--(axis cs:220,14493198.6036556)
--(axis cs:221,4586014.52352919)
--(axis cs:222,3800624.6565014)
--(axis cs:223,3625378.87934075)
--(axis cs:224,13364939.8241376)
--(axis cs:225,4912887.01395863)
--(axis cs:226,1866886.4274517)
--(axis cs:227,186615.888642257)
--(axis cs:228,4040268.64319032)
--(axis cs:229,5128655.01259346)
--(axis cs:230,-1247776.86322491)
--(axis cs:231,-1855985.14889759)
--(axis cs:232,-2483608.32837332)
--(axis cs:233,-1791423.17355327)
--(axis cs:234,-2264981.63776623)
--(axis cs:235,-2696237.3298425)
--(axis cs:236,-2703719.38903414)
--(axis cs:237,-2886793.1165264)
--(axis cs:238,-1224332.90299625)
--(axis cs:239,2659373.24187304)
--(axis cs:240,290883.90652868)
--(axis cs:241,4794522.27852834)
--(axis cs:242,-2411620.33328918)
--(axis cs:243,-2443538.60939043)
--(axis cs:244,-3835853.27667829)
--(axis cs:245,-1845998.20272431)
--(axis cs:246,4925938.49096104)
--(axis cs:247,10376679.5995322)
--(axis cs:248,5328463.23255601)
--(axis cs:249,-4213304.16613094)
--(axis cs:250,-5022408.28386499)
--(axis cs:251,-4914272.82100287)
--(axis cs:252,-5119602.03208105)
--(axis cs:253,-5497721.1271031)
--(axis cs:254,-2442975.38900373)
--(axis cs:255,1367776.87972946)
--(axis cs:256,-1661741.0191723)
--(axis cs:257,2398302.46172855)
--(axis cs:258,-2660052.39979639)
--(axis cs:259,-750373.208225159)
--(axis cs:260,-3028575.4429051)
--(axis cs:261,-4060563.8900129)
--(axis cs:262,-3963661.26425197)
--(axis cs:263,-4618477.17480335)
--(axis cs:264,-3862680.36005594)
--(axis cs:265,-3714634.06429792)
--(axis cs:266,-3662900.45513532)
--(axis cs:267,-3414510.4479601)
--(axis cs:268,-3108107.56735536)
--(axis cs:269,-3159011.97174962)
--(axis cs:270,-2146377.20028982)
--(axis cs:271,2183800.28552202)
--(axis cs:272,-1448706.47628226)
--(axis cs:273,522233.857812161)
--(axis cs:274,1591988.7889979)
--(axis cs:275,4815200.14648377)
--(axis cs:276,6470572.0082662)
--(axis cs:277,1618243.21691924)
--(axis cs:278,4546130.4942696)
--(axis cs:279,3131053.6330414)
--(axis cs:280,1787749.31547417)
--(axis cs:281,2588101.21016824)
--(axis cs:282,37103.9973795917)
--(axis cs:283,472405.48232968)
--(axis cs:284,8429328.89815577)
--(axis cs:285,4236779.26455399)
--(axis cs:286,7985316.67420399)
--(axis cs:287,3090801.74342229)
--(axis cs:288,-500641.26850551)
--(axis cs:289,-265787.108497286)
--(axis cs:290,-1305136.95552179)
--(axis cs:291,4506046.70509366)
--(axis cs:292,4075544.05402425)
--(axis cs:293,2659751.41534236)
--(axis cs:294,1197717.92233407)
--(axis cs:295,-134458.467444593)
--(axis cs:296,475366.556840254)
--(axis cs:297,3122538.83036122)
--(axis cs:298,2425781.90589592)
--(axis cs:299,1770789.27865894)
--(axis cs:300,5935303.07636428)
--(axis cs:301,430458.204777172)
--(axis cs:302,1934305.38471511)
--(axis cs:303,3777520.31348578)
--(axis cs:304,881370.25967041)
--(axis cs:305,-1229968.7964968)
--(axis cs:306,1087535.79577412)
--(axis cs:307,6088952.75678041)
--(axis cs:308,-130700.082635375)
--(axis cs:309,-1346452.14191096)
--(axis cs:310,-1135875.46141217)
--(axis cs:311,348474.76598726)
--(axis cs:312,-769490.197306531)
--(axis cs:313,3199852.21889307)
--(axis cs:314,11281294.0817896)
--(axis cs:315,12013050.8091445)
--(axis cs:316,3999398.49085605)
--(axis cs:317,1145144.98025232)
--(axis cs:318,-300770.962807514)
--(axis cs:319,-710310.096239293)
--(axis cs:320,204418.804410061)
--(axis cs:321,-467592.804261165)
--(axis cs:322,1274807.15355361)
--(axis cs:323,52739.4263860378)
--(axis cs:324,5894570.86159637)
--(axis cs:325,4542016.72808082)
--(axis cs:326,189448.919695729)
--(axis cs:327,662013.514965464)
--(axis cs:328,2760278.89556879)
--(axis cs:329,-406377.970876075)
--(axis cs:330,5005.34222960658)
--(axis cs:331,-632621.608866561)
--(axis cs:332,-660084.678761361)
--(axis cs:333,-540716.841857316)
--(axis cs:334,-1816262.78943646)
--(axis cs:335,-584.704716838896)
--(axis cs:336,2964311.28800744)
--(axis cs:337,-546661.516043665)
--(axis cs:338,3317780.89741051)
--(axis cs:339,688134.703477241)
--(axis cs:340,-486312.68669443)
--(axis cs:341,-275662.901845707)
--(axis cs:342,-277829.726799175)
--(axis cs:343,-2871818.25615153)
--(axis cs:344,-1677841.09251557)
--(axis cs:345,-1899433.10012714)
--(axis cs:346,-2436066.53572523)
--(axis cs:347,-3231636.91244571)
--(axis cs:348,4404954.32840096)
--(axis cs:349,5893492.1840755)
--(axis cs:350,-1645839.74027259)
--(axis cs:351,1075851.22105771)
--(axis cs:352,1901371.97020188)
--(axis cs:353,3862587.89178053)
--(axis cs:354,1209494.89465564)
--(axis cs:355,1654814.59243163)
--(axis cs:356,1656615.25219765)
--(axis cs:357,5512720.57682372)
--(axis cs:358,-1186934.10070688)
--(axis cs:359,-623056.532365343)
--(axis cs:360,3228808.0194847)
--(axis cs:361,546392.594264129)
--(axis cs:362,559758.191995264)
--(axis cs:363,1624303.51768999)
--(axis cs:364,7029.48937240988)
--(axis cs:365,-1727835.10299643)
--(axis cs:366,-3682894.93987112)
--(axis cs:367,-2190229.28352336)
--(axis cs:368,-1160751.06238406)
--(axis cs:369,-2370432.36275715)
--(axis cs:370,-2993548.09166752)
--(axis cs:371,-3192407.25685937)
--(axis cs:372,-2545242.5834131)
--(axis cs:373,-2873973.08569343)
--(axis cs:374,-1696456.18532576)
--(axis cs:375,-798848.883589353)
--(axis cs:376,986477.390315386)
--(axis cs:377,3348734.81300253)
--(axis cs:378,6626973.75186199)
--(axis cs:379,6371413.6612922)
--(axis cs:380,5795861.72877041)
--(axis cs:381,1233606.05986622)
--(axis cs:382,-943230.577345783)
--(axis cs:383,416917.761796283)
--(axis cs:384,1464552.83460537)
--(axis cs:385,-1877253.7968436)
--(axis cs:386,-1749839.28798172)
--(axis cs:387,-472953.423410408)
--(axis cs:388,-338111.773606827)
--(axis cs:389,-2035780.27819325)
--(axis cs:390,-2061481.09374961)
--(axis cs:391,-3158215.29913902)
--(axis cs:392,-3790071.36710928)
--(axis cs:393,-4543273.07540943)
--(axis cs:394,-4494690.39555643)
--(axis cs:395,-5077100.00496529)
--(axis cs:396,-5252485.54321641)
--(axis cs:397,-4818107.41185138)
--(axis cs:398,-4165937.63725661)
--(axis cs:399,-4171319.10070349)
--(axis cs:400,-3530634.85577306)
--(axis cs:401,-4692728.74109564)
--(axis cs:402,-5567470.36109978)
--(axis cs:403,-6260195.15455036)
--(axis cs:404,-5029231.41575883)
--(axis cs:405,-4486990.52348399)
--(axis cs:406,-4732619.47952825)
--(axis cs:407,-4816519.95932954)
--(axis cs:408,-4629872.73124915)
--(axis cs:409,-3431358.32832649)
--(axis cs:410,-3192557.67247197)
--(axis cs:411,-3786167.03564412)
--(axis cs:412,-3864635.41515271)
--(axis cs:413,-4237017.77484227)
--(axis cs:414,-4858452.62632265)
--(axis cs:415,-4563870.39261836)
--(axis cs:416,-3283594.54294286)
--(axis cs:417,-3848300.78644175)
--(axis cs:418,-4610091.62493011)
--(axis cs:419,-4527954.50175085)
--(axis cs:420,-5073830.19582114)
--(axis cs:421,-1404890.62381924)
--(axis cs:422,-3649822.26863315)
--(axis cs:423,-4317226.19752032)
--(axis cs:424,-4271162.62123753)
--(axis cs:425,-4733137.62389572)
--(axis cs:426,-4255460.18344971)
--(axis cs:427,-3409688.64856674)
--(axis cs:428,-2455926.27291062)
--(axis cs:429,-4376722.45194877)
--(axis cs:430,-4720339.42860565)
--(axis cs:431,-5164586.04975661)
--(axis cs:432,-4599185.12145516)
--(axis cs:433,-4719945.29499572)
--(axis cs:434,-4976550.82178102)
--(axis cs:435,-4895789.57661173)
--(axis cs:436,-5241464.09638324)
--(axis cs:437,-6005144.77906773)
--(axis cs:438,-5624977.89577927)
--(axis cs:439,-4349296.64044097)
--(axis cs:440,-5539845.80092224)
--(axis cs:440,14830480.7765201)
--(axis cs:440,14830480.7765201)
--(axis cs:439,16085864.2189607)
--(axis cs:438,14733672.3515671)
--(axis cs:437,14370420.6170347)
--(axis cs:436,15132078.7675993)
--(axis cs:435,15463508.5286588)
--(axis cs:434,15382641.3864916)
--(axis cs:433,15638299.7418372)
--(axis cs:432,15762064.3455102)
--(axis cs:431,15201924.6175205)
--(axis cs:430,15646458.1994131)
--(axis cs:429,15989053.6264464)
--(axis cs:428,17950658.6652244)
--(axis cs:427,16970908.620917)
--(axis cs:426,16126647.1168137)
--(axis cs:425,15688030.5127814)
--(axis cs:424,16111147.3340035)
--(axis cs:423,16062913.0814119)
--(axis cs:422,16733922.7583433)
--(axis cs:421,19000632.7912507)
--(axis cs:420,15348869.0982698)
--(axis cs:419,15873723.0288347)
--(axis cs:418,15763795.4476635)
--(axis cs:417,16548189.8778434)
--(axis cs:416,17106529.2992382)
--(axis cs:415,15816922.7580792)
--(axis cs:414,15513503.9916486)
--(axis cs:413,16140043.1837536)
--(axis cs:412,16522781.3280414)
--(axis cs:411,16581290.913388)
--(axis cs:410,17172966.0383705)
--(axis cs:409,16944012.618812)
--(axis cs:408,15728369.0235088)
--(axis cs:407,15543630.9187418)
--(axis cs:406,15636185.5746376)
--(axis cs:405,15905997.9709129)
--(axis cs:404,15364032.7068946)
--(axis cs:403,14131216.1826535)
--(axis cs:402,14814237.9791744)
--(axis cs:401,15684001.0332006)
--(axis cs:400,16842277.1451734)
--(axis cs:399,16202446.395964)
--(axis cs:398,16215597.6458939)
--(axis cs:397,15576209.1690017)
--(axis cs:396,15200989.6060951)
--(axis cs:395,15337288.9717207)
--(axis cs:394,15914268.9023479)
--(axis cs:393,15832019.5443323)
--(axis cs:392,16582494.3146612)
--(axis cs:391,17202216.5510524)
--(axis cs:390,18311909.3376179)
--(axis cs:389,18327551.0474831)
--(axis cs:388,20034563.7306281)
--(axis cs:387,19915513.1574234)
--(axis cs:386,18622805.0867952)
--(axis cs:385,18502063.7145788)
--(axis cs:384,21856671.08266)
--(axis cs:383,20810198.2039366)
--(axis cs:382,19443657.9328501)
--(axis cs:381,21617743.3552873)
--(axis cs:380,26243513.239791)
--(axis cs:379,26853662.370312)
--(axis cs:378,27080851.4684549)
--(axis cs:377,23785483.3611907)
--(axis cs:376,21426784.0669604)
--(axis cs:375,19580500.2699308)
--(axis cs:374,18691613.7831637)
--(axis cs:373,17514669.1181161)
--(axis cs:372,17860831.4732936)
--(axis cs:371,17186110.0095971)
--(axis cs:370,17371754.752866)
--(axis cs:369,18004721.9051835)
--(axis cs:368,19218030.7702658)
--(axis cs:367,18205032.8753521)
--(axis cs:366,16788763.500308)
--(axis cs:365,18639642.5078748)
--(axis cs:364,20407409.7737684)
--(axis cs:363,22014100.2354087)
--(axis cs:362,21005378.2246149)
--(axis cs:361,20929852.1314835)
--(axis cs:360,23664445.6467561)
--(axis cs:359,19772158.6029913)
--(axis cs:358,19230801.2840865)
--(axis cs:357,26001636.2785684)
--(axis cs:356,22059336.4700697)
--(axis cs:355,22045584.300872)
--(axis cs:354,21617758.3847529)
--(axis cs:353,24309171.2754407)
--(axis cs:352,22308225.670223)
--(axis cs:351,21480027.4053377)
--(axis cs:350,18832943.3400889)
--(axis cs:349,26358605.7316947)
--(axis cs:348,24846191.6434658)
--(axis cs:347,17143153.4877639)
--(axis cs:346,17940633.1587854)
--(axis cs:345,18465178.9143484)
--(axis cs:344,18712114.9038611)
--(axis cs:343,17512209.5680517)
--(axis cs:342,20144516.1309085)
--(axis cs:341,20120779.1878375)
--(axis cs:340,19886630.1825826)
--(axis cs:339,21068059.3368177)
--(axis cs:338,23746846.2518208)
--(axis cs:337,19860464.3611196)
--(axis cs:336,23390859.2171234)
--(axis cs:335,20367198.188842)
--(axis cs:334,18550167.1570829)
--(axis cs:333,19859735.7464643)
--(axis cs:332,19736296.0115599)
--(axis cs:331,19774861.651291)
--(axis cs:330,20382665.3228359)
--(axis cs:329,20070382.7040445)
--(axis cs:328,23242112.7902567)
--(axis cs:327,21063921.8552708)
--(axis cs:326,20575001.3001907)
--(axis cs:325,24915273.0025963)
--(axis cs:324,26306900.4911884)
--(axis cs:323,20485232.0669579)
--(axis cs:322,21746843.4253397)
--(axis cs:321,19919733.9150432)
--(axis cs:320,20570831.285638)
--(axis cs:319,19653850.5841018)
--(axis cs:318,20069819.7446534)
--(axis cs:317,21514844.1901031)
--(axis cs:316,24397904.4448526)
--(axis cs:315,32724049.9386586)
--(axis cs:314,31963134.3049702)
--(axis cs:313,23614622.1227809)
--(axis cs:312,19589240.5997069)
--(axis cs:311,20719378.4648599)
--(axis cs:310,19223958.4041887)
--(axis cs:309,19007594.3648602)
--(axis cs:308,20229465.5143636)
--(axis cs:307,26484136.1671795)
--(axis cs:306,21451405.4313406)
--(axis cs:305,19134335.9167842)
--(axis cs:304,21289101.590126)
--(axis cs:303,24205500.7335616)
--(axis cs:302,22319701.1266034)
--(axis cs:301,20818324.9557556)
--(axis cs:300,26357093.7655887)
--(axis cs:299,22176703.866486)
--(axis cs:298,22848376.6609318)
--(axis cs:297,23530351.867417)
--(axis cs:296,20834007.3436596)
--(axis cs:295,20225921.8359442)
--(axis cs:294,21577429.9612855)
--(axis cs:293,23083052.6194583)
--(axis cs:292,24482448.3495324)
--(axis cs:291,24892768.4077226)
--(axis cs:290,19057124.5607873)
--(axis cs:289,20160834.8373726)
--(axis cs:288,19865593.4329828)
--(axis cs:287,23473354.2317554)
--(axis cs:286,28439458.3247321)
--(axis cs:285,24644088.5649602)
--(axis cs:284,28929866.2891649)
--(axis cs:283,20877866.6085631)
--(axis cs:282,20431392.3334917)
--(axis cs:281,22995319.5136853)
--(axis cs:280,22205303.937515)
--(axis cs:279,23590282.9933437)
--(axis cs:278,24998560.6125365)
--(axis cs:277,22014783.1727723)
--(axis cs:276,26877264.2045079)
--(axis cs:275,25241547.9283964)
--(axis cs:274,21993147.8053683)
--(axis cs:273,20927662.9568114)
--(axis cs:272,18964277.877134)
--(axis cs:271,22729435.375454)
--(axis cs:270,18223607.5406626)
--(axis cs:269,17205315.8443742)
--(axis cs:268,17252032.3037264)
--(axis cs:267,16952438.1304502)
--(axis cs:266,16711170.3805308)
--(axis cs:265,16672453.7854996)
--(axis cs:264,16580284.0798781)
--(axis cs:263,15770620.2392254)
--(axis cs:262,16405521.973637)
--(axis cs:261,16337155.0065507)
--(axis cs:260,17591517.756832)
--(axis cs:259,19637014.9045162)
--(axis cs:258,17714781.603686)
--(axis cs:257,22825090.5587717)
--(axis cs:256,18730309.0511878)
--(axis cs:255,21768799.8565179)
--(axis cs:254,17994467.2119758)
--(axis cs:253,14992811.5084735)
--(axis cs:252,15407325.903205)
--(axis cs:251,15580270.4988643)
--(axis cs:250,15418156.7934177)
--(axis cs:249,16255378.4864552)
--(axis cs:248,25942993.0375729)
--(axis cs:247,30950167.6749981)
--(axis cs:246,25372425.1278865)
--(axis cs:245,18525425.3075182)
--(axis cs:244,16571691.8876285)
--(axis cs:243,17993506.1801967)
--(axis cs:242,18025039.0636296)
--(axis cs:241,25367081.1545941)
--(axis cs:240,20712985.0035551)
--(axis cs:239,23199319.8433877)
--(axis cs:238,19236885.7015874)
--(axis cs:237,17477018.5245853)
--(axis cs:236,17656391.1321659)
--(axis cs:235,17662985.2930218)
--(axis cs:234,18093080.8343726)
--(axis cs:233,18565386.0824384)
--(axis cs:232,17991423.3638597)
--(axis cs:231,18523910.5853597)
--(axis cs:230,19126169.9763011)
--(axis cs:229,25538583.012652)
--(axis cs:228,24429970.2656081)
--(axis cs:227,20585428.4901072)
--(axis cs:226,22299778.1224883)
--(axis cs:225,25289640.3886879)
--(axis cs:224,34145500.4277255)
--(axis cs:223,24033520.2444544)
--(axis cs:222,24291583.5118007)
--(axis cs:221,25188279.4989065)
--(axis cs:220,35543560.8184638)
--(axis cs:219,23368824.5850719)
--(axis cs:218,23910109.5360271)
--(axis cs:217,21853908.2019696)
--(axis cs:216,22890011.4923697)
--(axis cs:215,22858772.740177)
--(axis cs:214,30552419.5367069)
--(axis cs:213,22716532.2588531)
--(axis cs:212,22359859.690138)
--(axis cs:211,26415725.5554107)
--(axis cs:210,24474336.6472612)
--(axis cs:209,31444267.1891842)
--(axis cs:208,28852363.8254727)
--(axis cs:207,26678691.9203525)
--(axis cs:206,23037598.9126048)
--(axis cs:205,22546775.5092194)
--(axis cs:204,21685659.9933345)
--(axis cs:203,30995296.0759828)
--(axis cs:202,31165629.8262823)
--(axis cs:201,34515268.6991069)
--(axis cs:200,33037689.7103797)
--(axis cs:199,22293249.0876308)
--(axis cs:198,23250388.9484399)
--(axis cs:197,23923963.9942145)
--(axis cs:196,24492404.4764767)
--(axis cs:195,27712207.1185229)
--(axis cs:194,48906177.6658471)
--(axis cs:193,24478075.0922857)
--(axis cs:192,26459782.8840658)
--(axis cs:191,27593541.866312)
--(axis cs:190,19698381.6925849)
--(axis cs:189,21865619.5732777)
--(axis cs:188,23125250.6044197)
--(axis cs:187,26320197.6365015)
--(axis cs:186,32242694.4096063)
--(axis cs:185,22134609.26878)
--(axis cs:184,23283580.9107924)
--(axis cs:183,20564322.2739664)
--(axis cs:182,19333904.9949756)
--(axis cs:181,18604225.6238161)
--(axis cs:180,20037076.7443282)
--(axis cs:179,29659121.6919659)
--(axis cs:178,23768886.5259058)
--(axis cs:177,31034733.1662741)
--(axis cs:176,23951073.1126803)
--(axis cs:175,22148738.5480652)
--(axis cs:174,27816929.2999865)
--(axis cs:173,25101654.3524251)
--(axis cs:172,25898756.5551063)
--(axis cs:171,22559504.0179996)
--(axis cs:170,19982234.3956329)
--(axis cs:169,21127345.8615164)
--(axis cs:168,20339468.5412365)
--(axis cs:167,20150254.6871637)
--(axis cs:166,21212420.887957)
--(axis cs:165,18762438.307946)
--(axis cs:164,17879481.1034595)
--(axis cs:163,21440501.9147216)
--(axis cs:162,23991173.7332789)
--(axis cs:161,18205430.0406632)
--(axis cs:160,18283568.8180054)
--(axis cs:159,17158619.1328986)
--(axis cs:158,17065685.5775321)
--(axis cs:157,20539731.8037777)
--(axis cs:156,21101438.4852577)
--(axis cs:155,19557726.8029131)
--(axis cs:154,19257035.9093921)
--(axis cs:153,17700427.5706992)
--(axis cs:152,17547891.0716491)
--(axis cs:151,23245551.8935555)
--(axis cs:150,20568600.6782821)
--(axis cs:149,19877432.2309555)
--(axis cs:148,26618832.242101)
--(axis cs:147,19282147.0127552)
--(axis cs:146,20876995.0199392)
--(axis cs:145,23703065.1286156)
--(axis cs:144,26772890.0942129)
--(axis cs:143,19193571.9338843)
--(axis cs:142,19699752.7267255)
--(axis cs:141,17434127.9157116)
--(axis cs:140,20109856.1861686)
--(axis cs:139,18281001.9227786)
--(axis cs:138,21119503.4551924)
--(axis cs:137,22151044.3556919)
--(axis cs:136,17928776.2268629)
--(axis cs:135,21016942.8909668)
--(axis cs:134,22326564.6563311)
--(axis cs:133,20355963.3941826)
--(axis cs:132,23819113.3034069)
--(axis cs:131,22374446.3508875)
--(axis cs:130,21418013.4194908)
--(axis cs:129,24116887.1423835)
--(axis cs:128,26932329.7032267)
--(axis cs:127,24911443.2426018)
--(axis cs:126,27837018.2443708)
--(axis cs:125,25444211.0376399)
--(axis cs:124,21640548.8508261)
--(axis cs:123,14339963.2007857)
--(axis cs:122,17144429.8858538)
--(axis cs:121,18099585.795789)
--(axis cs:120,18462997.6545087)
--(axis cs:119,16462724.3211843)
--(axis cs:118,15978870.3314613)
--(axis cs:117,16077999.6932925)
--(axis cs:116,19932033.0557971)
--(axis cs:115,17900080.4040502)
--(axis cs:114,14634622.9197617)
--(axis cs:113,15257375.0696182)
--(axis cs:112,14289027.3995582)
--(axis cs:111,13955689.1891182)
--(axis cs:110,15141057.7776288)
--(axis cs:109,15635709.6119226)
--(axis cs:108,15514818.1450255)
--(axis cs:107,21918446.842534)
--(axis cs:106,23841109.7237801)
--(axis cs:105,16203980.6383082)
--(axis cs:104,15517994.4925496)
--(axis cs:103,14964417.5128652)
--(axis cs:102,14316221.496212)
--(axis cs:101,15251534.3009451)
--(axis cs:100,13155849.9858507)
--(axis cs:99,14998385.3552026)
--(axis cs:98,15186010.0063492)
--(axis cs:97,14534720.2712802)
--(axis cs:96,14457859.565939)
--(axis cs:95,15602095.0613439)
--(axis cs:94,14884491.9340412)
--(axis cs:93,14504924.7353939)
--(axis cs:92,14866780.0705446)
--(axis cs:91,15348816.4278384)
--(axis cs:90,17568520.3811705)
--(axis cs:89,16105507.7663636)
--(axis cs:88,15374194.9176835)
--(axis cs:87,15234817.0884177)
--(axis cs:86,15931827.9648273)
--(axis cs:85,18226246.7145771)
--(axis cs:84,15834037.5434274)
--(axis cs:83,16085912.4259221)
--(axis cs:82,15842066.8647745)
--(axis cs:81,15272515.1378741)
--(axis cs:80,14865440.5899156)
--(axis cs:79,14879774.86084)
--(axis cs:78,15194876.7241187)
--(axis cs:77,14897353.5663338)
--(axis cs:76,15250427.7003655)
--(axis cs:75,15258830.1053302)
--(axis cs:74,15837618.7244432)
--(axis cs:73,15846162.1131444)
--(axis cs:72,14844068.6041126)
--(axis cs:71,15441864.6355337)
--(axis cs:70,15154820.8399991)
--(axis cs:69,15171098.5194611)
--(axis cs:68,15224530.8219376)
--(axis cs:67,15378680.1352996)
--(axis cs:66,15764663.6825687)
--(axis cs:65,15844697.4158055)
--(axis cs:64,15349105.4361343)
--(axis cs:63,15898213.6599874)
--(axis cs:62,15993428.9267076)
--(axis cs:61,16129804.0048445)
--(axis cs:60,18150887.8790865)
--(axis cs:59,17947431.3838487)
--(axis cs:58,16308270.038825)
--(axis cs:57,17610549.9393124)
--(axis cs:56,14031176.6606279)
--(axis cs:55,15273237.6253415)
--(axis cs:54,15718528.6195391)
--(axis cs:53,15259471.2306792)
--(axis cs:52,15290785.5074787)
--(axis cs:51,15533454.853975)
--(axis cs:50,15522300.9873048)
--(axis cs:49,15714571.0347363)
--(axis cs:48,15975939.5666717)
--(axis cs:47,15835584.0316116)
--(axis cs:46,17224423.8028064)
--(axis cs:45,17909099.8231398)
--(axis cs:44,18928129.0479419)
--(axis cs:43,19225860.4357253)
--(axis cs:42,16532988.7484921)
--(axis cs:41,15708121.0837306)
--(axis cs:40,17003966.6886045)
--(axis cs:39,16128438.0231802)
--(axis cs:38,16923926.4709223)
--(axis cs:37,15672417.3533708)
--(axis cs:36,16950855.552143)
--(axis cs:35,16583851.9731915)
--(axis cs:34,16982424.8107828)
--(axis cs:33,17284114.7570414)
--(axis cs:32,16332871.361588)
--(axis cs:31,16218838.5389671)
--(axis cs:30,17207787.4549504)
--(axis cs:29,16972996.8778157)
--(axis cs:28,17566481.0647813)
--(axis cs:27,19348421.8071417)
--(axis cs:26,18722762.4062731)
--(axis cs:25,18142415.8711989)
--(axis cs:24,19822550.976893)
--(axis cs:23,18090605.9716623)
--(axis cs:22,19580520.8856769)
--(axis cs:21,22642485.4652698)
--(axis cs:20,18592430.9982338)
--(axis cs:19,19621888.9599521)
--(axis cs:18,20217979.1596348)
--(axis cs:17,21390918.1829544)
--(axis cs:16,21537339.8236532)
--(axis cs:15,17691971.7372764)
--(axis cs:14,17673887.0187252)
--(axis cs:13,18464249.4972517)
--(axis cs:12,18887271.918174)
--(axis cs:11,19429550.3994055)
--(axis cs:10,19548602.6839376)
--(axis cs:9,18965809.172831)
--(axis cs:8,19177200.333326)
--(axis cs:7,20591694.5033624)
--(axis cs:6,21072166.5176967)
--(axis cs:5,21480239.2343283)
--(axis cs:4,23180616.6122515)
--(axis cs:3,22980594.8171128)
--(axis cs:2,26623483.0509232)
--(axis cs:1,29097638.9813667)
--(axis cs:0,20095184.5598595)
--cycle;

\addplot [only marks, red, mark=asterisk, mark size=1.2]
table {%
0 29755500
1 20228000
2 8513700
3 9502900
4 8909200
5 7004800
6 6305500
7 6348400
8 7321100
9 6421300
10 6164800
11 5360500
12 5074700
13 5074700
14 6804600
15 9612000
16 9154100
17 5291400
18 5456200
19 5819200
20 12820200
21 5305900
22 5067000
23 11001200
24 7051300
25 7450000
26 9014400
27 5198200
28 4916300
29 5534800
30 4719000
31 4596600
32 6568000
33 6431200
34 5034500
35 6933000
36 4533400
37 4446000
38 5301600
39 4517400
40 4370200
41 6208600
42 7137600
43 4905200
44 5415600
45 4461400
46 4170200
47 4549000
48 5623500
49 4354800
50 6282500
51 6528100
52 4165200
53 3940800
54 3894500
55 4728400
56 12964300
57 7101500
58 9158000
59 10692100
60 4636300
61 5787600
62 5199600
63 4093600
64 3925700
65 3818200
66 3756600
67 3702000
68 3632400
69 3676500
70 3599700
71 4028000
72 6895400
73 4511900
74 5519200
75 3725000
76 3495900
77 3418000
78 3308300
79 3208300
80 3241700
81 7029700
82 5106300
83 4259800
84 11266900
85 7029700
86 3754100
87 3833100
88 5472000
89 6634600
90 3756400
91 3932500
92 3587100
93 3677400
94 3490000
95 3271000
96 3169400
97 3754100
98 3754100
99 3289100
100 3907600
101 3268800
102 3747400
103 4939300
104 5061200
105 34188900
106 27994500
107 6776800
108 5061200
109 4065600
110 3948300
111 5492400
112 6815200
113 4162700
114 10287200
115 19116000
116 5162800
117 5390800
118 9111000
119 8332200
120 6176400
121 6007000
122 5449500
123 9217100
124 26947100
125 23633100
126 7943900
127 9619000
128 7515000
129 7323100
130 6271200
131 10975700
132 5934800
133 13255700
134 8424800
135 5537500
136 20382400
137 15957800
138 9571500
139 11587400
140 5961900
141 7573700
142 7627900
143 34868400
144 19389200
145 7959700
146 5842300
147 22768500
148 7232800
149 7546600
150 16027800
151 5833200
152 5336600
153 7476600
154 7512800
155 6573700
156 4891900
157 5083800
158 4998000
159 5007000
160 5165000
161 24100400
162 19050500
163 6097300
164 8512800
165 14923900
166 9695700
167 9673100
168 13014100
169 9045600
170 15720800
171 25333000
172 21901700
173 29845600
174 12422700
175 11126900
176 29179700
177 12285000
178 31793800
179 7835600
180 7106400
181 9212600
182 14449900
183 22425400
184 24443600
185 36209300
186 23874700
187 11655200
188 11354900
189 8605400
190 31960800
191 32662900
192 23240300
193 113027800
194 30163900
195 17425200
196 9767900
197 13793000
198 9176500
199 48431100
200 56194500
201 19000900
202 34642700
203 14820100
204 11009500
205 16849500
206 26134400
207 33843500
208 32130100
209 16087400
210 17438700
211 14080400
212 15384500
213 40475900
214 18610300
215 11987000
216 10382000
217 12513000
218 23102600
219 67402600
220 16682500
221 15589900
222 22150000
223 64014200
224 18127200
225 11294600
226 10635400
227 18295300
228 23097100
229 8678100
230 8156300
231 14513900
232 7707800
233 7069100
234 6833500
235 6511900
236 6344300
237 7089500
238 23791600
239 10674900
240 33965900
241 8903700
242 8915100
243 6704400
244 6631900
245 30353300
246 41741700
247 41295500
248 7716900
249 7114400
250 6434900
251 6253700
252 6063400
253 6285400
254 14942200
255 8441700
256 15918400
257 6738400
258 6738400
259 9590000
260 6439400
261 6058900
262 5855000
263 5691900
264 5549300
265 5404300
266 5284200
267 5207200
268 5669300
269 5574200
270 7710100
271 5907100
272 10505100
273 12550400
274 26328400
275 21633000
276 8083800
277 18942200
278 7873100
279 8684000
280 8011300
281 6641000
282 6031700
283 34190200
284 19821000
285 31746200
286 18552600
287 7839200
288 12471100
289 8468800
290 22822100
291 17691900
292 8228700
293 7429200
294 9578700
295 11988600
296 19771200
297 13257000
298 12799500
299 26049800
300 8720300
301 18244600
302 16595700
303 8638700
304 7465400
305 14095100
306 23268300
307 8117800
308 7275200
309 6851600
310 8199300
311 6978500
312 21503900
313 50912700
314 44235500
315 13080400
316 10482400
317 7288800
318 7094000
319 7544700
320 9150600
321 8079300
322 7254800
323 23109800
324 19705500
325 9900300
326 8115500
327 12027200
328 11605900
329 8681700
330 8289900
331 9282000
332 9334100
333 6591200
334 9687400
335 12131300
336 7669300
337 22799500
338 11479000
339 6181200
340 10036200
341 8566200
342 6348800
343 8690800
344 6029400
345 5472200
346 5182300
347 25941000
348 22471100
349 6924100
350 10926400
351 11340900
352 11861800
353 10104200
354 10158500
355 11315900
356 29934200
357 8493800
358 7152900
359 22946700
360 11884500
361 9943400
362 10604700
363 7551500
364 6679500
365 10824400
366 8539100
367 10095100
368 6169900
369 6054300
370 5782500
371 5451900
372 7485800
373 8675000
374 5766700
375 7368000
376 17732700
377 15408800
378 14600200
379 16423500
380 10514100
381 6847100
382 11556000
383 14274000
384 6534500
385 6545900
386 7393000
387 12083800
388 6767800
389 6280800
390 5789300
391 5710100
392 5662500
393 5395200
394 5094000
395 4915100
396 4906000
397 5605900
398 6364700
399 8364600
400 5513000
401 4910500
402 4724800
403 4636500
404 4645500
405 4625100
406 4575300
407 4518700
408 6396400
409 5037400
410 4656800
411 5141600
412 4847100
413 4466600
414 4971700
415 7485800
416 4645500
417 4308000
418 4185700
419 4208400
420 9963700
421 4541300
422 4280900
423 4249100
424 4262700
425 4792700
426 4940000
427 12004500
428 4865200
429 4317100
430 4405400
431 4178900
432 4077000
433 3970500
434 3990900
435 4215200
436 3959200
437 3877700
438 4516400
439 4643300
440 3959200
};
\addplot [blue]
table {%
0 10054981.2110582
1 15809356.6477401
2 16836137.8022979
3 12680006.8252034
4 12053043.5501137
5 11619563.5329242
6 11547098.2527671
7 9966985.81912289
8 8469430.23308482
9 8308909.69247002
10 8342273.67417598
11 8986918.02883629
12 7714555.47338053
13 7349670.53312849
14 6888712.62983631
15 6911626.44544006
16 9051565.89487119
17 11544187.2395804
18 8043441.4630421
19 8165199.93052885
20 7430240.72286949
21 11616490.2601771
22 8225901.8112943
23 7137996.88263169
24 10475284.6642821
25 8096454.97135461
26 8675404.45321103
27 9606093.69501342
28 6737648.87259164
29 6432938.2301515
30 6462022.37591857
31 5609956.54654122
32 5553531.14808547
33 7214942.9248869
34 6572923.48336882
35 5877672.61571505
36 6989715.21171984
37 5094276.08598399
38 5488489.0605222
39 5582572.12748109
40 5719395.81432162
41 4774560.27848241
42 5858551.3562609
43 5816138.97320243
44 6080717.34005757
45 6191234.71044222
46 5374467.41793073
47 4623369.90770369
48 4706072.10202598
49 5453413.70356927
50 4643992.25251068
51 5871659.69062393
52 5950000.51788593
53 4720221.08730664
54 3808868.17891326
55 4042591.34310914
56 4632406.31846748
57 8753639.6691476
58 6470494.96791224
59 9187553.33576345
60 9137602.01150911
61 6254622.99751343
62 5820922.63384958
63 5547272.68138386
64 4372933.0711803
65 4468467.96423559
66 4456433.84743655
67 4173148.05491739
68 4017995.26332568
69 3956578.35113653
70 4109248.62801572
71 4095594.09579281
72 4347269.39862851
73 5796412.54002218
74 4959208.73978765
75 4919833.47411277
76 4056529.65330111
77 3684191.80037791
78 3794620.56390667
79 3698103.05991581
80 3566867.54078443
81 3780815.90477579
82 5440755.75688525
83 5035635.26682866
84 4822475.87813528
85 8390395.23894475
86 5717777.13280245
87 3945442.67057006
88 4178224.4001867
89 4951522.4183476
90 6050503.20528291
91 4061292.9888014
92 3954411.45244352
93 3560256.88939027
94 3640230.64979373
95 3412701.57189907
96 3388021.62759624
97 3142900.74003201
98 4487277.46386696
99 4282483.93248674
100 3618286.63198859
101 4541090.54115636
102 3969531.15760798
103 4364635.19520216
104 4670930.82920766
105 6288318.76503266
106 12004457.5749516
107 11607682.647235
108 6084515.64025539
109 4477774.80850155
110 5429177.95370432
111 5838578.32062274
112 5343429.49032497
113 6564769.07832737
114 4418307.2133838
115 9147150.63530318
116 10861654.7516882
117 5748191.49515906
118 6184772.97249525
119 7124945.7552795
120 8748725.20197144
121 8055571.22681504
122 7095665.75386219
123 6932909.6163178
124 12091976.0488649
125 15587017.7161032
126 15194731.4002861
127 12880488.8260844
128 14241123.9634915
129 12277105.1187554
130 9953278.86161329
131 11203737.6374524
132 12085683.7716119
133 8912828.11345523
134 10912757.5321852
135 11820473.8734847
136 7292680.7135492
137 14262116.2333253
138 12017633.5023344
139 9368724.05865186
140 11483865.9641636
141 7591536.61475472
142 9750172.60978321
143 9603946.91758487
144 13750976.4185203
145 13509330.4427871
146 11462445.8217293
147 9355918.02146749
148 17980485.9569202
149 10735070.3346046
150 10972271.60774
151 12987012.4257613
152 7842512.32853856
153 7939029.89428947
154 9770937.04258884
155 10114112.9567709
156 11568896.3521824
157 10202843.8443598
158 6564046.17523155
159 6544190.90594455
160 6626992.92081541
161 7620647.68828695
162 11286748.3167231
163 13508616.5401261
164 7590951.9338884
165 9169861.64827389
166 11872006.6882986
167 11282262.3824423
168 12361903.6472492
169 13484945.3971417
170 11248395.4841968
171 16048411.1907631
172 18555393.311561
173 14060504.0015114
174 17978080.5011367
175 14389096.369655
176 15703753.5901766
177 22305564.696526
178 16131584.193746
179 17475198.173337
180 10905250.1035996
181 9629567.24404327
182 9727049.30590247
183 13282126.0047677
184 11430498.5730266
185 12085109.9306029
186 17147033.6197352
187 13306235.0878384
188 13089240.8212588
189 12662927.3510426
190 9952327.37335085
191 14917210.6559072
192 13318412.8436167
193 14231346.4422219
194 8876495.42558645
195 14709529.7594958
196 15712793.2050807
197 13015864.0505096
198 15610818.9891855
199 13809337.232013
200 16234211.9740193
201 13700623.3034108
202 17040644.2711064
203 21054154.7618079
204 14383110.2900253
205 13476197.8707353
206 15906706.4118959
207 19355703.2399467
208 19074995.6633447
209 20372108.6492256
210 18675461.3864671
211 15043729.7628919
212 13043872.8515947
213 13251883.3555764
214 19208946.4910484
215 13679568.0862665
216 15306617.6429872
217 13951464.7546879
218 13913857.1602525
219 12125355.4546783
220 9266455.12271815
221 11328870.5771586
222 12295001.1803975
223 16027240.1358989
224 13693668.3890615
225 17148145.1048064
226 12330303.2804828
227 11464090.3472016
228 15853438.5853242
229 17200186.393431
230 10739400.3418854
231 9752528.07144827
232 9200603.34459613
233 9181552.70376439
234 8437929.4874013
235 7939624.06999314
236 7822666.12218395
237 7681504.80131611
238 8681070.67755751
239 11781371.2340276
240 9775113.70042782
241 12440365.5668639
242 8829341.88496283
243 7792740.78085354
244 6492990.82683774
245 8437417.62593567
246 13695669.1963617
247 14415058.9252391
248 10375133.0174038
249 6948569.35907781
250 6513154.22905017
251 6100983.80118833
252 6255829.97749803
253 5539936.34248061
254 8514639.91445777
255 12506585.9594444
256 9614324.31673441
257 12846657.6729774
258 8064475.80611475
259 10628168.491576
260 6984994.16637328
261 6717951.72590096
262 6470917.90769565
263 6039146.4311446
264 5896948.74456761
265 6043040.84727893
266 6225195.76495892
267 6418714.51472441
268 6507682.93858858
269 7249758.85811528
270 7721350.85768119
271 10286699.2169867
272 7731018.17550069
273 12068799.0891548
274 13241077.7478735
275 14494794.2149184
276 16912515.6257495
277 13497974.3742038
278 14660172.9430707
279 11159901.1528476
280 12712297.946919
281 13154006.8614324
282 10421952.4463667
283 10209317.2034649
284 12677060.0946351
285 14203390.8098705
286 16579971.8165129
287 15491620.8067473
288 10209877.6513049
289 11581237.3210281
290 10018475.4666538
291 14172319.9489963
292 15456096.9399246
293 13056064.5818932
294 11560130.9768474
295 11217157.4106688
296 12355896.1158725
297 11491144.3238356
298 13231453.3718711
299 11847574.1826062
300 15774616.1717691
301 11041157.4249809
302 14706995.5014862
303 16236788.9318893
304 12520378.7868318
305 9549510.11098187
306 12756123.3298404
307 16748960.1426266
308 10770572.8478642
309 9163745.18465884
310 8919430.31941202
311 10583570.6114004
312 9399287.59827643
313 11138418.7227971
314 12425865.9152035
315 15901113.616504
316 14004418.0074138
317 12542795.8529608
318 9827943.73571483
319 9211014.92326966
320 10403227.8795091
321 10695960.3447975
322 10125389.1582775
323 9446065.36017555
324 12475553.6503634
325 14433803.3302464
326 10760351.6344056
327 10946979.3854433
328 11238311.7072
329 9260405.59203176
330 10251113.6271912
331 9927558.3094797
332 9106204.55288936
333 10861621.2589261
334 8567260.77875342
335 11011706.8628292
336 12160945.1783146
337 10143641.9501035
338 14339880.845044
339 12436018.5357853
340 9514015.21958375
341 9152093.56292691
342 9140718.15080811
343 7865138.02310852
344 8456000.15525358
345 7973033.16837898
346 7464298.29214458
347 6837656.50120735
348 15855170.4898922
349 16104947.7761158
350 9316236.71694867
351 12903837.9012919
352 14943492.4340679
353 14158464.0597725
354 12082606.4210859
355 12647722.4482457
356 13597564.5142041
357 18067421.0530445
358 9627093.12541854
359 9562176.85276415
360 13489210.9470631
361 11648972.7839114
362 10280459.9770403
363 11245747.0946733
364 9277605.31223772
365 7950748.52833768
366 8694667.17434491
367 8639825.10566929
368 9679775.36758871
369 7193992.1029621
370 6675137.59388923
371 6086782.9256113
372 5884482.8252452
373 7470764.2422958
374 8237942.8809479
375 8138564.99913537
376 8850933.02991374
377 14139742.1501901
378 17799536.3770602
379 14594380.7917046
380 14521741.0849676
381 10843413.9285291
382 8016035.27265818
383 10021218.2408518
384 10518517.6810234
385 7422900.02916514
386 7601513.13238449
387 8505007.40237058
388 10251282.6532818
389 7601955.99911267
390 7789891.77710672
391 6393874.34302022
392 5897012.52561669
393 5777875.24872646
394 5426797.75535204
395 4988653.56785819
396 5494312.87946
397 5057602.83225843
398 5415370.14711242
399 5786985.11282077
400 6829510.39790567
401 4767026.69546303
402 4555982.0183673
403 3980369.66887837
404 4758999.60748926
405 4940927.69803512
406 4874056.33245328
407 4855335.15338745
408 4658443.0952399
409 6397033.91081679
410 6316187.57391557
411 5883752.05031363
412 4846534.22949867
413 5056109.98054353
414 4369517.53999858
415 5281865.8784673
416 6328619.23887379
417 4460551.11684384
418 4095617.66071411
419 4365865.83904842
420 4218809.90383208
421 7188096.35037887
422 5097594.03916658
423 4678507.06606024
424 4496846.03515255
425 4530638.85372892
426 5096033.12064491
427 5882907.18634648
428 9121521.44799462
429 5344590.84014927
430 4704604.98855027
431 4649662.44829788
432 4571986.7767646
433 4402367.68692441
434 4369021.41253375
435 4383334.76766141
436 4206512.78927942
437 4263500.61496703
438 3778425.86242553
439 4559530.69851574
440 4478475.19553038
};
\addplot [green01270]
table {%
0 12720887.6649943
1 15639773.5477111
2 16528156.8737364
3 14578130.5441398
4 15473797.0238421
5 14698586.7334707
6 15301952.2900671
7 14585622.5486763
8 13005187.7696172
9 14817777.3365178
10 12642167.9485143
11 10142330.9143853
12 10618412.6891114
13 11170532.8181873
14 10500892.4769025
15 11524737.6352854
16 14522713.9239559
17 13634313.1243527
18 14296906.2052619
19 12853134.3590212
20 13400190.4788114
21 14117133.4595784
22 11339635.6563661
23 9145182.15871798
24 9319094.91123744
25 9104795.83178669
26 9158452.76906919
27 9314299.66222099
28 8713697.94720265
29 8452384.64631421
30 8289331.41042146
31 7286473.99525564
32 7339911.94076293
33 9026744.06362321
34 7206120.91528108
35 7507776.85915609
36 8696046.72127687
37 7939612.41439815
38 9164994.4917942
39 8505525.2289693
40 8115287.39804909
41 7740491.00950869
42 9282282.47703274
43 11558436.6783473
44 13657042.5048416
45 11028423.12152
46 10950974.6078311
47 9005535.92450411
48 9876822.29065448
49 8750870.69594417
50 9241106.77081961
51 9185116.77285908
52 8867468.08482107
53 8872655.29323189
54 10141770.0957428
55 9388541.35833064
56 9082982.1499864
57 9717915.18116166
58 9123296.65843239
59 10357612.0505169
60 10402691.3852322
61 9642062.37956736
62 8204249.85546023
63 7881603.46412393
64 7294779.97729527
65 7132792.62965344
66 6859870.63248652
67 6246477.32237259
68 5958658.7880314
69 5760590.79026581
70 6359775.16929709
71 6093320.1641069
72 7495503.54286166
73 6766441.07569122
74 6635481.60190171
75 6635374.26420899
76 6183176.37422493
77 5929562.78462918
78 5953697.95606263
79 5896915.48454084
80 5314427.52521525
81 5695747.59085111
82 6057315.44083942
83 6320858.18304895
84 6855647.85116456
85 8330693.86549823
86 6703736.30937946
87 5681162.14576882
88 6251406.53548614
89 7194394.0643331
90 8637217.96897359
91 6854725.07983726
92 6967289.03649064
93 6484325.58434131
94 6384711.21305022
95 8090477.41367927
96 5740241.14466965
97 5988546.51092679
98 7508140.05995354
99 6736969.3903645
100 7006273.00478734
101 7095075.39174394
102 7760628.47594483
103 7631847.58288363
104 5945120.09347188
105 7510566.59907278
106 8520014.77739201
107 8432031.15237055
108 8139108.87715576
109 9454190.31850645
110 9623845.03052001
111 8237895.54170399
112 8598265.61953211
113 9077438.29336984
114 8079274.87407719
115 8138823.05934609
116 10022531.7523847
117 8831823.44206927
118 9355887.92168815
119 9238023.54240057
120 12636735.3299427
121 11299052.6476742
122 9163417.96943243
123 9125724.9657113
124 12292732.5861034
125 13149980.4925649
126 15857996.0694532
127 15213729.4434437
128 15231864.5210618
129 13460880.3751781
130 14844263.1202142
131 13415532.6962311
132 16400261.1006026
133 12053545.0992012
134 13092114.2431767
135 12114365.2153269
136 8809367.91830992
137 9206743.4197699
138 9840969.73312774
139 8546035.32707726
140 8455077.03343984
141 8214796.30381652
142 10772010.6391811
143 10233430.3487679
144 9723012.73347644
145 14144691.3720893
146 10763994.2192892
147 9257444.30611764
148 13526102.5848179
149 9894150.98345516
150 9966796.7594514
151 9791323.1379837
152 7857497.83135786
153 7948934.25617107
154 10128778.6986048
155 9952565.36234005
156 10418419.8935343
157 10622482.3686081
158 8581423.06663041
159 7677733.19840693
160 9698041.17051588
161 7838162.90139168
162 9343498.89494494
163 8528629.77133577
164 7277057.60894524
165 8053063.46715732
166 11271093.2589096
167 11922488.2726206
168 13803235.6102454
169 14752882.132407
170 13776309.6588014
171 14322574.4751359
172 16101353.0848686
173 16432816.8724258
174 15665849.5344024
175 15731263.9533989
176 15458891.229832
177 16152772.8321303
178 15854270.8690633
179 15084297.6596063
180 13215661.1756053
181 12259933.0162649
182 12544338.3783928
183 10730350.7172427
184 10733136.0887649
185 11669912.6967587
186 11831767.5364349
187 12777752.9279922
188 14724917.335154
189 11718768.8205222
190 10055849.6010243
191 11413502.5026263
192 12518313.5643225
193 10781833.0781585
194 10006743.0226444
195 12096828.81577
196 12259765.7691623
197 12687732.3830346
198 14109784.5719655
199 16117606.0085261
200 16138722.4113613
201 10884808.408948
202 17462595.558613
203 15510764.0420059
204 12813706.8946763
205 12208563.1935115
206 13945285.3975029
207 13560227.2343125
208 14522978.5238496
209 14543319.0166192
210 15477032.9640512
211 16016354.1133647
212 15723281.4405689
213 15156436.9098627
214 14473914.7973392
215 15740481.2657327
216 15210173.5703443
217 14067138.2249518
218 13536154.4685144
219 11961558.360694
220 9792463.91299698
221 14791556.5687527
222 12848994.785213
223 11550605.1871548
224 14603380.2802197
225 14017794.0891202
226 13433841.6775529
227 11507462.5073217
228 12529318.1740512
229 11794899.95125
230 10254974.6462425
231 10349147.3501448
232 9630943.93155885
233 8629666.78309291
234 7974594.41331246
235 7711298.06935298
236 7726629.05375831
237 8108698.9231023
238 10044102.0522073
239 9716817.8820743
240 10306290.904159
241 10346745.1492234
242 10155330.0082183
243 10136166.4857304
244 9271024.18839646
245 9368589.34064073
246 9901677.83113807
247 10437632.6124647
248 10260774.5219779
249 9496426.97252437
250 8715070.45514168
251 9465056.95977652
252 9751623.16903132
253 10050036.4120868
254 9746626.34981286
255 10316968.9243964
256 10036739.1233265
257 10569467.6373175
258 9202503.86693346
259 9179859.46211793
260 9754525.23909155
261 9131915.55272836
262 7350593.28930086
263 7981743.60289241
264 9621093.72207372
265 8063911.95878846
266 7360375.78867169
267 7261544.19206604
268 7079156.13021651
269 7494847.24359709
270 7424206.14462547
271 9715705.44962039
272 10212991.1510851
273 11277008.8970984
274 10365237.603113
275 11707559.1042721
276 13187848.6481024
277 12240964.6153968
278 11481806.7185748
279 13457803.7773578
280 13089311.8141142
281 13490409.4736893
282 10710840.0114038
283 10667019.1903338
284 11316583.4307238
285 11689252.6191757
286 12606588.415642
287 11816477.3237113
288 9255868.56382632
289 9459030.29156152
290 8639643.78002709
291 11855200.85249
292 12219652.2588667
293 12171019.5472775
294 10175642.6264709
295 8830223.30957548
296 9466243.86295523
297 11863130.1125112
298 12334938.3493319
299 11210373.8238755
300 11872610.3605884
301 9946524.28882423
302 8922115.0875076
303 10328852.3094796
304 10306895.1987275
305 8876409.09620457
306 9501703.31698428
307 11006752.5566433
308 8522516.09926422
309 7864508.06438988
310 8012005.863609
311 9141966.96838263
312 8077815.5182957
313 10899878.329779
314 9855181.0564264
315 10043554.9628045
316 12237795.6572362
317 9876487.27640231
318 8906427.66291531
319 8191241.60880075
320 9655595.30160562
321 10344674.4262347
322 13440076.8325945
323 11761316.9826697
324 11456760.0088284
325 10699609.0772376
326 10129416.06816
327 11048543.5304158
328 9860887.63573701
329 11211195.7060924
330 11076059.2151135
331 11124740.6030982
332 11623896.4454388
333 11256470.7864837
334 9257730.95600245
335 10325750.2293582
336 12327195.4741414
337 11100158.8153348
338 10712002.4203706
339 12348261.1203933
340 10261463.8962292
341 10677789.1548738
342 11484570.0401058
343 10558486.0136644
344 11024026.0204568
345 10715227.9596405
346 10268509.2844358
347 9172028.12693946
348 11003513.4961525
349 15754546.2478165
350 13870972.7669878
351 12435723.8663787
352 12529454.6343738
353 15442978.6249056
354 13434873.0346857
355 13831431.7243866
356 14756447.1583329
357 16074325.6202578
358 12037533.0612959
359 11715586.1331219
360 11761258.3213725
361 11385947.0863409
362 14432169.5876993
363 13553328.4993489
364 14164350.811359
365 11087859.2886735
366 13095594.8134762
367 12237481.5488719
368 11718691.0214969
369 11016920.098784
370 9215281.26474702
371 9264228.99478274
372 11478310.3388477
373 10896468.9019038
374 11737823.3411671
375 12721817.5089602
376 14027981.1850266
377 15979655.4394327
378 16815478.243183
379 16901873.4367673
380 15371111.4632192
381 12430093.629962
382 12820745.418263
383 13041135.9117754
384 12679057.5836903
385 10647130.5143637
386 10485755.2619947
387 10884006.7045532
388 9613907.76517364
389 9337120.98701259
390 8961861.39264418
391 8083108.45225366
392 8164065.09873097
393 7548877.83462095
394 9461927.57718971
395 10406719.0449344
396 10233511.9961408
397 8835809.70603344
398 8580966.15337063
399 7897046.98669159
400 8093453.64243611
401 9215318.37009768
402 8609873.69175424
403 9003522.14665768
404 9228554.26731426
405 9347016.5445927
406 8444486.17307309
407 7460284.29852382
408 7459952.09252524
409 8798766.24096639
410 8957977.83795478
411 8815211.11395713
412 10129104.8720587
413 9574249.55599222
414 8670313.34152339
415 9474622.65156938
416 10015503.9417654
417 10357900.5725708
418 8800646.1587251
419 9437841.55019407
420 10513503.0956557
421 10657636.2950476
422 9266965.50789299
423 9061299.25999676
424 8942097.6839588
425 9427169.35448265
426 8079596.67913655
427 8308210.05118421
428 8967515.39520705
429 7446043.33254839
430 6918684.61545611
431 7568728.85919106
432 6488950.2579546
433 6112237.80839171
434 6397807.24363669
435 6410615.7922632
436 7501288.96129153
437 6739488.91388034
438 6575314.66475881
439 9797084.01019702
440 6150540.30715094
};
\addplot [goldenrod1911910]
table {%
0 9883601.24631407
1 18811810.023743
2 16388978.9931058
3 12750193.3628457
4 12866885.1320491
5 11237114.2629386
6 10826095.7445589
7 10335315.4832238
8 8931911.98091591
9 8754929.70267
10 9350620.12209862
11 9248243.12725661
12 8697187.56883162
13 8269741.78577102
14 7477007.15529915
15 7502095.93802076
16 10298031.8552906
17 11174128.099268
18 10002145.5136478
19 9416707.74078847
20 8386948.11798123
21 12432482.3566744
22 9395577.18043257
23 7909797.72498885
24 9639865.79836171
25 7959614.60513868
26 8539349.39581477
27 9162816.56213214
28 7384904.60286837
29 6789562.56067839
30 7027449.35192787
31 6042084.5057625
32 6156670.07105656
33 7094534.97823199
34 6804379.35182453
35 6406502.84369703
36 6766676.76010171
37 5489960.85353035
38 6741362.9970491
39 5946657.13268363
40 6785239.85307427
41 5528755.51966556
42 6338092.4962577
43 9014329.0856417
44 8681494.50948096
45 7716263.89474382
46 7030215.77255734
47 5649231.08807507
48 5781315.70956694
49 5522342.62141692
50 5332214.19723308
51 5339593.56275009
52 5081672.13924033
53 5056371.07045892
54 5516535.20336208
55 5077110.67190656
56 3828313.35778867
57 7412833.03854046
58 6124972.24055523
59 7747066.00727195
60 7899182.62206936
61 5932799.19521987
62 5811711.17170761
63 5714076.24189959
64 5167149.47590995
65 5666147.44347491
66 5586329.57280885
67 5200702.00600738
68 5047809.46099761
69 4993496.59947334
70 4975760.74185699
71 5265837.08039109
72 4648591.34324701
73 5666653.40179355
74 5654209.06840183
75 5079806.57914127
76 5073864.50395922
77 4720171.72400742
78 5017465.29837006
79 4702664.05146985
80 4688717.52423691
81 5095916.57873519
82 5659515.26137955
83 5909979.41538448
84 5652888.07101034
85 8025000.6237779
86 5751228.39410115
87 5058581.46139111
88 5196901.27268788
89 5920428.25926722
90 7375807.20135061
91 5169316.60074717
92 4685961.08936232
93 4326577.28110919
94 4705690.78192487
95 5410146.94879127
96 4280392.05611399
97 4358867.52875636
98 5003532.99761446
99 4814801.16733541
100 2965631.82152845
101 5066343.73063825
102 4122426.86065628
103 4781305.8787124
104 5339911.35744345
105 6008692.05031023
106 13600852.0132033
107 11688215.3588332
108 5328220.54214851
109 5441167.6187234
110 4936077.03337252
111 3709303.59508975
112 4085310.49733785
113 5034188.73258281
114 4444087.32335599
115 7713742.93084358
116 9683628.51758535
117 5890932.97015814
118 5785620.45322312
119 6265039.97240321
120 8221454.80061613
121 7868669.60760719
122 6947082.94452928
123 4076166.6531123
124 11432539.9343826
125 15229823.7813093
126 17557683.8022553
127 14663897.4545074
128 16667418.3600006
129 13881618.8700189
130 11178502.3963488
131 12154266.950592
132 13542686.4178696
133 10145322.7390213
134 12112377.2859136
135 10802970.5973654
136 7741640.59027038
137 11951617.6981734
138 10905615.2990454
139 8080693.51682435
140 9912526.06998526
141 7235264.21136502
142 9497665.41649008
143 9000283.59891043
144 16529924.9355843
145 13455773.4132188
146 10684217.3137304
147 9096767.33941028
148 16387047.9021795
149 9688695.37720726
150 10378419.7999577
151 12934347.4920887
152 7365659.65646248
153 7493852.94966929
154 9057494.95903883
155 9367392.13226064
156 10874434.7235786
157 10341832.0410344
158 6876935.56711837
159 6980858.52135391
160 8097284.22266896
161 8027712.54100832
162 13773070.3913048
163 11257814.252372
164 7703441.41550751
165 8577633.26253775
166 11018475.6514006
167 9959792.13930823
168 10123301.4763114
169 10899526.1798265
170 9787501.7488349
171 12344297.6267245
172 15645087.1055887
173 14807380.659077
174 17577227.9435133
175 11935782.9765068
176 13735818.5289486
177 20811883.8331335
178 13548271.5298178
179 19418241.1356808
180 9846337.30625603
181 8420011.8758953
182 9130473.97163472
183 10377529.4552715
184 13041384.4635228
185 11889508.7653848
186 21978190.3183775
187 15991907.9106492
188 12910035.1080453
189 11672797.4894793
190 9511909.85956828
191 17371284.4345363
192 16226478.3414645
193 14265854.5507533
194 37965450.1406218
195 17463117.7548079
196 14287476.0759796
197 13706806.5025942
198 13046383.9492059
199 12076928.6206158
200 22697878.2994635
201 24145097.9871804
202 20886139.7145823
203 20768365.4947687
204 11484081.0579516
205 12323363.2679762
206 12836064.4674052
207 16471147.4035223
208 18606189.4407301
209 21186063.2479679
210 14264950.4849776
211 16152710.8981828
212 12077454.6603032
213 12476766.4611601
214 20296506.5132678
215 12611039.831124
216 12690110.7061249
217 11660632.7191603
218 13699591.3236088
219 13102683.652552
220 25018379.7110597
221 14887147.0112179
222 14046104.0841511
223 13829449.5618976
224 23755220.1259316
225 15101263.7013232
226 12083332.27497
227 10386022.1893748
228 14235119.4543992
229 15333619.0126227
230 8939196.55653809
231 8333962.71823106
232 7753907.51774318
233 8386981.45444258
234 7914049.59830319
235 7483373.98158964
236 7476335.87156586
237 7295112.70402944
238 9006276.39929558
239 12929346.5426304
240 10501934.4550419
241 15080801.7165612
242 7806709.36517019
243 7774983.78540315
244 6367919.30547511
245 8339713.55239693
246 15149181.8094238
247 20663423.6372651
248 15635728.1350645
249 6021037.16016211
250 5197874.25477637
251 5332998.83893073
252 5143861.93556199
253 4747545.19068519
254 7775745.91148603
255 11568288.3681237
256 8534284.01600774
257 12611696.5102501
258 7527364.60194482
259 9443320.84814551
260 7281471.15696344
261 6138295.55826891
262 6220930.35469251
263 5576071.53221101
264 6358801.85991107
265 6478909.86060084
266 6524134.96269775
267 6768963.84124506
268 7071962.36818552
269 7023151.93631229
270 8038615.17018639
271 12456617.830488
272 8757785.70042587
273 10724948.4073118
274 11792568.2971831
275 15028374.0374401
276 16673918.106387
277 11816513.1948458
278 14772345.5534031
279 13360668.3131925
280 11996526.6264946
281 12791710.3619268
282 10234248.1654357
283 10675136.0454464
284 18679597.5936604
285 14440433.9147571
286 18212387.499468
287 13282077.9875889
288 9682476.08223865
289 9947523.86443764
290 8875993.80263277
291 14699407.5564081
292 14278996.2017783
293 12871402.0174003
294 11387573.9418098
295 10045731.6842498
296 10654686.9502499
297 13326445.3488891
298 12637079.2834139
299 11973746.5725725
300 16146198.4209765
301 10624391.5802664
302 12127003.2556593
303 13991510.5235237
304 11085235.9248982
305 8952183.56014371
306 11269470.6135573
307 16286544.46198
308 10049382.7158641
309 8830571.1114746
310 9044041.47138826
311 10533926.6154236
312 9409875.20120019
313 13407237.170837
314 21622214.1933799
315 22368550.3739015
316 14198651.4678543
317 11329994.5851777
318 9884524.39092296
319 9471770.24393127
320 10387625.045024
321 9726070.55539101
322 11510825.2894466
323 10268985.746672
324 16100735.6763924
325 14728644.8653386
326 10382225.1099432
327 10862967.6851181
328 13001195.8429127
329 9832002.36658422
330 10193835.3325327
331 9571120.02121222
332 9538105.66639925
333 9659509.45230348
334 8366952.1838232
335 10183306.7420626
336 13177585.2525654
337 9656901.42253798
338 13532313.5746157
339 10878097.0201475
340 9700158.74794406
341 9922558.14299589
342 9933343.20205465
343 7320195.65595006
344 8517136.90567275
345 8282872.90711061
346 7752283.31153006
347 6955758.28765911
348 14625572.9859334
349 16126048.9578851
350 8593551.79990815
351 11277939.3131977
352 12104798.8202124
353 14085879.5836106
354 11413626.6397043
355 11850199.4466518
356 11857975.8611337
357 15757178.4276961
358 9021933.59168979
359 9574551.03531297
360 13446626.8331204
361 10738122.3628738
362 10782568.2083051
363 11819201.8765493
364 10207219.6315704
365 8455903.70243916
366 6552934.28021844
367 8007401.79591436
368 9028639.85394085
369 7817144.77121316
370 7189103.33059923
371 6996851.37636884
372 7657794.44494027
373 7320348.01621132
374 8497578.79891898
375 9390825.69317072
376 11206630.7286379
377 13567109.0870966
378 16853912.6101585
379 16612538.0158021
380 16019687.4842807
381 11425674.7075768
382 9250213.67775214
383 10613557.9828664
384 11660611.9586327
385 8312404.95886759
386 8436482.89940676
387 9721279.86700648
388 9848225.97851065
389 8145885.38464492
390 8125214.12193413
391 7022000.6259567
392 6396211.47377594
393 5644373.23446144
394 5709789.25339572
395 5130094.48337769
396 4974252.03143936
397 5379050.87857515
398 6024830.00431863
399 6015563.64763024
400 6655821.14470017
401 5495636.1460525
402 4623383.80903732
403 3935510.51405155
404 5167400.64556787
405 5709503.72371445
406 5451783.04755469
407 5363555.47970613
408 5549248.14612983
409 6756327.14524276
410 6990204.18294928
411 6397561.93887193
412 6329072.95644436
413 5951512.70445569
414 5327525.68266296
415 5626526.18273041
416 6911467.37814767
417 6349944.54570083
418 5576851.91136668
419 5672884.26354192
420 5137519.45122432
421 8797871.08371573
422 6542050.24485506
423 5872843.44194578
424 5919992.35638299
425 5477446.44444285
426 5935593.46668201
427 6780609.98617514
428 7747366.19615689
429 5806165.58724879
430 5463059.38540374
431 5018669.28388196
432 5581439.6120275
433 5459177.22342072
434 5203045.28235531
435 5283859.47602355
436 4945307.33560802
437 4182637.9189835
438 4554347.22789393
439 5868283.78925986
440 4645317.48779893
};
\addlegendentry{Test data}

\addlegendentry{LAR}

\addlegendentry{LSTM}
\addlegendentry{SVGP}
\addlegendimage{line width=5pt,draw=goldenrod1911910,opacity=0.5}
\addlegendentry{$95\%$ LAR CI}
\addlegendimage{line width=5pt,draw=blue,opacity=0.5}
\addlegendentry{$95\%$ SVGP CI}

\end{axis}


\end{tikzpicture}

		% This file was created with tikzplotlib v0.10.1.
\begin{tikzpicture}
\definecolor{green01270}{RGB}{0,127,0}
\definecolor{steelblue31119180}{RGB}{31,119,180}
\definecolor{darkgray176}{RGB}{176,176,176}
\definecolor{lightgray204}{RGB}{204,204,204}

\begin{axis}[
width=\figurewidth,
height=\figureheight,
axis background/.style={fill=background_color},
axis line style={white},
tick align=inside,
tick pos=left,
ylabel={Contributions (kWh)},
x grid style={white},
xmajorgrids,
xmin=150, xmax=300,
xtick style={color=white},
y grid style={white},
ymajorgrids,
xlabel={Time (days)},
xtick style={color=black},
ymin=-13237205.6599335, ymax=119040419.31714,
ytick style={color=black},
ytick={-20000000,0,20000000,40000000,60000000,80000000,100000000,120000000},
yticklabels={\ensuremath{-}0.2,0.0,0.2,0.4,0.6,0.8,1.0,1.2},
legend columns=3, 
legend style={
  nodes={scale=0.8, transform shape},
  fill opacity=0.8,
  draw opacity=1,
  text opacity=1,
  at={(0.03,0.97)},
  anchor=north west,
  legend pos=north east,
  draw=lightgray204,
  fill=background_color
},
]
\path [draw=blue, fill=blue, opacity=0.5]
(axis cs:0,31925052.2091835)
--(axis cs:0,1558183.05204632)
--(axis cs:1,-2349178.42999233)
--(axis cs:2,-5126183.6097403)
--(axis cs:3,7024335.79619604)
--(axis cs:4,-5739164.84421941)
--(axis cs:5,-3654257.77045015)
--(axis cs:6,-5519537.2407308)
--(axis cs:7,-5765991.04656563)
--(axis cs:8,-7934752.50211442)
--(axis cs:9,-6512049.72912347)
--(axis cs:10,-3992347.33414757)
--(axis cs:11,-5316205.56090487)
--(axis cs:12,-7888235.74583745)
--(axis cs:13,-4324348.01138568)
--(axis cs:14,-5820781.84119326)
--(axis cs:15,-6001510.58673185)
--(axis cs:16,-11084451.5735503)
--(axis cs:17,-8505380.47099489)
--(axis cs:18,-6317763.97851044)
--(axis cs:19,-6043994.00266101)
--(axis cs:20,-7252956.23418731)
--(axis cs:21,-9618673.39036574)
--(axis cs:22,-9508850.00971471)
--(axis cs:23,-7867763.64765651)
--(axis cs:24,-7686728.24722127)
--(axis cs:25,-8944613.44307322)
--(axis cs:26,-9203281.84058987)
--(axis cs:27,-12453972.7203834)
--(axis cs:28,-10550814.3180554)
--(axis cs:29,-9691860.75423588)
--(axis cs:30,-10218377.0003186)
--(axis cs:31,-9508590.36013648)
--(axis cs:32,-9857440.87546588)
--(axis cs:33,-9074201.31928582)
--(axis cs:34,-10317932.3402869)
--(axis cs:35,-9950043.87293975)
--(axis cs:36,-11075159.002645)
--(axis cs:37,-10174903.9501448)
--(axis cs:38,-10208930.3893856)
--(axis cs:39,-9248027.75112463)
--(axis cs:40,-8619619.34733264)
--(axis cs:41,-9614853.35479374)
--(axis cs:42,-8186434.7662968)
--(axis cs:43,-9915719.53105286)
--(axis cs:44,-12058437.8203891)
--(axis cs:45,-7871106.80845477)
--(axis cs:46,-7684395.08465428)
--(axis cs:47,-8683689.24909451)
--(axis cs:48,-7640804.68908913)
--(axis cs:49,-8386188.30279893)
--(axis cs:50,-6805722.15997268)
--(axis cs:51,-5590380.43944634)
--(axis cs:52,-8412910.03861637)
--(axis cs:53,-9695413.20699674)
--(axis cs:54,-10997308.6131284)
--(axis cs:55,-11806959.4750052)
--(axis cs:56,-9590038.68815796)
--(axis cs:57,-3881932.28977642)
--(axis cs:58,-2873247.41678162)
--(axis cs:59,999975.446154563)
--(axis cs:60,-2879697.2291535)
--(axis cs:61,-6746630.71858448)
--(axis cs:62,-8133593.36079635)
--(axis cs:63,-9808614.55604794)
--(axis cs:64,-8912413.63587536)
--(axis cs:65,-10595047.389887)
--(axis cs:66,-10176361.9092951)
--(axis cs:67,-10822283.2051554)
--(axis cs:68,-11819859.7472139)
--(axis cs:69,-12190168.3045937)
--(axis cs:70,-11239421.6797118)
--(axis cs:71,-10391904.972889)
--(axis cs:72,-8193895.91549887)
--(axis cs:73,-7480118.93072725)
--(axis cs:74,-6562568.40929183)
--(axis cs:75,-10190673.2652991)
--(axis cs:76,-10196818.2091364)
--(axis cs:77,-11388375.9572337)
--(axis cs:78,-10458419.1367197)
--(axis cs:79,-11476382.4226019)
--(axis cs:80,-12161082.352936)
--(axis cs:81,-10413567.1377354)
--(axis cs:82,-10181157.6861187)
--(axis cs:83,-9463024.05623508)
--(axis cs:84,-11226987.1542857)
--(axis cs:85,-8206325.73694822)
--(axis cs:86,-9750624.47650647)
--(axis cs:87,-11608829.2764832)
--(axis cs:88,-9898938.5198346)
--(axis cs:89,-11260335.1551459)
--(axis cs:90,-11675299.3615661)
--(axis cs:91,-10101100.4853032)
--(axis cs:92,-12278903.7568894)
--(axis cs:93,-12868703.5835604)
--(axis cs:94,-12989606.6385469)
--(axis cs:95,-14206425.8247409)
--(axis cs:96,-14002650.0815878)
--(axis cs:97,-12246115.1844074)
--(axis cs:98,-8961808.1432364)
--(axis cs:99,-10761121.7090604)
--(axis cs:100,-11053442.2467161)
--(axis cs:101,-12246281.4725702)
--(axis cs:102,-9879437.24328485)
--(axis cs:103,-12150198.4976553)
--(axis cs:104,-10383555.0509901)
--(axis cs:105,-366851.773162145)
--(axis cs:106,1682683.8349628)
--(axis cs:107,-7779164.36966727)
--(axis cs:108,-7923937.64003375)
--(axis cs:109,-11175532.2257438)
--(axis cs:110,-8963421.81197097)
--(axis cs:111,-11720419.2035788)
--(axis cs:112,-10939223.0135852)
--(axis cs:113,-12238728.7816918)
--(axis cs:114,-7866876.97170314)
--(axis cs:115,-1052507.0173686)
--(axis cs:116,-4289961.94679056)
--(axis cs:117,-9100466.9765065)
--(axis cs:118,-4813478.10293579)
--(axis cs:119,-8961011.58963506)
--(axis cs:120,-5252400.85605777)
--(axis cs:121,-9800714.86309974)
--(axis cs:122,-7715383.05326862)
--(axis cs:123,-5916935.19178989)
--(axis cs:124,-578362.954528984)
--(axis cs:125,593699.592696244)
--(axis cs:126,2314584.49466387)
--(axis cs:127,-2637917.72364703)
--(axis cs:128,-3266867.202149)
--(axis cs:129,316786.819850437)
--(axis cs:130,-7032436.79560497)
--(axis cs:131,3374600.8491869)
--(axis cs:132,-9156326.43965145)
--(axis cs:133,-6945960.33932743)
--(axis cs:134,-2144874.69709749)
--(axis cs:135,-2772530.35299499)
--(axis cs:136,-7732827.46535706)
--(axis cs:137,-1292919.52696036)
--(axis cs:138,-785066.349835197)
--(axis cs:139,-3445723.03501045)
--(axis cs:140,-203108.971859908)
--(axis cs:141,-1509748.63741338)
--(axis cs:142,379150.266811427)
--(axis cs:143,20709.6645837016)
--(axis cs:144,4744948.25751818)
--(axis cs:145,-3301579.0380511)
--(axis cs:146,45122.0313422345)
--(axis cs:147,1562533.69548005)
--(axis cs:148,-1698715.79518692)
--(axis cs:149,-905689.069364598)
--(axis cs:150,-4108932.19826762)
--(axis cs:151,-13079226.4851881)
--(axis cs:152,-3192418.99894056)
--(axis cs:153,2222276.85105362)
--(axis cs:154,-4791347.64706116)
--(axis cs:155,-248073.938784458)
--(axis cs:156,8837189.47384572)
--(axis cs:157,-6454322.05522016)
--(axis cs:158,-3229375.26130272)
--(axis cs:159,-6384249.25738256)
--(axis cs:160,-8708637.90086552)
--(axis cs:161,-6298587.35511497)
--(axis cs:162,-4707749.7346411)
--(axis cs:163,-3829980.91429561)
--(axis cs:164,-7337736.53589446)
--(axis cs:165,-5168010.10449969)
--(axis cs:166,745499.95747696)
--(axis cs:167,-4559099.62492303)
--(axis cs:168,-84431.084169345)
--(axis cs:169,-4303216.5154261)
--(axis cs:170,3918002.50425644)
--(axis cs:171,6985184.21120862)
--(axis cs:172,12675692.2032619)
--(axis cs:173,10618426.0757076)
--(axis cs:174,-814248.452844203)
--(axis cs:175,-2909265.61995627)
--(axis cs:176,-499015.022293478)
--(axis cs:177,-1620754.03801222)
--(axis cs:178,7351054.12826618)
--(axis cs:179,-8258343.85496712)
--(axis cs:180,-3280326.29834601)
--(axis cs:181,2131225.77778679)
--(axis cs:182,-1462677.88899643)
--(axis cs:183,3690627.24892992)
--(axis cs:184,5842298.98514461)
--(axis cs:185,16432308.6725312)
--(axis cs:186,21281139.8305038)
--(axis cs:187,-4472900.08934802)
--(axis cs:188,-2202226.28799476)
--(axis cs:189,-3350400.07992559)
--(axis cs:190,-2328484.81517838)
--(axis cs:191,6336578.61174068)
--(axis cs:192,8764073.82973653)
--(axis cs:193,25901757.9152522)
--(axis cs:194,20394524.7527422)
--(axis cs:195,2812282.90936599)
--(axis cs:196,4259804.9059528)
--(axis cs:197,1279836.27846912)
--(axis cs:198,-1252450.20391109)
--(axis cs:199,8475398.71222135)
--(axis cs:200,36825318.7224212)
--(axis cs:201,9021208.21018066)
--(axis cs:202,195298.788273076)
--(axis cs:203,435029.921350647)
--(axis cs:204,5678258.22176882)
--(axis cs:205,934244.1210443)
--(axis cs:206,7323924.95539739)
--(axis cs:207,7204040.78363095)
--(axis cs:208,12479063.4198653)
--(axis cs:209,-919939.564798279)
--(axis cs:210,5186715.38002818)
--(axis cs:211,21386553.9029942)
--(axis cs:212,-1579158.20875236)
--(axis cs:213,7108443.91654442)
--(axis cs:214,-5941755.69235916)
--(axis cs:215,-3625740.48573226)
--(axis cs:216,-2668942.69055533)
--(axis cs:217,-3838966.2631075)
--(axis cs:218,14468840.6339481)
--(axis cs:219,15672440.7025596)
--(axis cs:220,7939348.36185779)
--(axis cs:221,-3717474.32934823)
--(axis cs:222,3000447.0702942)
--(axis cs:223,12585582.5447384)
--(axis cs:224,2881721.21555357)
--(axis cs:225,3265751.28548337)
--(axis cs:226,-4541305.94211222)
--(axis cs:227,689822.42365418)
--(axis cs:228,-2261143.88942303)
--(axis cs:229,-383863.913648708)
--(axis cs:230,-4708664.91143827)
--(axis cs:231,-3142078.52519059)
--(axis cs:232,-7774746.54997696)
--(axis cs:233,-5282461.14722271)
--(axis cs:234,-7845759.78078065)
--(axis cs:235,-7582433.93274541)
--(axis cs:236,-4626678.36031134)
--(axis cs:237,-4017774.64428618)
--(axis cs:238,8733603.25581942)
--(axis cs:239,-1852390.11798437)
--(axis cs:240,15700827.0186618)
--(axis cs:241,-3003346.74903443)
--(axis cs:242,-1958493.87660743)
--(axis cs:243,-7322090.70865847)
--(axis cs:244,-5145740.85409056)
--(axis cs:245,2436020.753153)
--(axis cs:246,10920395.2591011)
--(axis cs:247,5124668.85916309)
--(axis cs:248,-4128684.09758325)
--(axis cs:249,-8816196.22128519)
--(axis cs:250,-8052250.91874349)
--(axis cs:251,-5077077.16364152)
--(axis cs:252,-6331015.13114138)
--(axis cs:253,-7982763.43616026)
--(axis cs:254,8079677.3621595)
--(axis cs:255,6919896.66031865)
--(axis cs:256,2057689.50909814)
--(axis cs:257,-224988.159488656)
--(axis cs:258,-927370.683494868)
--(axis cs:259,7303811.56633783)
--(axis cs:260,-6709166.44938457)
--(axis cs:261,-4198965.1839363)
--(axis cs:262,-5675773.8987025)
--(axis cs:263,-10747278.5740637)
--(axis cs:264,-14888096.7685547)
--(axis cs:265,-6543280.48251369)
--(axis cs:266,-10537126.7210858)
--(axis cs:267,-12836986.9649774)
--(axis cs:268,-7817245.99357252)
--(axis cs:269,-5655818.35049332)
--(axis cs:270,-3942387.54602846)
--(axis cs:271,-5472566.71012365)
--(axis cs:272,-1519103.24536341)
--(axis cs:273,-3633118.85919705)
--(axis cs:274,-1998811.21812547)
--(axis cs:275,4916528.65427052)
--(axis cs:276,-1150301.32548789)
--(axis cs:277,6658898.0258016)
--(axis cs:278,-2672208.59684595)
--(axis cs:279,-2954985.33492053)
--(axis cs:280,-471033.765561892)
--(axis cs:281,-3766971.25024104)
--(axis cs:282,-796979.105053034)
--(axis cs:283,-3500082.46696732)
--(axis cs:284,-2449309.53580593)
--(axis cs:285,-5397370.14119626)
--(axis cs:286,775122.449253762)
--(axis cs:287,-117921.479802124)
--(axis cs:288,-5808964.123311)
--(axis cs:289,-2865349.47467298)
--(axis cs:290,-3100333.76831)
--(axis cs:291,2893625.77789757)
--(axis cs:292,-2105200.55447249)
--(axis cs:293,-5986145.96593835)
--(axis cs:294,-5651700.59759589)
--(axis cs:295,-250026.212397311)
--(axis cs:296,1288371.56347809)
--(axis cs:297,3125219.5235871)
--(axis cs:298,1140581.10862351)
--(axis cs:299,518916.163653873)
--(axis cs:300,676044.219172895)
--(axis cs:301,-1605513.45558587)
--(axis cs:302,5454284.30100858)
--(axis cs:303,-572799.026433457)
--(axis cs:304,-7549702.51513581)
--(axis cs:305,-6714725.28570318)
--(axis cs:306,-1993622.99019448)
--(axis cs:307,-4447103.84677581)
--(axis cs:308,-6961875.57578766)
--(axis cs:309,-6957437.87394874)
--(axis cs:310,-4127747.68718458)
--(axis cs:311,-2645422.11963348)
--(axis cs:312,-407109.279278984)
--(axis cs:313,2437900.30792989)
--(axis cs:314,28822742.725176)
--(axis cs:315,1184600.58098856)
--(axis cs:316,-2066237.6451567)
--(axis cs:317,5878216.84809564)
--(axis cs:318,-6838110.7408238)
--(axis cs:319,-6180608.24577861)
--(axis cs:320,-4015879.7811871)
--(axis cs:321,-5092235.02825681)
--(axis cs:322,-6379543.09223432)
--(axis cs:323,-942438.842127299)
--(axis cs:324,856689.024505397)
--(axis cs:325,-1353666.58415581)
--(axis cs:326,-3720176.90352517)
--(axis cs:327,1959543.25176384)
--(axis cs:328,3864019.14557752)
--(axis cs:329,-3676171.24807593)
--(axis cs:330,-3590186.00506706)
--(axis cs:331,-32253.3700084165)
--(axis cs:332,-1616623.49092088)
--(axis cs:333,-4084865.89205346)
--(axis cs:334,-2895955.2276528)
--(axis cs:335,-1348947.35607245)
--(axis cs:336,-3466253.48573287)
--(axis cs:337,-2525303.4693416)
--(axis cs:338,-971093.557603233)
--(axis cs:339,-7099395.82810367)
--(axis cs:340,-4315499.32319484)
--(axis cs:341,62331.2809176389)
--(axis cs:342,-1806981.5735455)
--(axis cs:343,-6983769.5642515)
--(axis cs:344,-5724185.45903565)
--(axis cs:345,-6333634.46210776)
--(axis cs:346,-7603192.57254876)
--(axis cs:347,-3978146.26133446)
--(axis cs:348,4422665.37988386)
--(axis cs:349,-6565592.91870088)
--(axis cs:350,643051.459906681)
--(axis cs:351,3558832.00327458)
--(axis cs:352,2541305.64126211)
--(axis cs:353,-5214143.26673375)
--(axis cs:354,-2684596.3567352)
--(axis cs:355,-3807671.42530994)
--(axis cs:356,-339871.723502874)
--(axis cs:357,-5842403.1332742)
--(axis cs:358,-6136824.94669376)
--(axis cs:359,-4013118.53470249)
--(axis cs:360,-5766348.3628489)
--(axis cs:361,-3616617.3141033)
--(axis cs:362,-3759699.71121432)
--(axis cs:363,-6910818.05283808)
--(axis cs:364,-8880245.43715982)
--(axis cs:365,-4681542.96467946)
--(axis cs:366,-9369520.50944607)
--(axis cs:367,-4634835.48882621)
--(axis cs:368,-4339646.66726598)
--(axis cs:369,-6917627.07580572)
--(axis cs:370,-6979306.96651278)
--(axis cs:371,-7252840.70150443)
--(axis cs:372,-7955050.53245939)
--(axis cs:373,-6347402.33488897)
--(axis cs:374,-5555166.42674215)
--(axis cs:375,-2428721.77877922)
--(axis cs:376,1592489.14203584)
--(axis cs:377,-2142017.05971332)
--(axis cs:378,-7367561.03988581)
--(axis cs:379,-9900628.49364917)
--(axis cs:380,-11564176.4938226)
--(axis cs:381,-8924371.42075885)
--(axis cs:382,-6827682.90673274)
--(axis cs:383,-7438517.9373647)
--(axis cs:384,-6287035.23299627)
--(axis cs:385,-5966732.6596797)
--(axis cs:386,-5675633.57390694)
--(axis cs:387,-6727392.4676682)
--(axis cs:388,-6477449.466322)
--(axis cs:389,-6628169.29796252)
--(axis cs:390,-5203233.91729769)
--(axis cs:391,-8917686.83899321)
--(axis cs:392,-7918904.98372306)
--(axis cs:393,-9040797.2855427)
--(axis cs:394,-7688559.45463303)
--(axis cs:395,-9209191.2304942)
--(axis cs:396,-8958116.55621583)
--(axis cs:397,-9944916.01480553)
--(axis cs:398,-9179177.91944674)
--(axis cs:399,-8016894.61824447)
--(axis cs:400,-9727060.09788186)
--(axis cs:401,-9733658.99045403)
--(axis cs:402,-10792326.8587067)
--(axis cs:403,-9879890.6213648)
--(axis cs:404,-7081582.96955026)
--(axis cs:405,-7985604.0927485)
--(axis cs:406,-5700945.79484526)
--(axis cs:407,-8744974.86359898)
--(axis cs:408,-9913388.17994212)
--(axis cs:409,-9745405.08527431)
--(axis cs:410,-7312426.12676026)
--(axis cs:411,-7656846.79881373)
--(axis cs:412,-11131917.7448056)
--(axis cs:413,-9045677.80445304)
--(axis cs:414,-9448598.83601603)
--(axis cs:415,-7127846.21555243)
--(axis cs:416,-7054250.62536332)
--(axis cs:417,-10186030.2502215)
--(axis cs:418,-10530531.4496492)
--(axis cs:419,-12068525.8750916)
--(axis cs:420,-11804300.242239)
--(axis cs:421,-10854162.5215055)
--(axis cs:422,-10045247.4160132)
--(axis cs:423,-8869189.03546282)
--(axis cs:424,-10393591.1961976)
--(axis cs:425,-11178616.6971534)
--(axis cs:426,-10217213.559802)
--(axis cs:427,-11004154.3647045)
--(axis cs:428,-11694457.6975439)
--(axis cs:429,-10188609.9338197)
--(axis cs:430,-11077471.8828855)
--(axis cs:431,-7823382.36761869)
--(axis cs:432,-10426773.2511418)
--(axis cs:433,-10949175.1493761)
--(axis cs:434,-10452084.9037486)
--(axis cs:435,-11304130.2051657)
--(axis cs:436,-9560220.10306126)
--(axis cs:437,-9372052.86961551)
--(axis cs:438,-10830191.1953708)
--(axis cs:439,-11380050.1642548)
--(axis cs:440,-10675003.7696071)
--(axis cs:440,19269425.9866216)
--(axis cs:440,19269425.9866216)
--(axis cs:439,19640647.2693746)
--(axis cs:438,19020723.6416415)
--(axis cs:437,20639957.2108487)
--(axis cs:436,20335850.1715696)
--(axis cs:435,18494749.5862448)
--(axis cs:434,19331435.1674132)
--(axis cs:433,18806153.5582905)
--(axis cs:432,19321588.2045393)
--(axis cs:431,21962758.3859266)
--(axis cs:430,18737792.00468)
--(axis cs:429,19590998.7648917)
--(axis cs:428,18796852.7348172)
--(axis cs:427,18924937.8609915)
--(axis cs:426,19701731.6103114)
--(axis cs:425,18976156.7492356)
--(axis cs:424,19625265.6589473)
--(axis cs:423,21033271.999057)
--(axis cs:422,19864183.5279189)
--(axis cs:421,19297754.0258822)
--(axis cs:420,18486836.8761797)
--(axis cs:419,18010386.9957854)
--(axis cs:418,19605017.3381678)
--(axis cs:417,19938124.154487)
--(axis cs:416,22818711.9023564)
--(axis cs:415,22865368.0636351)
--(axis cs:414,20314695.1352992)
--(axis cs:413,20786963.2716108)
--(axis cs:412,18844625.0233313)
--(axis cs:411,22285915.021438)
--(axis cs:410,22545371.4460086)
--(axis cs:409,20282684.9283811)
--(axis cs:408,19837605.1452264)
--(axis cs:407,21073040.1654105)
--(axis cs:406,24143090.8492918)
--(axis cs:405,22031674.317949)
--(axis cs:404,22933365.1859467)
--(axis cs:403,20629032.8299722)
--(axis cs:402,19068772.2620246)
--(axis cs:401,20090180.6126877)
--(axis cs:400,20076308.1193671)
--(axis cs:399,21776675.7817981)
--(axis cs:398,20659158.1984162)
--(axis cs:397,19994713.4402426)
--(axis cs:396,21627355.9930479)
--(axis cs:395,21267056.8130254)
--(axis cs:394,22375090.1200801)
--(axis cs:393,20860582.8612837)
--(axis cs:392,21943204.3718551)
--(axis cs:391,20864468.0923135)
--(axis cs:390,24767274.2028565)
--(axis cs:389,23138626.6503239)
--(axis cs:388,23337419.0750379)
--(axis cs:387,23242643.421837)
--(axis cs:386,24160192.4090843)
--(axis cs:385,23869879.8195718)
--(axis cs:384,23735308.0349425)
--(axis cs:383,22641086.1736571)
--(axis cs:382,23289026.7615771)
--(axis cs:381,21092565.3437877)
--(axis cs:380,19739620.6703748)
--(axis cs:379,22038847.5769604)
--(axis cs:378,24460616.4125181)
--(axis cs:377,29179198.4091185)
--(axis cs:376,32685171.734137)
--(axis cs:375,27523658.1803533)
--(axis cs:374,24405625.7128298)
--(axis cs:373,23596860.378555)
--(axis cs:372,22093912.7632429)
--(axis cs:371,22822856.4234762)
--(axis cs:370,22828571.676611)
--(axis cs:369,22994587.4215753)
--(axis cs:368,25617930.7091296)
--(axis cs:367,25536670.6597363)
--(axis cs:366,22110841.9480306)
--(axis cs:365,25231117.1398694)
--(axis cs:364,21394499.3145963)
--(axis cs:363,23147919.8890693)
--(axis cs:362,27094361.9211554)
--(axis cs:361,26440493.2345099)
--(axis cs:360,24696215.6579044)
--(axis cs:359,26054020.9083346)
--(axis cs:358,24423935.3636656)
--(axis cs:357,25426024.2752054)
--(axis cs:356,29800757.7200575)
--(axis cs:355,26196266.3364167)
--(axis cs:354,27425956.0321598)
--(axis cs:353,25941003.6730595)
--(axis cs:352,33102025.3031111)
--(axis cs:351,33702948.4861921)
--(axis cs:350,31153767.8958894)
--(axis cs:349,24258232.6093165)
--(axis cs:348,34959542.9492213)
--(axis cs:347,26093936.7295133)
--(axis cs:346,22273672.9971589)
--(axis cs:345,23462700.7714909)
--(axis cs:344,24356343.5198528)
--(axis cs:343,23217328.7678497)
--(axis cs:342,28814516.3733742)
--(axis cs:341,30558253.7436295)
--(axis cs:340,25564664.3856557)
--(axis cs:339,22872123.5148956)
--(axis cs:338,29328759.5228704)
--(axis cs:337,27777727.1559723)
--(axis cs:336,28810332.4553675)
--(axis cs:335,28597954.5398856)
--(axis cs:334,27132910.4612543)
--(axis cs:333,26372968.1079824)
--(axis cs:332,28623613.0526695)
--(axis cs:331,30149163.9909585)
--(axis cs:330,26484238.963219)
--(axis cs:329,28222462.1412297)
--(axis cs:328,37712512.6380888)
--(axis cs:327,32291999.2118638)
--(axis cs:326,26485316.5664193)
--(axis cs:325,28854738.4941449)
--(axis cs:324,31801580.931825)
--(axis cs:323,29666976.2759318)
--(axis cs:322,24551786.4359335)
--(axis cs:321,25489170.0476276)
--(axis cs:320,26198136.3222077)
--(axis cs:319,23768798.0380697)
--(axis cs:318,23340056.6480787)
--(axis cs:317,36087476.0269351)
--(axis cs:316,28415596.3126493)
--(axis cs:315,34139567.6807584)
--(axis cs:314,60863838.3730434)
--(axis cs:313,33366257.4687737)
--(axis cs:312,29513458.4266879)
--(axis cs:311,27355663.4758028)
--(axis cs:310,25820949.9027381)
--(axis cs:309,23027934.6790122)
--(axis cs:308,23034383.5956228)
--(axis cs:307,25804698.4832121)
--(axis cs:306,27946791.5099236)
--(axis cs:305,23380529.9797668)
--(axis cs:304,23539078.1827214)
--(axis cs:303,30645727.9146455)
--(axis cs:302,35497595.6395359)
--(axis cs:301,28680658.4510061)
--(axis cs:300,31065619.5016689)
--(axis cs:299,30972133.9664309)
--(axis cs:298,31884585.8371734)
--(axis cs:297,34371213.3265747)
--(axis cs:296,31214388.7081396)
--(axis cs:295,29621514.9098342)
--(axis cs:294,24586386.6182542)
--(axis cs:293,25245090.0768507)
--(axis cs:292,28683499.901802)
--(axis cs:291,33347791.0430751)
--(axis cs:290,26928491.4321136)
--(axis cs:289,28410915.2395854)
--(axis cs:288,24202614.951331)
--(axis cs:287,30239234.1086275)
--(axis cs:286,31060097.8930927)
--(axis cs:285,24964239.1570495)
--(axis cs:284,28818711.79781)
--(axis cs:283,27117228.7172924)
--(axis cs:282,29569322.3469263)
--(axis cs:281,26525478.7476235)
--(axis cs:280,30021073.5912945)
--(axis cs:279,28112171.8573654)
--(axis cs:278,29338456.1984213)
--(axis cs:277,37195866.5460679)
--(axis cs:276,29222125.9950931)
--(axis cs:275,35252237.8132786)
--(axis cs:274,28642486.7615062)
--(axis cs:273,27104403.0099693)
--(axis cs:272,29208046.46708)
--(axis cs:271,27740715.2787662)
--(axis cs:270,26461553.5305262)
--(axis cs:269,24588390.1586955)
--(axis cs:268,22698203.8038261)
--(axis cs:267,17396306.8400042)
--(axis cs:266,19799969.3186595)
--(axis cs:265,23954282.8992644)
--(axis cs:264,16066991.6593592)
--(axis cs:263,19755390.250983)
--(axis cs:262,24791122.0648047)
--(axis cs:261,25972497.138684)
--(axis cs:260,25804846.0633435)
--(axis cs:259,37800480.2073878)
--(axis cs:258,29290885.9196329)
--(axis cs:257,30366227.4692394)
--(axis cs:256,32574565.9064221)
--(axis cs:255,37555802.3679874)
--(axis cs:254,39234226.8088434)
--(axis cs:253,24623182.9131283)
--(axis cs:252,26034806.0361543)
--(axis cs:251,26618422.2944383)
--(axis cs:250,22910499.7189953)
--(axis cs:249,22124243.4499929)
--(axis cs:248,28050519.8081102)
--(axis cs:247,39070036.2228537)
--(axis cs:246,41440270.7029597)
--(axis cs:245,32850043.6753078)
--(axis cs:244,25708014.0710282)
--(axis cs:243,23413756.7006655)
--(axis cs:242,28975712.7762289)
--(axis cs:241,30683941.561438)
--(axis cs:240,46259080.2631186)
--(axis cs:239,31496419.2008197)
--(axis cs:238,40174475.8829105)
--(axis cs:237,26171851.850441)
--(axis cs:236,25649024.6792549)
--(axis cs:235,22429623.0555503)
--(axis cs:234,22151997.690262)
--(axis cs:233,24615247.0072366)
--(axis cs:232,23223621.7653129)
--(axis cs:231,27071048.7250858)
--(axis cs:230,25423241.1624265)
--(axis cs:229,29880359.325936)
--(axis cs:228,27798465.4342151)
--(axis cs:227,30822339.7019223)
--(axis cs:226,26869858.9957163)
--(axis cs:225,33563535.0229065)
--(axis cs:224,35807777.4929778)
--(axis cs:223,42841456.8329565)
--(axis cs:222,34389869.6706888)
--(axis cs:221,29077177.9436712)
--(axis cs:220,41615414.1209878)
--(axis cs:219,47024780.6863132)
--(axis cs:218,45240399.5765361)
--(axis cs:217,26956051.066773)
--(axis cs:216,27944202.9746584)
--(axis cs:215,28588851.4750366)
--(axis cs:214,26400568.4765216)
--(axis cs:213,39034024.8959944)
--(axis cs:212,30956936.6976745)
--(axis cs:211,59561941.7908069)
--(axis cs:210,35896700.2708636)
--(axis cs:209,31790822.2941905)
--(axis cs:208,43942722.113551)
--(axis cs:207,37866539.9953383)
--(axis cs:206,38637210.6060093)
--(axis cs:205,32147143.8962265)
--(axis cs:204,36049708.5101149)
--(axis cs:203,31238223.0739249)
--(axis cs:202,33690031.0910221)
--(axis cs:201,42254320.3978042)
--(axis cs:200,69681832.0603227)
--(axis cs:199,40459630.5989072)
--(axis cs:198,29194091.5553931)
--(axis cs:197,32554661.9658286)
--(axis cs:196,35048740.8634254)
--(axis cs:195,35778857.7099913)
--(axis cs:194,53623946.9098459)
--(axis cs:193,56965672.0446213)
--(axis cs:192,39459513.125689)
--(axis cs:191,36702779.964499)
--(axis cs:190,27684794.8987143)
--(axis cs:189,26655822.9834415)
--(axis cs:188,28257644.1390571)
--(axis cs:187,29837530.6907703)
--(axis cs:186,54354530.3444295)
--(axis cs:185,48137229.6378947)
--(axis cs:184,37183067.4513279)
--(axis cs:183,33715366.2172923)
--(axis cs:182,29438337.936182)
--(axis cs:181,32464968.5799909)
--(axis cs:180,27265203.1957237)
--(axis cs:179,23920719.0678456)
--(axis cs:178,38431665.4868771)
--(axis cs:177,29278138.0427443)
--(axis cs:176,29986401.279054)
--(axis cs:175,27743578.5046792)
--(axis cs:174,30450307.7608892)
--(axis cs:173,43194608.107606)
--(axis cs:172,44664902.7554429)
--(axis cs:171,38157155.3617688)
--(axis cs:170,34346822.3520057)
--(axis cs:169,28050076.916275)
--(axis cs:168,30505794.2879546)
--(axis cs:167,25787863.0257074)
--(axis cs:166,30858385.5505781)
--(axis cs:165,24930015.6004494)
--(axis cs:164,22683276.2829642)
--(axis cs:163,26336136.6844731)
--(axis cs:162,26414245.1418528)
--(axis cs:161,23718828.2832113)
--(axis cs:160,21357599.358753)
--(axis cs:159,23562431.4407949)
--(axis cs:158,26939491.9777115)
--(axis cs:157,24609634.8091765)
--(axis cs:156,40474962.032026)
--(axis cs:155,29799652.454181)
--(axis cs:154,26407867.7423083)
--(axis cs:153,33036009.9893627)
--(axis cs:152,26954942.3494751)
--(axis cs:151,19875702.687444)
--(axis cs:150,26337795.1928648)
--(axis cs:149,29129267.012863)
--(axis cs:148,29633608.0840516)
--(axis cs:147,31473199.0275476)
--(axis cs:146,30282359.8630144)
--(axis cs:145,28264764.0445535)
--(axis cs:144,35293404.2318773)
--(axis cs:143,30028845.5890718)
--(axis cs:142,30530170.1826362)
--(axis cs:141,28586905.3030635)
--(axis cs:140,30477218.2223347)
--(axis cs:139,26978726.7507574)
--(axis cs:138,29717794.4357272)
--(axis cs:137,28774903.1597303)
--(axis cs:136,22297225.4582833)
--(axis cs:135,27578205.1716035)
--(axis cs:134,28318263.8842657)
--(axis cs:133,23239059.0042328)
--(axis cs:132,24599392.0679622)
--(axis cs:131,33751259.137669)
--(axis cs:130,23606544.2925807)
--(axis cs:129,30760848.5866496)
--(axis cs:128,27836049.9438673)
--(axis cs:127,28000257.0609644)
--(axis cs:126,34138121.0141403)
--(axis cs:125,31741096.9488111)
--(axis cs:124,29723786.0086987)
--(axis cs:123,25290957.1597012)
--(axis cs:122,22328731.4137595)
--(axis cs:121,20710143.34285)
--(axis cs:120,25819673.9663331)
--(axis cs:119,21097692.6045773)
--(axis cs:118,25228872.4266819)
--(axis cs:117,20792934.8938373)
--(axis cs:116,26478280.2006977)
--(axis cs:115,29075195.7027124)
--(axis cs:114,22055813.6310329)
--(axis cs:113,18393717.0254932)
--(axis cs:112,19114063.1724384)
--(axis cs:111,19577103.2045981)
--(axis cs:110,21218181.1188125)
--(axis cs:109,18929353.4185457)
--(axis cs:108,22042108.0200946)
--(axis cs:107,22859724.5604021)
--(axis cs:106,33278352.6834227)
--(axis cs:105,30245154.4428577)
--(axis cs:104,19423553.8682815)
--(axis cs:103,17768325.5590965)
--(axis cs:102,20164448.0430835)
--(axis cs:101,17647914.6207141)
--(axis cs:100,18944133.8403092)
--(axis cs:99,19165750.9856944)
--(axis cs:98,20974068.7558441)
--(axis cs:97,17630353.2494886)
--(axis cs:96,16104966.4232986)
--(axis cs:95,15839846.2715459)
--(axis cs:94,16897139.3979525)
--(axis cs:93,17026731.6626238)
--(axis cs:92,17689065.8713372)
--(axis cs:91,19792325.7248864)
--(axis cs:90,18327814.858991)
--(axis cs:89,18656719.4848975)
--(axis cs:88,19998077.2931614)
--(axis cs:87,18238201.5553992)
--(axis cs:86,20191434.886759)
--(axis cs:85,22453309.6165387)
--(axis cs:84,18800506.8532522)
--(axis cs:83,20381069.3561732)
--(axis cs:82,19721136.589673)
--(axis cs:81,19465282.1341626)
--(axis cs:80,17826147.803433)
--(axis cs:79,18400177.1220038)
--(axis cs:78,19343669.466588)
--(axis cs:77,18451222.6737694)
--(axis cs:76,19567594.7133333)
--(axis cs:75,19763529.2343535)
--(axis cs:74,23391787.4287728)
--(axis cs:73,22448232.5986782)
--(axis cs:72,22013188.6809488)
--(axis cs:71,19362252.3726255)
--(axis cs:70,19401354.2643512)
--(axis cs:69,17699739.5403701)
--(axis cs:68,17959493.5853639)
--(axis cs:67,19103861.3243429)
--(axis cs:66,19767251.5204452)
--(axis cs:65,19292319.0685236)
--(axis cs:64,20902683.4677022)
--(axis cs:63,20077393.1908966)
--(axis cs:62,21661919.9850012)
--(axis cs:61,23383391.9241443)
--(axis cs:60,27770806.2505149)
--(axis cs:59,31731351.2805811)
--(axis cs:58,27055950.7418718)
--(axis cs:57,26174852.42186)
--(axis cs:56,20586350.3810005)
--(axis cs:55,18190146.9387878)
--(axis cs:54,19171699.5483401)
--(axis cs:53,20535731.6871493)
--(axis cs:52,22158113.0477397)
--(axis cs:51,24506487.0540148)
--(axis cs:50,23222419.0863687)
--(axis cs:49,21653575.7628479)
--(axis cs:48,22326651.7200357)
--(axis cs:47,21183672.0007156)
--(axis cs:46,22233192.5736002)
--(axis cs:45,22088158.5384926)
--(axis cs:44,19480935.0467992)
--(axis cs:43,20579175.6859623)
--(axis cs:42,21963012.2324751)
--(axis cs:41,20153795.6548025)
--(axis cs:40,21947713.1863646)
--(axis cs:39,20519409.6062547)
--(axis cs:38,19579310.7835012)
--(axis cs:37,19686438.3265004)
--(axis cs:36,18881799.996467)
--(axis cs:35,19877241.1948065)
--(axis cs:34,19531546.293232)
--(axis cs:33,21059611.7424309)
--(axis cs:32,19972850.7981965)
--(axis cs:31,20323129.4298751)
--(axis cs:30,19706721.2221154)
--(axis cs:29,20258316.1972449)
--(axis cs:28,19562252.2555616)
--(axis cs:27,17572093.7839409)
--(axis cs:26,20645004.1439194)
--(axis cs:25,21146171.9279127)
--(axis cs:24,22268206.7737548)
--(axis cs:23,22150019.2535872)
--(axis cs:22,20421757.3667733)
--(axis cs:21,21009930.3828662)
--(axis cs:20,23537302.4460774)
--(axis cs:19,24163399.8554157)
--(axis cs:18,24095814.1359468)
--(axis cs:17,22073839.4432389)
--(axis cs:16,22581102.3097662)
--(axis cs:15,24024950.9419489)
--(axis cs:14,24325335.6721734)
--(axis cs:13,25746068.2520929)
--(axis cs:12,22434351.5923761)
--(axis cs:11,24566511.446166)
--(axis cs:10,26068500.8330347)
--(axis cs:9,23899564.4726533)
--(axis cs:8,22806903.39768)
--(axis cs:7,25040625.0923553)
--(axis cs:6,25345124.7440987)
--(axis cs:5,27166416.1786298)
--(axis cs:4,27464872.7561989)
--(axis cs:3,38738461.5213158)
--(axis cs:2,27359656.8054173)
--(axis cs:1,31938252.4538893)
--(axis cs:0,31925052.2091835)
--cycle;

\addplot [only marks, red, mark=asterisk, mark size=1.2]
table {%
0 29755500
1 20228000
2 8513700
3 9502900
4 8909200
5 7004800
6 6305500
7 6348400
8 7321100
9 6421300
10 6164800
11 5360500
12 5074700
13 5074700
14 6804600
15 9612000
16 9154100
17 5291400
18 5456200
19 5819200
20 12820200
21 5305900
22 5067000
23 11001200
24 7051300
25 7450000
26 9014400
27 5198200
28 4916300
29 5534800
30 4719000
31 4596600
32 6568000
33 6431200
34 5034500
35 6933000
36 4533400
37 4446000
38 5301600
39 4517400
40 4370200
41 6208600
42 7137600
43 4905200
44 5415600
45 4461400
46 4170200
47 4549000
48 5623500
49 4354800
50 6282500
51 6528100
52 4165200
53 3940800
54 3894500
55 4728400
56 12964300
57 7101500
58 9158000
59 10692100
60 4636300
61 5787600
62 5199600
63 4093600
64 3925700
65 3818200
66 3756600
67 3702000
68 3632400
69 3676500
70 3599700
71 4028000
72 6895400
73 4511900
74 5519200
75 3725000
76 3495900
77 3418000
78 3308300
79 3208300
80 3241700
81 7029700
82 5106300
83 4259800
84 11266900
85 7029700
86 3754100
87 3833100
88 5472000
89 6634600
90 3756400
91 3932500
92 3587100
93 3677400
94 3490000
95 3271000
96 3169400
97 3754100
98 3754100
99 3289100
100 3907600
101 3268800
102 3747400
103 4939300
104 5061200
105 34188900
106 27994500
107 6776800
108 5061200
109 4065600
110 3948300
111 5492400
112 6815200
113 4162700
114 10287200
115 19116000
116 5162800
117 5390800
118 9111000
119 8332200
120 6176400
121 6007000
122 5449500
123 9217100
124 26947100
125 23633100
126 7943900
127 9619000
128 7515000
129 7323100
130 6271200
131 10975700
132 5934800
133 13255700
134 8424800
135 5537500
136 20382400
137 15957800
138 9571500
139 11587400
140 5961900
141 7573700
142 7627900
143 34868400
144 19389200
145 7959700
146 5842300
147 22768500
148 7232800
149 7546600
150 16027800
151 5833200
152 5336600
153 7476600
154 7512800
155 6573700
156 4891900
157 5083800
158 4998000
159 5007000
160 5165000
161 24100400
162 19050500
163 6097300
164 8512800
165 14923900
166 9695700
167 9673100
168 13014100
169 9045600
170 15720800
171 25333000
172 21901700
173 29845600
174 12422700
175 11126900
176 29179700
177 12285000
178 31793800
179 7835600
180 7106400
181 9212600
182 14449900
183 22425400
184 24443600
185 36209300
186 23874700
187 11655200
188 11354900
189 8605400
190 31960800
191 32662900
192 23240300
193 113027800
194 30163900
195 17425200
196 9767900
197 13793000
198 9176500
199 48431100
200 56194500
201 19000900
202 34642700
203 14820100
204 11009500
205 16849500
206 26134400
207 33843500
208 32130100
209 16087400
210 17438700
211 14080400
212 15384500
213 40475900
214 18610300
215 11987000
216 10382000
217 12513000
218 23102600
219 67402600
220 16682500
221 15589900
222 22150000
223 64014200
224 18127200
225 11294600
226 10635400
227 18295300
228 23097100
229 8678100
230 8156300
231 14513900
232 7707800
233 7069100
234 6833500
235 6511900
236 6344300
237 7089500
238 23791600
239 10674900
240 33965900
241 8903700
242 8915100
243 6704400
244 6631900
245 30353300
246 41741700
247 41295500
248 7716900
249 7114400
250 6434900
251 6253700
252 6063400
253 6285400
254 14942200
255 8441700
256 15918400
257 6738400
258 6738400
259 9590000
260 6439400
261 6058900
262 5855000
263 5691900
264 5549300
265 5404300
266 5284200
267 5207200
268 5669300
269 5574200
270 7710100
271 5907100
272 10505100
273 12550400
274 26328400
275 21633000
276 8083800
277 18942200
278 7873100
279 8684000
280 8011300
281 6641000
282 6031700
283 34190200
284 19821000
285 31746200
286 18552600
287 7839200
288 12471100
289 8468800
290 22822100
291 17691900
292 8228700
293 7429200
294 9578700
295 11988600
296 19771200
297 13257000
298 12799500
299 26049800
300 8720300
301 18244600
302 16595700
303 8638700
304 7465400
305 14095100
306 23268300
307 8117800
308 7275200
309 6851600
310 8199300
311 6978500
312 21503900
313 50912700
314 44235500
315 13080400
316 10482400
317 7288800
318 7094000
319 7544700
320 9150600
321 8079300
322 7254800
323 23109800
324 19705500
325 9900300
326 8115500
327 12027200
328 11605900
329 8681700
330 8289900
331 9282000
332 9334100
333 6591200
334 9687400
335 12131300
336 7669300
337 22799500
338 11479000
339 6181200
340 10036200
341 8566200
342 6348800
343 8690800
344 6029400
345 5472200
346 5182300
347 25941000
348 22471100
349 6924100
350 10926400
351 11340900
352 11861800
353 10104200
354 10158500
355 11315900
356 29934200
357 8493800
358 7152900
359 22946700
360 11884500
361 9943400
362 10604700
363 7551500
364 6679500
365 10824400
366 8539100
367 10095100
368 6169900
369 6054300
370 5782500
371 5451900
372 7485800
373 8675000
374 5766700
375 7368000
376 17732700
377 15408800
378 14600200
379 16423500
380 10514100
381 6847100
382 11556000
383 14274000
384 6534500
385 6545900
386 7393000
387 12083800
388 6767800
389 6280800
390 5789300
391 5710100
392 5662500
393 5395200
394 5094000
395 4915100
396 4906000
397 5605900
398 6364700
399 8364600
400 5513000
401 4910500
402 4724800
403 4636500
404 4645500
405 4625100
406 4575300
407 4518700
408 6396400
409 5037400
410 4656800
411 5141600
412 4847100
413 4466600
414 4971700
415 7485800
416 4645500
417 4308000
418 4185700
419 4208400
420 9963700
421 4541300
422 4280900
423 4249100
424 4262700
425 4792700
426 4940000
427 12004500
428 4865200
429 4317100
430 4405400
431 4178900
432 4077000
433 3970500
434 3990900
435 4215200
436 3959200
437 3877700
438 4516400
439 4643300
440 3959200
};
\addplot [blue]
table {%
0 16741617.6306149
1 14794537.0119485
2 11116736.5978385
3 22881398.6587559
4 10862853.9559898
5 11756079.2040898
6 9912793.75168397
7 9637317.02289481
8 7436075.44778281
9 8693757.37176492
10 11038076.7494436
11 9625152.94263056
12 7273057.9232693
13 10710860.1203536
14 9252276.91549008
15 9011720.17760852
16 5748325.36810795
17 6784229.486122
18 8889025.07871816
19 9059702.92637735
20 8142173.10594504
21 5695628.49625022
22 5456453.67852928
23 7141127.80296532
24 7290739.26326678
25 6100779.24241973
26 5720861.15166478
27 2559060.53177878
28 4505718.9687531
29 5283227.72150452
30 4744172.11089837
31 5407269.5348693
32 5057704.96136532
33 5992705.21157252
34 4606806.97647257
35 4963598.66093337
36 3903320.49691103
37 4755767.18817779
38 4685190.19705777
39 5635690.92756503
40 6664046.91951597
41 5269471.15000439
42 6888288.73308916
43 5331728.07745474
44 3711248.61320507
45 7108525.86501894
46 7274398.74447294
47 6249991.37581053
48 7342923.5154733
49 6633693.73002447
50 8208348.463198
51 9458053.30728421
52 6872601.50456166
53 5420159.24007628
54 4087195.46760585
55 3191593.73189132
56 5498155.84642128
57 11146460.0660418
58 12091351.6625451
59 16365663.3633678
60 12445554.5106807
61 8318380.60277989
62 6764163.31210244
63 5134389.31742434
64 5995134.91591342
65 4348635.8393183
66 4795444.80557506
67 4140789.05959375
68 3069816.91907504
69 2754785.6178882
70 4080966.29231971
71 4485173.69986827
72 6909646.38272495
73 7484056.8339755
74 8414609.50974051
75 4786427.98452724
76 4685388.25209842
77 3531423.35826787
78 4442625.16493415
79 3461897.34970092
80 2832532.72524851
81 4525857.49821363
82 4769989.45177718
83 5459022.64996904
84 3786759.84948323
85 7123491.93979526
86 5220405.20512629
87 3314686.13945798
88 5049569.38666342
89 3698192.1648758
90 3326257.74871249
91 4845612.61979155
92 2705081.0572239
93 2079014.0395317
94 1953766.37970277
95 816710.223402533
96 1051158.17085538
97 2692119.03254059
98 6006130.30630387
99 4202314.63831697
100 3945345.79679654
101 2700816.57407194
102 5142505.3998993
103 2809063.53072061
104 4519999.40864573
105 14939151.3348478
106 17480518.2591927
107 7540280.09536742
108 7059085.1900304
109 3876910.59640093
110 6127379.65342075
111 3928342.00050963
112 4087420.07942659
113 3077494.12190068
114 7094468.32966486
115 14011344.3426719
116 11094159.1269536
117 5846233.95866538
118 10207697.1618731
119 6068340.5074711
120 10283636.5551377
121 5454714.23987512
122 7306674.18024546
123 9687010.98395565
124 14572711.5270849
125 16167398.2707537
126 18226352.7544021
127 12681169.6686587
128 12284591.3708591
129 15538817.70325
130 8287053.74848787
131 18562929.993428
132 7721532.81415535
133 8146549.33245267
134 13086694.5935841
135 12402837.4093042
136 7282198.99646311
137 13740991.816385
138 14466364.042946
139 11766501.8578735
140 15137054.6252374
141 13538578.3328251
142 15454660.2247238
143 15024777.6268277
144 20019176.2446977
145 12481592.5032512
146 15163740.9471783
147 16517866.3615138
148 13967446.1444323
149 14111788.9717492
150 11114431.4972986
151 3398238.10112796
152 11881261.6752673
153 17629143.4202082
154 10808260.0476236
155 14775789.2576983
156 24656075.7529358
157 9077656.37697818
158 11855058.3582044
159 8589091.09170616
160 6324480.72894372
161 8710120.46404816
162 10853247.7036058
163 11253077.8850887
164 7672769.87353489
165 9881002.74797484
166 15801942.7540276
167 10614381.7003922
168 15210681.6018926
169 11873430.2004245
170 19132412.4281311
171 22571169.7864887
172 28670297.4793524
173 26906517.0916568
174 14818029.6540225
175 12417156.4423615
176 14743693.1283802
177 13828692.002366
178 22891359.8075716
179 7831187.60643924
180 11992438.4486888
181 17298097.1788889
182 13987830.0235928
183 18702996.7331111
184 21512683.2182362
185 32284769.1552129
186 37817835.0874667
187 12682315.3007111
188 13027708.9255312
189 11652711.451758
190 12678155.041768
191 21519679.2881199
192 24111793.4777128
193 41433714.9799368
194 37009235.831294
195 19295570.3096787
196 19654272.8846891
197 16917249.1221489
198 13970820.675741
199 24467514.6555643
200 53253575.391372
201 25637764.3039924
202 16942664.9396476
203 15836626.4976378
204 20863983.3659418
205 16540694.0086354
206 22980567.7807034
207 22535290.3894846
208 28210892.7667081
209 15435441.3646961
210 20541707.8254459
211 40474247.8469006
212 14688889.2444611
213 23071234.4062694
214 10229406.3920812
215 12481555.4946522
216 12637630.1420516
217 11558542.4018327
218 29854620.1052421
219 31348610.6944364
220 24777381.2414228
221 12679851.8071615
222 18695158.3704915
223 27713519.6888474
224 19344749.3542657
225 18414643.1541949
226 11164276.526802
227 15756081.0627882
228 12768660.772396
229 14748247.7061436
230 10357288.1254941
231 11964485.0999476
232 7724437.60766799
233 9666392.93000696
234 7153118.95474069
235 7423594.56140242
236 10511173.1594718
237 11077038.6030774
238 24454039.569365
239 14822014.5414177
240 30979953.6408902
241 13840297.4062018
242 13508609.4498107
243 8045832.99600352
244 10281136.6084688
245 17643032.2142304
246 26180332.9810304
247 22097352.5410084
248 11960917.8552635
249 6654023.61435388
250 7429124.40012589
251 10770672.5653984
252 9851895.45250647
253 8320209.738484
254 23656952.0855015
255 22237849.514153
256 17316127.7077601
257 15070619.6548754
258 14181757.618069
259 22552145.8868628
260 9547839.80697949
261 10886765.9773739
262 9557674.08305113
263 4504055.83845966
264 589447.445402256
265 8705501.20837534
266 4631421.29878683
267 2279659.93751341
268 7440478.9051268
269 9466285.90410108
270 11259582.9922489
271 11134074.2843213
272 13844471.6108583
273 11735642.0753861
274 13321837.7716904
275 20084383.2337746
276 14035912.3348026
277 21927382.2859348
278 13333123.8007877
279 12578593.2612224
280 14775019.9128663
281 11379253.7486912
282 14386171.6209366
283 11808573.1251625
284 13184701.131002
285 9783434.50792663
286 15917610.1711732
287 15060656.3144127
288 9196825.41401001
289 12772782.8824562
290 11914078.8319018
291 18120708.4104864
292 13289149.6736648
293 9629472.0554562
294 9467343.01032915
295 14685744.3487184
296 16251380.1358089
297 18748216.4250809
298 16512583.4728985
299 15745525.0650424
300 15870831.8604209
301 13537572.4977101
302 20475939.9702723
303 15036464.444106
304 7994687.83379279
305 8332902.34703181
306 12976584.2598646
307 10678797.3182182
308 8036254.00991759
309 8035248.40253174
310 10846601.1077768
311 12355120.6780847
312 14553174.5737045
313 17902078.8883518
314 44843290.5491097
315 17662084.1308735
316 13174679.3337463
317 20982846.4375154
318 8250972.95362743
319 8794094.89614557
320 11091128.2705103
321 10198467.5096854
322 9086121.67184957
323 14362268.7169022
324 16329134.9781652
325 13750535.9549945
326 11382569.8314471
327 17125771.2318138
328 20788265.8918331
329 12273145.4465769
330 11447026.479076
331 15058455.3104751
332 13503494.7808743
333 11144051.1079645
334 12118477.6168008
335 13624503.5919066
336 12672039.4848173
337 12626211.8433153
338 14178832.9826336
339 7886363.84339595
340 10624582.5312304
341 15310292.5122736
342 13503767.3999144
343 8116779.60179909
344 9316079.03040858
345 8564533.15469158
346 7335240.21230509
347 11057895.2340894
348 19691104.1645526
349 8846319.84530781
350 15898409.677898
351 18630890.2447333
352 17821665.4721866
353 10363430.2031629
354 12370679.8377123
355 11194297.4555534
356 14730442.9982773
357 9791810.57096562
358 9143555.20848592
359 11020451.1868161
360 9464933.64752774
361 11411937.9602033
362 11667331.1049705
363 8118550.91811559
364 6257126.93871824
365 10274787.087595
366 6370660.71929226
367 10450917.585455
368 10639142.0209318
369 8038480.17288477
370 7924632.3550491
371 7785007.86098587
372 7069431.11539175
373 8624729.02183302
374 9425229.64304384
375 12547468.2007871
376 17138830.4380864
377 13518590.6747026
378 8546527.68631615
379 6069109.54165559
380 4087722.08827614
381 6084096.96151443
382 8230671.92742219
383 7601284.1181462
384 8724136.40097314
385 8951573.57994605
386 9242279.41758871
387 8257625.47708441
388 8429984.80435794
389 8255228.67618069
390 9782020.14277939
391 5973390.62666014
392 7012149.69406604
393 5909892.78787048
394 7343265.33272353
395 6028932.79126561
396 6334619.71841604
397 5024898.71271853
398 5739990.13948472
399 6879890.58177681
400 5174624.01074264
401 5178260.81111684
402 4138222.70165893
403 5374571.1043037
404 7925891.10819822
405 7023035.11260027
406 9221072.52722329
407 6164032.65090576
408 4962108.48264212
409 5268639.9215534
410 7616472.65962418
411 7314534.11131214
412 3856353.63926288
413 5870642.7335789
414 5433048.1496416
415 7868760.92404132
416 7882230.63849655
417 4876046.95213273
418 4537242.94425931
419 2970930.56034695
420 3341268.31697037
421 4221795.75218834
422 4909468.05595287
423 6082041.48179708
424 4615837.23137485
425 3898770.02604107
426 4742259.02525468
427 3960391.74814353
428 3551197.51863669
429 4701194.41553602
430 3830160.06089725
431 7069688.00915394
432 4447407.47669873
433 3928489.20445722
434 4439675.13183228
435 3595309.69053955
436 5387815.03425416
437 5633952.17061659
438 4095266.22313538
439 4130298.5525599
440 4297211.10850723
};


\addlegendentry{Test data}
\addlegendentry{LMCGP mean}
\addlegendimage{line width=5pt,draw=blue,opacity=0.5}
\addlegendentry{$95\%$ LMCGP CI}

\end{axis}

\end{tikzpicture}

	\end{figure}
	\vspace{-1.5em}
	\begin{block}{}
		The LMCGP model more accurately captures the peaks in the data, attributable to its ability to model complex behaviors through the incorporation of output correlations.
	\end{block}
\end{frame}



%\begin{frame}{To Conclude}
%		\justifying
%		\begin{block}{}
%			\justifying
%		The LMCGP effectively captures shared features and dynamics for multi-output tasks. However, increasing the number of independent GPs beyond a threshold leads to instability.
%		\end{block}
%		
%		\begin{block}{}
%			\justifying
%		The lengthscale matrix and task dependency coefficients, $a_{d,q}$, provide critical insights into feature selection, with some GPs specializing in specific tasks and others covering a broader range of outputs.
%		\end{block}
%		
%		\begin{block}{}
%			\justifying
%		To improve optimization performance, using Adam + NG optimizer proved superior to traditional methods, leading to more robust results.
%		\end{block}
%		
%		\begin{block}{}
%			\justifying
%		The LMCGP outperformed the IGP in terms of NLPD and MSLL across all horizons, emphasizing the benefits of task dependency modeling.
%		\end{block}
%		
%		\begin{block}{}
%			\justifying
%		The LMCGP's forecasting ability showed stronger learning of complex patterns by leveraging data from multiple tasks.
%		\end{block}
%\end{frame}
