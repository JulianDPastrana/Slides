\section{Multi-Output Gaussian Processes: Modeling Inter-Output Dependencies}

\subsection{Mathematical Framework}

\begin{frame}{Independent Gaussian Process (IGP)}
	
	\begin{columns}[T] % Top alignment
		\begin{column}{0.5\textwidth}
			
			\textbf{Independent Process} 
			\begin{equation*}
				u_{d}(\mathbf{x}) \sim \mathcal{GP}(0, k_{d}(\mathbf{x}, \mathbf{x}'))
			\end{equation*}
			
			\textbf{Latent Process}
			\begin{equation*}
				f_{d}(\mathbf{x}) = u_{d}(\mathbf{x})
			\end{equation*}
			
			
			\textbf{Multi-Output Model}
			\begin{equation*}
				\mathbf{f}_{*}
				\sim
				\mathcal{N} \left(
				\begin{array}{c}
					\mathbf{0}\\
				\end{array},
				\sum_{d=1}^D \mathbf{B}_d \otimes K_{d**}\right)
			\end{equation*}
			
			\begin{itemize}
				\item  $\mathbf{B}_d = (\delta_{d,d'}) \in \mathbb{R}^{D \times D}$.
				
				\item $K_{d**} \in \mathbb{R}^{N_* \times N_*}$ is the $d$-th covariance matrix at $X_*$ test inputs.
				
				\item $\otimes$ denotes the Kronecker product.
			\end{itemize}	
			
		\end{column}
		
		\begin{column}{0.5\textwidth}
			\setlength\figurewidth{0.8\textwidth} 
			\setlength\figureheight{0.32\textwidth}
			% NEURAL NETWORK activation
% https://www.youtube.com/watch?v=aircAruvnKk&list=PLZHQObOWTQDNU6R1_67000Dx_ZCJB-3pi&index=1
\begin{tikzpicture}[x=\figurewidth,y=\figureheight]
  \message{^^JNeural network activation}
  \def\NI{4} % number of nodes in input layers
  \def\NO{4} % number of nodes in output layers
  \def\yshift{0.4} % shift last node for dots
  
  % INPUT LAYER
  \foreach \i [evaluate={\c=int(\i==\NI); \y=\NI/2-\i-\c*\yshift; \index=(\i<\NI?int(\i):"D");}]
              in {1,...,\NI}{ % loop over nodes
    \node[node in,outer sep=0.6] (NI-\i) at (0,\y) {$u_{\index}$};
  }
  
  % OUTPUT LAYER
  \foreach \i [evaluate={\c=int(\i==\NO); \y=\NO/2-\i-\c*\yshift; \index=(\i<\NO?int(\i):"D");}]
    in {\NO,...,1}{ % loop over nodes
    \ifnum\i=1 % high-lighted node
      \node[node hidden]
        (NO-\i) at (1,\y) {$f_{\index}$};
      % \foreach \j [evaluate={\index=(\j<\NI?int(\j):"D");}] in {1,...,\NI}{ % loop over nodes in previous layer
        \draw[connect,white,line width=1.2] (NI-\i) -- (NO-\i);
        \draw[connect] (NI-\i) -- (NO-\i);
      %     % node[pos=0.50] {\contour{white}{$a_{1,\index}$}};
      % }
    \else % other light-colored nodes
      \node[node,blue!20!black!80,draw=myblue!20,fill=myblue!5]
        (NO-\i) at (1,\y) {$f_{\index}$};
      % \foreach \j in {1,...,\NI}{ % loop over nodes in previous layer
        %\draw[connect,white,line width=1.2] (NI-\j) -- (NO-\i);
        \draw[connect,myblue!20] (NI-\i) -- (NO-\i);
      % }
    \fi
  }
  
  % DOTS
  \path (NI-\NI) --++ (0,1+\yshift) node[midway,scale=1.2] {$\vdots$};
  \path (NO-\NO) --++ (0,1+\yshift) node[midway,scale=1.2] {$\vdots$};
  

  
\end{tikzpicture}
	
		\end{column}
		
	\end{columns}
\end{frame}


\begin{frame}{Linear Model of Coregionalization GP (LMCGP)}
	
	\begin{columns}[T] % Top alignment
	\begin{column}{0.5\textwidth}
		
		\textbf{Independent Process}
		\begin{equation*}
			u_{q}(\mathbf{x}) \sim \mathcal{GP}(0, k_{q}(\mathbf{x}, \mathbf{x}'))
		\end{equation*}	
		
		\textbf{Latent Process}
		\begin{equation*}
			f_{d}(\mathbf{x}) = \sum_{q=1}^Q a_{d,q} u_{q}(\mathbf{x})
		\end{equation*}
		
		\textbf{Multi-Output Model}
		\begin{equation*}
			\mathbf{f}_{*}
			\sim
			\mathcal{N} \left(
			\begin{array}{c}
				\mathbf{0}\\
			\end{array},
			\sum_{q=1}^Q \mathbf{B}_q \otimes K_{q**}\right)
		\end{equation*}
		
		\begin{itemize}
			\item  $\mathbf{B}_q = (a_{d,q}a_{d',q}) \in \mathbb{R}^{D \times D}$ is the $q$-th coregionalization matrix.
			
			\item $K_{q**} \in \mathbb{R}^{N_* \times N_*}$.
		\end{itemize}	

	\end{column}
	
	\begin{column}{0.5\textwidth}
		\setlength\figurewidth{0.8\textwidth} 
		\setlength\figureheight{0.32\textwidth}
		% NEURAL NETWORK activation
% https://www.youtube.com/watch?v=aircAruvnKk&list=PLZHQObOWTQDNU6R1_67000Dx_ZCJB-3pi&index=1
\begin{tikzpicture}[x=\figurewidth,y=\figureheight]
  \message{^^JNeural network activation}
  \def\NI{4} % number of nodes in input layers
  \def\NO{4} % number of nodes in output layers
  \def\yshift{0.4} % shift last node for dots
  
  % INPUT LAYER
  \foreach \i [evaluate={\c=int(\i==\NI); \y=\NI/2-\i-\c*\yshift; \index=(\i<\NI?int(\i):"Q");}]
              in {1,...,\NI}{ % loop over nodes
    \node[node in,outer sep=0.6] (NI-\i) at (0,\y) {$u_{\index}$};
  }
  
  % OUTPUT LAYER
  \foreach \i [evaluate={\c=int(\i==\NO); \y=\NO/2-\i-\c*\yshift; \index=(\i<\NO?int(\i):"D");}]
    in {\NO,...,1}{ % loop over nodes
    \ifnum\i=1 % high-lighted node
      \node[node hidden]
        (NO-\i) at (1,\y) {$f_{\index}$};
      \foreach \j [evaluate={\index=(\j<\NI?int(\j):"Q");}] in {1,...,\NI}{ % loop over nodes in previous layer
        \draw[connect,white,line width=1.2] (NI-\j) -- (NO-\i);
        \draw[connect] (NI-\j) -- (NO-\i)
          node[pos=0.50] {\contour{white}{$a_{1,\index}$}};
      }
    \else % other light-colored nodes
      \node[node,blue!20!black!80,draw=myblue!20,fill=myblue!5]
        (NO-\i) at (1,\y) {$f_{\index}$};
      \foreach \j in {1,...,\NI}{ % loop over nodes in previous layer
        %\draw[connect,white,line width=1.2] (NI-\j) -- (NO-\i);
        \draw[connect,myblue!20] (NI-\j) -- (NO-\i);
      }
    \fi
  }
  
  % DOTS
  \path (NI-\NI) --++ (0,1+\yshift) node[midway,scale=1.2] {$\vdots$};
  \path (NO-\NO) --++ (0,1+\yshift) node[midway,scale=1.2] {$\vdots$};
  

  
\end{tikzpicture}
	
	\end{column}
	
\end{columns}
\end{frame}

\begin{frame}{Variational Inference, ELBO, and Predictive Distribution}
	We extend variational inference to include the independent set, utilizing the inducing variables $u_q$ derived from independent processes. The ELBO is given by:
	
	\begin{equation*}\label{eq:lmc_elbo}
		\mathcal{L} = \sum_{d=1}^D\sum_{n=1}^{N} \mathbb{E}_{q(f_{dn})}\{\log p(y_{dn} \mid f_{dn})\} - \sum_{q=1}^Q \text{KL}\{q(\mathbf{u}_q)\parallel p(\mathbf{u}_q)\}
	\end{equation*}
	
	The posterior over test points $X_*$, $p(\mathbf{f}_* \mid \mathbf{y})$, is given by:
	
	\begin{equation*}
		p(\mathbf{f}_* \mid \mathbf{y}) \approx q(\mathbf{f}_*) = \int p(\mathbf{f}_* \mid \mathbf{u}) q(\mathbf{u}) d \mathbf{u}
	\end{equation*}
	
	Gaussian noise $\sigma_{Nd}^2$ is added to obtain the predictive distribution.
\end{frame}

\begin{frame}{Model Setup}

	\begin{block}{Covariance Function (LMCGP)}
		The LMCGP model uses a squared exponential kernel:
		\begin{equation*}
			k_{q}\left(\mathbf{x}, \mathbf{x'} \mid \Theta_q \right) = \exp\left(-\frac{1}{2}(\mathbf{x} - \mathbf{x'})^\top \Theta_q^{-2} (\mathbf{x} - \mathbf{x'})\right)
		\end{equation*}
		Here, \( \Theta_q \) is the lengthscale matrix, and \( \mathbf{B}_q \) works as outputscale.
	\end{block}
	
	\begin{block}{Optimization and Model Variants}
		Strong dependencies between parameters may cause poor local minima \cite{giraldo2021fully}. We address this by combining Natural Gradient (NG) to optimize variational parameters, and Adam for the rest \cite{pmlr-v84-salimbeni18a}.

		\textbf{Variants:}
		\begin{itemize}
			\item IGP: Independent GP (Adam).
			\item IGP+: Independent GP (Adam+NG).
			\item LMCGP: Correlated GP (Adam+NG)
		\end{itemize}
	\end{block}
\end{frame}



\subsection{Results and Discussions}

\subsubsection{Hyperparameter Tuning}

\begin{frame}{Tuning Q}
	\begin{figure}[htbp]
	\centering
	\setlength\figurewidth{0.5\columnwidth} 
	\setlength\figureheight{0.28\columnwidth}
	
	\subfloat[CRPS]{% This file was created with tikzplotlib v0.10.1.
\begin{tikzpicture}

\definecolor{darkgray176}{RGB}{176,176,176}
\definecolor{steelblue31119180}{RGB}{31,119,180}

\begin{axis}[
width=\figurewidth,
height=\figureheight,
axis background/.style={fill=background_color},
axis line style={white},
tick align=inside,
tick pos=left,
x grid style={white},
y grid style={white},
xmajorgrids,
ymajorgrids,
xmin=-1.25, xmax=48.25,
xtick style={color=black},
ymin=0.239929987456654, ymax=0.438839652236985,
ytick style={color=black},
xlabel style={font=\tiny},
tick label style={font=\tiny}
]
\addplot [semithick, steelblue31119180, mark=*, mark size=1, mark options={solid}]
table {%
1 0.429798303837879
2 0.374510552905612
3 0.345960005084818
4 0.323892519212847
5 0.311383175599372
6 0.299542632982873
7 0.297875424155548
8 0.28784981072712
9 0.286844690564962
10 0.279610258905049
11 0.279515690474725
12 0.273041146856969
13 0.271082282952741
14 0.271889987179417
15 0.270210632285698
16 0.26759372049199
17 0.266678895529016
18 0.264912747643612
19 0.26441644238815
20 0.309002018166923
21 0.257220132139326
22 0.291456973758776
23 0.2613559059979
24 0.287804261488934
25 0.285577216964857
26 0.287627602712463
27 0.300194088705333
28 0.24897133585576
29 0.293655299555358
30 0.287299736020118
31 0.296865072134467
32 0.291250790206968
33 0.307497232174908
34 0.289441525631671
35 0.294363581321622
36 0.291544275421493
37 0.305791393660394
38 0.296662343784081
39 0.308247437640682
40 0.298021172730888
41 0.29625972317634
42 0.307272161655389
43 0.303820351549033
44 0.297729094232099
45 0.299530085504188
46 0.293446316295075
};
\end{axis}

\end{tikzpicture}
}\hspace{0.1em}
	\subfloat[MSE]{% This file was created with tikzplotlib v0.10.1.
\begin{tikzpicture}

\definecolor{darkgray176}{RGB}{176,176,176}
\definecolor{steelblue31119180}{RGB}{31,119,180}

\begin{axis}[
width=\figurewidth,
height=\figureheight,
axis background/.style={fill=background_color},
axis line style={white},
tick align=inside,
tick pos=left,
x grid style={white},
y grid style={white},
xmajorgrids,
ymajorgrids,
xmin=-1.25, xmax=48.25,
xtick style={color=black},
ymin=0.259536486825269, ymax=0.731194606213278,
ytick style={color=black},
xlabel style={font=\tiny},
tick label style={font=\tiny}
]
\addplot [semithick, steelblue31119180, mark=*, mark size=1, mark options={solid}]
table {%
1 0.70975560078655
2 0.570005452889205
3 0.524956091611488
4 0.468590469300411
5 0.435353470792991
6 0.411259722878218
7 0.409096034731835
8 0.385485371917808
9 0.381016960281645
10 0.360132840745332
11 0.362921694624643
12 0.347941662750589
13 0.33894395440443
14 0.342577112243772
15 0.343123787309784
16 0.334095395016974
17 0.330290643463412
18 0.322687081772815
19 0.326757528243904
20 0.437607617331642
21 0.304623270578551
22 0.389408892826712
23 0.316685484093948
24 0.381513763041057
25 0.375840283026447
26 0.377810182111291
27 0.414747999915877
28 0.280975492251997
29 0.393662339819777
30 0.373947243132724
31 0.402964916925179
32 0.385485092508175
33 0.431905374190024
34 0.38280924563172
35 0.397469992032685
36 0.390029348756403
37 0.427310239658667
38 0.408634597214847
39 0.437765674807815
40 0.40614578730405
41 0.40052469480261
42 0.43521136441382
43 0.423471188409076
44 0.405748603781294
45 0.409759054806833
46 0.390775042857023
};
\end{axis}

\end{tikzpicture}
}\\
	\vspace{0.1em} % Adjust spacing between rows
	
	\subfloat[MSLL]{% This file was created with tikzplotlib v0.10.1.
\begin{tikzpicture}

\definecolor{darkgray176}{RGB}{176,176,176}
\definecolor{steelblue31119180}{RGB}{31,119,180}

\begin{axis}[
width=\figurewidth,
height=\figureheight,
axis background/.style={fill=background_color},
axis line style={white},
tick align=inside,
tick pos=left,
x grid style={white},
y grid style={white},
xmajorgrids,
ymajorgrids,
xmin=-1.25, xmax=48.25,
xtick style={color=black},
ymin=-1.06651615899946, ymax=-0.510154630298139,
ytick style={color=black},
xlabel={Number of Independent GPs (Q)},
xlabel style={font=\tiny},
tick label style={font=\tiny}
]
\addplot [semithick, steelblue31119180, mark=*, mark size=1, mark options={solid}]
table {%
1 -0.535443790693654
2 -0.656143802651295
3 -0.738178467100046
4 -0.800136632373368
5 -0.840255253429766
6 -0.875183228780147
7 -0.883212638166813
8 -0.916060696072918
9 -0.92732556592439
10 -0.941608614832543
11 -0.941373520408176
12 -0.963328948986684
13 -0.959595962050077
14 -0.96451770455758
15 -0.95974756275392
16 -0.971143985423901
17 -0.981752140746748
18 -0.952261104902915
19 -0.979157271636092
20 -0.642391297788161
21 -1.00087763478619
22 -0.720088184779209
23 -0.954280692847926
24 -0.728105102288219
25 -0.735847580651938
26 -0.735567459714915
27 -0.672249112260655
28 -1.04122699860395
29 -0.705361482919378
30 -0.729647651122807
31 -0.678594473020219
32 -0.711949327335191
33 -0.64218758924277
34 -0.719052195633076
35 -0.694440862143111
36 -0.703501481048209
37 -0.654029715100638
38 -0.668137210996388
39 -0.621833430969577
40 -0.671693822590583
41 -0.67987459279109
42 -0.638762954793032
43 -0.660630779693286
44 -0.675898190836243
45 -0.674508388989054
46 -0.695663541852463
};
\end{axis}

\end{tikzpicture}
}\hspace{0.1em}
	\subfloat[NLPD]{% This file was created with tikzplotlib v0.10.1.
\begin{tikzpicture}

\definecolor{darkgray176}{RGB}{176,176,176}
\definecolor{steelblue31119180}{RGB}{31,119,180}

\begin{axis}[
width=\figurewidth,
height=\figureheight,
axis background/.style={fill=background_color},
axis line style={white},
tick align=inside,
tick pos=left,
x grid style={white},
y grid style={white},
xmajorgrids,
ymajorgrids,
xmin=-1.25, xmax=48.25,
xtick style={color=black},
ymin=8.86884898711136, ymax=21.6651641472417,
ytick style={color=black},
xlabel={Number of Independent GPs (Q)},
]
\addplot [semithick, steelblue31119180, mark=*, mark size=1, mark options={solid}]
table {%
1 21.0835134581449
2 18.3074131831191
3 16.4206159007979
4 14.9955780995115
5 14.0728498152143
6 13.2695063821556
7 13.0848299662623
8 12.3293246344218
9 12.070232627838
10 11.7417225029505
11 11.7471296747109
12 11.2421548174052
13 11.3280135169472
14 11.2148134392746
15 11.3245267007588
16 11.0624089793492
17 10.8184214069237
18 11.4967152313319
19 10.8781033964688
20 18.6237207949712
21 10.3785350440165
22 16.8366923941771
23 11.4502647085967
24 16.6523032914699
25 16.4742262891044
26 16.4806690706559
27 17.9369910621039
28 9.4504996762082
29 17.1754065369533
30 16.6168246682744
31 17.7910477646339
32 17.0238861153895
33 18.6284060915152
34 16.8605201445382
35 17.4265808148074
36 17.2181865799902
37 18.3560371967843
38 18.031564791182
39 19.0965517317987
40 17.9497627245155
41 17.7616050099039
42 18.7071726838592
43 18.2042127111534
44 17.8530622548654
45 17.8850276973507
46 17.3984591814923
};
\end{axis}

\end{tikzpicture}
}
	\end{figure}
	\vspace{-1em}
	\begin{block}{}
		Performance metrics for LMCGP models as a function of the number of independent GPs. We select $Q=17$ as the proper parameter
	\end{block}
\end{frame}

\begin{frame}{Lengthscale and $a_{d,q}$ Values (H=1)}
	\begin{figure}[htbp]
		\centering
		\setlength\figurewidth{0.46\columnwidth} 
		\setlength\figureheight{0.46\columnwidth}
		\subfloat[]{% This file was created with tikzplotlib v0.10.1.
\begin{tikzpicture}

\definecolor{darkgray176}{RGB}{176,176,176}

\begin{axis}[
width=\figurewidth,%
height=\figureheight,%
title={$H=1$},
%colorbar,
%colorbar style={ylabel={}},
colormap/viridis,
point meta max=40,
point meta min=20,
%tick align=outside,
tick pos=left,
x grid style={darkgray176},
xmin=0, xmax=23,
xtick style={color=black},
xtick={0.5,2.5,4.5,6.5,8.5,10.5,12.5,14.5,16.5,18.5,20.5,22.5},
xticklabel style={rotate=0.0},
xticklabels={A,C,E,G,I,K,M,O,Q,S,U,W},
y dir=reverse,
y grid style={darkgray176},
ymin=0, ymax=17,
ytick style={color=black},
ytick={0.5,2.5,4.5,6.5,8.5,10.5,12.5,14.5,16.5,18.5,20.5,22.5},
yticklabel style={rotate=0.0},
yticklabels={1,3,5,7,9,11,13,15,17},
xlabel style={font=\tiny},
tick label style={font=\tiny}
]
\addplot graphics [includegraphics cmd=\pgfimage,xmin=0, xmax=23, ymin=17, ymax=0] {chp_lmc/figures/lengthscale_matrix_1-000.png};
\end{axis}

\end{tikzpicture}
}\hspace{-0.8em}
		\subfloat[]{% This file was created with tikzplotlib v0.10.1.
\begin{tikzpicture}

\definecolor{darkgray176}{RGB}{176,176,176}

\begin{axis}[
width=\figurewidth,%
height=\figureheight,%
colorbar,
colorbar style={ylabel={}},
colormap/viridis,
point meta max=1.5,
point meta min=0,
tick align=outside,
tick pos=left,
x grid style={darkgray176},
xmin=0, xmax=17,
xtick style={color=black},
xtick={0.5,2.5,4.5,6.5,8.5,10.5,12.5,14.5,16.5,18.5,20.5,22.5},
xticklabel style={rotate=0.0},
xticklabels={1,3,5,7,9,11,13,15,17},
y dir=reverse,
y grid style={darkgray176},
ymin=0, ymax=23,
ytick style={color=black},
ytick={0.5,2.5,4.5,6.5,8.5,10.5,12.5,14.5,16.5,18.5,20.5,22.5},
yticklabel style={rotate=0.0},
yticklabels={A,C,E,G,I,K,M,O,Q,S,U,W},
]
\addplot graphics [includegraphics cmd=\pgfimage,xmin=0, xmax=17, ymin=23, ymax=0] {chp_lmc/figures/coregionalization_1-000.png};
\end{axis}

\end{tikzpicture}
}
	\end{figure}
	\vspace{-2.0em}
	\begin{block}{}
		Lengthscale values (left) and coefficients $a_{d,q}$ (right) for horizon $H=1$ reveal two feature usage patterns: focused extraction from few features and broad dynamics capture from many, with smaller individual contributions.
	\end{block}
		
	
\end{frame}

\begin{frame}{Lengthscale and $a_{d,q}$ Values (H=30)}
	\begin{figure}[htbp]
		%\centering
		\setlength\figurewidth{0.46\columnwidth} 
		\setlength\figureheight{0.46\columnwidth}
		\subfloat[]{% This file was created with tikzplotlib v0.10.1.
\begin{tikzpicture}

\definecolor{darkgray176}{RGB}{176,176,176}

\begin{axis}[
width=\figurewidth,%
height=\figureheight,%
colorbar,
colorbar style={ylabel={}},
colormap/viridis,
point meta max=40,
point meta min=20,
tick align=outside,
tick pos=left,
x grid style={darkgray176},
xmin=0, xmax=23,
xtick style={color=black},
xtick={0.5,2.5,4.5,6.5,8.5,10.5,12.5,14.5,16.5,18.5,20.5,22.5},
xticklabel style={rotate=0.0},
xticklabels={A,C,E,G,I,K,M,O,Q,S,U,W},
y dir=reverse,
y grid style={darkgray176},
ymin=0, ymax=17,
ytick style={color=black},
ytick={0.5,2.5,4.5,6.5,8.5,10.5,12.5,14.5,16.5,18.5,20.5,22.5},
yticklabel style={rotate=0.0},
yticklabels={1,3,5,7,9,11,13,15,17},
xlabel style={font=\tiny},
tick label style={font=\tiny}
]
\addplot graphics [includegraphics cmd=\pgfimage,xmin=0, xmax=23, ymin=17, ymax=0] {chp_lmc/figures/lengthscale_matrix_30-000.png};
\end{axis}

\end{tikzpicture}
}\hspace{-0.8em}
		\subfloat[]{% This file was created with tikzplotlib v0.10.1.
\begin{tikzpicture}

\definecolor{darkgray176}{RGB}{176,176,176}

\begin{axis}[
width=\figurewidth,%
height=\figureheight,%
colorbar,
colorbar style={ylabel={}},
colormap/viridis,
point meta max=1.5,
point meta min=0,
tick align=outside,
tick pos=left,
x grid style={darkgray176},
xmin=0, xmax=17,
xtick style={color=black},
xtick={0.5,2.5,4.5,6.5,8.5,10.5,12.5,14.5,16.5,18.5,20.5,22.5},
xticklabel style={rotate=0.0},
xticklabels={1,3,5,7,9,11,13,15,17},
y dir=reverse,
y grid style={darkgray176},
ymin=0, ymax=23,
ytick style={color=black},
ytick={0.5,2.5,4.5,6.5,8.5,10.5,12.5,14.5,16.5,18.5,20.5,22.5},
yticklabel style={rotate=0.0},
yticklabels={A,C,E,G,I,K,M,O,Q,S,U,W},
xlabel style={font=\tiny},
tick label style={font=\tiny}
]
\addplot graphics [includegraphics cmd=\pgfimage,xmin=0, xmax=17, ymin=23, ymax=0] {chp_lmc/figures/coregionalization_30-000.png};
\end{axis}

\end{tikzpicture}
}
	\end{figure}
	\vspace{-2.0em}
	\begin{block}{}
		For horizon $H=30$, lengthscale values (left) and coefficients $a_{d,q}$ (right) show less selective input feature usage. All independent GPs incorporate more features due to the extended time gap. The $a_{d,q}$ coefficients are smaller, indicating weaker individual feature contributions to each output.
	\end{block}
	

\end{frame}

\subsubsection{Performance Analysis}

\begin{frame}{LMCGP vs IGP+ vs IGP}
	\begin{figure}[htbp]
		\centering
		\setlength\figurewidth{0.56\columnwidth}
		\setlength\figureheight{0.48\columnwidth}
		
		\subfloat[NLPD]{% This file was created with tikzplotlib v0.10.1.
\begin{tikzpicture}

\definecolor{darkgray176}{RGB}{176,176,176}
\definecolor{gray}{RGB}{128,128,128}
\definecolor{green01270}{RGB}{0,127,0}

\begin{axis}[
width=\figurewidth,%
height=\figureheight,%
axis background/.style={fill=background_color},
tick align=outside,
tick pos=left,
axis line style={white},
tick align=inside,
tick pos=left,
x grid style={white},
y grid style={white},
xmajorgrids,
ymajorgrids,
xlabel={Horizon},
xtick style={color=black},
xtick={0.25,1.25,2.25,3.25,4.25,5.25,6.25,7.25,8.25,9.25},
xticklabels={1,2,3,4,5,6,7,14,21,30},
ytick style={color=black},
xmin=-0.6125, xmax=10.1125,
ymin=0, ymax=29.8449605648958,
xlabel style={font=\tiny},
tick label style={font=\tiny}
]
\draw[draw=gray,fill=blue] (axis cs:-0.125,0) rectangle (axis cs:0.125,21.1569115523988);
\draw[draw=gray,fill=blue] (axis cs:0.875,0) rectangle (axis cs:1.125,23.5500376665929);
\draw[draw=gray,fill=blue] (axis cs:1.875,0) rectangle (axis cs:2.125,24.4596859590331);
\draw[draw=gray,fill=blue] (axis cs:2.875,0) rectangle (axis cs:3.125,25.3025311179923);
\draw[draw=gray,fill=blue] (axis cs:3.875,0) rectangle (axis cs:4.125,25.7392946447281);
\draw[draw=gray,fill=blue] (axis cs:4.875,0) rectangle (axis cs:5.125,25.8577961476045);
\draw[draw=gray,fill=blue] (axis cs:5.875,0) rectangle (axis cs:6.125,26.1182370685187);
\draw[draw=gray,fill=blue] (axis cs:6.875,0) rectangle (axis cs:7.125,27.0243046679054);
\draw[draw=gray,fill=blue] (axis cs:7.875,0) rectangle (axis cs:8.125,27.7866695852108);
\draw[draw=gray,fill=blue] (axis cs:8.875,0) rectangle (axis cs:9.125,28.4237719665674);
\draw[draw=gray,fill=green01270] (axis cs:0.125,0) rectangle (axis cs:0.375,11.7354752059831);

\draw[draw=gray,fill=green01270] (axis cs:1.125,0) rectangle (axis cs:1.375,14.4672232695331);
\draw[draw=gray,fill=green01270] (axis cs:2.125,0) rectangle (axis cs:2.375,15.1059474613183);
\draw[draw=gray,fill=green01270] (axis cs:3.125,0) rectangle (axis cs:3.375,15.8355933150668);
\draw[draw=gray,fill=green01270] (axis cs:4.125,0) rectangle (axis cs:4.375,15.7177185284159);
\draw[draw=gray,fill=green01270] (axis cs:5.125,0) rectangle (axis cs:5.375,16.2355496277881);
\draw[draw=gray,fill=green01270] (axis cs:6.125,0) rectangle (axis cs:6.375,16.1205779992887);
\draw[draw=gray,fill=green01270] (axis cs:7.125,0) rectangle (axis cs:7.375,16.4226456565551);
\draw[draw=gray,fill=green01270] (axis cs:8.125,0) rectangle (axis cs:8.375,17.167740838524);
\draw[draw=gray,fill=green01270] (axis cs:9.125,0) rectangle (axis cs:9.375,17.0291942568183);
\draw[draw=gray,fill=red] (axis cs:0.375,0) rectangle (axis cs:0.625,10.8184214069237);

\draw[draw=gray,fill=red] (axis cs:1.375,0) rectangle (axis cs:1.625,12.4999191375716);
\draw[draw=gray,fill=red] (axis cs:2.375,0) rectangle (axis cs:2.625,13.5104526402534);
\draw[draw=gray,fill=red] (axis cs:3.375,0) rectangle (axis cs:3.625,13.8888512386877);
\draw[draw=gray,fill=red] (axis cs:4.375,0) rectangle (axis cs:4.625,13.6898117257208);
\draw[draw=gray,fill=red] (axis cs:5.375,0) rectangle (axis cs:5.625,14.5583942742105);
\draw[draw=gray,fill=red] (axis cs:6.375,0) rectangle (axis cs:6.625,13.6712080903002);
\draw[draw=gray,fill=red] (axis cs:7.375,0) rectangle (axis cs:7.625,14.2560387120429);
\draw[draw=gray,fill=red] (axis cs:8.375,0) rectangle (axis cs:8.625,14.4738108583833);
\draw[draw=gray,fill=red] (axis cs:9.375,0) rectangle (axis cs:9.625,14.8850707781946);
\end{axis}

\end{tikzpicture}
}
		\hfill
		\subfloat[MSLL]{% This file was created with tikzplotlib v0.10.1.
\begin{tikzpicture}

\definecolor{darkgray176}{RGB}{176,176,176}
\definecolor{gray}{RGB}{128,128,128}
\definecolor{green01270}{RGB}{0,127,0}
\definecolor{lightgray204}{RGB}{204,204,204}

\begin{axis}[
width=\figurewidth,%
height=\figureheight,%
axis background/.style={fill=background_color},
tick align=outside,
tick pos=left,
axis line style={white},
tick align=inside,
tick pos=left,
x grid style={white},
y grid style={white},
xmajorgrids,
ymajorgrids,
xlabel={Horizon},
xmin=-0.6125, xmax=10.1125,
xtick style={color=black},
xtick={0.25,1.25,2.25,3.25,4.25,5.25,6.25,7.25,8.25,9.25},
xticklabels={1,2,3,4,5,6,7,14,21,30},
ymin=-1.03083974778409, ymax=0,
ytick style={color=black},
legend columns=3, 
legend style={
  nodes={scale=0.8, transform shape},
  fill opacity=1,
  draw opacity=1,
  text opacity=1,
  at={(0.03,0.97)},
  anchor=north west,
  legend pos=north west,
  draw=lightgray204,
  fill=background_color,
  font=\tiny
},
xlabel style={font=\tiny},
tick label style={font=\tiny}
]
\draw[draw=gray,fill=blue] (axis cs:-0.125,0) rectangle (axis cs:0.125,-0.532252569204355);
\addlegendimage{ybar,ybar legend,draw=gray,fill=blue}
\addlegendentry{IGP}

\draw[draw=gray,fill=blue] (axis cs:0.875,0) rectangle (axis cs:1.125,-0.428154570305303);
\draw[draw=gray,fill=blue] (axis cs:1.875,0) rectangle (axis cs:2.125,-0.388555233274582);
\draw[draw=gray,fill=blue] (axis cs:2.875,0) rectangle (axis cs:3.125,-0.351859932002441);
\draw[draw=gray,fill=blue] (axis cs:3.875,0) rectangle (axis cs:4.125,-0.332820583869734);
\draw[draw=gray,fill=blue] (axis cs:4.875,0) rectangle (axis cs:5.125,-0.327618686894435);
\draw[draw=gray,fill=blue] (axis cs:5.875,0) rectangle (axis cs:6.125,-0.316243282995642);
\draw[draw=gray,fill=blue] (axis cs:6.875,0) rectangle (axis cs:7.125,-0.272667322281103);
\draw[draw=gray,fill=blue] (axis cs:7.875,0) rectangle (axis cs:8.125,-0.238004115713702);
\draw[draw=gray,fill=blue] (axis cs:8.875,0) rectangle (axis cs:9.125,-0.209382344201211);
\draw[draw=gray,fill=green01270] (axis cs:0.125,0) rectangle (axis cs:0.375,-0.941880236439821);
\addlegendimage{ybar,ybar legend,draw=gray,fill=green01270}
\addlegendentry{IGP+}

\draw[draw=gray,fill=green01270] (axis cs:1.125,0) rectangle (axis cs:1.375,-0.823059544090509);
\draw[draw=gray,fill=green01270] (axis cs:2.125,0) rectangle (axis cs:2.375,-0.795239515783919);
\draw[draw=gray,fill=green01270] (axis cs:3.125,0) rectangle (axis cs:3.375,-0.763465923433985);
\draw[draw=gray,fill=green01270] (axis cs:4.125,0) rectangle (axis cs:4.375,-0.768541284578963);
\draw[draw=gray,fill=green01270] (axis cs:5.125,0) rectangle (axis cs:5.375,-0.745977231234278);
\draw[draw=gray,fill=green01270] (axis cs:6.125,0) rectangle (axis cs:6.375,-0.750924112092601);
\draw[draw=gray,fill=green01270] (axis cs:7.125,0) rectangle (axis cs:7.375,-0.733609018426772);
\draw[draw=gray,fill=green01270] (axis cs:8.125,0) rectangle (axis cs:8.375,-0.699696669917477);
\draw[draw=gray,fill=green01270] (axis cs:9.125,0) rectangle (axis cs:9.375,-0.704798766364212);
\draw[draw=gray,fill=red] (axis cs:0.375,0) rectangle (axis cs:0.625,-0.981752140746748);
\addlegendimage{ybar,ybar legend,draw=gray,fill=red}
\addlegendentry{LMCGP}

\draw[draw=gray,fill=red] (axis cs:1.375,0) rectangle (axis cs:1.625,-0.908594506349704);
\draw[draw=gray,fill=red] (axis cs:2.375,0) rectangle (axis cs:2.625,-0.86460885583022);
\draw[draw=gray,fill=red] (axis cs:3.375,0) rectangle (axis cs:3.625,-0.848106883276553);
\draw[draw=gray,fill=red] (axis cs:4.375,0) rectangle (axis cs:4.625,-0.856711145565708);
\draw[draw=gray,fill=red] (axis cs:5.375,0) rectangle (axis cs:5.625,-0.81889702921591);
\draw[draw=gray,fill=red] (axis cs:6.375,0) rectangle (axis cs:6.625,-0.857418455961666);
\draw[draw=gray,fill=red] (axis cs:7.375,0) rectangle (axis cs:7.625,-0.827809320362084);
\draw[draw=gray,fill=red] (axis cs:8.375,0) rectangle (axis cs:8.625,-0.816824060358377);
\draw[draw=gray,fill=red] (axis cs:9.375,0) rectangle (axis cs:9.625,-0.798021526304375);
\end{axis}

\end{tikzpicture}
}
		
	\end{figure}
	\vspace{-2.0em}
	\begin{block}{}
		Bar plots comparing LMCGP, IGP+, and IGP model performance across different horizons \(H\). The Adam+NG optimizer significantly boosts performance, and with LMCGP showing the most improvement, especially for larger horizons.
	\end{block}
	
	
\end{frame}

\begin{frame}{Model Forecasting}
	\begin{figure}[htbp]
		\centering
		\setlength\figurewidth{\columnwidth} 
		\setlength\figureheight{0.24\columnwidth}
		
		% This file was created with tikzplotlib v0.10.1.
\begin{tikzpicture}

\definecolor{darkgray176}{RGB}{176,176,176}

\begin{axis}[
width=\figurewidth,
height=\figureheight,
axis background/.style={fill=background_color},
axis line style={white},
tick align=inside,
tick pos=left,
x grid style={white},
y grid style={white},
xmajorgrids,
ymajorgrids,
ylabel={Contributions (kWh)},
xmin=100, xmax=250,
xtick style={color=white},
ymin=-1179761.11521924, ymax=11048199.3095146,
ytick style={color=black},
xtick style={color=black},
ytick={-2000000,0,2000000,4000000,6000000,8000000,10000000,12000000},
yticklabels={\ensuremath{-}0.2,0.0,0.2,0.4,0.6,0.8,1.0,1.2}
]
\path [draw=blue, fill=blue, opacity=0.5]
(axis cs:0,3298090.61341946)
--(axis cs:0,465857.6173202)
--(axis cs:1,557110.412891193)
--(axis cs:2,1452349.76271034)
--(axis cs:3,715623.832035669)
--(axis cs:4,762969.45564331)
--(axis cs:5,1186419.29064429)
--(axis cs:6,1304534.09692468)
--(axis cs:7,1274188.84901295)
--(axis cs:8,2165515.66489807)
--(axis cs:9,2355807.9244491)
--(axis cs:10,1846669.47567241)
--(axis cs:11,1247568.5013096)
--(axis cs:12,1145053.42612425)
--(axis cs:13,1567655.35431001)
--(axis cs:14,3614276.51806852)
--(axis cs:15,2935158.06320903)
--(axis cs:16,1660968.2328158)
--(axis cs:17,1484477.53928419)
--(axis cs:18,2282697.64972719)
--(axis cs:19,2116489.42723088)
--(axis cs:20,3951215.09447986)
--(axis cs:21,2745781.15369753)
--(axis cs:22,2955471.63485508)
--(axis cs:23,2006612.26702155)
--(axis cs:24,1950593.2539194)
--(axis cs:25,1720244.74831444)
--(axis cs:26,2679203.20005604)
--(axis cs:27,2154973.15750863)
--(axis cs:28,2114026.54175104)
--(axis cs:29,2364646.77992208)
--(axis cs:30,2154685.82960982)
--(axis cs:31,1817500.96179313)
--(axis cs:32,1771961.42586209)
--(axis cs:33,2402177.15542028)
--(axis cs:34,2019756.21832678)
--(axis cs:35,1992627.835539)
--(axis cs:36,1958118.88023124)
--(axis cs:37,3427363.3310928)
--(axis cs:38,3332772.96946622)
--(axis cs:39,2495729.75550952)
--(axis cs:40,2603769.59564675)
--(axis cs:41,2331994.79377589)
--(axis cs:42,3280602.91388125)
--(axis cs:43,3889632.66376676)
--(axis cs:44,5639991.69625302)
--(axis cs:45,5137865.0138126)
--(axis cs:46,4992687.95648802)
--(axis cs:47,4450416.92118277)
--(axis cs:48,5110856.98811604)
--(axis cs:49,4955033.32182764)
--(axis cs:50,5227197.8345418)
--(axis cs:51,4580517.45622274)
--(axis cs:52,5477888.24477971)
--(axis cs:53,4777803.94168347)
--(axis cs:54,5365431.10411067)
--(axis cs:55,4871024.28484342)
--(axis cs:56,5092490.3310705)
--(axis cs:57,5396332.65712777)
--(axis cs:58,4478385.06366616)
--(axis cs:59,3552832.17284692)
--(axis cs:60,3829359.07378284)
--(axis cs:61,3750211.16940225)
--(axis cs:62,2959465.16599263)
--(axis cs:63,2557649.35300701)
--(axis cs:64,2089108.44670904)
--(axis cs:65,1998354.31986427)
--(axis cs:66,2008098.0485728)
--(axis cs:67,1525724.93036105)
--(axis cs:68,1423046.22922697)
--(axis cs:69,1523191.96694318)
--(axis cs:70,1340375.29506976)
--(axis cs:71,1497826.36456112)
--(axis cs:72,1728968.34804088)
--(axis cs:73,1353035.61885676)
--(axis cs:74,1328429.90370304)
--(axis cs:75,1396977.29124161)
--(axis cs:76,1230522.48866863)
--(axis cs:77,1353132.41386272)
--(axis cs:78,1165077.68253592)
--(axis cs:79,1125395.81424651)
--(axis cs:80,1017387.33005536)
--(axis cs:81,704126.086095834)
--(axis cs:82,993911.940520618)
--(axis cs:83,835301.048116284)
--(axis cs:84,666715.747644076)
--(axis cs:85,538744.787723697)
--(axis cs:86,741901.702136838)
--(axis cs:87,738813.99874532)
--(axis cs:88,680435.957240663)
--(axis cs:89,1529615.13643135)
--(axis cs:90,2507927.59956951)
--(axis cs:91,1498768.45097802)
--(axis cs:92,1296057.37274499)
--(axis cs:93,1721177.18815594)
--(axis cs:94,1264602.43459754)
--(axis cs:95,2625766.63293761)
--(axis cs:96,1960011.01789385)
--(axis cs:97,1962797.46878254)
--(axis cs:98,3020694.77427961)
--(axis cs:99,3456617.83577721)
--(axis cs:100,2968345.01282953)
--(axis cs:101,3164164.06616781)
--(axis cs:102,4234913.18236029)
--(axis cs:103,3258352.39684647)
--(axis cs:104,2210397.97598546)
--(axis cs:105,2161124.51800842)
--(axis cs:106,2048184.73690876)
--(axis cs:107,2570527.48059903)
--(axis cs:108,2670893.93005862)
--(axis cs:109,4884818.59003846)
--(axis cs:110,4745934.31298702)
--(axis cs:111,4313754.85423278)
--(axis cs:112,4283116.0161001)
--(axis cs:113,3753368.72380145)
--(axis cs:114,3443326.65113712)
--(axis cs:115,3677598.30248844)
--(axis cs:116,3122532.96839209)
--(axis cs:117,2865839.31254292)
--(axis cs:118,4140775.51058325)
--(axis cs:119,3860547.50247544)
--(axis cs:120,4153292.40561136)
--(axis cs:121,4036034.45461627)
--(axis cs:122,3751551.87234028)
--(axis cs:123,4700782.92128616)
--(axis cs:124,3794390.5627783)
--(axis cs:125,4058138.61186678)
--(axis cs:126,4569032.65784055)
--(axis cs:127,4450053.55582914)
--(axis cs:128,3846058.29630588)
--(axis cs:129,3450764.11565759)
--(axis cs:130,4192530.3965044)
--(axis cs:131,3948030.36897039)
--(axis cs:132,3372005.16079636)
--(axis cs:133,3234217.48525013)
--(axis cs:134,3425711.00222692)
--(axis cs:135,2792819.38561302)
--(axis cs:136,2626454.194124)
--(axis cs:137,2306633.55767691)
--(axis cs:138,2060044.65879852)
--(axis cs:139,1968116.08306185)
--(axis cs:140,1717023.22591117)
--(axis cs:141,2094104.21669595)
--(axis cs:142,1937118.33217648)
--(axis cs:143,1625103.76374674)
--(axis cs:144,1409577.69272765)
--(axis cs:145,1964076.9845819)
--(axis cs:146,1375688.58092843)
--(axis cs:147,1467165.56094546)
--(axis cs:148,1559679.05036303)
--(axis cs:149,1226697.58792301)
--(axis cs:150,1402173.11191927)
--(axis cs:151,1254587.69173763)
--(axis cs:152,1286567.84734836)
--(axis cs:153,1334792.24417443)
--(axis cs:154,1328164.68310166)
--(axis cs:155,1126362.16178536)
--(axis cs:156,668237.73703561)
--(axis cs:157,1127235.3107449)
--(axis cs:158,1632162.26347461)
--(axis cs:159,1177382.97256945)
--(axis cs:160,2015071.89465325)
--(axis cs:161,1409394.24631645)
--(axis cs:162,1393465.20162315)
--(axis cs:163,1246436.50357643)
--(axis cs:164,1164174.86709969)
--(axis cs:165,1818972.3616609)
--(axis cs:166,2572327.27937035)
--(axis cs:167,3591667.93111539)
--(axis cs:168,4471828.65761357)
--(axis cs:169,3802915.2376127)
--(axis cs:170,3999554.35588635)
--(axis cs:171,3693915.64077473)
--(axis cs:172,3450507.15600844)
--(axis cs:173,2587541.55995633)
--(axis cs:174,2851319.45736168)
--(axis cs:175,3000122.99419351)
--(axis cs:176,2784899.96953439)
--(axis cs:177,3090930.80028584)
--(axis cs:178,2370631.37142123)
--(axis cs:179,2449776.9024309)
--(axis cs:180,2812025.77571664)
--(axis cs:181,2809224.97907856)
--(axis cs:182,2331774.47781231)
--(axis cs:183,2225789.18763243)
--(axis cs:184,3268048.66694268)
--(axis cs:185,3971785.85025822)
--(axis cs:186,2545962.64444085)
--(axis cs:187,2434425.22641129)
--(axis cs:188,2959160.56045036)
--(axis cs:189,2400352.61845487)
--(axis cs:190,1955953.24854113)
--(axis cs:191,2001996.07749886)
--(axis cs:192,1923452.51808021)
--(axis cs:193,1750896.49170196)
--(axis cs:194,1783351.09243177)
--(axis cs:195,2103688.7191065)
--(axis cs:196,1909038.88820211)
--(axis cs:197,2806870.25954512)
--(axis cs:198,2692834.18644369)
--(axis cs:199,2296519.40057696)
--(axis cs:200,2486692.92401046)
--(axis cs:201,2095584.14292209)
--(axis cs:202,1675946.25068925)
--(axis cs:203,3072677.23535931)
--(axis cs:204,2305872.10793784)
--(axis cs:205,1639998.14110069)
--(axis cs:206,1897297.82938119)
--(axis cs:207,2516455.42050456)
--(axis cs:208,3903285.58329427)
--(axis cs:209,4614650.35955348)
--(axis cs:210,3772744.38162123)
--(axis cs:211,3333513.40115669)
--(axis cs:212,3917495.00462976)
--(axis cs:213,3577034.8852332)
--(axis cs:214,4797351.27444669)
--(axis cs:215,4384681.05463923)
--(axis cs:216,3730573.68154308)
--(axis cs:217,3158027.8605921)
--(axis cs:218,2458030.70401212)
--(axis cs:219,3331020.86389005)
--(axis cs:220,2712134.48499815)
--(axis cs:221,2162928.66360811)
--(axis cs:222,1825457.45935997)
--(axis cs:223,1885707.94813202)
--(axis cs:224,1886071.96198466)
--(axis cs:225,2182964.85406941)
--(axis cs:226,1580157.74598131)
--(axis cs:227,1832839.18291419)
--(axis cs:228,1650589.54554133)
--(axis cs:229,1422781.53193569)
--(axis cs:230,1464085.7989485)
--(axis cs:231,2031780.65421068)
--(axis cs:232,1494382.40455315)
--(axis cs:233,914932.671018968)
--(axis cs:234,792357.374605939)
--(axis cs:235,406409.814634516)
--(axis cs:236,370238.623763876)
--(axis cs:237,544921.119918156)
--(axis cs:238,521168.054969305)
--(axis cs:239,854306.085872763)
--(axis cs:240,886181.638629842)
--(axis cs:241,1152215.62288188)
--(axis cs:242,1278290.91268209)
--(axis cs:243,1128883.95354332)
--(axis cs:244,1284923.80252441)
--(axis cs:245,621442.024858651)
--(axis cs:246,286106.255669645)
--(axis cs:247,240806.764632077)
--(axis cs:248,874744.001826515)
--(axis cs:249,1641675.98452366)
--(axis cs:250,1713985.45309159)
--(axis cs:251,1466695.50901054)
--(axis cs:252,3040566.45057753)
--(axis cs:253,2835475.63500972)
--(axis cs:254,1913732.06891227)
--(axis cs:255,998113.841010088)
--(axis cs:256,620908.671919839)
--(axis cs:257,818401.272013301)
--(axis cs:258,615685.327660873)
--(axis cs:259,507548.092708218)
--(axis cs:260,346183.727785249)
--(axis cs:261,274009.675780073)
--(axis cs:262,96520.374304387)
--(axis cs:263,93212.5446859284)
--(axis cs:264,1487285.42437898)
--(axis cs:265,476921.827417952)
--(axis cs:266,107317.404680169)
--(axis cs:267,-49109.5949719502)
--(axis cs:268,-256252.56616734)
--(axis cs:269,-92308.1344758887)
--(axis cs:270,-116927.535948643)
--(axis cs:271,289428.084634251)
--(axis cs:272,214772.391202526)
--(axis cs:273,683480.504893524)
--(axis cs:274,677103.117495926)
--(axis cs:275,737202.92812135)
--(axis cs:276,382316.680265599)
--(axis cs:277,491924.208372013)
--(axis cs:278,744502.970654319)
--(axis cs:279,785033.442219134)
--(axis cs:280,1310937.1378614)
--(axis cs:281,899419.373767827)
--(axis cs:282,886448.605576929)
--(axis cs:283,410066.195098474)
--(axis cs:284,550048.890537803)
--(axis cs:285,633917.600305023)
--(axis cs:286,1852138.47447478)
--(axis cs:287,1712086.1486982)
--(axis cs:288,900275.814193739)
--(axis cs:289,824841.450928469)
--(axis cs:290,624369.062091679)
--(axis cs:291,599295.59612115)
--(axis cs:292,355185.225048213)
--(axis cs:293,34733.2008654126)
--(axis cs:294,92431.0383837889)
--(axis cs:295,426808.760290373)
--(axis cs:296,216678.998100189)
--(axis cs:297,-146330.915896008)
--(axis cs:298,810525.429412987)
--(axis cs:299,494472.196479666)
--(axis cs:300,472174.475589318)
--(axis cs:301,-100042.343622442)
--(axis cs:302,359019.650107192)
--(axis cs:303,-10437.220865659)
--(axis cs:304,330309.912235414)
--(axis cs:305,243159.372548005)
--(axis cs:306,35649.0509497572)
--(axis cs:307,-69216.3980846831)
--(axis cs:308,35573.1193388849)
--(axis cs:309,59020.773665301)
--(axis cs:310,-109515.508267104)
--(axis cs:311,-249525.466476568)
--(axis cs:312,-169541.978879317)
--(axis cs:313,-125311.351773892)
--(axis cs:314,-7169.06574247451)
--(axis cs:315,215372.760864698)
--(axis cs:316,301627.284971812)
--(axis cs:317,-62786.9158317149)
--(axis cs:318,-151332.582301052)
--(axis cs:319,-125098.604727891)
--(axis cs:320,-250343.662122739)
--(axis cs:321,-190860.835058795)
--(axis cs:322,996503.998952574)
--(axis cs:323,56626.9940408827)
--(axis cs:324,303555.537268104)
--(axis cs:325,-46544.1350105214)
--(axis cs:326,110582.563576289)
--(axis cs:327,809112.400838642)
--(axis cs:328,116966.319132173)
--(axis cs:329,567074.537513901)
--(axis cs:330,876738.386369152)
--(axis cs:331,1839390.49743992)
--(axis cs:332,1433527.89080109)
--(axis cs:333,1749136.50406472)
--(axis cs:334,1879242.59839379)
--(axis cs:335,1493305.89561125)
--(axis cs:336,1051901.80835731)
--(axis cs:337,1130190.69075129)
--(axis cs:338,2076165.53290727)
--(axis cs:339,2021509.79054042)
--(axis cs:340,1596954.35367818)
--(axis cs:341,1466990.59365905)
--(axis cs:342,1713500.97930155)
--(axis cs:343,2648892.51235437)
--(axis cs:344,2222420.28803067)
--(axis cs:345,2657133.0290612)
--(axis cs:346,2319800.44494188)
--(axis cs:347,2221716.93476701)
--(axis cs:348,2150355.29409068)
--(axis cs:349,2262123.79639333)
--(axis cs:350,2435489.66829177)
--(axis cs:351,2078649.88417262)
--(axis cs:352,2278948.99284515)
--(axis cs:353,3445548.20831634)
--(axis cs:354,2121047.00536887)
--(axis cs:355,3088416.03023497)
--(axis cs:356,3591471.93880473)
--(axis cs:357,3720880.59324315)
--(axis cs:358,3505634.02254362)
--(axis cs:359,2829668.04436118)
--(axis cs:360,2360173.10676984)
--(axis cs:361,2467165.76810063)
--(axis cs:362,4739696.85890337)
--(axis cs:363,3823018.36218221)
--(axis cs:364,4766238.67829979)
--(axis cs:365,4385499.48385948)
--(axis cs:366,4613315.55529414)
--(axis cs:367,4206021.1300104)
--(axis cs:368,4514126.28047213)
--(axis cs:369,4290082.13697938)
--(axis cs:370,3738966.76141103)
--(axis cs:371,4134325.4983705)
--(axis cs:372,5349979.05545172)
--(axis cs:373,4950181.03603409)
--(axis cs:374,4960653.32549454)
--(axis cs:375,4775781.48942897)
--(axis cs:376,5325251.55750143)
--(axis cs:377,5525453.35776351)
--(axis cs:378,4496532.66554729)
--(axis cs:379,5178482.14598779)
--(axis cs:380,4888676.3271879)
--(axis cs:381,4574404.72069025)
--(axis cs:382,5104435.06043462)
--(axis cs:383,5116207.9772937)
--(axis cs:384,4153381.18398648)
--(axis cs:385,3746572.52007473)
--(axis cs:386,3936721.3327407)
--(axis cs:387,3726538.33215469)
--(axis cs:388,3456334.67733559)
--(axis cs:389,3517902.43557617)
--(axis cs:390,2758560.83532127)
--(axis cs:391,2812207.22723949)
--(axis cs:392,3052128.3315753)
--(axis cs:393,3229603.72141612)
--(axis cs:394,4933626.03439596)
--(axis cs:395,5939912.52869829)
--(axis cs:396,4376137.13960106)
--(axis cs:397,4730744.71261998)
--(axis cs:398,4257010.90412776)
--(axis cs:399,3927911.1480924)
--(axis cs:400,3291062.58615156)
--(axis cs:401,4010163.52733104)
--(axis cs:402,3489733.10078748)
--(axis cs:403,3537983.80933249)
--(axis cs:404,3945746.19946215)
--(axis cs:405,4305727.44320176)
--(axis cs:406,3810503.53413486)
--(axis cs:407,3099738.01031865)
--(axis cs:408,2836482.82344633)
--(axis cs:409,3405332.32492256)
--(axis cs:410,3190494.86776237)
--(axis cs:411,2734253.41266685)
--(axis cs:412,3880051.54824113)
--(axis cs:413,3400295.76288321)
--(axis cs:414,4244003.75631128)
--(axis cs:415,4857840.99732225)
--(axis cs:416,4604343.31230494)
--(axis cs:417,5328657.59484081)
--(axis cs:418,4756015.07549877)
--(axis cs:419,5299544.3000856)
--(axis cs:420,5228341.43253966)
--(axis cs:421,3602603.60766712)
--(axis cs:422,4240729.25956026)
--(axis cs:423,4140617.34334184)
--(axis cs:424,4647414.20373685)
--(axis cs:425,4931493.82257261)
--(axis cs:426,3930453.22709928)
--(axis cs:427,4177297.12696187)
--(axis cs:428,3903227.25896235)
--(axis cs:429,3288664.58525577)
--(axis cs:430,2838745.48696528)
--(axis cs:431,3099383.5690177)
--(axis cs:432,2455833.39817416)
--(axis cs:433,2070546.82136891)
--(axis cs:434,1811859.68566384)
--(axis cs:435,1871092.18009062)
--(axis cs:436,2310805.5691393)
--(axis cs:437,2808049.16530327)
--(axis cs:438,2545052.10990325)
--(axis cs:439,4372246.25248682)
--(axis cs:440,2706499.68497308)
--(axis cs:440,5504119.28405238)
--(axis cs:440,5504119.28405238)
--(axis cs:439,7348164.46811499)
--(axis cs:438,5286321.26697825)
--(axis cs:437,5607797.91163651)
--(axis cs:436,5083428.9867568)
--(axis cs:435,4619588.26570217)
--(axis cs:434,4561655.24112269)
--(axis cs:433,4815110.06394573)
--(axis cs:432,5201767.10490132)
--(axis cs:431,5851827.49856792)
--(axis cs:430,5598515.30658438)
--(axis cs:429,6040119.95728033)
--(axis cs:428,6741982.55760494)
--(axis cs:427,6955345.25831563)
--(axis cs:426,6717961.29414324)
--(axis cs:425,7868607.16085864)
--(axis cs:424,7434653.08875435)
--(axis cs:423,6901942.35863715)
--(axis cs:422,7009141.49599133)
--(axis cs:421,6429698.089284)
--(axis cs:420,8186587.6514956)
--(axis cs:419,8143010.53787903)
--(axis cs:418,7527272.15929921)
--(axis cs:417,8134688.03926404)
--(axis cs:416,7369238.35499058)
--(axis cs:415,7639877.42367109)
--(axis cs:414,6989660.34428997)
--(axis cs:413,6148106.14464119)
--(axis cs:412,6685287.18125434)
--(axis cs:411,5487179.59001654)
--(axis cs:410,5937474.10630416)
--(axis cs:409,6195514.05749783)
--(axis cs:408,5570379.77486612)
--(axis cs:407,5838321.06373918)
--(axis cs:406,6558668.93631367)
--(axis cs:405,7108186.6629631)
--(axis cs:404,6731652.27785305)
--(axis cs:403,6306399.76878425)
--(axis cs:402,6247148.45159061)
--(axis cs:401,6770691.33145946)
--(axis cs:400,6033880.76060417)
--(axis cs:399,6677905.93783705)
--(axis cs:398,7020568.4074165)
--(axis cs:397,7534803.22650369)
--(axis cs:396,7936744.27234537)
--(axis cs:395,8830447.35317147)
--(axis cs:394,7748854.85546531)
--(axis cs:393,6002639.36331449)
--(axis cs:392,5806391.87536515)
--(axis cs:391,5547083.26859188)
--(axis cs:390,5516142.18772878)
--(axis cs:389,6254483.58148969)
--(axis cs:388,6200330.73726554)
--(axis cs:387,6514729.11103135)
--(axis cs:386,6689011.75766431)
--(axis cs:385,6499389.32246309)
--(axis cs:384,6963092.48876329)
--(axis cs:383,7908843.6530553)
--(axis cs:382,7924607.22813377)
--(axis cs:381,7352583.85417601)
--(axis cs:380,7879032.76907708)
--(axis cs:379,8199724.62759792)
--(axis cs:378,7517077.12243853)
--(axis cs:377,8468693.43212983)
--(axis cs:376,8327295.53515512)
--(axis cs:375,7549406.26723802)
--(axis cs:374,7719402.0488679)
--(axis cs:373,7711754.5470669)
--(axis cs:372,8177092.12121585)
--(axis cs:371,6908170.77756871)
--(axis cs:370,6477137.01278685)
--(axis cs:369,7042697.09045853)
--(axis cs:368,7281645.30815505)
--(axis cs:367,7014421.5739207)
--(axis cs:366,7659540.39793703)
--(axis cs:365,7151993.03585822)
--(axis cs:364,7686859.63258696)
--(axis cs:363,6593550.91887438)
--(axis cs:362,7620890.60157898)
--(axis cs:361,5239984.52715461)
--(axis cs:360,5177636.45552424)
--(axis cs:359,5619514.72846648)
--(axis cs:358,6373990.44087953)
--(axis cs:357,6679338.01239519)
--(axis cs:356,6375744.4816086)
--(axis cs:355,5842529.72128119)
--(axis cs:354,4904610.99803981)
--(axis cs:353,6333623.19821547)
--(axis cs:352,5077783.13813826)
--(axis cs:351,4865548.65847679)
--(axis cs:350,5262189.01964392)
--(axis cs:349,5087447.93479344)
--(axis cs:348,4966332.57601505)
--(axis cs:347,4991478.19731552)
--(axis cs:346,5084307.69813742)
--(axis cs:345,5399016.48330295)
--(axis cs:344,5010942.20606527)
--(axis cs:343,5439240.97774197)
--(axis cs:342,4614402.99991725)
--(axis cs:341,4308308.86391225)
--(axis cs:340,4346774.31659758)
--(axis cs:339,4777441.42026733)
--(axis cs:338,4864458.93092161)
--(axis cs:337,3951128.48961581)
--(axis cs:336,3885637.85551985)
--(axis cs:335,4245544.5565213)
--(axis cs:334,4638084.59867995)
--(axis cs:333,4573410.70500372)
--(axis cs:332,4250527.49093969)
--(axis cs:331,4641992.45021931)
--(axis cs:330,3680152.80522091)
--(axis cs:329,3418723.61323159)
--(axis cs:328,2988705.44430047)
--(axis cs:327,3672121.77156958)
--(axis cs:326,2911721.70938931)
--(axis cs:325,2733572.27849695)
--(axis cs:324,3173148.88887797)
--(axis cs:323,2892954.34595979)
--(axis cs:322,3973876.77586973)
--(axis cs:321,2641168.46176282)
--(axis cs:320,2558417.13977218)
--(axis cs:319,2653166.05730056)
--(axis cs:318,2661036.1768631)
--(axis cs:317,2721861.25696069)
--(axis cs:316,3101355.30139582)
--(axis cs:315,3408879.06998406)
--(axis cs:314,3050146.65200765)
--(axis cs:313,2778700.50048071)
--(axis cs:312,2599210.18060143)
--(axis cs:311,2544321.369809)
--(axis cs:310,2661565.21226322)
--(axis cs:309,2815020.28381152)
--(axis cs:308,2794710.90735719)
--(axis cs:307,2748523.98529866)
--(axis cs:306,2787696.66327486)
--(axis cs:305,2998879.14493493)
--(axis cs:304,3187499.9289648)
--(axis cs:303,2845643.03832852)
--(axis cs:302,3152814.73996759)
--(axis cs:301,2672140.44560134)
--(axis cs:300,3279347.04097069)
--(axis cs:299,3308777.6879206)
--(axis cs:298,3701188.93315594)
--(axis cs:297,2796760.15662307)
--(axis cs:296,2995871.38088872)
--(axis cs:295,3185110.67732702)
--(axis cs:294,2874896.77821536)
--(axis cs:293,2878522.58006788)
--(axis cs:292,3205977.79173376)
--(axis cs:291,3406278.06113206)
--(axis cs:290,3383030.43339321)
--(axis cs:289,3686850.9818056)
--(axis cs:288,3649393.25938962)
--(axis cs:287,4487092.15411888)
--(axis cs:286,4664180.24236407)
--(axis cs:285,3432622.11164265)
--(axis cs:284,3470034.02008172)
--(axis cs:283,3273007.85141328)
--(axis cs:282,3700927.90391652)
--(axis cs:281,3725007.47631598)
--(axis cs:280,4144299.13806607)
--(axis cs:279,3788916.29685904)
--(axis cs:278,3686611.18964805)
--(axis cs:277,3329460.28965378)
--(axis cs:276,3180383.66152208)
--(axis cs:275,3563904.20235497)
--(axis cs:274,3493061.33396807)
--(axis cs:273,3562818.18779604)
--(axis cs:272,3097831.58883847)
--(axis cs:271,3296576.02287542)
--(axis cs:270,2688129.24419712)
--(axis cs:269,2700107.41257458)
--(axis cs:268,2534841.74255336)
--(axis cs:267,2735256.35146191)
--(axis cs:266,2890122.54901592)
--(axis cs:265,3270291.42955106)
--(axis cs:264,4419314.255679)
--(axis cs:263,2894518.53209849)
--(axis cs:262,2876666.34036665)
--(axis cs:261,3068210.93868055)
--(axis cs:260,3345641.40547627)
--(axis cs:259,3297469.72722383)
--(axis cs:258,3382331.03723258)
--(axis cs:257,3631957.62646092)
--(axis cs:256,3415704.69586198)
--(axis cs:255,3779638.74393355)
--(axis cs:254,4781760.06380218)
--(axis cs:253,5944225.30021846)
--(axis cs:252,6073481.91469693)
--(axis cs:251,4383683.41150189)
--(axis cs:250,4563581.76479201)
--(axis cs:249,4525859.17730639)
--(axis cs:248,3941435.70622865)
--(axis cs:247,3212976.20315207)
--(axis cs:246,3104101.46677242)
--(axis cs:245,3386561.99160867)
--(axis cs:244,4090027.20597709)
--(axis cs:243,3982885.81066212)
--(axis cs:242,4118904.53705816)
--(axis cs:241,4203988.01415746)
--(axis cs:240,3677832.98959689)
--(axis cs:239,3904950.32311942)
--(axis cs:238,3401823.70076356)
--(axis cs:237,3293964.0840191)
--(axis cs:236,3117042.23639277)
--(axis cs:235,3148859.89666613)
--(axis cs:234,3533638.43346279)
--(axis cs:233,3654096.68422278)
--(axis cs:232,4383571.54380299)
--(axis cs:231,4813357.27486779)
--(axis cs:230,4227681.20121493)
--(axis cs:229,4200991.94754879)
--(axis cs:228,4415629.43984443)
--(axis cs:227,4613183.82966591)
--(axis cs:226,4376707.46338543)
--(axis cs:225,4961332.60785765)
--(axis cs:224,5084316.00854482)
--(axis cs:223,4667615.77116273)
--(axis cs:222,4723937.68055006)
--(axis cs:221,5205275.47307821)
--(axis cs:220,5937911.18495707)
--(axis cs:219,6268043.60606182)
--(axis cs:218,5269201.81729281)
--(axis cs:217,5967820.05425842)
--(axis cs:216,6561731.11460597)
--(axis cs:215,7486859.4565432)
--(axis cs:214,7812503.12232471)
--(axis cs:213,6629849.64491808)
--(axis cs:212,7171583.04267045)
--(axis cs:211,6643344.41206696)
--(axis cs:210,6605674.75512065)
--(axis cs:209,7705030.27967994)
--(axis cs:208,6858174.61580623)
--(axis cs:207,5328876.72345733)
--(axis cs:206,4789238.26228286)
--(axis cs:205,4485499.10787516)
--(axis cs:204,5123334.74027629)
--(axis cs:203,5901384.03120099)
--(axis cs:202,4665838.34784664)
--(axis cs:201,5269060.70087227)
--(axis cs:200,5549486.39017438)
--(axis cs:199,5171563.84644327)
--(axis cs:198,5530184.8768302)
--(axis cs:197,5843405.44470577)
--(axis cs:196,4742444.1988198)
--(axis cs:195,5046754.81007176)
--(axis cs:194,5253372.06472122)
--(axis cs:193,4566789.9796232)
--(axis cs:192,4827138.74113361)
--(axis cs:191,4800988.97551979)
--(axis cs:190,4719413.36941007)
--(axis cs:189,5161230.65863742)
--(axis cs:188,5750955.36486478)
--(axis cs:187,5758550.79697207)
--(axis cs:186,5612093.6461845)
--(axis cs:185,6919078.26850697)
--(axis cs:184,6164948.15067698)
--(axis cs:183,4984209.15299668)
--(axis cs:182,5164039.50598038)
--(axis cs:181,5620731.69317635)
--(axis cs:180,5615136.52982753)
--(axis cs:179,5382890.82138788)
--(axis cs:178,5231875.75627967)
--(axis cs:177,5950947.55694576)
--(axis cs:176,5607979.42866071)
--(axis cs:175,5840346.17328674)
--(axis cs:174,5730697.03131017)
--(axis cs:173,5601802.12335811)
--(axis cs:172,6423223.7286551)
--(axis cs:171,6586450.34311486)
--(axis cs:170,6801509.00878203)
--(axis cs:169,6842849.70651975)
--(axis cs:168,7361864.09960174)
--(axis cs:167,6391502.43046139)
--(axis cs:166,5365733.31765595)
--(axis cs:165,4581650.99364249)
--(axis cs:164,3901268.45255561)
--(axis cs:163,4031983.39036851)
--(axis cs:162,4241447.42167771)
--(axis cs:161,4153223.36775793)
--(axis cs:160,4818688.87487267)
--(axis cs:159,3918462.94067343)
--(axis cs:158,4426223.7129722)
--(axis cs:157,3946922.16352153)
--(axis cs:156,3530092.00608419)
--(axis cs:155,3905434.78084464)
--(axis cs:154,4138292.14590654)
--(axis cs:153,4223972.00573887)
--(axis cs:152,4034344.67787265)
--(axis cs:151,4250176.73690378)
--(axis cs:150,4189009.01388185)
--(axis cs:149,3988526.1737746)
--(axis cs:148,4421564.57825788)
--(axis cs:147,4222610.95411911)
--(axis cs:146,4141611.7497089)
--(axis cs:145,4950434.08304539)
--(axis cs:144,4267196.11002066)
--(axis cs:143,4389912.88652535)
--(axis cs:142,4723198.51581179)
--(axis cs:141,4888589.45033503)
--(axis cs:140,4514675.75928041)
--(axis cs:139,4755768.97523567)
--(axis cs:138,4881784.58821812)
--(axis cs:137,5090050.5033033)
--(axis cs:136,5386632.13397601)
--(axis cs:135,5628273.51473552)
--(axis cs:134,6294688.49274948)
--(axis cs:133,6018507.61381107)
--(axis cs:132,6341720.13041212)
--(axis cs:131,6754957.58507503)
--(axis cs:130,7029225.55972997)
--(axis cs:129,6283777.08540735)
--(axis cs:128,6721933.49022095)
--(axis cs:127,7306059.31580851)
--(axis cs:126,7579552.36988872)
--(axis cs:125,6954805.89620807)
--(axis cs:124,6607449.69132247)
--(axis cs:123,7882425.0989099)
--(axis cs:122,6543855.091075)
--(axis cs:121,6871325.40823765)
--(axis cs:120,7085615.23886283)
--(axis cs:119,6651192.87907239)
--(axis cs:118,6915443.22198535)
--(axis cs:117,5629881.95900547)
--(axis cs:116,6030460.04698004)
--(axis cs:115,6448748.77771057)
--(axis cs:114,6205707.49803054)
--(axis cs:113,6616676.06963156)
--(axis cs:112,7090098.79940921)
--(axis cs:111,7437902.49806632)
--(axis cs:110,7568415.51296793)
--(axis cs:109,7689879.79448929)
--(axis cs:108,5429293.30455139)
--(axis cs:107,5431755.61640051)
--(axis cs:106,4930832.88224804)
--(axis cs:105,4967974.80176967)
--(axis cs:104,4963295.02227129)
--(axis cs:103,6021036.05663061)
--(axis cs:102,7023406.92785154)
--(axis cs:101,5928315.48727363)
--(axis cs:100,5755756.09236342)
--(axis cs:99,6227558.51886065)
--(axis cs:98,5775487.10976456)
--(axis cs:97,4706088.52816465)
--(axis cs:96,4721492.50978843)
--(axis cs:95,5449695.50159825)
--(axis cs:94,4024699.63002554)
--(axis cs:93,4477189.10968006)
--(axis cs:92,4067459.07348675)
--(axis cs:91,4255978.94507249)
--(axis cs:90,5315405.64514978)
--(axis cs:89,4344021.51570716)
--(axis cs:88,3457315.35393321)
--(axis cs:87,3511800.25754385)
--(axis cs:86,3540574.47747443)
--(axis cs:85,3390735.90610928)
--(axis cs:84,3452617.8644648)
--(axis cs:83,3595023.42182856)
--(axis cs:82,3779709.25776138)
--(axis cs:81,3469431.52073134)
--(axis cs:80,3784385.61294715)
--(axis cs:79,3879287.437987)
--(axis cs:78,3915950.95355021)
--(axis cs:77,4103563.2643981)
--(axis cs:76,3974503.23666443)
--(axis cs:75,4152494.36930141)
--(axis cs:74,4098543.82266536)
--(axis cs:73,4112396.95566143)
--(axis cs:72,4597251.34304712)
--(axis cs:71,4238022.25705543)
--(axis cs:70,4094259.43020541)
--(axis cs:69,4273986.57810092)
--(axis cs:68,4166771.2168624)
--(axis cs:67,4274595.42047132)
--(axis cs:66,4753574.15960222)
--(axis cs:65,4739572.73856728)
--(axis cs:64,4841119.87232656)
--(axis cs:63,5314410.35551768)
--(axis cs:62,5699726.10984118)
--(axis cs:61,6551591.48704189)
--(axis cs:60,6720274.69292626)
--(axis cs:59,6357250.61424643)
--(axis cs:58,7232861.00619755)
--(axis cs:57,8198808.22485866)
--(axis cs:56,7948237.17212604)
--(axis cs:55,7653831.85609762)
--(axis cs:54,8180481.51931398)
--(axis cs:53,7724435.62094894)
--(axis cs:52,8445880.95089847)
--(axis cs:51,7367204.84434589)
--(axis cs:50,8022289.58859808)
--(axis cs:49,7761905.98179495)
--(axis cs:48,7893730.14804959)
--(axis cs:47,7201439.42834343)
--(axis cs:46,7759637.54571478)
--(axis cs:45,7927207.22924079)
--(axis cs:44,8988458.36817636)
--(axis cs:43,6815947.08414072)
--(axis cs:42,6073283.40021956)
--(axis cs:41,5066361.32759724)
--(axis cs:40,5481217.22011891)
--(axis cs:39,5234755.09483368)
--(axis cs:38,6075141.26917497)
--(axis cs:37,6184577.80196725)
--(axis cs:36,4719884.98859688)
--(axis cs:35,4729565.06486421)
--(axis cs:34,4755587.359074)
--(axis cs:33,5203498.99504928)
--(axis cs:32,4507364.73886795)
--(axis cs:31,4554499.62979632)
--(axis cs:30,4901292.57839067)
--(axis cs:29,5121845.09679022)
--(axis cs:28,4874718.55821018)
--(axis cs:27,4925370.70434273)
--(axis cs:26,5442923.45173927)
--(axis cs:25,4482317.13504413)
--(axis cs:24,4704040.47601498)
--(axis cs:23,4753961.08035306)
--(axis cs:22,5709270.56070734)
--(axis cs:21,5611571.90550764)
--(axis cs:20,6855644.41182983)
--(axis cs:19,4900145.26699665)
--(axis cs:18,5141814.00554462)
--(axis cs:17,4346456.87159604)
--(axis cs:16,5104613.17118294)
--(axis cs:15,5698795.9478887)
--(axis cs:14,6424183.37634539)
--(axis cs:13,4361122.99937119)
--(axis cs:12,3931055.60812131)
--(axis cs:11,3990108.40878331)
--(axis cs:10,4644905.70223763)
--(axis cs:9,5161587.91478171)
--(axis cs:8,5074838.58300285)
--(axis cs:7,4208046.13104687)
--(axis cs:6,4228889.00813326)
--(axis cs:5,4061615.98691241)
--(axis cs:4,4012820.97685055)
--(axis cs:3,3625753.7163068)
--(axis cs:2,4367140.4523824)
--(axis cs:1,3740117.56686952)
--(axis cs:0,3298090.61341946)
--cycle;

\addplot [only marks, red, mark=asterisk, mark size=1.2]
table {%
0 1811900
1 2950200
2 1817300
3 1816900
4 2271100
5 2510500
6 2288800
7 3174300
8 3646500
9 3002900
10 2336100
11 2254700
12 2661500
13 4982300
14 4045900
15 3433400
16 3294700
17 3166200
18 3009200
19 6084000
20 4332200
21 4207900
22 3374700
23 3295100
24 3073300
25 4142000
26 3655200
27 3744100
28 3744100
29 3641900
30 3253400
31 3031800
32 3558000
33 3319200
34 3328800
35 3451900
36 4412400
37 4396000
38 3925600
39 3467300
40 3754500
41 4058500
42 4092200
43 7297200
44 7012300
45 6235800
46 5705800
47 7073900
48 6717800
49 6939200
50 6193200
51 7487800
52 8148200
53 6703700
54 6683500
55 7868100
56 6811200
57 5505000
58 4985200
59 5481700
60 6089200
61 4231100
62 3903800
63 3434500
64 3285400
65 3375500
66 2838300
67 2825600
68 2842800
69 2717600
70 2797300
71 2643500
72 2659400
73 2669600
74 2752700
75 2646200
76 2619200
77 2538600
78 2454500
79 2186800
80 2033500
81 2029000
82 2041600
83 1891300
84 1766300
85 1760600
86 1875700
87 1820500
88 2204200
89 3592500
90 2735200
91 2598200
92 2856700
93 2593200
94 3478900
95 3199200
96 3019100
97 4171900
98 4427600
99 4213500
100 4365900
101 5690400
102 4459800
103 3720400
104 3635100
105 3411600
106 3340600
107 3688500
108 6044300
109 6861500
110 6231400
111 6058400
112 5697500
113 5012600
114 4644800
115 4554200
116 4556500
117 5831000
118 5064100
119 5524500
120 5076700
121 4941600
122 6442400
123 5240200
124 5782200
125 6059500
126 6629700
127 5507700
128 4775000
129 5671900
130 5773300
131 4902900
132 4809100
133 4293600
134 4435000
135 4055400
136 3638800
137 3305700
138 3474600
139 2917400
140 3147700
141 3083300
142 2956800
143 2946200
144 2864800
145 2884100
146 2908400
147 2944200
148 2762200
149 2850900
150 2479600
151 2637600
152 2255300
153 2770100
154 2642000
155 2306400
156 2395300
157 2684400
158 2534800
159 3081400
160 2917600
161 2672300
162 2742700
163 2510800
164 2874400
165 3502600
166 5254700
167 5111400
168 6195900
169 5648700
170 5800300
171 5547500
172 4379000
173 4311100
174 4581900
175 4392900
176 4709000
177 4125600
178 3863700
179 4489700
180 4446700
181 4004700
182 3373200
183 3897200
184 5274200
185 4474300
186 4310400
187 4085500
188 3586100
189 3133600
190 3207100
191 2884800
192 3267300
193 3172800
194 3172400
195 3248800
196 3604100
197 3358500
198 3837600
199 3962600
200 4147200
201 3334000
202 4282800
203 3251900
204 3390500
205 3135000
206 3631800
207 5172900
208 6400000
209 5456400
210 5464100
211 5115400
212 5065500
213 5812500
214 6945000
215 5561700
216 4547200
217 4139500
218 4247100
219 4097600
220 4039400
221 3659800
222 3320100
223 3539800
224 3605800
225 3208300
226 3151200
227 3226800
228 2968500
229 3008100
230 3908200
231 3058200
232 2550400
233 2345200
234 1914100
235 1868200
236 2182400
237 2060100
238 1896000
239 2427900
240 2419000
241 2551000
242 2413800
243 2982200
244 2174900
245 1916600
246 1694000
247 1910600
248 2973400
249 2991500
250 2385300
251 4807600
252 5394900
253 3829100
254 2475700
255 2190300
256 2382000
257 1975400
258 1972900
259 1562100
260 1395600
261 1208000
262 1366300
263 3296700
264 1998800
265 1506000
266 1217300
267 1125000
268 1065000
269 974300
270 1621800
271 1085100
272 1563400
273 1855000
274 1754600
275 1705500
276 1725600
277 1559600
278 2031400
279 2515800
280 1955600
281 1988800
282 1627900
283 1727200
284 1817700
285 3388400
286 3130700
287 2240500
288 2280000
289 1737400
290 1703900
291 1898000
292 1732700
293 1540100
294 1595600
295 1481700
296 1251700
297 2006700
298 2034500
299 1855400
300 1320300
301 1380900
302 1487200
303 1633800
304 1553700
305 1452900
306 1291700
307 1335100
308 1180000
309 1294500
310 1145300
311 1197200
312 1196800
313 1351300
314 1527900
315 1541900
316 1141000
317 1096800
318 1060700
319 1125400
320 1064000
321 2156300
322 1732100
323 1386600
324 1198600
325 1379500
326 1809800
327 1940200
328 2223800
329 2445300
330 3258400
331 3025800
332 3715900
333 3465000
334 2927100
335 2545400
336 2406300
337 3339500
338 3323700
339 3033700
340 2808200
341 2855500
342 4484000
343 3796400
344 3992500
345 4025700
346 3529700
347 3506600
348 4210800
349 4138000
350 3929200
351 3971800
352 4943700
353 3703700
354 4286300
355 4712500
356 5405600
357 4591900
358 3988600
359 3633700
360 3445000
361 5647700
362 4572400
363 7786400
364 5313100
365 7995100
366 6747200
367 6405700
368 5925900
369 4952600
370 4883300
371 7171600
372 6939800
373 6449700
374 6123600
375 6851600
376 7203000
377 7068600
378 6852800
379 6643100
380 5865500
381 7082600
382 6590000
383 5693500
384 5029500
385 5025600
386 5159400
387 4680900
388 4698500
389 3975400
390 3958600
391 4126500
392 4172900
393 6345300
394 7193900
395 10066000
396 7074000
397 6074800
398 5300600
399 4628000
400 4796300
401 4861800
402 4713500
403 5040400
404 5581800
405 5028200
406 4294400
407 4095700
408 4346300
409 4683300
410 4195600
411 4379800
412 4570400
413 5224900
414 5962800
415 5969900
416 6426400
417 5434800
418 7391400
419 8254800
420 4183500
421 5586700
422 5466200
423 5908900
424 7937300
425 5826200
426 5692100
427 5829200
428 4836800
429 4460300
430 4207400
431 3905100
432 3502300
433 3373200
434 3400400
435 3284400
436 3720900
437 3594000
438 4594100
439 3260600
440 5988400
};
\addplot [blue]
table {%
0 1881974.11536983
1 2148613.98988036
2 2909745.10754637
3 2170688.77417123
4 2387895.21624693
5 2624017.63877835
6 2766711.55252897
7 2741117.49002991
8 3620177.12395046
9 3758697.9196154
10 3245787.58895502
11 2618838.45504646
12 2538054.51712278
13 2964389.1768406
14 5019229.94720695
15 4316977.00554887
16 3382790.70199937
17 2915467.20544011
18 3712255.8276359
19 3508317.34711376
20 5403429.75315485
21 4178676.52960259
22 4332371.09778121
23 3380286.67368731
24 3327316.86496719
25 3101280.94167928
26 4061063.32589765
27 3540171.93092568
28 3494372.54998061
29 3743245.93835615
30 3527989.20400024
31 3186000.29579472
32 3139663.08236502
33 3802838.07523478
34 3387671.78870039
35 3361096.4502016
36 3339001.93441406
37 4805970.56653003
38 4703957.1193206
39 3865242.4251716
40 4042493.40788283
41 3699178.06068657
42 4676943.15705041
43 5352789.87395374
44 7314225.03221469
45 6532536.1215267
46 6376162.7511014
47 5825928.1747631
48 6502293.56808282
49 6358469.6518113
50 6624743.71156994
51 5973861.15028432
52 6961884.59783909
53 6251119.78131621
54 6772956.31171232
55 6262428.07047052
56 6520363.75159827
57 6797570.44099322
58 5855623.03493186
59 4955041.39354668
60 5274816.88335455
61 5150901.32822207
62 4329595.6379169
63 3936029.85426234
64 3465114.1595178
65 3368963.52921577
66 3380836.10408751
67 2900160.17541619
68 2794908.72304469
69 2898589.27252205
70 2717317.36263759
71 2867924.31080828
72 3163109.845544
73 2732716.28725909
74 2713486.8631842
75 2774735.83027151
76 2602512.86266653
77 2728347.83913041
78 2540514.31804306
79 2502341.62611676
80 2400886.47150125
81 2086778.80341358
82 2386810.599141
83 2215162.23497242
84 2059666.80605444
85 1964740.34691649
86 2141238.08980563
87 2125307.12814458
88 2068875.65558694
89 2936818.32606926
90 3911666.62235965
91 2877373.69802526
92 2681758.22311587
93 3099183.148918
94 2644651.03231154
95 4037731.06726793
96 3340751.76384114
97 3334442.99847359
98 4398090.94202208
99 4842088.17731893
100 4362050.55259648
101 4546239.77672072
102 5629160.05510592
103 4639694.22673854
104 3586846.49912837
105 3564549.65988904
106 3489508.8095784
107 4001141.54849977
108 4050093.61730501
109 6287349.19226388
110 6157174.91297747
111 5875828.67614955
112 5686607.40775465
113 5185022.39671651
114 4824517.07458383
115 5063173.54009951
116 4576496.50768607
117 4247860.63577419
118 5528109.3662843
119 5255870.19077392
120 5619453.8222371
121 5453679.93142696
122 5147703.48170764
123 6291604.01009803
124 5200920.12705038
125 5506472.25403742
126 6074292.51386463
127 5878056.43581882
128 5283995.89326341
129 4867270.60053247
130 5610877.97811719
131 5351493.97702271
132 4856862.64560424
133 4626362.5495306
134 4860199.7474882
135 4210546.45017427
136 4006543.16405
137 3698342.03049011
138 3470914.62350832
139 3361942.52914876
140 3115849.49259579
141 3491346.83351549
142 3330158.42399414
143 3007508.32513605
144 2838386.90137416
145 3457255.53381365
146 2758650.16531866
147 2844888.25753229
148 2990621.81431046
149 2607611.88084881
150 2795591.06290056
151 2752382.2143207
152 2660456.2626105
153 2779382.12495665
154 2733228.4145041
155 2515898.471315
156 2099164.8715599
157 2537078.73713321
158 3029192.98822341
159 2547922.95662144
160 3416880.38476296
161 2781308.80703719
162 2817456.31165043
163 2639209.94697247
164 2532721.65982765
165 3200311.67765169
166 3969030.29851315
167 4991585.18078839
168 5916846.37860766
169 5322882.47206623
170 5400531.68233419
171 5140182.9919448
172 4936865.44233177
173 4094671.84165722
174 4291008.24433593
175 4420234.58374012
176 4196439.69909755
177 4520939.1786158
178 3801253.56385045
179 3916333.86190939
180 4213581.15277209
181 4214978.33612745
182 3747906.99189635
183 3604999.17031456
184 4716498.40880983
185 5445432.05938259
186 4079028.14531268
187 4096488.01169168
188 4355057.96265757
189 3780791.63854614
190 3337683.3089756
191 3401492.52650933
192 3375295.62960691
193 3158843.23566258
194 3518361.5785765
195 3575221.76458913
196 3325741.54351096
197 4325137.85212545
198 4111509.53163694
199 3734041.62351011
200 4018089.65709242
201 3682322.42189718
202 3170892.29926794
203 4487030.63328015
204 3714603.42410706
205 3062748.62448793
206 3343268.04583202
207 3922666.07198094
208 5380730.09955025
209 6159840.31961671
210 5189209.56837094
211 4988428.90661183
212 5544539.0236501
213 5103442.26507564
214 6304927.1983857
215 5935770.25559122
216 5146152.39807453
217 4562923.95742526
218 3863616.26065246
219 4799532.23497593
220 4325022.83497761
221 3684102.06834316
222 3274697.56995501
223 3276661.85964737
224 3485193.98526474
225 3572148.73096353
226 2978432.60468337
227 3223011.50629005
228 3033109.49269288
229 2811886.73974224
230 2845883.50008171
231 3422568.96453924
232 2938976.97417807
233 2284514.67762088
234 2162997.90403436
235 1777634.85565033
236 1743640.43007832
237 1919442.60196863
238 1961495.87786643
239 2379628.20449609
240 2282007.31411336
241 2678101.81851967
242 2698597.72487013
243 2555884.88210272
244 2687475.50425075
245 2004002.00823366
246 1695103.86122103
247 1726891.48389207
248 2408089.85402758
249 3083767.58091503
250 3138783.6089418
251 2925189.46025621
252 4557024.18263723
253 4389850.46761409
254 3347746.06635723
255 2388876.29247182
256 2018306.68389091
257 2225179.44923711
258 1999008.18244673
259 1902508.90996603
260 1845912.56663076
261 1671110.30723031
262 1486593.35733552
263 1493865.53839221
264 2953299.84002899
265 1873606.62848451
266 1498719.97684804
267 1343073.37824498
268 1139294.58819301
269 1303899.63904935
270 1285600.85412424
271 1793002.05375484
272 1656301.9900205
273 2123149.34634478
274 2085082.225732
275 2150553.56523816
276 1781350.17089384
277 1910692.2490129
278 2215557.08015118
279 2286974.86953908
280 2727618.13796373
281 2312213.4250419
282 2293688.25474672
283 1841537.02325588
284 2010041.45530976
285 2033269.85597384
286 3258159.35841943
287 3099589.15140854
288 2274834.53679168
289 2255846.21636704
290 2003699.74774244
291 2002786.8286266
292 1780581.50839099
293 1456627.89046665
294 1483663.90829957
295 1805959.71880869
296 1606275.18949445
297 1325214.62036353
298 2255857.18128446
299 1901624.94220013
300 1875760.75828
301 1286049.05098945
302 1755917.19503739
303 1417602.90873143
304 1758904.92060011
305 1621019.25874147
306 1411672.85711231
307 1339653.79360699
308 1415142.01334804
309 1437020.52873841
310 1276024.85199806
311 1147397.95166621
312 1214834.10086106
313 1326694.57435341
314 1521488.79313259
315 1812125.91542438
316 1701491.29318382
317 1329537.17056449
318 1254851.79728102
319 1264033.72628633
320 1154036.73882472
321 1225153.81335201
322 2485190.38741115
323 1474790.67000034
324 1738352.21307304
325 1343514.07174321
326 1511152.1364828
327 2240617.08620411
328 1552835.88171632
329 1992899.07537274
330 2278445.59579503
331 3240691.47382962
332 2842027.69087039
333 3161273.60453422
334 3258663.59853687
335 2869425.22606628
336 2468769.83193858
337 2540659.59018355
338 3470312.23191444
339 3399475.60540388
340 2971864.33513788
341 2887649.72878565
342 3163951.9896094
343 4044066.74504817
344 3616681.24704797
345 4028074.75618207
346 3702054.07153965
347 3606597.56604127
348 3558343.93505286
349 3674785.86559339
350 3848839.34396785
351 3472099.27132471
352 3678366.06549171
353 4889585.70326591
354 3512829.00170434
355 4465472.87575808
356 4983608.21020667
357 5200109.30281917
358 4939812.23171157
359 4224591.38641383
360 3768904.78114704
361 3853575.14762762
362 6180293.73024117
363 5208284.64052829
364 6226549.15544337
365 5768746.25985885
366 6136427.97661558
367 5610221.35196555
368 5897885.79431359
369 5666389.61371896
370 5108051.88709894
371 5521248.13796961
372 6763535.58833379
373 6330967.7915505
374 6340027.68718122
375 6162593.87833349
376 6826273.54632827
377 6997073.39494667
378 6006804.89399291
379 6689103.38679285
380 6383854.54813249
381 5963494.28743313
382 6514521.1442842
383 6512525.8151745
384 5558236.83637489
385 5122980.92126891
386 5312866.5452025
387 5120633.72159302
388 4828332.70730056
389 4886193.00853293
390 4137351.51152503
391 4179645.24791568
392 4429260.10347023
393 4616121.54236531
394 6341240.44493064
395 7385179.94093488
396 6156440.70597321
397 6132773.96956183
398 5638789.65577213
399 5302908.54296473
400 4662471.67337786
401 5390427.42939525
402 4868440.77618904
403 4922191.78905837
404 5338699.2386576
405 5706957.05308243
406 5184586.23522427
407 4469029.53702892
408 4203431.29915623
409 4800423.1912102
410 4563984.48703327
411 4110716.50134169
412 5282669.36474773
413 4774200.9537622
414 5616832.05030063
415 6248859.21049667
416 5986790.83364776
417 6731672.81705243
418 6141643.61739899
419 6721277.41898231
420 6707464.54201763
421 5016150.84847556
422 5624935.3777758
423 5521279.85098949
424 6041033.6462456
425 6400050.49171562
426 5324207.26062126
427 5566321.19263875
428 5322604.90828365
429 4664392.27126805
430 4218630.39677483
431 4475605.53379281
432 3828800.25153774
433 3442828.44265732
434 3186757.46339327
435 3245340.22289639
436 3697117.27794805
437 4207923.53846989
438 3915686.68844075
439 5860205.3603009
440 4105309.48451273
};
\end{axis}

\end{tikzpicture}
\hfill
		% This file was created with tikzplotlib v0.10.1.
\begin{tikzpicture}

\definecolor{darkgray176}{RGB}{176,176,176}

\begin{axis}[
width=\figurewidth,
height=\figureheight,
axis background/.style={fill=background_color},
axis line style={white},
tick align=inside,
tick pos=left,
ylabel={Contributions (kWh)},
x grid style={white},
xmajorgrids,
xmin=50, xmax=200,
xtick style={color=black},
y grid style={white},
ymajorgrids,
ymin=-3414130.1187411, ymax=12497779.277438,
ytick style={color=black},
ytick={-4000000,-2000000,0,2000000,4000000,6000000,8000000,10000000,12000000,14000000},
]
\path [draw=blue, fill=blue, opacity=0.5]
(axis cs:0,7679984.24705729)
--(axis cs:0,2917489.43104134)
--(axis cs:1,2982475.2004108)
--(axis cs:2,3412409.28949479)
--(axis cs:3,3203869.38989402)
--(axis cs:4,2912864.85847748)
--(axis cs:5,3294361.27001976)
--(axis cs:6,2504446.52562919)
--(axis cs:7,2455496.5359375)
--(axis cs:8,2063730.61385061)
--(axis cs:9,2471527.08738114)
--(axis cs:10,2087061.93914864)
--(axis cs:11,2068758.04016285)
--(axis cs:12,1653180.33918535)
--(axis cs:13,3859736.02751677)
--(axis cs:14,4537897.20831926)
--(axis cs:15,3277620.00039835)
--(axis cs:16,1695284.85875567)
--(axis cs:17,1966959.62473449)
--(axis cs:18,3134134.78771934)
--(axis cs:19,2828709.56750116)
--(axis cs:20,3876449.59478391)
--(axis cs:21,2848093.35253936)
--(axis cs:22,3600507.52062665)
--(axis cs:23,3022347.70935306)
--(axis cs:24,2905676.13474874)
--(axis cs:25,2481211.30106356)
--(axis cs:26,2564268.35362987)
--(axis cs:27,2030429.25208577)
--(axis cs:28,2294638.86030943)
--(axis cs:29,3322096.97925028)
--(axis cs:30,2397133.9884203)
--(axis cs:31,2053047.60767589)
--(axis cs:32,2311818.05999175)
--(axis cs:33,2697391.55598018)
--(axis cs:34,2157240.38803831)
--(axis cs:35,2401106.17088506)
--(axis cs:36,2299723.00322058)
--(axis cs:37,3800216.20377714)
--(axis cs:38,3703461.88262605)
--(axis cs:39,3019779.81502086)
--(axis cs:40,3413201.44657928)
--(axis cs:41,2881025.07959644)
--(axis cs:42,4291389.08062548)
--(axis cs:43,4060142.8237833)
--(axis cs:44,3481997.07579305)
--(axis cs:45,3891492.14582372)
--(axis cs:46,4110304.48482329)
--(axis cs:47,3532827.15842848)
--(axis cs:48,3634901.65677719)
--(axis cs:49,2869282.93787973)
--(axis cs:50,4494561.93350498)
--(axis cs:51,3216161.45753051)
--(axis cs:52,4034747.58598494)
--(axis cs:53,3317984.37464376)
--(axis cs:54,3594001.88822107)
--(axis cs:55,3117094.45043948)
--(axis cs:56,3065348.27955064)
--(axis cs:57,4650171.09004554)
--(axis cs:58,3310559.73327812)
--(axis cs:59,2809454.99209545)
--(axis cs:60,2996747.62458147)
--(axis cs:61,2426189.79359569)
--(axis cs:62,2813148.52216307)
--(axis cs:63,2811657.34525501)
--(axis cs:64,3335729.43511588)
--(axis cs:65,2537340.37919239)
--(axis cs:66,2164190.19690098)
--(axis cs:67,1654852.02503031)
--(axis cs:68,1399690.08274928)
--(axis cs:69,1237163.36781349)
--(axis cs:70,1209226.19800315)
--(axis cs:71,1436966.65871377)
--(axis cs:72,3513173.94671417)
--(axis cs:73,2820010.91609189)
--(axis cs:74,2353687.92180664)
--(axis cs:75,1902465.34870207)
--(axis cs:76,1624368.23086607)
--(axis cs:77,2095649.46641945)
--(axis cs:78,1574748.80152594)
--(axis cs:79,1683137.36360548)
--(axis cs:80,1229993.81825738)
--(axis cs:81,1066761.80772511)
--(axis cs:82,3111364.08941993)
--(axis cs:83,1701936.78221748)
--(axis cs:84,1831707.83885457)
--(axis cs:85,2562040.28739348)
--(axis cs:86,3113548.58271729)
--(axis cs:87,1877180.53775377)
--(axis cs:88,2281224.20268079)
--(axis cs:89,2586727.81812645)
--(axis cs:90,1952498.79589295)
--(axis cs:91,1802213.10017979)
--(axis cs:92,1288777.44523775)
--(axis cs:93,1105400.53027211)
--(axis cs:94,1120194.97881654)
--(axis cs:95,1275151.52814121)
--(axis cs:96,935119.511475899)
--(axis cs:97,2125290.88469931)
--(axis cs:98,2730128.1621721)
--(axis cs:99,2341227.18780366)
--(axis cs:100,1493226.36825061)
--(axis cs:101,1161461.47985358)
--(axis cs:102,1129224.43332816)
--(axis cs:103,1059155.30887248)
--(axis cs:104,1341415.34656189)
--(axis cs:105,983744.755914505)
--(axis cs:106,1664866.5549998)
--(axis cs:107,2420746.70900163)
--(axis cs:108,1655608.96741228)
--(axis cs:109,1695656.71105454)
--(axis cs:110,1453713.18470758)
--(axis cs:111,605962.766181772)
--(axis cs:112,1163845.7545825)
--(axis cs:113,577191.472753175)
--(axis cs:114,1516529.69718251)
--(axis cs:115,3358582.923284)
--(axis cs:116,2702482.98495262)
--(axis cs:117,2015092.41675752)
--(axis cs:118,1997318.90056699)
--(axis cs:119,1965145.60743524)
--(axis cs:120,2010626.58147187)
--(axis cs:121,1299869.15564852)
--(axis cs:122,1532434.83647141)
--(axis cs:123,2135592.4581391)
--(axis cs:124,2314684.70553052)
--(axis cs:125,2846091.68720357)
--(axis cs:126,2305832.55878519)
--(axis cs:127,2467803.82965228)
--(axis cs:128,1921492.81952138)
--(axis cs:129,2458008.96321451)
--(axis cs:130,2325944.66787883)
--(axis cs:131,2429545.75979456)
--(axis cs:132,1815136.82704659)
--(axis cs:133,2264282.25902703)
--(axis cs:134,2786897.04742897)
--(axis cs:135,1242794.96059963)
--(axis cs:136,1365941.21273608)
--(axis cs:137,1553723.35083154)
--(axis cs:138,1289906.37631477)
--(axis cs:139,788932.163283331)
--(axis cs:140,940767.004119291)
--(axis cs:141,1105773.15870677)
--(axis cs:142,1851531.81487777)
--(axis cs:143,1980425.71421969)
--(axis cs:144,2270204.48383211)
--(axis cs:145,2405585.91819742)
--(axis cs:146,1675951.01406862)
--(axis cs:147,1429079.19995716)
--(axis cs:148,3332519.62155422)
--(axis cs:149,1970398.27253148)
--(axis cs:150,2212308.53497331)
--(axis cs:151,731321.690293832)
--(axis cs:152,742401.862329931)
--(axis cs:153,696481.653166119)
--(axis cs:154,844265.149900103)
--(axis cs:155,965383.799802638)
--(axis cs:156,1448044.83825602)
--(axis cs:157,1038315.64862774)
--(axis cs:158,2121144.6338735)
--(axis cs:159,1381243.99103028)
--(axis cs:160,1617778.68211952)
--(axis cs:161,1290676.22403618)
--(axis cs:162,1594225.52892163)
--(axis cs:163,1180084.83564248)
--(axis cs:164,1263719.17512619)
--(axis cs:165,1232759.83590769)
--(axis cs:166,2241237.79852954)
--(axis cs:167,2547958.48127936)
--(axis cs:168,3979832.30590487)
--(axis cs:169,2684384.55364342)
--(axis cs:170,3287709.67916509)
--(axis cs:171,3432549.29471511)
--(axis cs:172,3448630.89848808)
--(axis cs:173,3017009.45836024)
--(axis cs:174,2746960.6610996)
--(axis cs:175,2735298.36892186)
--(axis cs:176,3064884.2230575)
--(axis cs:177,3645100.07441874)
--(axis cs:178,2753781.31003333)
--(axis cs:179,2463743.76010546)
--(axis cs:180,1904047.78126948)
--(axis cs:181,2267442.74088608)
--(axis cs:182,2062314.14372937)
--(axis cs:183,2132300.28432497)
--(axis cs:184,3747428.53643517)
--(axis cs:185,4033076.96487381)
--(axis cs:186,3605624.41010253)
--(axis cs:187,2356068.7823238)
--(axis cs:188,2976651.89941397)
--(axis cs:189,2440791.2162352)
--(axis cs:190,2306345.2306331)
--(axis cs:191,3351800.0534161)
--(axis cs:192,2261309.62684204)
--(axis cs:193,1819279.53589965)
--(axis cs:194,1464569.85686153)
--(axis cs:195,1698819.78271284)
--(axis cs:196,2188578.81351247)
--(axis cs:197,2694323.45912929)
--(axis cs:198,3474396.36982183)
--(axis cs:199,2938411.45915798)
--(axis cs:200,3869974.85491382)
--(axis cs:201,3120349.86858124)
--(axis cs:202,2683422.49830802)
--(axis cs:203,3724868.45667511)
--(axis cs:204,3245032.53837236)
--(axis cs:205,2132785.10054987)
--(axis cs:206,2974878.85806041)
--(axis cs:207,3196914.03575752)
--(axis cs:208,4199883.83119749)
--(axis cs:209,3451290.728535)
--(axis cs:210,3105455.07919413)
--(axis cs:211,3128299.23365977)
--(axis cs:212,2433218.17485911)
--(axis cs:213,2769390.67800514)
--(axis cs:214,3050205.51425521)
--(axis cs:215,2879942.76903341)
--(axis cs:216,2537096.49545219)
--(axis cs:217,1821998.67525135)
--(axis cs:218,1294846.54573787)
--(axis cs:219,2022770.21523863)
--(axis cs:220,1550706.08933325)
--(axis cs:221,689285.743333825)
--(axis cs:222,1123458.94219666)
--(axis cs:223,2091519.73641097)
--(axis cs:224,2221600.6607308)
--(axis cs:225,2588898.8711105)
--(axis cs:226,2063643.09785299)
--(axis cs:227,1802367.97327628)
--(axis cs:228,1249671.50296711)
--(axis cs:229,736612.423290445)
--(axis cs:230,102494.637866587)
--(axis cs:231,75155.2371517671)
--(axis cs:232,-145397.268552344)
--(axis cs:233,494222.041646762)
--(axis cs:234,616546.236871024)
--(axis cs:235,342706.927493482)
--(axis cs:236,159769.545503355)
--(axis cs:237,112298.063791818)
--(axis cs:238,-391148.456104159)
--(axis cs:239,-504510.593138072)
--(axis cs:240,372555.188068123)
--(axis cs:241,922619.762367975)
--(axis cs:242,-498242.310442616)
--(axis cs:243,-834211.405757753)
--(axis cs:244,-544254.336859646)
--(axis cs:245,-261888.349011427)
--(axis cs:246,257954.199219426)
--(axis cs:247,-196752.683902787)
--(axis cs:248,-1064551.72945586)
--(axis cs:249,-1060880.93599958)
--(axis cs:250,-788384.100683442)
--(axis cs:251,-1418055.27242892)
--(axis cs:252,-1190275.53927862)
--(axis cs:253,-978849.892340882)
--(axis cs:254,-817894.323294024)
--(axis cs:255,-58360.6458918303)
--(axis cs:256,-418422.068856314)
--(axis cs:257,-286019.019800927)
--(axis cs:258,-675867.655906836)
--(axis cs:259,-566296.756191047)
--(axis cs:260,-1617590.49495109)
--(axis cs:261,-1100575.73697092)
--(axis cs:262,-768917.460955109)
--(axis cs:263,-1671845.75501035)
--(axis cs:264,-1352822.13736866)
--(axis cs:265,-1315059.16599414)
--(axis cs:266,-1200662.26779044)
--(axis cs:267,-1192114.44657334)
--(axis cs:268,-1009209.38367926)
--(axis cs:269,-251308.632440381)
--(axis cs:270,269169.755357018)
--(axis cs:271,-961498.785289288)
--(axis cs:272,-1024219.01019484)
--(axis cs:273,1115751.80977258)
--(axis cs:274,1077096.35865129)
--(axis cs:275,2807879.67676026)
--(axis cs:276,1773298.47465474)
--(axis cs:277,3209451.75713373)
--(axis cs:278,1967719.89400045)
--(axis cs:279,2417301.04230861)
--(axis cs:280,2822802.70517876)
--(axis cs:281,3081151.00053684)
--(axis cs:282,2550851.12953923)
--(axis cs:283,1682341.6431942)
--(axis cs:284,2367219.44165324)
--(axis cs:285,2592886.59713286)
--(axis cs:286,2225712.97144436)
--(axis cs:287,2481842.45757226)
--(axis cs:288,2055939.50134904)
--(axis cs:289,2560241.14735253)
--(axis cs:290,2011428.63100231)
--(axis cs:291,1592480.28189806)
--(axis cs:292,2724382.7163194)
--(axis cs:293,2054264.09948943)
--(axis cs:294,1891202.54059085)
--(axis cs:295,1501417.80656133)
--(axis cs:296,1293635.38956902)
--(axis cs:297,998888.85364853)
--(axis cs:298,1687795.56441785)
--(axis cs:299,1156354.96763287)
--(axis cs:300,2045969.09083358)
--(axis cs:301,1309088.90913885)
--(axis cs:302,1346530.20706875)
--(axis cs:303,583675.187761881)
--(axis cs:304,423265.493671787)
--(axis cs:305,444675.38511499)
--(axis cs:306,1027462.28289674)
--(axis cs:307,1276332.78455816)
--(axis cs:308,667446.147486211)
--(axis cs:309,441178.537505197)
--(axis cs:310,361036.530339733)
--(axis cs:311,359848.942644017)
--(axis cs:312,388001.013866929)
--(axis cs:313,2120319.55307975)
--(axis cs:314,3532925.51091316)
--(axis cs:315,1668211.78606755)
--(axis cs:316,1189355.03839513)
--(axis cs:317,774533.518755079)
--(axis cs:318,484053.943311654)
--(axis cs:319,431951.171165565)
--(axis cs:320,452226.455621021)
--(axis cs:321,646648.558764158)
--(axis cs:322,656320.749266473)
--(axis cs:323,824994.071183939)
--(axis cs:324,1664474.82067434)
--(axis cs:325,1420105.02429988)
--(axis cs:326,1443991.67860116)
--(axis cs:327,1943207.27550379)
--(axis cs:328,672602.616723479)
--(axis cs:329,59565.213614882)
--(axis cs:330,972909.308632754)
--(axis cs:331,1928137.10569285)
--(axis cs:332,1367180.63002391)
--(axis cs:333,2496436.94992171)
--(axis cs:334,1646695.27865691)
--(axis cs:335,1746751.84391372)
--(axis cs:336,1716044.32219471)
--(axis cs:337,1660888.20296363)
--(axis cs:338,1730947.72984214)
--(axis cs:339,1493665.66778162)
--(axis cs:340,1290860.94633602)
--(axis cs:341,1674580.04091971)
--(axis cs:342,1002324.79565706)
--(axis cs:343,801582.916510339)
--(axis cs:344,1853343.79561231)
--(axis cs:345,1590230.56587636)
--(axis cs:346,1381269.06669185)
--(axis cs:347,1507060.94565381)
--(axis cs:348,2325146.24438242)
--(axis cs:349,2345382.89679721)
--(axis cs:350,2174982.0761801)
--(axis cs:351,2190262.48084903)
--(axis cs:352,2014551.65643376)
--(axis cs:353,2645309.35506452)
--(axis cs:354,2125941.13618584)
--(axis cs:355,2220225.50154373)
--(axis cs:356,2704948.47310419)
--(axis cs:357,3159007.19823957)
--(axis cs:358,2433518.07616621)
--(axis cs:359,2059444.41484432)
--(axis cs:360,2064578.35051255)
--(axis cs:361,2151582.14322951)
--(axis cs:362,2693986.49320545)
--(axis cs:363,3315718.48649008)
--(axis cs:364,2856572.253689)
--(axis cs:365,3439388.09630195)
--(axis cs:366,1940223.42255239)
--(axis cs:367,2272070.3962789)
--(axis cs:368,3117417.83929134)
--(axis cs:369,2332994.13105902)
--(axis cs:370,3146596.38026869)
--(axis cs:371,3468065.85506536)
--(axis cs:372,4086706.28704517)
--(axis cs:373,3759930.65356188)
--(axis cs:374,3391430.10968748)
--(axis cs:375,4034443.79796314)
--(axis cs:376,3959819.23371528)
--(axis cs:377,4235317.34346317)
--(axis cs:378,4137485.1030638)
--(axis cs:379,3496015.34363113)
--(axis cs:380,2181152.63113452)
--(axis cs:381,3092585.75026244)
--(axis cs:382,3868144.3505766)
--(axis cs:383,4011925.16037722)
--(axis cs:384,3172547.53836776)
--(axis cs:385,3890018.75265465)
--(axis cs:386,3772780.42133329)
--(axis cs:387,3254358.07719239)
--(axis cs:388,3362255.26897569)
--(axis cs:389,3492707.89090529)
--(axis cs:390,3166640.42813365)
--(axis cs:391,2892856.89934353)
--(axis cs:392,4092752.53567151)
--(axis cs:393,3771200.82051161)
--(axis cs:394,4646147.65032412)
--(axis cs:395,4364989.99027751)
--(axis cs:396,3506508.07958188)
--(axis cs:397,3323581.45236484)
--(axis cs:398,2812454.85617854)
--(axis cs:399,3998005.03209272)
--(axis cs:400,3582073.38247949)
--(axis cs:401,3763943.21289781)
--(axis cs:402,2834831.10336599)
--(axis cs:403,3506516.10781756)
--(axis cs:404,4311980.64557083)
--(axis cs:405,4283312.42445459)
--(axis cs:406,4042480.6427551)
--(axis cs:407,3340675.95059722)
--(axis cs:408,3279738.95930303)
--(axis cs:409,4198697.81156371)
--(axis cs:410,3524230.56256518)
--(axis cs:411,3050900.49603025)
--(axis cs:412,3960144.5076375)
--(axis cs:413,3613529.19174047)
--(axis cs:414,4048265.56227727)
--(axis cs:415,4317130.13664659)
--(axis cs:416,3474861.10805085)
--(axis cs:417,3519516.34854784)
--(axis cs:418,3138067.57331856)
--(axis cs:419,2803690.17493645)
--(axis cs:420,2348548.99812738)
--(axis cs:421,2263872.12187547)
--(axis cs:422,2237429.69808405)
--(axis cs:423,2851102.42285992)
--(axis cs:424,2431060.61794783)
--(axis cs:425,2043255.3426351)
--(axis cs:426,1875798.78806705)
--(axis cs:427,1980876.27960109)
--(axis cs:428,1074587.97042321)
--(axis cs:429,1194882.20566753)
--(axis cs:430,1448647.15581405)
--(axis cs:431,3885615.18217673)
--(axis cs:432,2395431.24552416)
--(axis cs:433,1726188.80308921)
--(axis cs:434,1518835.88124687)
--(axis cs:435,1369885.93183798)
--(axis cs:436,3963059.87060979)
--(axis cs:437,3867796.84753531)
--(axis cs:438,3174165.87244418)
--(axis cs:439,4290185.78046883)
--(axis cs:440,3256120.16507369)
--(axis cs:440,8022356.31003915)
--(axis cs:440,8022356.31003915)
--(axis cs:439,9691203.45842095)
--(axis cs:438,7866419.53012419)
--(axis cs:437,8710386.69306469)
--(axis cs:436,8729814.6984532)
--(axis cs:435,6064181.06058305)
--(axis cs:434,6210636.1558785)
--(axis cs:433,6410236.85286229)
--(axis cs:432,7076933.46019765)
--(axis cs:431,8597866.44141556)
--(axis cs:430,6147228.07995845)
--(axis cs:429,5897067.85148066)
--(axis cs:428,5855273.49397249)
--(axis cs:427,6691312.28447242)
--(axis cs:426,6595059.88172283)
--(axis cs:425,6820851.89778449)
--(axis cs:424,7161416.94647428)
--(axis cs:423,7551280.76531755)
--(axis cs:422,6948631.19252782)
--(axis cs:421,7048335.28621586)
--(axis cs:420,7133489.86256111)
--(axis cs:419,7541349.78049977)
--(axis cs:418,7850287.46441707)
--(axis cs:417,8258372.3385759)
--(axis cs:416,8176402.06116528)
--(axis cs:415,9038374.97633848)
--(axis cs:414,8740661.28399606)
--(axis cs:413,8308951.35332157)
--(axis cs:412,8741203.63788666)
--(axis cs:411,7745851.07282461)
--(axis cs:410,8210095.86799942)
--(axis cs:409,8970816.99739047)
--(axis cs:408,7955101.18326571)
--(axis cs:407,8024635.58924367)
--(axis cs:406,8746584.26977279)
--(axis cs:405,9263590.65680052)
--(axis cs:404,9163335.29814785)
--(axis cs:403,8241139.79166751)
--(axis cs:402,7536376.66274259)
--(axis cs:401,8474613.8239344)
--(axis cs:400,8267840.11074639)
--(axis cs:399,8690879.77504601)
--(axis cs:398,7500819.66104002)
--(axis cs:397,8028173.07387978)
--(axis cs:396,8344917.75035064)
--(axis cs:395,9512468.64800802)
--(axis cs:394,9672833.67191725)
--(axis cs:393,8527351.07688676)
--(axis cs:392,8829111.93168207)
--(axis cs:391,7563055.79117382)
--(axis cs:390,7882167.05603462)
--(axis cs:389,8171488.55926914)
--(axis cs:388,8044000.56481757)
--(axis cs:387,7980366.180205)
--(axis cs:386,8470653.80442838)
--(axis cs:385,8595415.37318593)
--(axis cs:384,7901820.00849699)
--(axis cs:383,8741656.22342907)
--(axis cs:382,8592232.92436555)
--(axis cs:381,7800415.71810708)
--(axis cs:380,7110305.73232726)
--(axis cs:379,8460103.66663251)
--(axis cs:378,9200822.92396957)
--(axis cs:377,9124663.08237261)
--(axis cs:376,8864320.06526396)
--(axis cs:375,8750093.95695064)
--(axis cs:374,8082852.41242962)
--(axis cs:373,8453698.03346741)
--(axis cs:372,8839859.85695364)
--(axis cs:371,8185143.61466612)
--(axis cs:370,7821365.91307012)
--(axis cs:369,7017415.705728)
--(axis cs:368,7807572.4543371)
--(axis cs:367,6998019.46415688)
--(axis cs:366,6845152.37655341)
--(axis cs:365,8136843.28837586)
--(axis cs:364,7601084.20756068)
--(axis cs:363,8027974.65681743)
--(axis cs:362,7529218.4817143)
--(axis cs:361,6877212.73448065)
--(axis cs:360,6851887.02627603)
--(axis cs:359,6800988.46511969)
--(axis cs:358,7276617.9784904)
--(axis cs:357,8127833.33000954)
--(axis cs:356,7431637.60907517)
--(axis cs:355,6922444.82617644)
--(axis cs:354,6874955.51636513)
--(axis cs:353,7468384.46957303)
--(axis cs:352,6773385.30364753)
--(axis cs:351,6937483.15904326)
--(axis cs:350,6997257.5453839)
--(axis cs:349,7161727.86128985)
--(axis cs:348,7105470.07659208)
--(axis cs:347,6223681.96220715)
--(axis cs:346,6084913.93935312)
--(axis cs:345,6272301.29539062)
--(axis cs:344,6585660.21973364)
--(axis cs:343,5532374.23038601)
--(axis cs:342,5872041.8784377)
--(axis cs:341,6485634.62363162)
--(axis cs:340,5988356.17167386)
--(axis cs:339,6196712.2547643)
--(axis cs:338,6480575.78146034)
--(axis cs:337,6446053.10374163)
--(axis cs:336,6574919.70291316)
--(axis cs:335,6441446.09879662)
--(axis cs:334,6352856.7679818)
--(axis cs:333,7294094.45899704)
--(axis cs:332,6147059.7659915)
--(axis cs:331,6681151.99913654)
--(axis cs:330,5749354.00157312)
--(axis cs:329,4902034.73928589)
--(axis cs:328,5575140.32927956)
--(axis cs:327,6724172.21912285)
--(axis cs:326,6185848.04308381)
--(axis cs:325,6136087.94975728)
--(axis cs:324,6474423.21248024)
--(axis cs:323,5628376.34755757)
--(axis cs:322,5586059.78764098)
--(axis cs:321,5414616.19288484)
--(axis cs:320,5213347.95859968)
--(axis cs:319,5140517.35701155)
--(axis cs:318,5225350.38348705)
--(axis cs:317,5489158.71378632)
--(axis cs:316,5935537.49263528)
--(axis cs:315,7029171.1370364)
--(axis cs:314,8539768.54562119)
--(axis cs:313,6946513.07394425)
--(axis cs:312,5099785.40781596)
--(axis cs:311,5100793.0596111)
--(axis cs:310,5074612.0960969)
--(axis cs:309,5128503.53265326)
--(axis cs:308,5362234.55971839)
--(axis cs:307,6030681.9891503)
--(axis cs:306,5720210.09157174)
--(axis cs:305,5137178.88732451)
--(axis cs:304,5248585.24454535)
--(axis cs:303,5395877.84086812)
--(axis cs:302,6063216.58703186)
--(axis cs:301,6021727.77580961)
--(axis cs:300,6809391.43228823)
--(axis cs:299,5937227.24141796)
--(axis cs:298,6538314.61786301)
--(axis cs:297,5884606.19087736)
--(axis cs:296,6004482.08810512)
--(axis cs:295,6180307.90881648)
--(axis cs:294,6615756.21228378)
--(axis cs:293,6862426.82672978)
--(axis cs:292,7604555.75380777)
--(axis cs:291,6342216.36659028)
--(axis cs:290,6710358.89109285)
--(axis cs:289,7490808.33116744)
--(axis cs:288,6752426.02442935)
--(axis cs:287,7214057.49616407)
--(axis cs:286,6987570.7206934)
--(axis cs:285,7351296.7302903)
--(axis cs:284,7270179.03615997)
--(axis cs:283,6473017.82257516)
--(axis cs:282,7313202.84727305)
--(axis cs:281,7868196.99427038)
--(axis cs:280,7631796.17498143)
--(axis cs:279,7358026.72616718)
--(axis cs:278,6895203.86710297)
--(axis cs:277,8039495.4315072)
--(axis cs:276,6517470.84407483)
--(axis cs:275,7600518.37380187)
--(axis cs:274,5862226.22870356)
--(axis cs:273,5918627.17450617)
--(axis cs:272,3799402.5583802)
--(axis cs:271,4144691.39833375)
--(axis cs:270,4997222.53172352)
--(axis cs:269,4474377.23469855)
--(axis cs:268,3763895.7140109)
--(axis cs:267,3569744.86627753)
--(axis cs:266,3570175.51008694)
--(axis cs:265,3471600.53079333)
--(axis cs:264,3568839.35642576)
--(axis cs:263,3103638.5953311)
--(axis cs:262,3979428.0755234)
--(axis cs:261,3660072.03999091)
--(axis cs:260,3582017.31357931)
--(axis cs:259,4215693.98740307)
--(axis cs:258,4058865.95164573)
--(axis cs:257,4555947.83435833)
--(axis cs:256,4363657.76015522)
--(axis cs:255,4710584.64527587)
--(axis cs:254,4089524.97193181)
--(axis cs:253,4177425.72182261)
--(axis cs:252,3879893.12886764)
--(axis cs:251,3564783.85836179)
--(axis cs:250,4057534.5185709)
--(axis cs:249,3810589.1321966)
--(axis cs:248,4095326.64372283)
--(axis cs:247,4945073.40193915)
--(axis cs:246,5073915.87791109)
--(axis cs:245,4469787.05552874)
--(axis cs:244,4225886.88042692)
--(axis cs:243,3994193.8450365)
--(axis cs:242,4336683.66130794)
--(axis cs:241,5979827.12068389)
--(axis cs:240,5125492.40553625)
--(axis cs:239,4645917.94904521)
--(axis cs:238,4540225.07378281)
--(axis cs:237,4809571.21403723)
--(axis cs:236,4855362.01881906)
--(axis cs:235,5030360.20909669)
--(axis cs:234,5300132.03578323)
--(axis cs:233,5175127.29155159)
--(axis cs:232,4727029.98233835)
--(axis cs:231,4797343.38275115)
--(axis cs:230,4816296.87730774)
--(axis cs:229,5473531.17287031)
--(axis cs:228,5953622.24253577)
--(axis cs:227,6541505.17458394)
--(axis cs:226,6845008.74534634)
--(axis cs:225,7317777.98370203)
--(axis cs:224,7477608.63668965)
--(axis cs:223,6829629.63183662)
--(axis cs:222,5982290.0699364)
--(axis cs:221,5861995.45939569)
--(axis cs:220,6972680.34843589)
--(axis cs:219,6942842.35688233)
--(axis cs:218,6084359.90535575)
--(axis cs:217,6607539.83881067)
--(axis cs:216,7312393.95659604)
--(axis cs:215,7900605.9830517)
--(axis cs:214,8028995.86076962)
--(axis cs:213,7839179.97006602)
--(axis cs:212,7604756.13279482)
--(axis cs:211,8608018.08291848)
--(axis cs:210,7916329.93450038)
--(axis cs:209,8467271.64633882)
--(axis cs:208,9465953.67527066)
--(axis cs:207,7973369.03780106)
--(axis cs:206,7867600.79482359)
--(axis cs:205,7013974.55411695)
--(axis cs:204,8058118.41554364)
--(axis cs:203,8566881.28533567)
--(axis cs:202,7762703.9048781)
--(axis cs:201,8506678.63455296)
--(axis cs:200,8999991.03141956)
--(axis cs:199,7834257.12383957)
--(axis cs:198,8293559.26307454)
--(axis cs:197,7671251.96914194)
--(axis cs:196,7018499.83210553)
--(axis cs:195,6708084.48817406)
--(axis cs:194,7092479.59374169)
--(axis cs:193,6623697.25765675)
--(axis cs:192,7128797.73804549)
--(axis cs:191,8122079.22580202)
--(axis cs:190,7005831.45343372)
--(axis cs:189,7142485.30690615)
--(axis cs:188,7734119.90854094)
--(axis cs:187,7667217.27026429)
--(axis cs:186,8760889.25525625)
--(axis cs:185,8969829.48438775)
--(axis cs:184,8696524.60397946)
--(axis cs:183,6830897.01066938)
--(axis cs:182,6848667.03451863)
--(axis cs:181,7030307.72378642)
--(axis cs:180,6666893.86124969)
--(axis cs:179,7413270.73944194)
--(axis cs:178,7608298.51197949)
--(axis cs:177,8454965.59160259)
--(axis cs:176,7851560.0645369)
--(axis cs:175,7539087.83161556)
--(axis cs:174,7597022.79454206)
--(axis cs:173,8111629.08014248)
--(axis cs:172,8460508.1652476)
--(axis cs:171,8269135.93007002)
--(axis cs:170,8030037.5577928)
--(axis cs:169,7843572.20304232)
--(axis cs:168,8855513.83080661)
--(axis cs:167,7277434.23525222)
--(axis cs:166,6981310.7156479)
--(axis cs:165,5949064.48217294)
--(axis cs:164,5941589.08559362)
--(axis cs:163,5913077.36251603)
--(axis cs:162,6456745.70609371)
--(axis cs:161,5974388.27070282)
--(axis cs:160,6348246.07492212)
--(axis cs:159,6055094.9635329)
--(axis cs:158,6852948.12814345)
--(axis cs:157,5802820.07066115)
--(axis cs:156,6360101.54756293)
--(axis cs:155,5688410.40971653)
--(axis cs:154,5617267.20853434)
--(axis cs:153,5517640.80548702)
--(axis cs:152,5434638.04990845)
--(axis cs:151,5821721.89087053)
--(axis cs:150,6954165.21558927)
--(axis cs:149,6673093.01398498)
--(axis cs:148,8266053.16671285)
--(axis cs:147,6133498.37307363)
--(axis cs:146,6394821.79600721)
--(axis cs:145,7337268.85564099)
--(axis cs:144,7104010.23216164)
--(axis cs:143,6690035.52343236)
--(axis cs:142,6580301.95706702)
--(axis cs:141,5857267.6137795)
--(axis cs:140,5708734.07977597)
--(axis cs:139,5552060.95484431)
--(axis cs:138,6100646.02124984)
--(axis cs:137,6284286.97887067)
--(axis cs:136,6076872.34192094)
--(axis cs:135,6057978.31506259)
--(axis cs:134,7633551.11817586)
--(axis cs:133,7005642.82215158)
--(axis cs:132,6871682.35362213)
--(axis cs:131,7195455.61239436)
--(axis cs:130,7144874.59020811)
--(axis cs:129,7264017.00082095)
--(axis cs:128,6808651.9720635)
--(axis cs:127,7275645.98244101)
--(axis cs:126,7273809.96141489)
--(axis cs:125,7656584.12289611)
--(axis cs:124,7086409.23081471)
--(axis cs:123,7316064.61168085)
--(axis cs:122,6282566.86410872)
--(axis cs:121,6108860.49712517)
--(axis cs:120,6960947.94839998)
--(axis cs:119,6704489.41414463)
--(axis cs:118,6719501.71596854)
--(axis cs:117,6715890.0598456)
--(axis cs:116,7702009.8260955)
--(axis cs:115,8094022.27309296)
--(axis cs:114,6224478.86133071)
--(axis cs:113,5403289.30444477)
--(axis cs:112,5921970.01045392)
--(axis cs:111,5773587.5852807)
--(axis cs:110,6197216.15960382)
--(axis cs:109,6455005.24173789)
--(axis cs:108,6363503.93425422)
--(axis cs:107,7253276.86750313)
--(axis cs:106,6537754.5198939)
--(axis cs:105,5801660.3046735)
--(axis cs:104,6037670.00341317)
--(axis cs:103,5773836.71712868)
--(axis cs:102,5867265.40550506)
--(axis cs:101,5872362.67622323)
--(axis cs:100,6228839.39756818)
--(axis cs:99,7054917.96675261)
--(axis cs:98,7431220.63180688)
--(axis cs:97,6809495.36445855)
--(axis cs:96,5668234.35987965)
--(axis cs:95,6071114.35838793)
--(axis cs:94,5829748.90535881)
--(axis cs:93,5804871.37892263)
--(axis cs:92,6000743.31709687)
--(axis cs:91,6494569.94744923)
--(axis cs:90,6698150.86011434)
--(axis cs:89,7308882.20505177)
--(axis cs:88,6987550.6117297)
--(axis cs:87,6581850.45899539)
--(axis cs:86,7855726.59571309)
--(axis cs:85,7394539.45379704)
--(axis cs:84,6554035.45594732)
--(axis cs:83,6394975.67781797)
--(axis cs:82,7850606.08262153)
--(axis cs:81,5779838.93524948)
--(axis cs:80,5949856.55831945)
--(axis cs:79,6383454.03809912)
--(axis cs:78,6271410.28224737)
--(axis cs:77,6791990.24153959)
--(axis cs:76,6308177.97159257)
--(axis cs:75,6601279.44819712)
--(axis cs:74,7085841.84714626)
--(axis cs:73,7532699.33918604)
--(axis cs:72,8629621.44408564)
--(axis cs:71,6124766.7654831)
--(axis cs:70,5922364.37606678)
--(axis cs:69,5944655.5857685)
--(axis cs:68,6092849.17724834)
--(axis cs:67,6350420.35214876)
--(axis cs:66,6853546.93339051)
--(axis cs:65,7223705.95219183)
--(axis cs:64,8042296.51818)
--(axis cs:63,7511761.62983328)
--(axis cs:62,7494078.48788662)
--(axis cs:61,7159752.87235976)
--(axis cs:60,7979838.39618029)
--(axis cs:59,7575224.59594134)
--(axis cs:58,8015850.69312602)
--(axis cs:57,9405281.85405485)
--(axis cs:56,7820355.94316565)
--(axis cs:55,7832005.57964444)
--(axis cs:54,8350857.76497807)
--(axis cs:53,8065462.53297554)
--(axis cs:52,9052961.64103819)
--(axis cs:51,7950741.33034869)
--(axis cs:50,9233994.24260177)
--(axis cs:49,7605572.95366764)
--(axis cs:48,8337490.74567383)
--(axis cs:47,8221615.13451748)
--(axis cs:46,8830784.18158457)
--(axis cs:45,8608553.29241206)
--(axis cs:44,8720181.54875344)
--(axis cs:43,8975446.78313228)
--(axis cs:42,9106663.87325399)
--(axis cs:41,7552354.96632876)
--(axis cs:40,8254323.21389199)
--(axis cs:39,7697260.87393141)
--(axis cs:38,8387255.9309216)
--(axis cs:37,8513166.48357755)
--(axis cs:36,6996659.79328977)
--(axis cs:35,7071389.41326878)
--(axis cs:34,6827846.4819399)
--(axis cs:33,7441741.47343554)
--(axis cs:32,6982897.01352642)
--(axis cs:31,6723379.5961083)
--(axis cs:30,7082756.28995205)
--(axis cs:29,8024871.52077783)
--(axis cs:28,6995268.95702948)
--(axis cs:27,6743084.58716952)
--(axis cs:26,7259051.64086478)
--(axis cs:25,7184740.06182591)
--(axis cs:24,7602442.59011252)
--(axis cs:23,7712469.15797597)
--(axis cs:22,8302425.66899976)
--(axis cs:21,7715006.54235887)
--(axis cs:20,8728141.64079677)
--(axis cs:19,7556424.53620037)
--(axis cs:18,7926601.75760115)
--(axis cs:17,6800967.931237)
--(axis cs:16,7766133.5586506)
--(axis cs:15,7986714.32678255)
--(axis cs:14,9321241.78467989)
--(axis cs:13,8641134.16753677)
--(axis cs:12,6389653.07323644)
--(axis cs:11,6749407.69242221)
--(axis cs:10,6827428.59381815)
--(axis cs:9,7232825.07862385)
--(axis cs:8,6953409.42891104)
--(axis cs:7,7326635.23231807)
--(axis cs:6,7414173.69228216)
--(axis cs:5,8129876.40119676)
--(axis cs:4,8144611.9796668)
--(axis cs:3,8093423.58946947)
--(axis cs:2,8324253.76512555)
--(axis cs:1,8271244.11171538)
--(axis cs:0,7679984.24705729)
--cycle;

\addplot [only marks, red, mark=asterisk, mark size=1.2]
table {%
0 7149100
1 5658700
2 5278900
3 6046300
4 5661300
5 5928400
6 5221600
7 4428300
8 4796900
9 4144900
10 3767800
11 3453700
12 7709600
13 7709600
14 5534200
15 5959400
16 4790300
17 4820800
18 4843600
19 7379500
20 7029500
21 6528800
22 5612900
23 5680700
24 4828500
25 4393800
26 4046200
27 4235000
28 5846400
29 4380600
30 3798100
31 4207000
32 4958500
33 3948900
34 4204500
35 4362400
36 6925600
37 6458900
38 5364400
39 5995100
40 4883700
41 8473900
42 7491200
43 7058500
44 6415100
45 6576600
46 5782900
47 5479600
48 4524700
49 7920800
50 5019300
51 8852200
52 5798500
53 5956900
54 5249100
55 6044900
56 8281700
57 5647300
58 4850400
59 7208100
60 4797400
61 5228600
62 5174700
63 6327900
64 4725400
65 3995200
66 3364900
67 3007100
68 2773200
69 2709100
70 2954100
71 10084800
72 5496600
73 4964200
74 4087900
75 3579900
76 4194200
77 3393200
78 3388600
79 2876100
80 2652000
81 6194500
82 3557200
83 3872300
84 5395800
85 6194500
86 3897200
87 4466600
88 5014200
89 4057300
90 3780500
91 3121200
92 3035300
93 2766900
94 2847700
95 2480800
96 4124200
97 4759100
98 4759100
99 3726000
100 3217600
101 3130400
102 3074100
103 2881200
104 2418100
105 2747900
106 4268200
107 3733900
108 3835400
109 3862200
110 3172400
111 3248600
112 2834700
113 3209700
114 6202700
115 7724000
116 4160700
117 3946100
118 4168600
119 4587900
120 3794300
121 3508300
122 4960300
123 5194600
124 4757200
125 5037300
126 4577300
127 4343500
128 4838900
129 5547100
130 4331800
131 4591600
132 3884200
133 4230800
134 3919900
135 3447000
136 3375500
137 3653600
138 2934400
139 2704200
140 2984100
141 4224400
142 4231700
143 3532700
144 5728500
145 3739000
146 3302000
147 6888800
148 4376700
149 4427800
150 3371200
151 3079700
152 3038200
153 3045500
154 2783900
155 4552900
156 3226900
157 5103700
158 3608000
159 3633800
160 3034400
161 3512000
162 2962500
163 3088900
164 2882200
165 4026900
166 4617400
167 5124600
168 7773700
169 5641200
170 6349000
171 6525700
172 5305400
173 4846800
174 5394400
175 5407800
176 5351500
177 5015200
178 5513000
179 4517400
180 4271000
181 4220700
182 4023200
183 5952700
184 6192100
185 6109600
186 6124700
187 6094300
188 4790000
189 4472500
190 5605800
191 5186300
192 4291800
193 4173200
194 4022700
195 4806200
196 4205000
197 6726900
198 5977600
199 5216300
200 7415200
201 6250300
202 6979200
203 7303600
204 5190000
205 5815200
206 5235700
207 9781600
208 6692300
209 6353100
210 5431300
211 5204300
212 4729600
213 5068200
214 6122000
215 5059400
216 3967400
217 3688700
218 3742700
219 3118400
220 2941300
221 3205200
222 3930500
223 4250700
224 5111600
225 4421400
226 3775000
227 3451100
228 2984700
229 2712000
230 2500800
231 2312500
232 2314000
233 2447400
234 2275900
235 2090500
236 2015900
237 1912100
238 1875000
239 2692700
240 4866700
241 2094200
242 1805300
243 1607100
244 1474500
245 1870800
246 1663200
247 1451800
248 1333600
249 1245100
250 1207500
251 1180600
252 1154200
253 1123100
254 2155000
255 1463400
256 1873100
257 1340100
258 1340100
259 1230700
260 1160200
261 1104100
262 1055500
263 1022100
264 989200
265 959500
266 932600
267 905300
268 1649700
269 2359000
270 1346500
271 1139800
272 3925300
273 3750000
274 6550700
275 4871300
276 7055400
277 5808600
278 5128100
279 6022300
280 6574800
281 4805000
282 3766700
283 4804600
284 5674600
285 4063700
286 5243600
287 4373900
288 6620200
289 4486100
290 4234800
291 7840200
292 4896400
293 4088400
294 3589600
295 3241900
296 3224300
297 3727800
298 3196100
299 4726700
300 3278200
301 3079800
302 3213200
303 3031100
304 2591200
305 3070900
306 3489500
307 2517000
308 2212500
309 1990900
310 1849000
311 1947800
312 4983000
313 3274900
314 7221400
315 3673500
316 2890600
317 2486900
318 2228700
319 2038600
320 2631000
321 3451500
322 2727900
323 4411500
324 3607800
325 3217900
326 4213100
327 3042600
328 2697800
329 2699600
330 3654500
331 3325900
332 5257800
333 3309200
334 3438000
335 3834900
336 3367600
337 3426900
338 3866300
339 3354100
340 3635000
341 3162800
342 3024600
343 3957700
344 3484400
345 3246600
346 3211900
347 4457900
348 5736300
349 4331200
350 4098200
351 3952600
352 3992500
353 3673100
354 3894600
355 4672000
356 7140300
357 4707200
358 3695700
359 3964600
360 3656800
361 4355000
362 5632900
363 4689600
364 5576800
365 4433700
366 4036500
367 5008900
368 3930400
369 5319100
370 5797900
371 6756400
372 6488500
373 5668600
374 6442200
375 5849400
376 7113400
377 8568000
378 6312900
379 5103100
380 5300900
381 5946300
382 6725000
383 5804400
384 7214000
385 6788500
386 5741800
387 5898600
388 6116900
389 5599500
390 5021900
391 8005800
392 7747600
393 10203800
394 10338700
395 6655400
396 5403000
397 4469400
398 6673500
399 6047800
400 7027700
401 5099400
402 6563300
403 9234600
404 10016500
405 7341900
406 5782100
407 5582800
408 8059100
409 5688600
410 4980300
411 7363700
412 6216500
413 6770000
414 6974300
415 5061300
416 5165700
417 4762300
418 4461900
419 4335500
420 3995200
421 3494100
422 4353000
423 3605400
424 3149300
425 2922100
426 3169300
427 2875400
428 2509600
429 2772800
430 7051400
431 4182000
432 3320200
433 2970400
434 2779800
435 8009900
436 8343200
437 5654700
438 10685900
439 6363000
440 5326000
};
\addplot [blue]
table {%
0 5298736.83904932
1 5626859.65606309
2 5868331.52731017
3 5648646.48968174
4 5528738.41907214
5 5712118.83560826
6 4959310.10895567
7 4891065.88412778
8 4508570.02138083
9 4852176.08300249
10 4457245.2664834
11 4409082.86629253
12 4021416.7062109
13 6250435.09752677
14 6929569.49649957
15 5632167.16359045
16 4730709.20870313
17 4383963.77798575
18 5530368.27266025
19 5192567.05185076
20 6302295.61779034
21 5281549.94744911
22 5951466.59481321
23 5367408.43366451
24 5254059.36243063
25 4832975.68144474
26 4911659.99724732
27 4386756.91962765
28 4644953.90866946
29 5673484.25001405
30 4739945.13918617
31 4388213.6018921
32 4647357.53675909
33 5069566.51470786
34 4492543.4349891
35 4736247.79207692
36 4648191.39825518
37 6156691.34367734
38 6045358.90677382
39 5358520.34447614
40 5833762.33023563
41 5216690.0229626
42 6699026.47693974
43 6517794.80345779
44 6101089.31227324
45 6250022.71911789
46 6470544.33320393
47 5877221.14647298
48 5986196.20122551
49 5237427.94577368
50 6864278.08805338
51 5583451.3939396
52 6543854.61351157
53 5691723.45380965
54 5972429.82659957
55 5474550.01504196
56 5442852.11135815
57 7027726.4720502
58 5663205.21320207
59 5192339.79401839
60 5488293.01038088
61 4792971.33297773
62 5153613.50502484
63 5161709.48754414
64 5689012.97664794
65 4880523.16569211
66 4508868.56514575
67 4002636.18858953
68 3746269.62999881
69 3590909.476791
70 3565795.28703497
71 3780866.71209844
72 6071397.6953999
73 5176355.12763896
74 4719764.88447645
75 4251872.3984496
76 3966273.10122932
77 4443819.85397952
78 3923079.54188665
79 4033295.7008523
80 3589925.18828841
81 3423300.37148729
82 5480985.08602073
83 4048456.23001773
84 4192871.64740095
85 4978289.87059526
86 5484637.58921519
87 4229515.49837458
88 4634387.40720525
89 4947805.01158911
90 4325324.82800364
91 4148391.52381451
92 3644760.38116731
93 3455135.95459737
94 3474971.94208767
95 3673132.94326457
96 3301676.93567778
97 4467393.12457893
98 5080674.39698949
99 4698072.57727813
100 3861032.88290939
101 3516912.07803841
102 3498244.91941661
103 3416496.01300058
104 3689542.67498753
105 3392702.530294
106 4101310.53744685
107 4837011.78825238
108 4009556.45083325
109 4075330.97639622
110 3825464.6721557
111 3189775.17573124
112 3542907.88251821
113 2990240.38859897
114 3870504.27925661
115 5726302.59818848
116 5202246.40552406
117 4365491.23830156
118 4358410.30826777
119 4334817.51078993
120 4485787.26493593
121 3704364.82638684
122 3907500.85029006
123 4725828.53490997
124 4700546.96817261
125 5251337.90504984
126 4789821.26010004
127 4871724.90604665
128 4365072.39579244
129 4861012.98201773
130 4735409.62904347
131 4812500.68609446
132 4343409.59033436
133 4634962.54058931
134 5210224.08280241
135 3650386.63783111
136 3721406.77732851
137 3919005.16485111
138 3695276.19878231
139 3170496.55906382
140 3324750.54194763
141 3481520.38624314
142 4215916.88597239
143 4335230.61882603
144 4687107.35799687
145 4871427.3869192
146 4035386.40503791
147 3781288.78651539
148 5799286.39413354
149 4321745.64325823
150 4583236.87528129
151 3276521.79058218
152 3088519.95611919
153 3107061.22932657
154 3230766.17921722
155 3326897.10475958
156 3904073.19290947
157 3420567.85964444
158 4487046.38100847
159 3718169.47728159
160 3983012.37852082
161 3632532.2473695
162 4025485.61750767
163 3546581.09907926
164 3602654.13035991
165 3590912.15904031
166 4611274.25708872
167 4912696.35826579
168 6417673.06835574
169 5263978.37834287
170 5658873.61847895
171 5850842.61239257
172 5954569.53186784
173 5564319.26925136
174 5171991.72782083
175 5137193.10026871
176 5458222.1437972
177 6050032.83301066
178 5181039.91100641
179 4938507.2497737
180 4285470.82125958
181 4648875.23233625
182 4455490.589124
183 4481598.64749718
184 6221976.57020731
185 6501453.22463078
186 6183256.83267939
187 5011643.02629405
188 5355385.90397746
189 4791638.26157068
190 4656088.34203341
191 5736939.63960906
192 4695053.68244377
193 4221488.3967782
194 4278524.72530161
195 4203452.13544345
196 4603539.322809
197 5182787.71413561
198 5883977.81644818
199 5386334.29149877
200 6434982.94316669
201 5813514.2515671
202 5223063.20159306
203 6145874.87100539
204 5651575.476958
205 4573379.82733341
206 5421239.826442
207 5585141.53677929
208 6832918.75323408
209 5959281.18743691
210 5510892.50684725
211 5868158.65828912
212 5018987.15382696
213 5304285.32403558
214 5539600.68751242
215 5390274.37604256
216 4924745.22602412
217 4214769.25703101
218 3689603.22554681
219 4482806.28606048
220 4261693.21888457
221 3275640.60136476
222 3552874.50606653
223 4460574.6841238
224 4849604.64871022
225 4953338.42740627
226 4454325.92159966
227 4171936.57393011
228 3601646.87275144
229 3105071.79808038
230 2459395.75758716
231 2436249.30995146
232 2290816.356893
233 2834674.66659918
234 2958339.13632713
235 2686533.56829509
236 2507565.78216121
237 2460934.63891452
238 2074538.30883932
239 2070703.67795357
240 2749023.79680219
241 3451223.44152593
242 1919220.67543266
243 1579991.21963937
244 1840816.27178364
245 2103949.35325866
246 2665935.03856526
247 2374160.35901818
248 1515387.45713348
249 1374854.09809851
250 1634575.20894373
251 1073364.29296644
252 1344808.79479451
253 1599287.91474086
254 1635815.3243189
255 2326111.99969202
256 1972617.84564945
257 2134964.4072787
258 1691499.14786945
259 1824698.61560601
260 982213.409314109
261 1279748.15150999
262 1605255.30728415
263 715896.420160378
264 1108008.60952855
265 1078270.6823996
266 1184756.62114825
267 1188815.20985209
268 1377343.16516582
269 2111534.30112908
270 2633196.14354027
271 1591596.30652223
272 1387591.77409268
273 3517189.49213937
274 3469661.29367743
275 5204199.02528106
276 4145384.65936478
277 5624473.59432047
278 4431461.88055171
279 4887663.88423789
280 5227299.44008009
281 5474673.99740361
282 4932026.98840614
283 4077679.73288468
284 4818699.2389066
285 4972091.66371158
286 4606641.84606888
287 4847949.97686817
288 4404182.76288919
289 5025524.73925998
290 4360893.76104758
291 3967348.32424417
292 5164469.23506358
293 4458345.4631096
294 4253479.37643732
295 3840862.8576889
296 3649058.73883707
297 3441747.52226294
298 4113055.09114043
299 3546791.10452542
300 4427680.26156091
301 3665408.34247423
302 3704873.3970503
303 2989776.514315
304 2835925.36910857
305 2790927.13621975
306 3373836.18723424
307 3653507.38685423
308 3014840.3536023
309 2784841.03507923
310 2717824.31321831
311 2730321.00112756
312 2743893.21084144
313 4533416.313512
314 6036347.02826717
315 4348691.46155198
316 3562446.26551521
317 3131846.1162707
318 2854702.16339935
319 2786234.26408856
320 2832787.20711035
321 3030632.3758245
322 3121190.26845373
323 3226685.20937076
324 4069449.01657729
325 3778096.48702858
326 3814919.86084248
327 4333689.74731332
328 3123871.47300152
329 2480799.97645038
330 3361131.65510294
331 4304644.55241469
332 3757120.19800771
333 4895265.70445937
334 3999776.02331935
335 4094098.97135517
336 4145482.01255394
337 4053470.65335263
338 4105761.75565124
339 3845188.96127296
340 3639608.55900494
341 4080107.33227566
342 3437183.33704738
343 3166978.57344817
344 4219502.00767298
345 3931265.93063349
346 3733091.50302249
347 3865371.45393048
348 4715308.16048725
349 4753555.37904353
350 4586119.810782
351 4563872.81994614
352 4393968.48004065
353 5056846.91231878
354 4500448.32627549
355 4571335.16386009
356 5068293.04108968
357 5643420.26412455
358 4855068.02732831
359 4430216.43998201
360 4458232.68839429
361 4514397.43885508
362 5111602.48745987
363 5671846.57165375
364 5228828.23062484
365 5788115.69233891
366 4392687.8995529
367 4635044.93021789
368 5462495.14681422
369 4675204.91839351
370 5483981.14666941
371 5826604.73486574
372 6463283.07199941
373 6106814.34351465
374 5737141.26105855
375 6392268.87745689
376 6412069.64948962
377 6679990.21291789
378 6669154.01351669
379 5978059.50513182
380 4645729.18173089
381 5446500.73418476
382 6230188.63747107
383 6376790.69190314
384 5537183.77343238
385 6242717.06292029
386 6121717.11288083
387 5617362.1286987
388 5703127.91689663
389 5832098.22508721
390 5524403.74208413
391 5227956.34525868
392 6460932.23367679
393 6149275.94869918
394 7159490.66112068
395 6938729.31914276
396 5925712.91496626
397 5675877.26312231
398 5156637.25860928
399 6344442.40356936
400 5924956.74661294
401 6119278.5184161
402 5185603.88305429
403 5873827.94974253
404 6737657.97185934
405 6773451.54062755
406 6394532.45626395
407 5682655.76992045
408 5617420.07128437
409 6584757.40447709
410 5867163.2152823
411 5398375.78442743
412 6350674.07276208
413 5961240.27253102
414 6394463.42313667
415 6677752.55649254
416 5825631.58460806
417 5888944.34356187
418 5494177.51886781
419 5172519.97771811
420 4741019.43034424
421 4656103.70404567
422 4593030.44530593
423 5201191.59408874
424 4796238.78221106
425 4432053.62020979
426 4235429.33489494
427 4336094.28203675
428 3464930.73219785
429 3545975.02857409
430 3797937.61788625
431 6241740.81179615
432 4736182.35286091
433 4068212.82797575
434 3864736.01856268
435 3717033.49621051
436 6346437.28453149
437 6289091.7703
438 5520292.70128419
439 6990694.61944489
440 5639238.23755642
};
\end{axis}

\end{tikzpicture}
\hfill
		% This file was created with tikzplotlib v0.10.1.
\begin{tikzpicture}
\definecolor{green01270}{RGB}{0,127,0}
\definecolor{steelblue31119180}{RGB}{31,119,180}
\definecolor{darkgray176}{RGB}{176,176,176}
\definecolor{lightgray204}{RGB}{204,204,204}

\begin{axis}[
width=\figurewidth,
height=\figureheight,
axis background/.style={fill=background_color},
axis line style={white},
tick align=inside,
tick pos=left,
ylabel={Contributions (kWh)},
x grid style={white},
xmajorgrids,
xmin=150, xmax=300,
xtick style={color=white},
y grid style={white},
ymajorgrids,
xlabel={Time (days)},
xtick style={color=black},
ymin=-13237205.6599335, ymax=119040419.31714,
ytick style={color=black},
ytick={-20000000,0,20000000,40000000,60000000,80000000,100000000,120000000},
yticklabels={\ensuremath{-}0.2,0.0,0.2,0.4,0.6,0.8,1.0,1.2},
legend columns=3, 
legend style={
  nodes={scale=0.8, transform shape},
  fill opacity=0.8,
  draw opacity=1,
  text opacity=1,
  at={(0.03,0.97)},
  anchor=north west,
  legend pos=north east,
  draw=lightgray204,
  fill=background_color
},
]
\path [draw=blue, fill=blue, opacity=0.5]
(axis cs:0,31925052.2091835)
--(axis cs:0,1558183.05204632)
--(axis cs:1,-2349178.42999233)
--(axis cs:2,-5126183.6097403)
--(axis cs:3,7024335.79619604)
--(axis cs:4,-5739164.84421941)
--(axis cs:5,-3654257.77045015)
--(axis cs:6,-5519537.2407308)
--(axis cs:7,-5765991.04656563)
--(axis cs:8,-7934752.50211442)
--(axis cs:9,-6512049.72912347)
--(axis cs:10,-3992347.33414757)
--(axis cs:11,-5316205.56090487)
--(axis cs:12,-7888235.74583745)
--(axis cs:13,-4324348.01138568)
--(axis cs:14,-5820781.84119326)
--(axis cs:15,-6001510.58673185)
--(axis cs:16,-11084451.5735503)
--(axis cs:17,-8505380.47099489)
--(axis cs:18,-6317763.97851044)
--(axis cs:19,-6043994.00266101)
--(axis cs:20,-7252956.23418731)
--(axis cs:21,-9618673.39036574)
--(axis cs:22,-9508850.00971471)
--(axis cs:23,-7867763.64765651)
--(axis cs:24,-7686728.24722127)
--(axis cs:25,-8944613.44307322)
--(axis cs:26,-9203281.84058987)
--(axis cs:27,-12453972.7203834)
--(axis cs:28,-10550814.3180554)
--(axis cs:29,-9691860.75423588)
--(axis cs:30,-10218377.0003186)
--(axis cs:31,-9508590.36013648)
--(axis cs:32,-9857440.87546588)
--(axis cs:33,-9074201.31928582)
--(axis cs:34,-10317932.3402869)
--(axis cs:35,-9950043.87293975)
--(axis cs:36,-11075159.002645)
--(axis cs:37,-10174903.9501448)
--(axis cs:38,-10208930.3893856)
--(axis cs:39,-9248027.75112463)
--(axis cs:40,-8619619.34733264)
--(axis cs:41,-9614853.35479374)
--(axis cs:42,-8186434.7662968)
--(axis cs:43,-9915719.53105286)
--(axis cs:44,-12058437.8203891)
--(axis cs:45,-7871106.80845477)
--(axis cs:46,-7684395.08465428)
--(axis cs:47,-8683689.24909451)
--(axis cs:48,-7640804.68908913)
--(axis cs:49,-8386188.30279893)
--(axis cs:50,-6805722.15997268)
--(axis cs:51,-5590380.43944634)
--(axis cs:52,-8412910.03861637)
--(axis cs:53,-9695413.20699674)
--(axis cs:54,-10997308.6131284)
--(axis cs:55,-11806959.4750052)
--(axis cs:56,-9590038.68815796)
--(axis cs:57,-3881932.28977642)
--(axis cs:58,-2873247.41678162)
--(axis cs:59,999975.446154563)
--(axis cs:60,-2879697.2291535)
--(axis cs:61,-6746630.71858448)
--(axis cs:62,-8133593.36079635)
--(axis cs:63,-9808614.55604794)
--(axis cs:64,-8912413.63587536)
--(axis cs:65,-10595047.389887)
--(axis cs:66,-10176361.9092951)
--(axis cs:67,-10822283.2051554)
--(axis cs:68,-11819859.7472139)
--(axis cs:69,-12190168.3045937)
--(axis cs:70,-11239421.6797118)
--(axis cs:71,-10391904.972889)
--(axis cs:72,-8193895.91549887)
--(axis cs:73,-7480118.93072725)
--(axis cs:74,-6562568.40929183)
--(axis cs:75,-10190673.2652991)
--(axis cs:76,-10196818.2091364)
--(axis cs:77,-11388375.9572337)
--(axis cs:78,-10458419.1367197)
--(axis cs:79,-11476382.4226019)
--(axis cs:80,-12161082.352936)
--(axis cs:81,-10413567.1377354)
--(axis cs:82,-10181157.6861187)
--(axis cs:83,-9463024.05623508)
--(axis cs:84,-11226987.1542857)
--(axis cs:85,-8206325.73694822)
--(axis cs:86,-9750624.47650647)
--(axis cs:87,-11608829.2764832)
--(axis cs:88,-9898938.5198346)
--(axis cs:89,-11260335.1551459)
--(axis cs:90,-11675299.3615661)
--(axis cs:91,-10101100.4853032)
--(axis cs:92,-12278903.7568894)
--(axis cs:93,-12868703.5835604)
--(axis cs:94,-12989606.6385469)
--(axis cs:95,-14206425.8247409)
--(axis cs:96,-14002650.0815878)
--(axis cs:97,-12246115.1844074)
--(axis cs:98,-8961808.1432364)
--(axis cs:99,-10761121.7090604)
--(axis cs:100,-11053442.2467161)
--(axis cs:101,-12246281.4725702)
--(axis cs:102,-9879437.24328485)
--(axis cs:103,-12150198.4976553)
--(axis cs:104,-10383555.0509901)
--(axis cs:105,-366851.773162145)
--(axis cs:106,1682683.8349628)
--(axis cs:107,-7779164.36966727)
--(axis cs:108,-7923937.64003375)
--(axis cs:109,-11175532.2257438)
--(axis cs:110,-8963421.81197097)
--(axis cs:111,-11720419.2035788)
--(axis cs:112,-10939223.0135852)
--(axis cs:113,-12238728.7816918)
--(axis cs:114,-7866876.97170314)
--(axis cs:115,-1052507.0173686)
--(axis cs:116,-4289961.94679056)
--(axis cs:117,-9100466.9765065)
--(axis cs:118,-4813478.10293579)
--(axis cs:119,-8961011.58963506)
--(axis cs:120,-5252400.85605777)
--(axis cs:121,-9800714.86309974)
--(axis cs:122,-7715383.05326862)
--(axis cs:123,-5916935.19178989)
--(axis cs:124,-578362.954528984)
--(axis cs:125,593699.592696244)
--(axis cs:126,2314584.49466387)
--(axis cs:127,-2637917.72364703)
--(axis cs:128,-3266867.202149)
--(axis cs:129,316786.819850437)
--(axis cs:130,-7032436.79560497)
--(axis cs:131,3374600.8491869)
--(axis cs:132,-9156326.43965145)
--(axis cs:133,-6945960.33932743)
--(axis cs:134,-2144874.69709749)
--(axis cs:135,-2772530.35299499)
--(axis cs:136,-7732827.46535706)
--(axis cs:137,-1292919.52696036)
--(axis cs:138,-785066.349835197)
--(axis cs:139,-3445723.03501045)
--(axis cs:140,-203108.971859908)
--(axis cs:141,-1509748.63741338)
--(axis cs:142,379150.266811427)
--(axis cs:143,20709.6645837016)
--(axis cs:144,4744948.25751818)
--(axis cs:145,-3301579.0380511)
--(axis cs:146,45122.0313422345)
--(axis cs:147,1562533.69548005)
--(axis cs:148,-1698715.79518692)
--(axis cs:149,-905689.069364598)
--(axis cs:150,-4108932.19826762)
--(axis cs:151,-13079226.4851881)
--(axis cs:152,-3192418.99894056)
--(axis cs:153,2222276.85105362)
--(axis cs:154,-4791347.64706116)
--(axis cs:155,-248073.938784458)
--(axis cs:156,8837189.47384572)
--(axis cs:157,-6454322.05522016)
--(axis cs:158,-3229375.26130272)
--(axis cs:159,-6384249.25738256)
--(axis cs:160,-8708637.90086552)
--(axis cs:161,-6298587.35511497)
--(axis cs:162,-4707749.7346411)
--(axis cs:163,-3829980.91429561)
--(axis cs:164,-7337736.53589446)
--(axis cs:165,-5168010.10449969)
--(axis cs:166,745499.95747696)
--(axis cs:167,-4559099.62492303)
--(axis cs:168,-84431.084169345)
--(axis cs:169,-4303216.5154261)
--(axis cs:170,3918002.50425644)
--(axis cs:171,6985184.21120862)
--(axis cs:172,12675692.2032619)
--(axis cs:173,10618426.0757076)
--(axis cs:174,-814248.452844203)
--(axis cs:175,-2909265.61995627)
--(axis cs:176,-499015.022293478)
--(axis cs:177,-1620754.03801222)
--(axis cs:178,7351054.12826618)
--(axis cs:179,-8258343.85496712)
--(axis cs:180,-3280326.29834601)
--(axis cs:181,2131225.77778679)
--(axis cs:182,-1462677.88899643)
--(axis cs:183,3690627.24892992)
--(axis cs:184,5842298.98514461)
--(axis cs:185,16432308.6725312)
--(axis cs:186,21281139.8305038)
--(axis cs:187,-4472900.08934802)
--(axis cs:188,-2202226.28799476)
--(axis cs:189,-3350400.07992559)
--(axis cs:190,-2328484.81517838)
--(axis cs:191,6336578.61174068)
--(axis cs:192,8764073.82973653)
--(axis cs:193,25901757.9152522)
--(axis cs:194,20394524.7527422)
--(axis cs:195,2812282.90936599)
--(axis cs:196,4259804.9059528)
--(axis cs:197,1279836.27846912)
--(axis cs:198,-1252450.20391109)
--(axis cs:199,8475398.71222135)
--(axis cs:200,36825318.7224212)
--(axis cs:201,9021208.21018066)
--(axis cs:202,195298.788273076)
--(axis cs:203,435029.921350647)
--(axis cs:204,5678258.22176882)
--(axis cs:205,934244.1210443)
--(axis cs:206,7323924.95539739)
--(axis cs:207,7204040.78363095)
--(axis cs:208,12479063.4198653)
--(axis cs:209,-919939.564798279)
--(axis cs:210,5186715.38002818)
--(axis cs:211,21386553.9029942)
--(axis cs:212,-1579158.20875236)
--(axis cs:213,7108443.91654442)
--(axis cs:214,-5941755.69235916)
--(axis cs:215,-3625740.48573226)
--(axis cs:216,-2668942.69055533)
--(axis cs:217,-3838966.2631075)
--(axis cs:218,14468840.6339481)
--(axis cs:219,15672440.7025596)
--(axis cs:220,7939348.36185779)
--(axis cs:221,-3717474.32934823)
--(axis cs:222,3000447.0702942)
--(axis cs:223,12585582.5447384)
--(axis cs:224,2881721.21555357)
--(axis cs:225,3265751.28548337)
--(axis cs:226,-4541305.94211222)
--(axis cs:227,689822.42365418)
--(axis cs:228,-2261143.88942303)
--(axis cs:229,-383863.913648708)
--(axis cs:230,-4708664.91143827)
--(axis cs:231,-3142078.52519059)
--(axis cs:232,-7774746.54997696)
--(axis cs:233,-5282461.14722271)
--(axis cs:234,-7845759.78078065)
--(axis cs:235,-7582433.93274541)
--(axis cs:236,-4626678.36031134)
--(axis cs:237,-4017774.64428618)
--(axis cs:238,8733603.25581942)
--(axis cs:239,-1852390.11798437)
--(axis cs:240,15700827.0186618)
--(axis cs:241,-3003346.74903443)
--(axis cs:242,-1958493.87660743)
--(axis cs:243,-7322090.70865847)
--(axis cs:244,-5145740.85409056)
--(axis cs:245,2436020.753153)
--(axis cs:246,10920395.2591011)
--(axis cs:247,5124668.85916309)
--(axis cs:248,-4128684.09758325)
--(axis cs:249,-8816196.22128519)
--(axis cs:250,-8052250.91874349)
--(axis cs:251,-5077077.16364152)
--(axis cs:252,-6331015.13114138)
--(axis cs:253,-7982763.43616026)
--(axis cs:254,8079677.3621595)
--(axis cs:255,6919896.66031865)
--(axis cs:256,2057689.50909814)
--(axis cs:257,-224988.159488656)
--(axis cs:258,-927370.683494868)
--(axis cs:259,7303811.56633783)
--(axis cs:260,-6709166.44938457)
--(axis cs:261,-4198965.1839363)
--(axis cs:262,-5675773.8987025)
--(axis cs:263,-10747278.5740637)
--(axis cs:264,-14888096.7685547)
--(axis cs:265,-6543280.48251369)
--(axis cs:266,-10537126.7210858)
--(axis cs:267,-12836986.9649774)
--(axis cs:268,-7817245.99357252)
--(axis cs:269,-5655818.35049332)
--(axis cs:270,-3942387.54602846)
--(axis cs:271,-5472566.71012365)
--(axis cs:272,-1519103.24536341)
--(axis cs:273,-3633118.85919705)
--(axis cs:274,-1998811.21812547)
--(axis cs:275,4916528.65427052)
--(axis cs:276,-1150301.32548789)
--(axis cs:277,6658898.0258016)
--(axis cs:278,-2672208.59684595)
--(axis cs:279,-2954985.33492053)
--(axis cs:280,-471033.765561892)
--(axis cs:281,-3766971.25024104)
--(axis cs:282,-796979.105053034)
--(axis cs:283,-3500082.46696732)
--(axis cs:284,-2449309.53580593)
--(axis cs:285,-5397370.14119626)
--(axis cs:286,775122.449253762)
--(axis cs:287,-117921.479802124)
--(axis cs:288,-5808964.123311)
--(axis cs:289,-2865349.47467298)
--(axis cs:290,-3100333.76831)
--(axis cs:291,2893625.77789757)
--(axis cs:292,-2105200.55447249)
--(axis cs:293,-5986145.96593835)
--(axis cs:294,-5651700.59759589)
--(axis cs:295,-250026.212397311)
--(axis cs:296,1288371.56347809)
--(axis cs:297,3125219.5235871)
--(axis cs:298,1140581.10862351)
--(axis cs:299,518916.163653873)
--(axis cs:300,676044.219172895)
--(axis cs:301,-1605513.45558587)
--(axis cs:302,5454284.30100858)
--(axis cs:303,-572799.026433457)
--(axis cs:304,-7549702.51513581)
--(axis cs:305,-6714725.28570318)
--(axis cs:306,-1993622.99019448)
--(axis cs:307,-4447103.84677581)
--(axis cs:308,-6961875.57578766)
--(axis cs:309,-6957437.87394874)
--(axis cs:310,-4127747.68718458)
--(axis cs:311,-2645422.11963348)
--(axis cs:312,-407109.279278984)
--(axis cs:313,2437900.30792989)
--(axis cs:314,28822742.725176)
--(axis cs:315,1184600.58098856)
--(axis cs:316,-2066237.6451567)
--(axis cs:317,5878216.84809564)
--(axis cs:318,-6838110.7408238)
--(axis cs:319,-6180608.24577861)
--(axis cs:320,-4015879.7811871)
--(axis cs:321,-5092235.02825681)
--(axis cs:322,-6379543.09223432)
--(axis cs:323,-942438.842127299)
--(axis cs:324,856689.024505397)
--(axis cs:325,-1353666.58415581)
--(axis cs:326,-3720176.90352517)
--(axis cs:327,1959543.25176384)
--(axis cs:328,3864019.14557752)
--(axis cs:329,-3676171.24807593)
--(axis cs:330,-3590186.00506706)
--(axis cs:331,-32253.3700084165)
--(axis cs:332,-1616623.49092088)
--(axis cs:333,-4084865.89205346)
--(axis cs:334,-2895955.2276528)
--(axis cs:335,-1348947.35607245)
--(axis cs:336,-3466253.48573287)
--(axis cs:337,-2525303.4693416)
--(axis cs:338,-971093.557603233)
--(axis cs:339,-7099395.82810367)
--(axis cs:340,-4315499.32319484)
--(axis cs:341,62331.2809176389)
--(axis cs:342,-1806981.5735455)
--(axis cs:343,-6983769.5642515)
--(axis cs:344,-5724185.45903565)
--(axis cs:345,-6333634.46210776)
--(axis cs:346,-7603192.57254876)
--(axis cs:347,-3978146.26133446)
--(axis cs:348,4422665.37988386)
--(axis cs:349,-6565592.91870088)
--(axis cs:350,643051.459906681)
--(axis cs:351,3558832.00327458)
--(axis cs:352,2541305.64126211)
--(axis cs:353,-5214143.26673375)
--(axis cs:354,-2684596.3567352)
--(axis cs:355,-3807671.42530994)
--(axis cs:356,-339871.723502874)
--(axis cs:357,-5842403.1332742)
--(axis cs:358,-6136824.94669376)
--(axis cs:359,-4013118.53470249)
--(axis cs:360,-5766348.3628489)
--(axis cs:361,-3616617.3141033)
--(axis cs:362,-3759699.71121432)
--(axis cs:363,-6910818.05283808)
--(axis cs:364,-8880245.43715982)
--(axis cs:365,-4681542.96467946)
--(axis cs:366,-9369520.50944607)
--(axis cs:367,-4634835.48882621)
--(axis cs:368,-4339646.66726598)
--(axis cs:369,-6917627.07580572)
--(axis cs:370,-6979306.96651278)
--(axis cs:371,-7252840.70150443)
--(axis cs:372,-7955050.53245939)
--(axis cs:373,-6347402.33488897)
--(axis cs:374,-5555166.42674215)
--(axis cs:375,-2428721.77877922)
--(axis cs:376,1592489.14203584)
--(axis cs:377,-2142017.05971332)
--(axis cs:378,-7367561.03988581)
--(axis cs:379,-9900628.49364917)
--(axis cs:380,-11564176.4938226)
--(axis cs:381,-8924371.42075885)
--(axis cs:382,-6827682.90673274)
--(axis cs:383,-7438517.9373647)
--(axis cs:384,-6287035.23299627)
--(axis cs:385,-5966732.6596797)
--(axis cs:386,-5675633.57390694)
--(axis cs:387,-6727392.4676682)
--(axis cs:388,-6477449.466322)
--(axis cs:389,-6628169.29796252)
--(axis cs:390,-5203233.91729769)
--(axis cs:391,-8917686.83899321)
--(axis cs:392,-7918904.98372306)
--(axis cs:393,-9040797.2855427)
--(axis cs:394,-7688559.45463303)
--(axis cs:395,-9209191.2304942)
--(axis cs:396,-8958116.55621583)
--(axis cs:397,-9944916.01480553)
--(axis cs:398,-9179177.91944674)
--(axis cs:399,-8016894.61824447)
--(axis cs:400,-9727060.09788186)
--(axis cs:401,-9733658.99045403)
--(axis cs:402,-10792326.8587067)
--(axis cs:403,-9879890.6213648)
--(axis cs:404,-7081582.96955026)
--(axis cs:405,-7985604.0927485)
--(axis cs:406,-5700945.79484526)
--(axis cs:407,-8744974.86359898)
--(axis cs:408,-9913388.17994212)
--(axis cs:409,-9745405.08527431)
--(axis cs:410,-7312426.12676026)
--(axis cs:411,-7656846.79881373)
--(axis cs:412,-11131917.7448056)
--(axis cs:413,-9045677.80445304)
--(axis cs:414,-9448598.83601603)
--(axis cs:415,-7127846.21555243)
--(axis cs:416,-7054250.62536332)
--(axis cs:417,-10186030.2502215)
--(axis cs:418,-10530531.4496492)
--(axis cs:419,-12068525.8750916)
--(axis cs:420,-11804300.242239)
--(axis cs:421,-10854162.5215055)
--(axis cs:422,-10045247.4160132)
--(axis cs:423,-8869189.03546282)
--(axis cs:424,-10393591.1961976)
--(axis cs:425,-11178616.6971534)
--(axis cs:426,-10217213.559802)
--(axis cs:427,-11004154.3647045)
--(axis cs:428,-11694457.6975439)
--(axis cs:429,-10188609.9338197)
--(axis cs:430,-11077471.8828855)
--(axis cs:431,-7823382.36761869)
--(axis cs:432,-10426773.2511418)
--(axis cs:433,-10949175.1493761)
--(axis cs:434,-10452084.9037486)
--(axis cs:435,-11304130.2051657)
--(axis cs:436,-9560220.10306126)
--(axis cs:437,-9372052.86961551)
--(axis cs:438,-10830191.1953708)
--(axis cs:439,-11380050.1642548)
--(axis cs:440,-10675003.7696071)
--(axis cs:440,19269425.9866216)
--(axis cs:440,19269425.9866216)
--(axis cs:439,19640647.2693746)
--(axis cs:438,19020723.6416415)
--(axis cs:437,20639957.2108487)
--(axis cs:436,20335850.1715696)
--(axis cs:435,18494749.5862448)
--(axis cs:434,19331435.1674132)
--(axis cs:433,18806153.5582905)
--(axis cs:432,19321588.2045393)
--(axis cs:431,21962758.3859266)
--(axis cs:430,18737792.00468)
--(axis cs:429,19590998.7648917)
--(axis cs:428,18796852.7348172)
--(axis cs:427,18924937.8609915)
--(axis cs:426,19701731.6103114)
--(axis cs:425,18976156.7492356)
--(axis cs:424,19625265.6589473)
--(axis cs:423,21033271.999057)
--(axis cs:422,19864183.5279189)
--(axis cs:421,19297754.0258822)
--(axis cs:420,18486836.8761797)
--(axis cs:419,18010386.9957854)
--(axis cs:418,19605017.3381678)
--(axis cs:417,19938124.154487)
--(axis cs:416,22818711.9023564)
--(axis cs:415,22865368.0636351)
--(axis cs:414,20314695.1352992)
--(axis cs:413,20786963.2716108)
--(axis cs:412,18844625.0233313)
--(axis cs:411,22285915.021438)
--(axis cs:410,22545371.4460086)
--(axis cs:409,20282684.9283811)
--(axis cs:408,19837605.1452264)
--(axis cs:407,21073040.1654105)
--(axis cs:406,24143090.8492918)
--(axis cs:405,22031674.317949)
--(axis cs:404,22933365.1859467)
--(axis cs:403,20629032.8299722)
--(axis cs:402,19068772.2620246)
--(axis cs:401,20090180.6126877)
--(axis cs:400,20076308.1193671)
--(axis cs:399,21776675.7817981)
--(axis cs:398,20659158.1984162)
--(axis cs:397,19994713.4402426)
--(axis cs:396,21627355.9930479)
--(axis cs:395,21267056.8130254)
--(axis cs:394,22375090.1200801)
--(axis cs:393,20860582.8612837)
--(axis cs:392,21943204.3718551)
--(axis cs:391,20864468.0923135)
--(axis cs:390,24767274.2028565)
--(axis cs:389,23138626.6503239)
--(axis cs:388,23337419.0750379)
--(axis cs:387,23242643.421837)
--(axis cs:386,24160192.4090843)
--(axis cs:385,23869879.8195718)
--(axis cs:384,23735308.0349425)
--(axis cs:383,22641086.1736571)
--(axis cs:382,23289026.7615771)
--(axis cs:381,21092565.3437877)
--(axis cs:380,19739620.6703748)
--(axis cs:379,22038847.5769604)
--(axis cs:378,24460616.4125181)
--(axis cs:377,29179198.4091185)
--(axis cs:376,32685171.734137)
--(axis cs:375,27523658.1803533)
--(axis cs:374,24405625.7128298)
--(axis cs:373,23596860.378555)
--(axis cs:372,22093912.7632429)
--(axis cs:371,22822856.4234762)
--(axis cs:370,22828571.676611)
--(axis cs:369,22994587.4215753)
--(axis cs:368,25617930.7091296)
--(axis cs:367,25536670.6597363)
--(axis cs:366,22110841.9480306)
--(axis cs:365,25231117.1398694)
--(axis cs:364,21394499.3145963)
--(axis cs:363,23147919.8890693)
--(axis cs:362,27094361.9211554)
--(axis cs:361,26440493.2345099)
--(axis cs:360,24696215.6579044)
--(axis cs:359,26054020.9083346)
--(axis cs:358,24423935.3636656)
--(axis cs:357,25426024.2752054)
--(axis cs:356,29800757.7200575)
--(axis cs:355,26196266.3364167)
--(axis cs:354,27425956.0321598)
--(axis cs:353,25941003.6730595)
--(axis cs:352,33102025.3031111)
--(axis cs:351,33702948.4861921)
--(axis cs:350,31153767.8958894)
--(axis cs:349,24258232.6093165)
--(axis cs:348,34959542.9492213)
--(axis cs:347,26093936.7295133)
--(axis cs:346,22273672.9971589)
--(axis cs:345,23462700.7714909)
--(axis cs:344,24356343.5198528)
--(axis cs:343,23217328.7678497)
--(axis cs:342,28814516.3733742)
--(axis cs:341,30558253.7436295)
--(axis cs:340,25564664.3856557)
--(axis cs:339,22872123.5148956)
--(axis cs:338,29328759.5228704)
--(axis cs:337,27777727.1559723)
--(axis cs:336,28810332.4553675)
--(axis cs:335,28597954.5398856)
--(axis cs:334,27132910.4612543)
--(axis cs:333,26372968.1079824)
--(axis cs:332,28623613.0526695)
--(axis cs:331,30149163.9909585)
--(axis cs:330,26484238.963219)
--(axis cs:329,28222462.1412297)
--(axis cs:328,37712512.6380888)
--(axis cs:327,32291999.2118638)
--(axis cs:326,26485316.5664193)
--(axis cs:325,28854738.4941449)
--(axis cs:324,31801580.931825)
--(axis cs:323,29666976.2759318)
--(axis cs:322,24551786.4359335)
--(axis cs:321,25489170.0476276)
--(axis cs:320,26198136.3222077)
--(axis cs:319,23768798.0380697)
--(axis cs:318,23340056.6480787)
--(axis cs:317,36087476.0269351)
--(axis cs:316,28415596.3126493)
--(axis cs:315,34139567.6807584)
--(axis cs:314,60863838.3730434)
--(axis cs:313,33366257.4687737)
--(axis cs:312,29513458.4266879)
--(axis cs:311,27355663.4758028)
--(axis cs:310,25820949.9027381)
--(axis cs:309,23027934.6790122)
--(axis cs:308,23034383.5956228)
--(axis cs:307,25804698.4832121)
--(axis cs:306,27946791.5099236)
--(axis cs:305,23380529.9797668)
--(axis cs:304,23539078.1827214)
--(axis cs:303,30645727.9146455)
--(axis cs:302,35497595.6395359)
--(axis cs:301,28680658.4510061)
--(axis cs:300,31065619.5016689)
--(axis cs:299,30972133.9664309)
--(axis cs:298,31884585.8371734)
--(axis cs:297,34371213.3265747)
--(axis cs:296,31214388.7081396)
--(axis cs:295,29621514.9098342)
--(axis cs:294,24586386.6182542)
--(axis cs:293,25245090.0768507)
--(axis cs:292,28683499.901802)
--(axis cs:291,33347791.0430751)
--(axis cs:290,26928491.4321136)
--(axis cs:289,28410915.2395854)
--(axis cs:288,24202614.951331)
--(axis cs:287,30239234.1086275)
--(axis cs:286,31060097.8930927)
--(axis cs:285,24964239.1570495)
--(axis cs:284,28818711.79781)
--(axis cs:283,27117228.7172924)
--(axis cs:282,29569322.3469263)
--(axis cs:281,26525478.7476235)
--(axis cs:280,30021073.5912945)
--(axis cs:279,28112171.8573654)
--(axis cs:278,29338456.1984213)
--(axis cs:277,37195866.5460679)
--(axis cs:276,29222125.9950931)
--(axis cs:275,35252237.8132786)
--(axis cs:274,28642486.7615062)
--(axis cs:273,27104403.0099693)
--(axis cs:272,29208046.46708)
--(axis cs:271,27740715.2787662)
--(axis cs:270,26461553.5305262)
--(axis cs:269,24588390.1586955)
--(axis cs:268,22698203.8038261)
--(axis cs:267,17396306.8400042)
--(axis cs:266,19799969.3186595)
--(axis cs:265,23954282.8992644)
--(axis cs:264,16066991.6593592)
--(axis cs:263,19755390.250983)
--(axis cs:262,24791122.0648047)
--(axis cs:261,25972497.138684)
--(axis cs:260,25804846.0633435)
--(axis cs:259,37800480.2073878)
--(axis cs:258,29290885.9196329)
--(axis cs:257,30366227.4692394)
--(axis cs:256,32574565.9064221)
--(axis cs:255,37555802.3679874)
--(axis cs:254,39234226.8088434)
--(axis cs:253,24623182.9131283)
--(axis cs:252,26034806.0361543)
--(axis cs:251,26618422.2944383)
--(axis cs:250,22910499.7189953)
--(axis cs:249,22124243.4499929)
--(axis cs:248,28050519.8081102)
--(axis cs:247,39070036.2228537)
--(axis cs:246,41440270.7029597)
--(axis cs:245,32850043.6753078)
--(axis cs:244,25708014.0710282)
--(axis cs:243,23413756.7006655)
--(axis cs:242,28975712.7762289)
--(axis cs:241,30683941.561438)
--(axis cs:240,46259080.2631186)
--(axis cs:239,31496419.2008197)
--(axis cs:238,40174475.8829105)
--(axis cs:237,26171851.850441)
--(axis cs:236,25649024.6792549)
--(axis cs:235,22429623.0555503)
--(axis cs:234,22151997.690262)
--(axis cs:233,24615247.0072366)
--(axis cs:232,23223621.7653129)
--(axis cs:231,27071048.7250858)
--(axis cs:230,25423241.1624265)
--(axis cs:229,29880359.325936)
--(axis cs:228,27798465.4342151)
--(axis cs:227,30822339.7019223)
--(axis cs:226,26869858.9957163)
--(axis cs:225,33563535.0229065)
--(axis cs:224,35807777.4929778)
--(axis cs:223,42841456.8329565)
--(axis cs:222,34389869.6706888)
--(axis cs:221,29077177.9436712)
--(axis cs:220,41615414.1209878)
--(axis cs:219,47024780.6863132)
--(axis cs:218,45240399.5765361)
--(axis cs:217,26956051.066773)
--(axis cs:216,27944202.9746584)
--(axis cs:215,28588851.4750366)
--(axis cs:214,26400568.4765216)
--(axis cs:213,39034024.8959944)
--(axis cs:212,30956936.6976745)
--(axis cs:211,59561941.7908069)
--(axis cs:210,35896700.2708636)
--(axis cs:209,31790822.2941905)
--(axis cs:208,43942722.113551)
--(axis cs:207,37866539.9953383)
--(axis cs:206,38637210.6060093)
--(axis cs:205,32147143.8962265)
--(axis cs:204,36049708.5101149)
--(axis cs:203,31238223.0739249)
--(axis cs:202,33690031.0910221)
--(axis cs:201,42254320.3978042)
--(axis cs:200,69681832.0603227)
--(axis cs:199,40459630.5989072)
--(axis cs:198,29194091.5553931)
--(axis cs:197,32554661.9658286)
--(axis cs:196,35048740.8634254)
--(axis cs:195,35778857.7099913)
--(axis cs:194,53623946.9098459)
--(axis cs:193,56965672.0446213)
--(axis cs:192,39459513.125689)
--(axis cs:191,36702779.964499)
--(axis cs:190,27684794.8987143)
--(axis cs:189,26655822.9834415)
--(axis cs:188,28257644.1390571)
--(axis cs:187,29837530.6907703)
--(axis cs:186,54354530.3444295)
--(axis cs:185,48137229.6378947)
--(axis cs:184,37183067.4513279)
--(axis cs:183,33715366.2172923)
--(axis cs:182,29438337.936182)
--(axis cs:181,32464968.5799909)
--(axis cs:180,27265203.1957237)
--(axis cs:179,23920719.0678456)
--(axis cs:178,38431665.4868771)
--(axis cs:177,29278138.0427443)
--(axis cs:176,29986401.279054)
--(axis cs:175,27743578.5046792)
--(axis cs:174,30450307.7608892)
--(axis cs:173,43194608.107606)
--(axis cs:172,44664902.7554429)
--(axis cs:171,38157155.3617688)
--(axis cs:170,34346822.3520057)
--(axis cs:169,28050076.916275)
--(axis cs:168,30505794.2879546)
--(axis cs:167,25787863.0257074)
--(axis cs:166,30858385.5505781)
--(axis cs:165,24930015.6004494)
--(axis cs:164,22683276.2829642)
--(axis cs:163,26336136.6844731)
--(axis cs:162,26414245.1418528)
--(axis cs:161,23718828.2832113)
--(axis cs:160,21357599.358753)
--(axis cs:159,23562431.4407949)
--(axis cs:158,26939491.9777115)
--(axis cs:157,24609634.8091765)
--(axis cs:156,40474962.032026)
--(axis cs:155,29799652.454181)
--(axis cs:154,26407867.7423083)
--(axis cs:153,33036009.9893627)
--(axis cs:152,26954942.3494751)
--(axis cs:151,19875702.687444)
--(axis cs:150,26337795.1928648)
--(axis cs:149,29129267.012863)
--(axis cs:148,29633608.0840516)
--(axis cs:147,31473199.0275476)
--(axis cs:146,30282359.8630144)
--(axis cs:145,28264764.0445535)
--(axis cs:144,35293404.2318773)
--(axis cs:143,30028845.5890718)
--(axis cs:142,30530170.1826362)
--(axis cs:141,28586905.3030635)
--(axis cs:140,30477218.2223347)
--(axis cs:139,26978726.7507574)
--(axis cs:138,29717794.4357272)
--(axis cs:137,28774903.1597303)
--(axis cs:136,22297225.4582833)
--(axis cs:135,27578205.1716035)
--(axis cs:134,28318263.8842657)
--(axis cs:133,23239059.0042328)
--(axis cs:132,24599392.0679622)
--(axis cs:131,33751259.137669)
--(axis cs:130,23606544.2925807)
--(axis cs:129,30760848.5866496)
--(axis cs:128,27836049.9438673)
--(axis cs:127,28000257.0609644)
--(axis cs:126,34138121.0141403)
--(axis cs:125,31741096.9488111)
--(axis cs:124,29723786.0086987)
--(axis cs:123,25290957.1597012)
--(axis cs:122,22328731.4137595)
--(axis cs:121,20710143.34285)
--(axis cs:120,25819673.9663331)
--(axis cs:119,21097692.6045773)
--(axis cs:118,25228872.4266819)
--(axis cs:117,20792934.8938373)
--(axis cs:116,26478280.2006977)
--(axis cs:115,29075195.7027124)
--(axis cs:114,22055813.6310329)
--(axis cs:113,18393717.0254932)
--(axis cs:112,19114063.1724384)
--(axis cs:111,19577103.2045981)
--(axis cs:110,21218181.1188125)
--(axis cs:109,18929353.4185457)
--(axis cs:108,22042108.0200946)
--(axis cs:107,22859724.5604021)
--(axis cs:106,33278352.6834227)
--(axis cs:105,30245154.4428577)
--(axis cs:104,19423553.8682815)
--(axis cs:103,17768325.5590965)
--(axis cs:102,20164448.0430835)
--(axis cs:101,17647914.6207141)
--(axis cs:100,18944133.8403092)
--(axis cs:99,19165750.9856944)
--(axis cs:98,20974068.7558441)
--(axis cs:97,17630353.2494886)
--(axis cs:96,16104966.4232986)
--(axis cs:95,15839846.2715459)
--(axis cs:94,16897139.3979525)
--(axis cs:93,17026731.6626238)
--(axis cs:92,17689065.8713372)
--(axis cs:91,19792325.7248864)
--(axis cs:90,18327814.858991)
--(axis cs:89,18656719.4848975)
--(axis cs:88,19998077.2931614)
--(axis cs:87,18238201.5553992)
--(axis cs:86,20191434.886759)
--(axis cs:85,22453309.6165387)
--(axis cs:84,18800506.8532522)
--(axis cs:83,20381069.3561732)
--(axis cs:82,19721136.589673)
--(axis cs:81,19465282.1341626)
--(axis cs:80,17826147.803433)
--(axis cs:79,18400177.1220038)
--(axis cs:78,19343669.466588)
--(axis cs:77,18451222.6737694)
--(axis cs:76,19567594.7133333)
--(axis cs:75,19763529.2343535)
--(axis cs:74,23391787.4287728)
--(axis cs:73,22448232.5986782)
--(axis cs:72,22013188.6809488)
--(axis cs:71,19362252.3726255)
--(axis cs:70,19401354.2643512)
--(axis cs:69,17699739.5403701)
--(axis cs:68,17959493.5853639)
--(axis cs:67,19103861.3243429)
--(axis cs:66,19767251.5204452)
--(axis cs:65,19292319.0685236)
--(axis cs:64,20902683.4677022)
--(axis cs:63,20077393.1908966)
--(axis cs:62,21661919.9850012)
--(axis cs:61,23383391.9241443)
--(axis cs:60,27770806.2505149)
--(axis cs:59,31731351.2805811)
--(axis cs:58,27055950.7418718)
--(axis cs:57,26174852.42186)
--(axis cs:56,20586350.3810005)
--(axis cs:55,18190146.9387878)
--(axis cs:54,19171699.5483401)
--(axis cs:53,20535731.6871493)
--(axis cs:52,22158113.0477397)
--(axis cs:51,24506487.0540148)
--(axis cs:50,23222419.0863687)
--(axis cs:49,21653575.7628479)
--(axis cs:48,22326651.7200357)
--(axis cs:47,21183672.0007156)
--(axis cs:46,22233192.5736002)
--(axis cs:45,22088158.5384926)
--(axis cs:44,19480935.0467992)
--(axis cs:43,20579175.6859623)
--(axis cs:42,21963012.2324751)
--(axis cs:41,20153795.6548025)
--(axis cs:40,21947713.1863646)
--(axis cs:39,20519409.6062547)
--(axis cs:38,19579310.7835012)
--(axis cs:37,19686438.3265004)
--(axis cs:36,18881799.996467)
--(axis cs:35,19877241.1948065)
--(axis cs:34,19531546.293232)
--(axis cs:33,21059611.7424309)
--(axis cs:32,19972850.7981965)
--(axis cs:31,20323129.4298751)
--(axis cs:30,19706721.2221154)
--(axis cs:29,20258316.1972449)
--(axis cs:28,19562252.2555616)
--(axis cs:27,17572093.7839409)
--(axis cs:26,20645004.1439194)
--(axis cs:25,21146171.9279127)
--(axis cs:24,22268206.7737548)
--(axis cs:23,22150019.2535872)
--(axis cs:22,20421757.3667733)
--(axis cs:21,21009930.3828662)
--(axis cs:20,23537302.4460774)
--(axis cs:19,24163399.8554157)
--(axis cs:18,24095814.1359468)
--(axis cs:17,22073839.4432389)
--(axis cs:16,22581102.3097662)
--(axis cs:15,24024950.9419489)
--(axis cs:14,24325335.6721734)
--(axis cs:13,25746068.2520929)
--(axis cs:12,22434351.5923761)
--(axis cs:11,24566511.446166)
--(axis cs:10,26068500.8330347)
--(axis cs:9,23899564.4726533)
--(axis cs:8,22806903.39768)
--(axis cs:7,25040625.0923553)
--(axis cs:6,25345124.7440987)
--(axis cs:5,27166416.1786298)
--(axis cs:4,27464872.7561989)
--(axis cs:3,38738461.5213158)
--(axis cs:2,27359656.8054173)
--(axis cs:1,31938252.4538893)
--(axis cs:0,31925052.2091835)
--cycle;

\addplot [only marks, red, mark=asterisk, mark size=1.2]
table {%
0 29755500
1 20228000
2 8513700
3 9502900
4 8909200
5 7004800
6 6305500
7 6348400
8 7321100
9 6421300
10 6164800
11 5360500
12 5074700
13 5074700
14 6804600
15 9612000
16 9154100
17 5291400
18 5456200
19 5819200
20 12820200
21 5305900
22 5067000
23 11001200
24 7051300
25 7450000
26 9014400
27 5198200
28 4916300
29 5534800
30 4719000
31 4596600
32 6568000
33 6431200
34 5034500
35 6933000
36 4533400
37 4446000
38 5301600
39 4517400
40 4370200
41 6208600
42 7137600
43 4905200
44 5415600
45 4461400
46 4170200
47 4549000
48 5623500
49 4354800
50 6282500
51 6528100
52 4165200
53 3940800
54 3894500
55 4728400
56 12964300
57 7101500
58 9158000
59 10692100
60 4636300
61 5787600
62 5199600
63 4093600
64 3925700
65 3818200
66 3756600
67 3702000
68 3632400
69 3676500
70 3599700
71 4028000
72 6895400
73 4511900
74 5519200
75 3725000
76 3495900
77 3418000
78 3308300
79 3208300
80 3241700
81 7029700
82 5106300
83 4259800
84 11266900
85 7029700
86 3754100
87 3833100
88 5472000
89 6634600
90 3756400
91 3932500
92 3587100
93 3677400
94 3490000
95 3271000
96 3169400
97 3754100
98 3754100
99 3289100
100 3907600
101 3268800
102 3747400
103 4939300
104 5061200
105 34188900
106 27994500
107 6776800
108 5061200
109 4065600
110 3948300
111 5492400
112 6815200
113 4162700
114 10287200
115 19116000
116 5162800
117 5390800
118 9111000
119 8332200
120 6176400
121 6007000
122 5449500
123 9217100
124 26947100
125 23633100
126 7943900
127 9619000
128 7515000
129 7323100
130 6271200
131 10975700
132 5934800
133 13255700
134 8424800
135 5537500
136 20382400
137 15957800
138 9571500
139 11587400
140 5961900
141 7573700
142 7627900
143 34868400
144 19389200
145 7959700
146 5842300
147 22768500
148 7232800
149 7546600
150 16027800
151 5833200
152 5336600
153 7476600
154 7512800
155 6573700
156 4891900
157 5083800
158 4998000
159 5007000
160 5165000
161 24100400
162 19050500
163 6097300
164 8512800
165 14923900
166 9695700
167 9673100
168 13014100
169 9045600
170 15720800
171 25333000
172 21901700
173 29845600
174 12422700
175 11126900
176 29179700
177 12285000
178 31793800
179 7835600
180 7106400
181 9212600
182 14449900
183 22425400
184 24443600
185 36209300
186 23874700
187 11655200
188 11354900
189 8605400
190 31960800
191 32662900
192 23240300
193 113027800
194 30163900
195 17425200
196 9767900
197 13793000
198 9176500
199 48431100
200 56194500
201 19000900
202 34642700
203 14820100
204 11009500
205 16849500
206 26134400
207 33843500
208 32130100
209 16087400
210 17438700
211 14080400
212 15384500
213 40475900
214 18610300
215 11987000
216 10382000
217 12513000
218 23102600
219 67402600
220 16682500
221 15589900
222 22150000
223 64014200
224 18127200
225 11294600
226 10635400
227 18295300
228 23097100
229 8678100
230 8156300
231 14513900
232 7707800
233 7069100
234 6833500
235 6511900
236 6344300
237 7089500
238 23791600
239 10674900
240 33965900
241 8903700
242 8915100
243 6704400
244 6631900
245 30353300
246 41741700
247 41295500
248 7716900
249 7114400
250 6434900
251 6253700
252 6063400
253 6285400
254 14942200
255 8441700
256 15918400
257 6738400
258 6738400
259 9590000
260 6439400
261 6058900
262 5855000
263 5691900
264 5549300
265 5404300
266 5284200
267 5207200
268 5669300
269 5574200
270 7710100
271 5907100
272 10505100
273 12550400
274 26328400
275 21633000
276 8083800
277 18942200
278 7873100
279 8684000
280 8011300
281 6641000
282 6031700
283 34190200
284 19821000
285 31746200
286 18552600
287 7839200
288 12471100
289 8468800
290 22822100
291 17691900
292 8228700
293 7429200
294 9578700
295 11988600
296 19771200
297 13257000
298 12799500
299 26049800
300 8720300
301 18244600
302 16595700
303 8638700
304 7465400
305 14095100
306 23268300
307 8117800
308 7275200
309 6851600
310 8199300
311 6978500
312 21503900
313 50912700
314 44235500
315 13080400
316 10482400
317 7288800
318 7094000
319 7544700
320 9150600
321 8079300
322 7254800
323 23109800
324 19705500
325 9900300
326 8115500
327 12027200
328 11605900
329 8681700
330 8289900
331 9282000
332 9334100
333 6591200
334 9687400
335 12131300
336 7669300
337 22799500
338 11479000
339 6181200
340 10036200
341 8566200
342 6348800
343 8690800
344 6029400
345 5472200
346 5182300
347 25941000
348 22471100
349 6924100
350 10926400
351 11340900
352 11861800
353 10104200
354 10158500
355 11315900
356 29934200
357 8493800
358 7152900
359 22946700
360 11884500
361 9943400
362 10604700
363 7551500
364 6679500
365 10824400
366 8539100
367 10095100
368 6169900
369 6054300
370 5782500
371 5451900
372 7485800
373 8675000
374 5766700
375 7368000
376 17732700
377 15408800
378 14600200
379 16423500
380 10514100
381 6847100
382 11556000
383 14274000
384 6534500
385 6545900
386 7393000
387 12083800
388 6767800
389 6280800
390 5789300
391 5710100
392 5662500
393 5395200
394 5094000
395 4915100
396 4906000
397 5605900
398 6364700
399 8364600
400 5513000
401 4910500
402 4724800
403 4636500
404 4645500
405 4625100
406 4575300
407 4518700
408 6396400
409 5037400
410 4656800
411 5141600
412 4847100
413 4466600
414 4971700
415 7485800
416 4645500
417 4308000
418 4185700
419 4208400
420 9963700
421 4541300
422 4280900
423 4249100
424 4262700
425 4792700
426 4940000
427 12004500
428 4865200
429 4317100
430 4405400
431 4178900
432 4077000
433 3970500
434 3990900
435 4215200
436 3959200
437 3877700
438 4516400
439 4643300
440 3959200
};
\addplot [blue]
table {%
0 16741617.6306149
1 14794537.0119485
2 11116736.5978385
3 22881398.6587559
4 10862853.9559898
5 11756079.2040898
6 9912793.75168397
7 9637317.02289481
8 7436075.44778281
9 8693757.37176492
10 11038076.7494436
11 9625152.94263056
12 7273057.9232693
13 10710860.1203536
14 9252276.91549008
15 9011720.17760852
16 5748325.36810795
17 6784229.486122
18 8889025.07871816
19 9059702.92637735
20 8142173.10594504
21 5695628.49625022
22 5456453.67852928
23 7141127.80296532
24 7290739.26326678
25 6100779.24241973
26 5720861.15166478
27 2559060.53177878
28 4505718.9687531
29 5283227.72150452
30 4744172.11089837
31 5407269.5348693
32 5057704.96136532
33 5992705.21157252
34 4606806.97647257
35 4963598.66093337
36 3903320.49691103
37 4755767.18817779
38 4685190.19705777
39 5635690.92756503
40 6664046.91951597
41 5269471.15000439
42 6888288.73308916
43 5331728.07745474
44 3711248.61320507
45 7108525.86501894
46 7274398.74447294
47 6249991.37581053
48 7342923.5154733
49 6633693.73002447
50 8208348.463198
51 9458053.30728421
52 6872601.50456166
53 5420159.24007628
54 4087195.46760585
55 3191593.73189132
56 5498155.84642128
57 11146460.0660418
58 12091351.6625451
59 16365663.3633678
60 12445554.5106807
61 8318380.60277989
62 6764163.31210244
63 5134389.31742434
64 5995134.91591342
65 4348635.8393183
66 4795444.80557506
67 4140789.05959375
68 3069816.91907504
69 2754785.6178882
70 4080966.29231971
71 4485173.69986827
72 6909646.38272495
73 7484056.8339755
74 8414609.50974051
75 4786427.98452724
76 4685388.25209842
77 3531423.35826787
78 4442625.16493415
79 3461897.34970092
80 2832532.72524851
81 4525857.49821363
82 4769989.45177718
83 5459022.64996904
84 3786759.84948323
85 7123491.93979526
86 5220405.20512629
87 3314686.13945798
88 5049569.38666342
89 3698192.1648758
90 3326257.74871249
91 4845612.61979155
92 2705081.0572239
93 2079014.0395317
94 1953766.37970277
95 816710.223402533
96 1051158.17085538
97 2692119.03254059
98 6006130.30630387
99 4202314.63831697
100 3945345.79679654
101 2700816.57407194
102 5142505.3998993
103 2809063.53072061
104 4519999.40864573
105 14939151.3348478
106 17480518.2591927
107 7540280.09536742
108 7059085.1900304
109 3876910.59640093
110 6127379.65342075
111 3928342.00050963
112 4087420.07942659
113 3077494.12190068
114 7094468.32966486
115 14011344.3426719
116 11094159.1269536
117 5846233.95866538
118 10207697.1618731
119 6068340.5074711
120 10283636.5551377
121 5454714.23987512
122 7306674.18024546
123 9687010.98395565
124 14572711.5270849
125 16167398.2707537
126 18226352.7544021
127 12681169.6686587
128 12284591.3708591
129 15538817.70325
130 8287053.74848787
131 18562929.993428
132 7721532.81415535
133 8146549.33245267
134 13086694.5935841
135 12402837.4093042
136 7282198.99646311
137 13740991.816385
138 14466364.042946
139 11766501.8578735
140 15137054.6252374
141 13538578.3328251
142 15454660.2247238
143 15024777.6268277
144 20019176.2446977
145 12481592.5032512
146 15163740.9471783
147 16517866.3615138
148 13967446.1444323
149 14111788.9717492
150 11114431.4972986
151 3398238.10112796
152 11881261.6752673
153 17629143.4202082
154 10808260.0476236
155 14775789.2576983
156 24656075.7529358
157 9077656.37697818
158 11855058.3582044
159 8589091.09170616
160 6324480.72894372
161 8710120.46404816
162 10853247.7036058
163 11253077.8850887
164 7672769.87353489
165 9881002.74797484
166 15801942.7540276
167 10614381.7003922
168 15210681.6018926
169 11873430.2004245
170 19132412.4281311
171 22571169.7864887
172 28670297.4793524
173 26906517.0916568
174 14818029.6540225
175 12417156.4423615
176 14743693.1283802
177 13828692.002366
178 22891359.8075716
179 7831187.60643924
180 11992438.4486888
181 17298097.1788889
182 13987830.0235928
183 18702996.7331111
184 21512683.2182362
185 32284769.1552129
186 37817835.0874667
187 12682315.3007111
188 13027708.9255312
189 11652711.451758
190 12678155.041768
191 21519679.2881199
192 24111793.4777128
193 41433714.9799368
194 37009235.831294
195 19295570.3096787
196 19654272.8846891
197 16917249.1221489
198 13970820.675741
199 24467514.6555643
200 53253575.391372
201 25637764.3039924
202 16942664.9396476
203 15836626.4976378
204 20863983.3659418
205 16540694.0086354
206 22980567.7807034
207 22535290.3894846
208 28210892.7667081
209 15435441.3646961
210 20541707.8254459
211 40474247.8469006
212 14688889.2444611
213 23071234.4062694
214 10229406.3920812
215 12481555.4946522
216 12637630.1420516
217 11558542.4018327
218 29854620.1052421
219 31348610.6944364
220 24777381.2414228
221 12679851.8071615
222 18695158.3704915
223 27713519.6888474
224 19344749.3542657
225 18414643.1541949
226 11164276.526802
227 15756081.0627882
228 12768660.772396
229 14748247.7061436
230 10357288.1254941
231 11964485.0999476
232 7724437.60766799
233 9666392.93000696
234 7153118.95474069
235 7423594.56140242
236 10511173.1594718
237 11077038.6030774
238 24454039.569365
239 14822014.5414177
240 30979953.6408902
241 13840297.4062018
242 13508609.4498107
243 8045832.99600352
244 10281136.6084688
245 17643032.2142304
246 26180332.9810304
247 22097352.5410084
248 11960917.8552635
249 6654023.61435388
250 7429124.40012589
251 10770672.5653984
252 9851895.45250647
253 8320209.738484
254 23656952.0855015
255 22237849.514153
256 17316127.7077601
257 15070619.6548754
258 14181757.618069
259 22552145.8868628
260 9547839.80697949
261 10886765.9773739
262 9557674.08305113
263 4504055.83845966
264 589447.445402256
265 8705501.20837534
266 4631421.29878683
267 2279659.93751341
268 7440478.9051268
269 9466285.90410108
270 11259582.9922489
271 11134074.2843213
272 13844471.6108583
273 11735642.0753861
274 13321837.7716904
275 20084383.2337746
276 14035912.3348026
277 21927382.2859348
278 13333123.8007877
279 12578593.2612224
280 14775019.9128663
281 11379253.7486912
282 14386171.6209366
283 11808573.1251625
284 13184701.131002
285 9783434.50792663
286 15917610.1711732
287 15060656.3144127
288 9196825.41401001
289 12772782.8824562
290 11914078.8319018
291 18120708.4104864
292 13289149.6736648
293 9629472.0554562
294 9467343.01032915
295 14685744.3487184
296 16251380.1358089
297 18748216.4250809
298 16512583.4728985
299 15745525.0650424
300 15870831.8604209
301 13537572.4977101
302 20475939.9702723
303 15036464.444106
304 7994687.83379279
305 8332902.34703181
306 12976584.2598646
307 10678797.3182182
308 8036254.00991759
309 8035248.40253174
310 10846601.1077768
311 12355120.6780847
312 14553174.5737045
313 17902078.8883518
314 44843290.5491097
315 17662084.1308735
316 13174679.3337463
317 20982846.4375154
318 8250972.95362743
319 8794094.89614557
320 11091128.2705103
321 10198467.5096854
322 9086121.67184957
323 14362268.7169022
324 16329134.9781652
325 13750535.9549945
326 11382569.8314471
327 17125771.2318138
328 20788265.8918331
329 12273145.4465769
330 11447026.479076
331 15058455.3104751
332 13503494.7808743
333 11144051.1079645
334 12118477.6168008
335 13624503.5919066
336 12672039.4848173
337 12626211.8433153
338 14178832.9826336
339 7886363.84339595
340 10624582.5312304
341 15310292.5122736
342 13503767.3999144
343 8116779.60179909
344 9316079.03040858
345 8564533.15469158
346 7335240.21230509
347 11057895.2340894
348 19691104.1645526
349 8846319.84530781
350 15898409.677898
351 18630890.2447333
352 17821665.4721866
353 10363430.2031629
354 12370679.8377123
355 11194297.4555534
356 14730442.9982773
357 9791810.57096562
358 9143555.20848592
359 11020451.1868161
360 9464933.64752774
361 11411937.9602033
362 11667331.1049705
363 8118550.91811559
364 6257126.93871824
365 10274787.087595
366 6370660.71929226
367 10450917.585455
368 10639142.0209318
369 8038480.17288477
370 7924632.3550491
371 7785007.86098587
372 7069431.11539175
373 8624729.02183302
374 9425229.64304384
375 12547468.2007871
376 17138830.4380864
377 13518590.6747026
378 8546527.68631615
379 6069109.54165559
380 4087722.08827614
381 6084096.96151443
382 8230671.92742219
383 7601284.1181462
384 8724136.40097314
385 8951573.57994605
386 9242279.41758871
387 8257625.47708441
388 8429984.80435794
389 8255228.67618069
390 9782020.14277939
391 5973390.62666014
392 7012149.69406604
393 5909892.78787048
394 7343265.33272353
395 6028932.79126561
396 6334619.71841604
397 5024898.71271853
398 5739990.13948472
399 6879890.58177681
400 5174624.01074264
401 5178260.81111684
402 4138222.70165893
403 5374571.1043037
404 7925891.10819822
405 7023035.11260027
406 9221072.52722329
407 6164032.65090576
408 4962108.48264212
409 5268639.9215534
410 7616472.65962418
411 7314534.11131214
412 3856353.63926288
413 5870642.7335789
414 5433048.1496416
415 7868760.92404132
416 7882230.63849655
417 4876046.95213273
418 4537242.94425931
419 2970930.56034695
420 3341268.31697037
421 4221795.75218834
422 4909468.05595287
423 6082041.48179708
424 4615837.23137485
425 3898770.02604107
426 4742259.02525468
427 3960391.74814353
428 3551197.51863669
429 4701194.41553602
430 3830160.06089725
431 7069688.00915394
432 4447407.47669873
433 3928489.20445722
434 4439675.13183228
435 3595309.69053955
436 5387815.03425416
437 5633952.17061659
438 4095266.22313538
439 4130298.5525599
440 4297211.10850723
};


\addlegendentry{Test data}
\addlegendentry{LMCGP mean}
\addlegendimage{line width=5pt,draw=blue,opacity=0.5}
\addlegendentry{$95\%$ LMCGP CI}

\end{axis}

\end{tikzpicture}

	\end{figure}
	\vspace{-1.3em}
	\begin{block}{}
		Test data for reservoirs T, A, and I (top to bottom) in one day ahead LMCGP model prediction. The model more accurately follows the peaks due its complex behavior.
	\end{block}
\end{frame}



\begin{frame}{To Conclude}
		\justifying
		\begin{block}{}
		The LMCGP effectively captures shared features and dynamics for multi-output tasks. However, increasing the number of independent GPs beyond a threshold leads to instability.
		\end{block}
		
		\begin{block}{}
		The lengthscale matrix and task dependency coefficients, $a_{d,q}$, provide critical insights into feature selection, with some GPs specializing in specific tasks and others covering a broader range of outputs.
		\end{block}
		
		\begin{block}{}
		To improve optimization performance, using Adam + NG optimizer proved superior to traditional methods, leading to more robust results.
		\end{block}
		
		\begin{block}{}
		The LMCGP outperformed the IGP in terms of NLPD and MSLL across all horizons, emphasizing the benefits of task dependency modeling.
		\end{block}
		
		\begin{block}{}
		The LMCGP's forecasting ability showed stronger learning of complex patterns by leveraging data from multiple tasks.
		\end{block}
\end{frame}
