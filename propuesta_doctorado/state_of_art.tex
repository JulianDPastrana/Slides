\section{Estado del arte}

Diversos modelos de aprendizaje automático han sido propuestos para apoyar el diagnóstico del TDAH. Por ejemplo, en \cite{10.1007/978-3-030-89691-1_41} se implementó un modelo basado en un regresor logístico que utiliza la transformada wavelet para la clasificación de registros electroencefalográficos, permitiendo resaltar características relevantes. Asimismo, en \cite{Vaidya202051} se han empleado máquinas de soporte vectorial, las cuales ofrecen resultados interpretables para orientar las decisiones clínicas. Además, en \cite{Tor2021} se presentó un sistema automatizado de clasificación para diferenciar entre TDAH y trastorno de conducta, utilizando técnicas de descomposición para extraer características discriminatorias.


Por otro lado, en \cite{Mao20191} se utilizaron modelos de aprendizaje profundo con características espacio-temporales, combinando capas convolucionales y de memoria LSTM, lo que sugiere una mejora respecto a los métodos tradicionales de evaluación del TDAH. Además, en \cite{Vahid2019} se empleó una arquitectura basada en EEGNet, alcanzando una precisión del 83\% en la detección de TDAH en comparación con sujetos de control, aunque se reportaron dificultades para diferenciar entre los distintos subtipos de TDAH. Por último, en \cite{AHMADI2021102227} se propone una red convolucional profunda para extraer características discriminatorias y distinguir entre dos subtipos de TDAH, logrando una precisión cercana al 99.46\%.

Por su parte, los modelos fundacionales emergen como herramientas de inteligencia artificial preentrenadas a gran escala, capaces de adaptarse a tareas específicas y, en consecuencia, de posibilitar diagnósticos personalizados en el ámbito clínico. En \cite{Zhang2024} se ilustra un amplio espectro de modelos fundacionales aplicados a la medicina, evidenciando su potencial en aplicaciones de salud basadas en imágenes. Asimismo, en \cite{Wang2023} se proponen estos modelos para la clasificación de imágenes provenientes de múltiples fuentes, logrando un rendimiento aceptable y eficiencia en términos de costo para aplicaciones clínicas. Además, en \cite{Abbaspourazad2024} se demuestra que el empleo de aprendizaje auto-supervisado aplicado a grandes volúmenes de datos de fotopletismografía y electrocardiograma permite entrenar modelos fundacionales capaces de extraer información relevante sobre el estado de salud de los usuarios. Estos modelos proporcionan biomarcadores digitales adicionales que facilitan la detección temprana del TDAH y el seguimiento de sus comorbilidades.

Los procesos gaussianos han sido ampliamente utilizados en diversos estudios para mejorar la precisión en el diagnóstico y la comprensión del TDAH, aprovechando sus propiedades no paramétricas y estocásticas. Por ejemplo, en \cite{Lim2013} se implementó un clasificador basado en este modelo, el cual empleó datos relativos al volumen de la materia gris para discriminar entre individuos sanos, alcanzando una precisión cercana al 79.3\%. De manera similar, en \cite{Hart20143083} se aplicó un clasificador fundamentado en procesos gaussianos para predecir el diagnóstico individual de TDAH mediante un análisis de patrones multivariantes durante una tarea de atención, logrando una precisión global del 77\%. Estos resultados han permitido establecer correlaciones entre la severidad de los síntomas y el diagnóstico de TDAH, contribuyendo a una mejor comprensión de sus manifestaciones clínicas.