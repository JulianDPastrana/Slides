\section{Estado del arte}

Diversos modelos de aprendizaje automático han sido propuestos para apoyar el diagnóstico del TDAH. Por ejemplo, en \cite{10.1007/978-3-030-89691-1_41} se implementó un modelo basado en un regresor logístico que utiliza la transformada wavelet para la clasificación de registros electroencefalográficos, permitiendo resaltar características relevantes. Asimismo, en \cite{Vaidya202051} se han empleado máquinas de soporte vectorial, las cuales ofrecen resultados interpretables para orientar las decisiones clínicas. Además, en \cite{Tor2021} se presentó un sistema automatizado de clasificación para diferenciar entre TDAH y trastorno de conducta, utilizando técnicas de descomposición para extraer características discriminatorias.


Por otro lado, en \cite{Mao20191} se utilizaron modelos de aprendizaje profundo con características espacio-temporales, combinando capas convolucionales y de memoria LSTM, lo que sugiere una mejora respecto a los métodos tradicionales de evaluación del TDAH. Además, en \cite{Vahid2019} se empleó una arquitectura basada en EEGNet, alcanzando una precisión del 83\% en la detección de TDAH en comparación con sujetos de control, aunque se reportaron dificultades para diferenciar entre los distintos subtipos de TDAH. Por último, en \cite{AHMADI2021102227} se propone una red convolucional profunda para extraer características discriminatorias y distinguir entre dos subtipos de TDAH, logrando una precisión cercana al 99.46\%.



\subsection{Aplicaciones de Biosensores en el Diagnóstico del TDAH}
% Overview of how EEG and other biosignals are used for ADHD diagnosis.

\subsection{Aprendizaje Automático y Profundo en el Análisis de Biosenales}
% Review of machine learning and deep learning applications in biosignal analysis.

\subsection{Modelos Fundacionales y Aprendizaje por Transferencia}
% Explanation of foundation models and their advantages in transfer learning.

\subsection{Procesos Gaussianos para la Modelización de la Incertidumbre}
% Discussion on Gaussian processes and their role in capturing uncertainty.

\subsection{Identificación de Brechas y Oportunidades de Innovación}
% Summary of gaps in current research and potential innovation areas.

\subsection{Conclusiones del Estado del Arte}
% Overall summary and justification for the proposed approach.
