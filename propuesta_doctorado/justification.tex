\section{Justificación}

El Plan Nacional de Desarrollo 2022-2026 (PND), Colombia Potencia Mundial de la Vida, busca trabajar mediante  un enfoque preventivo y predictivo, estrategias orientadas a promover la capacidad de agencia y cuidado de la salud. En su sección, Salud y bienestar para las juventudes, propone fomentar el bienestar mental, con el objetivo de abordar las causas que inciden en la salud mental de los jovenetes y fomentar las prácticas profesionales que potencien su acompañamiento.

En Colombia, el CONPES 3992 establece estrategias para la promoción de la salud mental, incluyendo la investigación y la integración de sistemas de información en el área. En particular, el Departamento Nacional de Planeación (DNP) desarrolla un modelo predictor de determinantes sociales incidentes en la salud mental para orientar políticas de promoción. Asimismo, el Ministerio de Salud y Protección Social promueve estrategias para la atención en salud e inclusión social.

Desde el ámbito de la salud pública, los trastornos mentales son considerados una prioridad mundial debido a sus implicaciones en el desarrollo social. Uno de los principales trastornos del neurodesarrollo es el Trastorno por Déficit de Atención con Hiperactividad (TDAH), que usualmente persiste hasta la adultez y se caracteriza por inatención e hiperactividad \cite{APA_TDAH}. Su pronóstico incluye impulsividad, fracaso escolar, comportamientos antisociales e incluso delincuencia. El TDAH presenta subtipos según su gravedad: leve, moderado o grave. Su etiología está relacionada con factores genéticos, neurobiológicos, prenatales, perinatales y posnatales, cuya interacción contribuye a su manifestación clínica \cite{Saxena2013}. 

Aunque la prevalencia mundial del TDAH oscila entre el 5\% y 10\% en niños en edad escolar \cite{GalianaSimal2020}, en Colombia se han reportado valores del 15-17\%, ubicándose entre las cifras más altas de Latinoamérica \cite{Pineda2003}. Se estima que la prevalencia por rango de edad es del 62\% en niños entre 6 y 11 años, debido a alteraciones en la maduración cerebral, lo que se traduce en un retraso de entre 2 y 3 años con respecto a niños sin la condición \cite{LlanosLizcano2019}.

El diagnóstico del TDAH se basa en los criterios del DSM-5, a partir de información aportada por el paciente o su entorno sobre síntomas percibidos de inatención, impulsividad e hiperactividad, los cuales deben mantenerse durante al menos seis meses, no ser acordes al nivel de desarrollo y afectar el desempeño social, académico o laboral. Sin embargo, la ponderación subjetiva de los síntomas conlleva restricciones como discrepancias en las percepciones entre padres, maestros y profesionales \cite{Narad2015}, falta de uniformidad en la adopción de guías clínicas y criterios insuficientes del DSM-5 \cite{Eslami2021,Yeh2012}. Además, la cobertura diagnóstica es limitada: solo el 7.4\% de los casos recibe confirmación mediante entrevista psiquiátrica estructurada, y solo el 6.6\% recibe tratamiento \cite{Polanczyk2014}, lo que genera una alta tasa de diagnósticos erróneos en la población infantil de la región \cite{DeLaViudaSuarez2021}.

Paralelamente, se han desarrollado plataformas digitales e interfaces avanzadas para la interacción con pacientes, las cuales pueden facilitar el diagnóstico y tratamiento de diversas patologías. Entre estas, los videojuegos han cobrado relevancia como herramientas asistenciales, debido a su capacidad para mejorar la flexibilidad cognitiva y la memoria de trabajo, aspectos clave en la atención y concentración. La Política Nacional de Ciencia, Tecnología e Innovación 2021–2030 fomenta el desarrollo de tecnologías innovadoras en este ámbito. En este contexto, los grupos de investigación en Automática, Control y Procesamiento Digital de Señales, Neuroaprendizaje y el Centro BIOS han centrado su trabajo en bioseñales, visión por computador, modelamiento estocástico y aprendizaje automático, con aplicaciones en salud.

En conclusión, es necesario fortalecer los programas de salud mental infantil para detectar a tiempo los factores asociados al TDAH y mejorar los indicadores de prevalencia. Además, resulta importante optimizar la calidad en el diagnóstico, evaluación e intervención de los pacientes. En este sentido, esta propuesta se enfoca en enfrentar estos desafíos, reconociendo que se trata de un problema de salud pública de gran magnitud, mediante el desarrollo de una herramienta de apoyo para el diagnóstico del TDAH basada en señales biológicas, que no solo contribuya a generar nuevos conocimientos, sino que también impulse la innovación en los sectores de la salud, la ingeniería y la ciencia.
