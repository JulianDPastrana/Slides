
\section{Planteamiento del problema y pregunta de investigación}

Los trastornos mentales afectan la cognición, el comportamiento y las emociones de millones de personas en el mundo. Se estima que alrededor de 350 millones de individuos padecen algún trastorno mental \cite{Dehghan-Bonari2023}. La impulsividad, como característica sintomática central, está presente en diversas condiciones mentales, entre las que destacan el Trastorno Negativista Desafiante (TND), el Trastorno de Desregulación del Estado de Ánimo Disruptivo (TDEAD) y el Trastorno Bipolar (TB). Dentro de este grupo, el Trastorno por Déficit de Atención e Hiperactividad (TDAH) representa un desafío clínico significativo, con una prevalencia aproximada del 5\% en población infantil y adolescente, y del 2.5\% en adultos, constituyendo así uno de los trastornos neuropsiquiátricos más comunes durante la infancia y la adolescencia. Estos trastornos no solo dificultan el rendimiento académico y laboral \cite{Ayano2020}, sino que también aumenta el riesgo de presentar otros problemas psiquiátricos, conductas delictivas y adicciones \cite{Faraone2015}.

La detección e intervención temprana de la impulsividad es esencial para brindar un tratamiento oportuno y efectivo \cite{Kivumbi2019}. Sin embargo, las técnicas diagnósticas actuales presentan limitaciones importantes: muchas pruebas requieren un seguimiento prolongado \cite{Zhou2015-bg}, pueden estar sujetas a interpretaciones subjetivas \cite{LOHANI2023111689}, omitirse debido a la gran variación en las presentaciones clínicas y síntomas \cite{adam2013mental}, y muchos criterios resultan ineficientes para el diagnostico en adultos \cite{Sibley21042021}. Además, el acceso a un tratamiento clínico y seguimiento es frecuentemente restringido debido a la falta de recursos o la escasez de especialistas \cite{Asherson2022,Pallanti2020}.

El avance en el desarrollo de modelos de aprendizaje automático ha permitido la creación de herramientas que apoyan el diagnóstico y tratamiendo de diversas enfermedades neurológicas y transtornos mentales \cite{article}. Generalmente, estos modelos utilizan señales electroencefalográficas (EEG) para clasificar registros según la presencia o ausencia dela enfermedad, mediante un enfoque de clasificación binaria. No obstante, la naturaleza no estacionaria y la complejidad inherente a las señales EEG complican su análisis utilizando modelos tradicionales \cite{KHARE2023106676,Loh2024}.

El uso de técnicas de aprendizaje profundo se presenta como una alternativa prometedora, ya que permite extraer características abstractas de las señales EEG mediante una gran cantidad de parámetros \cite{Jahani2024,Garcia-Argibay2023,Moghaddari2020}. No obstante, el entrenamiento de estos modelos demanda grandes volúmenes de datos etiquetados en un entorno supervisado, lo cual frecuentemente es costoso o inviable, y su potencial mejora se ve limitada por la falta de interpretabilidad \cite{emam2021statedatacomputervision,healthcare11030285}.

Para superar esta limitación, han surgido los modelos fundacionales, que permiten preentrenar con conjuntos de datos no etiquetados y, posteriormente, ajustar el modelo para tareas específicas utilizando una cantidad significativamente menor de datos etiquetados \cite{Mathew2024,Abbaspourazad2023}. Este enfoque aprovecha la estructura subyacente presente en datos sin clasificar, permitiendo que el modelo aprenda representaciones generales robustas. Gracias a este preentrenamiento, el modelo puede transferir conocimientos a tareas concretas, lo que reduce la dependencia de grandes volúmenes de datos anotados y agiliza el proceso de adaptación a nuevos escenarios. 

Un aspecto relevante en el preentrenamiento de estos modelos es la variabilidad inherente a los conjuntos de datos no etiquetados. Estas bases de datos pueden diferir en los estándares de adquisición, lo que se manifiesta en la presencia de valores ausentes, variaciones en la tasa de muestreo, diferencias en el número de canales e incluso problemas de generalización al cambiar de sujetos o sesiones (conocido como Dataset Shift) \cite{APICELLA2024128354}. Descartar estos datos supondría la pérdida de información potencialmente valiosa; por ello, es importante desarrollar un modelo fundacional capaz de gestionar eficazmente dichas inconsistencias y maximizar el aprovechamiento de los datos disponibles.

Lo anterior también induce una incertidumbre en el modelo, por lo que en el ámbito clínico no es suficiente obtener una predicción puntual en el diagnóstico, sino también disponer de una medida de la incertidumbre asociada que cuantifique el nivel de riesgo a la hora de tomar una decisión basada en la inferencia del modelo. En este sentido, los procesos Gaussianos ofrecen una solución al generar una distribución de probabilidad predictiva mediante un enfoque Bayesiano en lugar de una predicción puntual \cite{GUTIERREZBECKER2018246,Jahani2024}. Sin embargo, la integración de procesos Gaussianos con modelos fundacionales aún es un área poco explorada y con un gran potencial de mejora.

\textbf{Pregunta de investigación:}  
¿Cómo desarrollar un modelo fundacional estocástico para la inferencia a partir de señales biológicas que integre procesos Gaussianos, para el soporte a la práctica clínica mediante el análisis de señales EEG, considerando la variabilidad e inconsistencias presentes en los conjuntos de datos?
