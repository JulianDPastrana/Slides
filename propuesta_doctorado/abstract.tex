\begin{abstract}
	El diagnóstico de trastornos mentales y algunas terapias tradicionalmente se basan en evaluaciones clínicas, entrevistas y pruebas estandarizadas, lo cual puede conllevar interpretaciones subjetivas y requerir un seguimiento prolongado. Además, la generación de modelos de inteligencia artificial que apoyen la práctica clínica suele depender de grandes volúmenes de datos etiquetados, restringe su trasferencia a ambientes reales. En este contexto, el presente proyecto propone el desarrollo de un modelo fundacional estocástico basado en procesos Gaussianos para el entrenamiento de modelos de Inteligencia Artificial (IA) a partir de múltiples bases de datos de señales electroencefalográficas (EEG) con características de adquisición variantes, que integre datos no etiquetados mediante técnicas de autoaprendizaje y datos etiquetados para su ajuste fino. Este enfoque no solo busca extraer representaciones generales a partir de la estructura de las señales biológicas, sino que también incorpora un componente que modela la incertidumbre en las predicciones, ofreciendo una medida cuantitativa de la confiabilidad del modelo. Se espera que los resultados impulsen la generación de nuevo conocimiento y la innovación en los ámbitos de la salud, la ingeniería y la ciencia.
\end{abstract}
