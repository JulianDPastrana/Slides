\begin{abstract}
	El diagnóstico de trastornos mentales y algunas terapias tradicionalmente se basan en evaluaciones clínicas, entrevistas y pruebas estandarizadas, lo cual puede conllevar interpretaciones subjetivas y requerir un seguimiento prolongado. Además, la generación de modelos de inteligencia artificial que apoyen la práctica clínica suele depender de grandes volúmenes de datos etiquetados, restringe su trasferencia a ambientes reales. En este contexto, el presente proyecto propone el desarrollo de un modelo fundacional estocástico basado en procesos Gaussianos para el entrenamiento de modelos de Inteligencia Artificial (IA) a partir de múltiples bases de datos de señales electroencefalográficas (EEG) con características de adquisición variantes, que integre datos no etiquetados mediante técnicas de autoaprendizaje y datos etiquetados para su ajuste fino. Este enfoque no solo busca extraer representaciones generales a partir de la estructura de las señales biológicas, sino que también incorpora un componente que modela la incertidumbre en las predicciones, ofreciendo una medida cuantitativa de la confiabilidad del modelo. Se espera que los resultados impulsen la generación de nuevo conocimiento y la innovación en los ámbitos de la salud, la ingeniería y la ciencia.\\
	
	\begin{center}
		\textbf{Abstract}
	\end{center}
	
	The diagnosis of mental disorders and certain therapies have traditionally relied on clinical evaluations, interviews, and standardized tests, which can lead to subjective interpretations and require prolonged follow-up. Moreover, the development of artificial intelligence models that support clinical practice typically depends on large volumes of labeled data, limiting their transfer to real-world settings. In this context, the present project proposes the development of a stochastic foundational model based on Gaussian processes for training AI models using multiple databases of electroencephalographic (EEG) signals with varying acquisition characteristics. This model integrates unlabeled data through self-learning techniques and labeled data for fine-tuning. This approach not only seeks to extract general representations from the structure of biological signals but also incorporates a component that models prediction uncertainty, providing a quantitative measure of the model's reliability. It is expected that the results will spur the generation of new knowledge and innovation in the fields of health, engineering, and science.
	
\end{abstract}
