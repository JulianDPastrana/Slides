\begin{abstract}
	El diagnóstico del TDAH tradicionalmente se basa en evaluaciones clínicas, entrevistas y pruebas estandarizadas, lo cual puede conllevar interpretaciones subjetivas y requerir un seguimiento prolongado. Además, la generación de modelos de inteligencia artificial que apoyen el diagnóstico asistido suele depender de grandes volúmenes de datos etiquetados, lo que constituye un cuello de botella en la práctica clínica. En este contexto, el presente proyecto propone el desarrollo de un modelo fundacional estocástico basado en procesos Gaussianos para la clasificación de señales biológicas relacionadas con registros EEG, que integre datos no etiquetados mediante técnicas de autoaprendizaje y datos etiquetados para su ajuste fino. Este enfoque no solo busca extraer representaciones generales robustas a partir de la estructura subyacente de las señales EEG, sino que también incorpora un componente que modela la incertidumbre en las predicciones, ofreciendo una medida cuantitativa de la confiabilidad diagnóstica. Se espera que los avances resultantes impulsen la generación de nuevo conocimiento y la innovación en los ámbitos de la salud, la ingeniería y la ciencia, contribuyendo a mejorar la precisión diagnóstica y, en última instancia, la calidad de vida de los pacientes.
\end{abstract}
