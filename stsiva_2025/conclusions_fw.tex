\section{Conlusions and Future Work}

\begin{frame}{Conlclusions}
	
	\begin{block}{}
	\begin{itemize}
		\item The Chained LMC GP incorporates \textcolor{BrandTeal}{\textbf{latent Gaussian processes to model likelihood parameters}}, enabling more flexible and adaptive uncertainty quantification. 
		\item We evaluated the approach on \textcolor{BrandTeal}{\textbf{daily streamflow time series from 23 Colombian reservoirs}}, framing each reservoir as a related task in a multi-output forecasting scenario.  
		\item Across all forecasting horizons, the Chained LMC GP achieved \textcolor{BrandTeal}{\textbf{significant improvements over the standard LMC GP}}. 
		\item Its main advantage is its ability to \textcolor{BrandTeal}{\textbf{model data noise with a GP}}, which better explains peaks and smooth trends." 
	\end{itemize}
	\end{block}
\end{frame}

\begin{frame}{Future Work}
	\begin{block}{}
		In future work, we plan to extend this research in three ways.
		
		\begin{itemize}
			 \item Validate the Chained LMC GP with non-Gaussian likelihood functions for modeling amplitude-constrained time series. 
			 \item Enhance the LMC GP by introducing non-instantaneous mixing, similar to a process convolution, which will generalize the Chained model to better capture temporal dependencies.
			 \item Incorporate a deep learning approach, such as deep Gaussian Processes, to capture more complex patterns and improve prediction accuracy.
		\end{itemize} 
	\end{block}
\end{frame}

\begin{frame}{Acknowledgment}
	\begin{block}{}
		\begin{itemize}
			\item Thanks to the \emph{Doctorate in Engineering}, program from the Universidad Tecnológica de Pereira, for their support.
			\item This work was carried out under grants of the project 1110-890-82356, funded by MINCIENCIAS.
			\item Acknowledgment to the Automatics Research Group from the Universidad Tecnológica de Pereira for their invaluable support in the development of the research.
		\end{itemize}
	\end{block}
\end{frame}
