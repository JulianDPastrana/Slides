\section{Conlusions and Future Work}

\begin{frame}{Conlclusions}
	
	\begin{block}{}
	\begin{itemize}
		\item The Chained LMC GP incorporates \textcolor{BrandTeal}{\textbf{latent Gaussian processes to model likelihood parameters}}, enabling more flexible and adaptive uncertainty quantification. 
		\item We evaluated the approach on \textcolor{BrandTeal}{\textbf{daily streamflow time series from 23 Colombian reservoirs}}, framing each reservoir as a related task in a multi-output forecasting scenario.  
		\item Across all forecasting horizons, the Chained LMC GP achieved \textcolor{BrandTeal}{\textbf{significant improvements over the standard LMC GP}}. 
		\item Its advantage lies in the \textcolor{BrandTeal}{\textbf{ability to model input-dependent noise}}, which enhances the exploitation of patterns such as periodic trends in the data. 
	\end{itemize}
	\end{block}
\end{frame}

\begin{frame}{Future Work}
	\begin{block}{}
		In future work, we plan to extend this research in three ways.
		
		\begin{itemize}
			 \item Validate the Chained LMC GP with non-Gaussian likelihood functions for modeling amplitude constrained time-series. 
			 \item Aim to enhance the LMC GP by introducing non-instantaneous mixing, similar to a process convolution, which will generalize the Chained model for extracting time dependencies.
			 \item Intend to incorporate a deep learning approach, such as deep Gaussian Processes, to capture more complex patterns and improve prediction accuracy.
		\end{itemize} 
	\end{block}
\end{frame}

\begin{frame}{Acknowledgment}
	\begin{block}{}
		\begin{itemize}
			\item Gratitude to the \emph{Doctorado en Ingenier\'ia}, graduate program of the Universidad Tecnológica de Pereira, for their support.
			\item Special thanks to the project 111089082356, funded by MINCIENCIAS.
			\item Acknowledgment to the Automatics Research Group of the Universidad Tecnológica de Pereira for their invaluable support in the development of the research.
		\end{itemize}
	\end{block}
\end{frame}
